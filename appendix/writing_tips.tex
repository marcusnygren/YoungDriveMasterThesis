Förstadiet}

Analysera. Samla stoff. Sortera. Strukturera och planera.

\textbf{Bestäm syftet med rapporten}
Vad ska rapporten framhäva, och vem skriver jag för?

1. Grovsortering av källmaterial
2. Ställ upp preliminär disposition av den tänkta rapporten
3. Huvudsyftet med rapporten
4. Dela in i avsnitt och ge innehållsmässigt illustrativa rubriker.

OBS: ägna inte för mycket tid åt det språkliga. Det kan jag komma tillbaka till senare.

\subsection{Skrivstadiet}

Formulera

När du börjar skriva själva rapporten behöver du inte börja från början.

Skriv de avsnitt som du har en god bild av just för tillfället och spara de oklara avsnitten till senare. Du kommer antagligen att få revidera dispositionen flera gånger under skrivandet – det är ingen nackdel, utan snarare ett naturligt led i skrivprocessen. När du har ett utkast till hela rapporten kan du sätta igång med efterarbetet. Användandet av formatmallar underlättar skrivarbetet.

\subsection{Efterstadiet}

Bearbeta. Språkgranska. Korrekturläsa. Trycka.

\subsection{Språkliga råd}

%* Använd ett så enkelt och entydigt språk som möjligt med hänsyn till syftet och målgruppen. Variera meningslängden.  Om du upptäcker att vissa meningar blir väl långa kan du försöka att bryta upp dem i mindre delar så att bara en huvudtanke uttrycks i varje mening. Skriv hellre i aktiv än i passiv form.

%* Undvik vaga, flertydiga, oprecisa och betydelsetomma ord och uttryck.

%* Skriv fullständigt. Om du argumenterar för en viss sak måste argumentationen vara fullständig, logisk och sammanhängande. Undvik att bli alltför detaljerad om det inte behövs. Oväsentlig information leder lätt in läsaren på fel spår.

%* Använd samma tempus, gärna dåtid, förutom när det gäller kunskap som inte är tidsbunden och handlingar som inte är avslutade (till exempel: processen pågår) då du bör använda presens.

\textbf{Tips}

% Ett gott råd är att vara sparsam med förkortningar i brödtexten

% Tänk på att alltid förklara en förkortning eller en ny term första gången som den används, om den inte är allmänt vedertagen.

% Du kanske inte får flyt i texten i första versionen och upptäcker när du läser igenom arbetet att något saknas eller att textflödet haltar. En orsak kan vara att dispositionen inte är den rätta. Du måste kanske stuva om ordningen mellan avsnitten eller mellan de olika styckena i ett avsnitt. En annan orsak kan vara att meningarna står som ”isolerade öar” utan kontakt med den övriga texten.

% Gör inte indrag direkt efter rubriker, punktlistor, bilder och så vidare

\textbf{Citeringar}

% Citera personer med (”), inte tumtecken (")

% Du får alltså inte ändra språket i ett citat, även om det finns grammatik- eller stavfel i ursprungstexten.

% Utelämnas enstaka ord inom ett citat markeras de med hjälp av 3 punkter ... med blanksteg före och efter punkterna. Om en hel mening eller flera meningar utelämnas används 3 tankstreck – – –. När kravet på tydlighet är stort kan de 3 punkterna eller 3 tankstrecken omges med hakparentes eller snedstreck