\section{Original Time Plan and Activities}
% en tidplan för examensarbetets genomförande inklusive planerade datum för halvtidskontroll och framläggning

\subsection{Before Uganda}

\begin{center}
    \begin{tabular}{ | l | p{10cm} |}
    \hline
    Week & Focus \\ \hline
    2 & Workshop with Lena Tibell and Konrad Schönborn on Research questions \& Proposal of method. \\ \hline
    3 & Start writing "Planeringsrapport".
    Study interaction design
    via guest lecture Jonas Löwgren,
    and reading the book "Thoughtful Interaction Design". \\ \hline
    4 & Interview with Take Aanstoot, Social entrepreneur in Kenya. Submission "Planeringsrapport". Education day in Service design in Stockholm (by Expedition Mondial). Meet Joachim Svärdh about Entrepreneurship research. \\ \hline
    5 & Approval "Planeringsrapport" with Camilla Forsell. Meeting with Lena Tibell and Konrad Schönborn (2016-02-02). Travel to Uganda. \\ \hline

    \end{tabular}
\end{center}

\subsection{In Uganda}

Times specified are in local time to where the master thesis was done at the time. Uganda time (EAT - Eastern Africa Time) is 2 hours forward of Swedish time (CET - Central European Time). Meetings with Swedish partners are generally done via Skype, where Uganda meetings are preferably done in person. Note that during all of this time, writing the master thesis will progress. After the time in Uganda, the report will be a 100\% focus. 1 day per week will be spent on report writing, including Analysis work for the meetings.

\begin{center}
    \begin{tabular}{ | l | p{10cm} |}
    \hline
    Week & Focus \\ \hline
    6 & \textbf{Cultural adaption}. Land, set up wifi, set up the apartment, learn about the YoungDrive organization, meet people. Be prepared for stomach disease. Get familiar with the transportation system in Kampala. Get familiar with the city.

    \textbf{Iteration 1}. Prepare Iteration 1 with Iliana. % sätta plan och förutsättningar för Iteration 1 (sätta i kontakt med coacher eller coach-möte, inleda kontakt om jag kan följa med ut i fält). Briefa Iliana i min projektplan.
    Start-up meeting with partners. % Uppstartsmöte. Ta del av deras kunskap, dröm, målbild - stämma av så att vi står på samma sida. Lyssna på områden inför frågeguiden.  % Lärdom från Kenya, att boka möte med Iliana och Josefina, projektledare, och gå igenom noggrant med stakeholders
    Start report writing: analyze, collect material, sort, structure and plan.
    % Analysera, samla stoff, sortera, strukturera och planera
    \\ \hline
    7 & \textbf{Iteration 1}. Prepare Interactions. Analyze Start-up meeting with partners. Write on report. % hypoteser vad vi tror att utmaningen är.
    in order to create \textit{Questionairee guide}. Understand technical tools, without working on an app solution - the goal is to get familiar with the tools. % (öppen, undersökande, ej specifika frågor untan diskusionsunderlag). Göra backend.
    \\ \hline
    8 & \textbf{Iteration 1}. Travel for Interactions. Do 8 face-to-face interviews, with no digital focus, hypotethical situations. Do minimum 2 field visits to understand the coach's situation, ideally living in Kamuli or Tororo a couple of days. This is a good opportunity to learn coaches how the tables and smartphones work.  \\ \hline % "jaså, det är så här det går till". Jag kommer få insikten "vänta nu, är det så här det går till?".

    9 & \textbf{Iteration 1}. Analysis \& Compilation. Thursday: Expert meeting (March 3rd, 6-7 PM). Friday: Partner meeting (March 4th, 11-12 AM).

    \textbf{Iteration 2}. Determine Needs. Ideation. Create low-fi Trigger material (pen and paper) and determine what the hi-fi (digital app) material should be.
    % Några coacher kanske verkar självgående, andra behöver mycket stöd. Det verkar som om den behöver fungera offline och online. Utifrån analysen skapar jag en egen behovsbeskrivning, sedan kreations-fas då jag kommer på idéer. Sedan gör jag ett low hi-fi triggermaterial, och det detaljerade triggermaterialet min digitala lösning.
    \\ \hline
    10 & \textbf{Iteration 2}. Design and Develop the hi-fi trigger material. \textit{Half-time check-up with examiner.} \\ \hline
    11 & \textbf{Iteration 2}. Interactions, control group \#1 \& \#2. \\ \hline
    12 & \textbf{Iteration 2}. Interactions, control group \#1 \& \#2. % Några gör jag på huvudkontoret med 8 personer. Några tar jag med mig trigger-materialet till, ut på fält.
    \\ \hline
    13 & \textbf{Vacation}. \\ \hline
    14 & \textbf{Iteration 2}. \textit{Analysis \#2} (What choices needs to be made? What path should be taken? Start formulate Customer path. If needed, document how people see apps, document limitations, document experience needs, document risks.) % Kundresa: ("What does a day/session look like for a coach? What are their basic needs?").
    \& Compilation. Thursday: Expert meeting (April 7th, 4 PM). Friday: Partner meeting (April 8th, 11-12 AM). Continued Development Creative Brief. Determine what actions needs to be taken outside of the development of the app. Create Behovsgrupper. \\ \hline

    15 & \textbf{Iteration 3}. Develop and Modifications phase. \\ \hline
    16 & \textbf{Iteration 3}. Develop and Modifications phase. Interactions: App Tests with Interviews \& Measurements (with time allocated for late arrivals and missing participants). \\ \hline
    17 & \textbf{Iteration 3}. Interactions: App Tests with Interviews \& Measurements.
    Analysis \& Compilation.
    Friday: Partner meeting (April 29th, 11 AM) \& Expert meeting (April 29th, 4 PM). \\ \hline
    18 & \textbf{Final analysis}. Finalize the app. Travel back to Sweden. \\ \hline
    \end{tabular}
\end{center}

\subsection{After Uganda}
\begin{center}
    \begin{tabular}{ | l | p{10cm} |}
    \hline
    Week & Focus \\ \hline
    19 & Write on Master thesis report. Attend Auscultations. \\ \hline
    20 & Write on Master thesis report. Attend Auscultations. \\ \hline
    21 & Write on Master thesis report. Attend Auscultations. Find opponent for Master thesis. \\ \hline
    22 & Submission of report to examiner, after approval by supervisor. Examiner decides on date and time for presentation. Send report to opponent, and get the opponent's report. \\ \hline
    \end{tabular}
\end{center}

\subsection{After Semester}

\begin{center}
    \begin{tabular}{ | l | p{10cm} |}
    \hline
    Week & Focus \\ \hline
    35 & Presentation of Master thesis, with supervisor, examiner and opponent. Hand over publication approval to the administrator. \\ \hline
    36 & Opposition of another person's Master thesis. \\ \hline
    37 & Do changes to report if requested. Upload report to X-sys for approval (within 10 days). Write Reflections document and submit on X-sys within the 10 days. Publish master thesis in X-sys. \\ \hline
    \end{tabular}
\end{center}
