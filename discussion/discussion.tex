\chapter{Discussion}\label{cha:discussion}

% %Ingen rapport kan göra anspråk på att ha löst problemen inom ett område på ett uttömmande sätt. Därför är det viktigt att visa på vilket sätt du själv eller  andra kan använda dina resultat i andra studier i framtiden. Den avslutande  diskussionen blir till sin karaktär mer subjektiv, men se till att du inte  spekulerar alltför vilt.

\section{Discussion of method}

\subsection{Consequences of involving end users and stakeholders throughout the whole process}

\subsubsection{Product Benefit from involving users and stakeholders}
Design thinking, human-centered design and service design, has been proven to be crucial for the success of this project. Service design thinking and methods, gave a framework to have all of these perspectives in balance and consideration, always with the end user as the most important person.

\subsubsection{Support Benefit from inolving users and stakeholders}
The fact that the end users and stakeholders has been involved from the start, made them feel ownership of the product. This has many benefits, among others that they \textit{everyone} involved is satisfied with the \textit{final} app, since they think that their opinions and expertise has been taken into consideration and implemented. This further increases trust, and the the likelihood of them supporting future work. Even more so, the end users are more likely to use the app, as they have been co-creators of the product.

\subsubsection{Complications}
I was not a designer. I was a computer expert with social skills, now needing to design and develop an app for a cultural and socio-economic context very different from my own. 

In this regard, the technical aspect was but one. I \textit{did} need to learn how to develop hybrid apps in JavaScript that worked offline, and had an online back-end. Those was the technical demands. 

But more so, I needed to quickly become a good designer. Not mainly from a perspective of graphic design or interaction design, but \textit{how} to explore, design, and implements what the user needs from the requirements "fun, user friendly, and good for learning". The approach to learn design from these perspective was to read extensive literature, consult a diverse set of experts, and be very humble and curious in interactions with the end-users and stakeholders. 

This took me a long way, to the point where research, experiments, and constant improvements could lead to increasingly well-informed decisions.

I now have new-found skills in:
\begin{itemize}
\item etnologicy (getting to know and learn from people in a different culture)
\item human-centered design
\item design thinking
\item service design thinking
\item interaction design
\item digital learning
\item data analysis
\end{itemize}

It has placed high psychological pressure and leadership demands on me as a new designer, to:
\begin{itemize}
\item always be in charge of balancing all the different perspectives, with the end user's best in mind
\item be able to change the planned process when new learnings or opportunities emerge (leading an agile design process)
\item always implement new functionality from customer needs instead of designer or engineer bias
\item continually design and run workshops and tests suitable for the target groups
\end{itemize}

The reason why this has been especially hard, is that simultaneously to learning design and technological skills, I have been in a different cultural setting than the designer is used to. This has also been extremely rewarding, at the same time exhausting.

\section{Discussion of result}
In three months time, an app was developed with precision to the needs and context of the end users. The design has been heavily influenced by the end users, from day 1 of the project, in conjunction with relevant research, and in balance to stakeholder goals and considerations, and supervisor advice. 