\chapter{Discussion}\label{cha:discussion}

% %Ingen rapport kan göra anspråk på att ha löst problemen inom ett område på ett uttömmande sätt. Därför är det viktigt att visa på vilket sätt du själv eller  andra kan använda dina resultat i andra studier i framtiden. Den avslutande  diskussionen blir till sin karaktär mer subjektiv, men se till att du inte  spekulerar alltför vilt.

%\todo{Alla källor jag nämner, de behöver jag egentligen diskutera i Discussion-delen. Då börjar jag vrända och vida på saker, och hitta olika tankemönster = Vad som stämmer och inte stämmer i det jag sett jämfört med teorin}

The discussion section is framed by revisiting the five research questions. For each question, important aspects are considered, often comparing with the literature. This then leads to the conclusions of the master thesis, and future work.

%\section{Technical development}
\section{How is the Development Affected by the Technical Possibilities?}
%\section{What in the development has been affected by the technical possibilities?}

As devices were limited, a goal was to make the app available on as many devices as possible. Creating a hybrid app using web technologies using Meteor made the app available both as a native app on Android and iOS devices, as well as on the web. On the other hand, this is not enough: the pre-evaluation showed that only 3 out of 16 had a smartphone today \cite{youngdrive-statistics}.

As internet is accessible but expensive and often used seldom, the app does not provide rich media or simulations, but focuses instead on creative design possibilities using multiple-choice, also having cognitive load and scaffolding in mind. The interviews and observations from iteration 4 details that the coaches are happy with the user friendliness of the app, and that the training in its current form has high value for the coaches. On the other hand, images could probably be used to lower misconceptions in language, and a future wish of the coaches is to have the manuals accessible via mobile as well, which includes both text and images.

Most of the coaches have been first-time smartphone users. Letting them continuously test and co-create the app has created a tailor-made app from their needs and conditions. It may be surprising how simple design solutions using text and clear visuals can provide rich learning feedback, mentioned by Nocol \cite{nicol}. On the other hand, since it is so tailor-made, it needs to be examined if to make the app work in other countries than Uganda and Zambia, where design and technology preferences might be different.

That the app should work offline, and still be able to push quiz results when it gets online, has been a challenge. It can be hard to find good existing approaches for some technical platforms, but for Meteor plugins such as GroundDB proved very usable, since it is also automatic. In other apps in developing countries sometimes the user decides themselves when to push data, but in this case, quiz results are so small in size that it was unnecessary. This might be reconsidered in the future, for example if answers are no longer solely multiple-choice.

Below, reasons the development was negatively affected by the technical constrains are highlighted.

\subsection{Online Data Collection was Needed Earlier}
To test on all of the coaches in Uganda, it would have been preferable if data collection would have happened via the app instead of manually already in iteration 3, since there would be more than 10 test subjects, which had been the limit in Zambia. This was planned for, but technical implications with Meteor made it delayed. Done manually, not all data was recoded in iteration 3, which made it harder to draw conclusions from the quiz results. Both Lopez \cite{une-terre} and \cite{ropinski} explains how visualization techniques (like parallell coordinates) are more suitable for large data sets. This can be read more about below.

\subsubsection{Problems with Internet Access}
In day 3 of the Zambia coach training in iteration 2, iOS no longer allowed uncertified app installs from computer: you needed to have paid a license even for unreleased apps, being a "Trusted developer". This stopped the app from being able to be installed on all the iOS devices, so that only the web version could be used. Thus, only the web app was tested from Wednesday and onwards. This was a problem, as the app regularly crashed at refresh because of low internet capacity. Sometimes, it was neeed to go to the other office where there was wifi, to refresh the webpage, and go back to the location. This of course would not work for Josefina. While it was positive that these issues were found thanks to testing, valuable time testing the actual functionality of the app was lost, with less feedback for continued development as a result.

\subsection{Backwards Capability Issues} \label{backwards-capability}
Upgrading from version 1.2 to 1.3 during Iteration 3 was a good example of technical limitations. It took a lot of time, but when it was discovered that version 1.2 did not work for old Android devices, the changes needed to be reverted. In another project, new Andorid versions might have been acceptable, but here a "better" version of software was not viable.

Meteor 1.2 had several disadvantages: while it worked for all devices, it did not support React.js Meteor 1.3 was released, which promised a better developer experience, with JavaScript ES6 support, and access to Node Package Manager (npm), plus official support for React.js. In 1.2, only some npm packages had been adapted for Meteor, and tools such as Webpack could not be used.

The downsides was discovered after implementation: there were missing backward compatibility to the older of the Android devices. The backup would be the web version, but at the time of iteration 3, there were no Heroku build-pack for Meteor 1.3, making the website to crash. This was however fixed before iteration 4, which is why Meteor 1.3 was kept.

\subsection{No Time Assigned for Writing Automatic Tests}
The project would have benefited from passing automatic tests before doing user tests. While automatic tests were never written because of time constratints, since iteration 3, beta releases and production releases were seperated into different domains, using Heroku's staging environement, with a different GitHub branch for each new iteration.

Even so, doing automated tests would could have helped finding things that had worked previously but not in a new version, or finding bugs with new functionality like client-server communication. This would have made interactions with coaches more efficient, since the users would have been exposed to an app with less accidental errors.

\subsection{Difficulties Comparing Quiz Results between Iterations}
It would be highly interesting to compare the quiz results between different itreations of the app, to measure how much learning has increased. However, the educational range and knowledge between Zambia and Uganda is too large to draw such conclusions: while all of the Zambia coaches had 100\% correct answers on quiz 3 "Financial literacy" (iteration 2), the same number for Uganda was 91.8\% (iteration 4). This makes further conclusions very hard, more than from informed guesses and observations. See more about this in section \ref{future-work}. How to compare empirically measure learning effectiveness between different countries could be interesting future work.



%\section{Study development}

\section{How is the Design Affected by the Contextual Constraints, e.g. Young Entrepreneurs, Entrepreneurship Education, and Culture?} % such as X, Y, Z, \Å?

  An insight is how quickly the coaches have increased their fluency with using the smartphone and the app. Even though the design at first needs to be very simple, as long as features are introduced slowly and as intuitive, the app could become more complex over time, to the point where no compromises needs to be made.

  When it comes to design constraints in regards to overcoming cultural differences, learning from the expertise of local partners and tech companies can not be overestimated. They have been very willing to share previous mistakes, learnings and successes. This has saved a lot of time, and made the sparse amount of interactions and development time so much more efficient. Also, the value of getting to know the coaches on a first-hand basis has been greatly beneficial. The app has been designed together with them as co-creators, with a developed mutual interest and understanding, having a common goal of creating an app that works for their needs.

  The design was heavily influenced by starting the project with an iteration to truly understand the target group and context, and see the needs before starting thinking about ideas. It was good that this iteration did not have a digital focus, but even questioned if the best way to solve the needs would be an app altogether. Thanks to service design, everything in the app is informed by needs, which is an example of being a thoughtful designer \citep{lowgren} \citep{stickdorn}).

  Service design methodology allowed the design to be embraced by the contextual constraints. Since service design involves looking at the whole context, both the digital context and the physical context, it was possible to understand in what situation the app can be used, and the situation of the people using it.

  When it comes to designing for entrepreneurship, the focus for this thesis was on the educators: both the teacher, and the YoungDrive coaches. This goes in line with the insights from \cite{ruskovaara}, where the teacher seemed to be the main factor for qualitative entrepreneurship education. However, contrary to \cite{ruskovaara}, women did perform better than the male counterparts in the quiz results. Similar results have been found in other developing countries as well: from the pre-test, it is noticed that women puts more work into preparation, even if they have a lower school level. This could be one reason why their results where better.

  More so, the interviews with coaches shows that the app has been fun to use, which was a recommendation from both YoungDrive and the research from \cite{dickson} on recommendations for entrepreneurship education. Secondly, the app has focused on another best practice from \cite{dickson}, which is that the app should develop the competences as a mentor, enabler or coach. This can be noticable by interview answers such as "open's up ones mind" and that it "evaluates my performance" with "how I've understood", "how I'm performing" and that "the feedback is excellent".

  \subsubsection{From Computer Expert into a Digital Service Designer}
  The study method took the project a long way, to the point where research, experiments, and constant improvements could lead to increasingly well-informed decisions.

  New-found skills are acquired within:
  \begin{itemize}
  \item ethnology (getting to know and learn from people in a different culture)
  \item human-centered design
  \item design thinking
  \item service design thinking
  \item interaction design
  \item digital learning
  \item data analysis
  \end{itemize}

  It has placed high psychological pressure and leadership demands as a new designer, to:
  \begin{itemize}
  \item always be in charge of balancing all the different perspectives, with the end user's best in mind
  \item be able to change the planned process when new learnings or opportunities emerge (leading an agile design process)
  \item always implement new functionality from customer needs instead of designer or engineer bias
  \item continually design and run workshops and tests suitable for the target groups
  \end{itemize}

  The reason why this has been especially hard, is that simultaneously to learning design and technological skills, it has been in a different cultural setting than the designer is used to. This has also been extremely rewarding, at the same time exhausting.

  \subsubsection{Having Many People Involved in the Design Process to Adapt to the Contextual Constraints}

  The contextual constraints affected the design in the way that more expertise and guidance was needed than otherwise, from a diverse set of people. Service design research proposes to have a diverse team to build with a holistic perspective \citep{stickdorn}, and this has been followed.

  In the project, working with service design in another culture has been one of the hardest and most exciting challenges. When the interactions for iteration 1 were cancelled the day before the trip because of local elections, experts could empower and affirm that this was not out of the ordinary. A learning was that it is only possible to plan to a certain extent, but then changing the plan in the last second, is needed more than when one's own culture. This support has been very helpful, having several people familiar with working in a different cultural context before.


  To seek out people and situations that were not obvious, gave new insights into the work.  The project partner Plan International, did not only back up the interactions and provide their own expertise, but also allowed interviews with experienced consultants from Grameen Foundation and Designers without Borders. Another examples includes involving Expedition Mondial in the design process, testing the app on university students and refugee innovators and at startup hubs, and involving the local YoungDrive project leaders more than originally thought.

  It was very valuable to combine having a diverse team with the designer having a clear direction and caring for the vision of the product. Otherwise, the product might have ended up in many different directions

  The fact that the end users and stakeholders has been involved from the start, made them feel and have actual ownership of the product. This has many benefits, among others that \textit{everyone} involved is satisfied with the \textit{final} app, since their opinions and expertise has been taken into consideration and implemented. The fact that they can notice this further increases trust, and the the likelihood of them supporting future work, which goes in line with \cite{stickdorn}. To conclude, the design has been affected heavily by the contextual constraints, to the point where the end users are more likely to use the app as they have contributed to making a tailor-made product for themselves.

  \subsubsection{Spending Time in the Real-World Training}
  The original time plan stated that the interactions for Iteration \#2 would have been in Tororo, and that it would not be possible to test the app during coach training whatsoever.

  %Therefore, Iliana Björling from YoungDrive did questions during Iteration \#2 initially for only two sessions, guided by the YoungDrive manuals and Bloom's Taxonomy educational objectives.

  However, during a Skype meeting with YoungDrive project leader Josefina, it was announced that it would be possible to participate in the coach training in Zambia during Iteration \#3.

  A new work plan was created, which would allow travel to Zambia and to develop the app and participate in the YoungDrive coach training together with the coaches.

  Now, it was shown already at Iteration \#2, that if the app would have been created solely by the designer, it would have been assumed necessary with more functionality. The field hackathon, designing and developing together with the users, was fantastic - already for iteration 1, the purpose "Validate the coaches' level of knowledge during their education" could be fulfilled. The two following iterations could not focus on "Train the coaches on distance", and "Certify all staff".

  Also, without the 5 days of the training, questions for each topic would not have needed to be created, and this would then have been a must-have for Future work. %Thus, Josefina gradually created questions for all the days, but because of the lack of time, she did not use Bloom's Taxonomy to analyze the questions against educational objectives.

  The intense training in Zambia gave a lot of time to discuss and interact with the trainer, Josefina Lönn. One important contextual constraint that was noticed, was that Josefina did not want to be replaced, but appreciated having the YoungDrive to the point where the app should not replace her, even if it in the future would benefit YoungDrive in terms of for example monetary reasons. This might not only be personal preference, as \cite{ruskovaara} claims how important the role of the teacher is for effective entrepreneurship education.

  Without visiting the Zambia training, it would have been much harder to focus on the coach training purpose of the app, since the Uganda coach training was already over. A consequence might have been that the app would have been more focused solely on "Train the coaches on distance". This shows that spending time in the real-world context of the situation you are developing for, is very important. - especially when it is unfamiliar to you. This goes in line with the service design principles \citep{stickdorn}.

  \subsection{Having Low Scores on the Quizzes Might Lower Motivation}
  Interestingly from iteration 4, the coaches scoring from 0-53\% on quizzes included two coaches being top performers in Iteration 4, but also the two coaches that did not show up for Iteration 4. Further research could be done, if the difference was on motivation or fixed versus growth mindset. It could be for some, that low quiz results in iteration 3 led to not wanting to use the app any more. The app was designed to be more empowering in iteration 4, but naturally, the app could not be tested on the two coaches that did not show up.


%\section{Learning development}

\section{How can test questions be developed to support entrepreneurship learning?} % Bloom

  The problem identified with multiple-choice questions, regardless if the recommendations by Nicol \cite{nicol} were taken, is that they first could only measure lower-order learning objectives, see figure \ref{fig:revised-bloom}. While entrepreneurial \textit{knowledge} objectives might be considered A-B 1-2, building entrepreneurial \textit{skills} is definitely related to C-D 3-6.

  When assessing the first question sets according to Bloom's revised taxonomy, some characteristics were shown, which guided future creation of questions. Most notably, to reach C-D 3-6 on Bloom, there were some techniques: intelligent multiple-answers could encourage the coach to \textit{evaluate} instead of using process of elminination or encouraging guessing. Putting the coach in a coach scenario (how to act in X situation?), the coach could be tested on a \textit{procedural} and \textit{metacognitive} level to \textit{apply}, \textit{analyse} and \textit{evaluate} skills, and get feedback.

  Previous research by for example Nicol \cite{nicol} had already shown how multiple-choice can be powerful, for example by following the principles in figure \ref{fig:multiple-choice}. The same articles mentions the approach taken with using a confidence-meter similar to "Are you sure?". These recommendations have been utilized.

  Some bad questions have still existed, where coaches did not understand and failed to interpret the question, because too advanced English language was used. This points out the value of testing.

  %What is new in this research, is that 1) the confidence-meter is used for feedback, not only assessment, and 2) multiple-choice questions are formulated in a way that supports higher-learning objectives (see Bloom), like entrepreneurship learning.

  %To achieve 2), Bloom's revised taxonomy was used to identify that the standard multiple-choice format did often practice lower order learning objectives (like Factual/Conceptual or Remember/Understand), which fits for entrepreneurial knowledge, but not for entrepreneurship skills. By constructing questions from real-world situations, multiple-choice questions can test procedural knowledge, and the coach may need to also analyse scenarios. The only limitation, is that the coach can not apply the procedural knowledge in the app, but also in real life. Also, as long as the questions are not constructed by the coach herself, she will not reach the \textit{create} level on the cognitive process dimension.

  %\subsubsection{The benefit of having quizzes for the whole YoungDrive training}

  Regarding testing and improving the quality of questions, the initial plan was that YoungDrive would only produce questions for two YoungDrive training weeks, not all 10. To have questions for all of the weeks have greatly benefited the master thesis, and increased the value of the final product. If not all quizzes would have been developed and tested, this would have been a Future Work.

  From a question assessment, it is shown that all of Bloom's levels can now be reached via the app, but two: \textit{create}, and \textit{apply}. This is because users can not create anything in the app, and because of the multiple answers are shown immediately, they are not encouraged to apply their own thinking to the question, before seeing the alternatives.

  % Kritik
  \subsection{Learning effect}
  To the largest extent, the questions have been praised, in regards to formulation and challenge, which can be seen in figure \ref{fig:learning} and \ref{fig:interactiondesign}.

  The lack of a post-test makes it hard to see if the test questions in the app has a real-world effect. Ideally, the post-test in Uganda would have been to observe coaches having their youth session, and compare their correctness and confidence behaviour when not having used the app. %In Zambia, coaches were observed by Josefina, the teacher, but the results are not comparable since all coaches used the app during the training.

  The lack of data (not least from the pre-quiz), makes it hard to draw reliable conclusions from the data. Regarding the pre-quiz paper submissions, the submissions should have been checked for blanks before handed in. Another mistake was that school level was not always exactly specified, and this means that school level and quiz results might have a stronger correlation that can be shown now.

  Regarding recording test question results, this should have been done manually already in iteration 3. The fact that it was done automatically in the app for iteration 4, made so that more data could be recorded, more reliably, than when the project leaders filled them in by hand whenever a coach raised her hand to say she was finished with a quiz.

  The biggest evidence for a learning effect, is the coaches who had a low score on their first try with a quiz, but after the training could pass the certification test, getting 100\% in 1 try. \todo{Improve this section}


%\section{Learning development}

\section{How does design affect usability and learning done via the app?}

  The quiz design is made with Kathy Sierra's model of deliberate practice in mind. \cite{sierra} Depending on the combination of correct and sure, the coach knowledge for each question is put into three different buckets: "Can't do", "Can do with effort" och "Can do effortlessly". By retaking the "Can't do" questions with Try again, reflecting on the "Can do with effort" questions (for example: correct but unsure), and waiting to test the "Can do effortlessly" questions until the certification, all questions are eventually put into "Can do effortlessly".

  The coach can choose to leave the training without doing the certification quiz, later repeating the test. This is to make the learning self-directed and just-in-time, and to allow the coach to do it's own scaffolding. If the coach uses the app before a youth session, and has a low result, the coach will get feedback as such (for example: "Nice effort, but you still need to practice and prepare yourself even more! How can you do that?").

  The goal is to allow the coach to move knowledge from "Can't do" to "Can do effortlessly" to "Certified" in a pace that suits the coach. This might be neccessary, instead of a one-fits-all solution, as the coaches' preferences are so different.

  The learning goes faster in iteration 4 than in iteration 3, which is most notably shown by the fact that the speed from try 1 into getting 100\% in 1 try has increased. This is largely because usability issues has been addressed, and because design choices has been made that stimulates and makes learning more efficient. Most notably, the score board has been improved to show which questions the coach is correct and sure of ("Can do efforlessly"), unsure but correct on ("Can do with effort"), and were wrong on ("Can't do"). One mistake made, was no not think about reducing cognitive load \cite{sierra} until iteration 4, when usability issues were identified as serious problems in iteration 3.

  The YoungDrive app is the first known application which uses a confidence metric (in this case "Are you sure?") for the student's own sake, and not only assessment, like in the case study detailed by Nicol \cite{nicol}. The effect is reaching meta-cognitive on Bloom's.

  %\subsubsection{Benefits of "Are you sure?"}
  %"Det här kan motverka %traditioner och ”så här vi alltid gjort det” genom att %tvinga en att reflektera över varför inte ens rätta svar %korrelerar med hur empowered du känner dig. Bryter normer, %sätter sig emot lathet och agerar proaktivt för en skarpare %utveckling tillsammans."



  %Lena frågades också om vilken skala jag ska ha på 5, 4 eller %vad jag inte tänkt på, 2-gradig skala. 5 eller 4 är vilket %som enligt literatur, det finns två olika skolor. 2-gradiga %skalan bedömde jag vara bäst, p.g.a. användarvänlighet, tydligt för coacherna

  \subsection{Benefits with confidence level and correctness in combination}
  If a coach is wrong and sure about a lot of questions, it might be the indication that the coach is teaching the wrong information to the youth, which might potentially hurt hundreds of their youth's businesses.
  %Fanns extremt många fördelar med denna, och kom bara fram %till ännu fler efter diskussioner med människor och Lena %Tibell, framför allt hur denna kan förbättra utbildningen %och 1-on-1 coachning, och bli väldigt bra självreflektion %för coachen.

  If the coach is correct but not confident, it could be considered a guess. In Sierra's framework of building expertise, it would be called "Can do with effort". In the app during iteration 3, dampened for iteration 4, there were troubles when coaches passing the training with too many correct guesses, knowledge that were not yet "Can do effortlessly". This meant that they would fail the Certification, because they could not answer reliably, providing the wrong alternative. This in turn meant, that they were put back into training instead of getting a medal, which was not motivating. The conclusion was that the coach should not start the Certification mode before being truly ready.

  One solution could have been that "Improve" would not only include repeating wrong answers, but also answers where the coach had been correct but said "No" on "Are you sure?". However, coaches seldom believed they would be wrong, or at least did not determine it worthwhile to be honest. Some preventative measures were taken to try to make the coach more honest. Unfortunately, for iteration 3, only some coaches took notice of number of tries as an indication that they should pay more attention. For iteration 4, the design had improved by giving coaches minus points for being sure and wrong, but here as well, not everyone payed attention. This goes in-line with research both from deliberate practice being a desirable difficulty (learning being hard might make the coach more likely to want to "cheat", or take an easier route like guessing instead of putting more effort).

  The bigger problem was that the knowledge were not yet reliant, not that coaches were not honest. The solution for iteration 4 was to improve learning in the app, partly by a more personal score board with feedback, which could show the coach which kind of questions she needed to repeat ("Can't do" or "Can do with effort"), but instead of forcing the coach to redo correct guesses ("Can do with effort"), she could reinforce the correct answers by personal feedback. Sierra \cite{sierra} calls this "high pay-off tips", which can be very effective.

  \subsection{Deciding on learning methods}
  There was a lot of work behind choosing the learning design methods in Iteration \#3. The way to progress, was to brainstorm various solution, discuss them with experts, and then create trigger material and test some of these approaches.

  Retaking questions that were wrong ("Try again", called "Imrpove" in iteration 3) was inspired by deliberate practice \cite{sierra}, and is already common in e-learning driver license softwares to learn traffic signs or how to act in various situations.

  Showing the coach how many quiz tries they have done, was inspired by Linköping University's work with the e-learning tool NTA Digital, where they reward students with badges for getting 100\% in few tries. Their goal with this kind of "gamification", is to reward students for studying before taking the test. Similarly, for the training mode, the coach seeing number of tries was a method of studying the correct answers more thoroughly. For the certification, where the coach was supposed to have trained before taking the test, badges worked with the same purpose as NTA Digital, to reward students that had studied properly.

  %In their application, this is so subtle to the user that only performance-driven students might take notice, whereas students that are not motivated by this are not discouraged.

  "Are you sure?" was inspired by a Swedish teacher, and has been used before by others. \cite{nicol} It has then been used to determine if a right answer should award a point or not. Nicol and the pedagogical expert for this thesis, Henrik Marklund, suggested that the teacher had overlooked a learning benefit of this approach: the student reflecting on their own knowledge, which is proved great for learning. This was extended in this thesis where personalised feedback has the goal of the coach getting both confident and correct. In a school situation, this might not be neccessary, but in the YoungDrive context, the purpose of the app was to build both correctness and confidence with the material. To pass the certification after training, getting 100\% without faults, made the coach feel that confidence, at the same time reassuring the teacher that the coach had learned the material.

  After iteration 4 a bonus test was made with Plan Tororo staff, which showed the relevancy of the certification mode: one group that were 100\% correct on the first try, did get 100\% correct on their second try, meaning guesses had been present. On the other hand, a person have 1 wrong answer, passing the training on the second try, did then pass the certification on the first try, with confidence. It can therefore be determined, that when all of the answers are answered correctly, after having gotten all answers correct once, that the coach has both correct information and confident - this is a good example of deliberate practice: the information has gotten reliable.

  Other learning design methods were considered, as previously discussed. It was discussed if multiple-choice answers should be completely abandoned, replaced with flashcards. Challenges were that the coaches had no previous knowledge of typing on a keyboard, and analyzing recorded answers would be too technically demanding. The integration and benefits of flashcards was however discussed in various ways, see Future work.


%\section{Study development}

\section{How can users' feedback be used to inform modifications of the app?}

  The number one time waster, has been spending time building ideas that is not informed by the users' feedback, and not immediately realizing the mistake. Such ideas will not be used, are often not supported by research, or does not fit the context. This is the same finding that Stickdorn has made \cite{stickdorn}, and is a violation from service design principles of making the process user-centered. Luckily, almost all of the work with the master thesis has indeed been user-centered.

  Getting enough feedback to evaluate each component of the app has been important. Enough feedback, means that modifications to the app can be made with confidence. The faster this can happen, the faster the app gets better. This is why the field hackathon, and service mini-sprints, were so beneficial, see \ref{digital-service-design}. It ensured that fast iterations could happen with enough feedback to provide effective modifications and additions. The use of service mini-sprints was so effective that the interactions in Tororo could be expanded with even more days, as long as this is properly planned for.

  As long as the users understand the purpose of the app, and how it should benefit them, they can be part of the creation process, not only the testing. This is an extraordinary opportunity, that should not be underestimated. To involve the users and get to know them, is essential to understand the true needs of the coaches, and can greatly inform modifications of the app.

  The app evaluation shows that not only should users' feedback inform modifications of the app, they should be the \textit{basis} of the app. To do this, this report has provided but one approach, via Digital Service Design. It is also a service design principle, making the process co-creative \cite{stickdorn}.

  The presence of different perspectives (from users, stakeholders, experts, and the designers care for the vision) lead to a holistic view when designing, inline with Stickdorn \cite{stickdorn}. The designer can balance these, as long as the views are present during the whole development process, and not only in the end of the project. For the questionnaire made for iteration 1, stakeholder views were very important. From their views, the questions could be improved and put into context, and lessons were learned which otherwise would not have happened. In iteration 1 it was felt that maybe too much time was spent with stakeholders, almost to the point where the process where no longer user-centered.

  By asking Why-why-why continually, the true needs of the coaches could be gradually exposed. Similarly, by hosting co-creation workshops, the coaches can be part of designing lo-fi app prototypes that addresses these needs. The role of the designer then, becomes putting these ideas into a context, and comparing with best practices and relevant research (in this case, most notably learning design), modifying, developing, and testing. As Löwgren and Stolterman says, the output will not be better than the designer carrying out the process \cite{lowgren}. To embrace the role of a socio-technical expert, more so than the computer expert or political agent, has been one of the enablers for co-creation. Service design thinking and methods, gave a framework to have all of these perspectives in balance and consideration, always with the end user as the most important person.

  %Discuss how user's feedback should be used to inform modifications of the app, compared to what I have done

  %Do this by comparing the results with research, and analyzing the method

  %What research was used to let users' feedback inform modifications of the app?
  %Research recommends when developing an app, that the app is regularly tested with the thought users. Both qualitative and quantitative data can inform modifications.

  %To evaluate usability, the interaction design principles of desirability, utility, usability and pleasurability can be used.

  %To evaluate learning, interviews and comparing pre-test and test results (and ideally also a post-test).


  %When evaluating the app, there are many different frameworks that can be used.

  %Service design takes this one step further, putting the users in the center of the product development, and letting the users be co-creators.



  %Users feedback was analysed in terms of interaction design principles (evaluate desirability, utility, usability and pleasurability), service design methodology (qualitative data analysis) and



\section{Summary}

As of now, the app can not be said to evaluate how good the coaches are with teaching. Neither, it helps them assess their ability to learn (it sure would be valuable for coaches to track their ability to get better and remember, see chapter \ref{cha:future-work} Future Work). The app does however help the coaches to assess their knowledge level and what areas they need to train on.

Moreover, the coaches seems to like the app, which is important. There are tendencies that the app works better for some coaches, especially those who take time to reflect on the feedback given by the app. This speaks for designing the app for different need groups. If motivation and confidence is high, it could indicate them becoming better teachers, but as of know there are no evidence of this more than that the coaches themselves has identified that confidence is important for having a good youth lecture.

A bonus result is that the app allows also the teacher to assess the knowledge level of the coaches on a day to day basis during the training, giving insight into how well their teaching has been received. The teacher can use this data to understand what they need to repeat the following day, or if adapting their teaching will lead to higher results. Allowing the teacher to analyse their test questions according to Bloom's Revised Taxonomy created awareness of what knowledge level they were assessing the coaches with by their questions, and motivated matching the formulation of a question to the knowledge and cognitive process dimension suitable for the educational objective.
