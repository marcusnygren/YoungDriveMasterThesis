\chapter{Discussion}\label{cha:discussion}

% %Ingen rapport kan göra anspråk på att ha löst problemen inom ett område på ett uttömmande sätt. Därför är det viktigt att visa på vilket sätt du själv eller  andra kan använda dina resultat i andra studier i framtiden. Den avslutande  diskussionen blir till sin karaktär mer subjektiv, men se till att du inte  spekulerar alltför vilt.

\section{Discussion of method}

\subsection{Iteration 1}
Skype interview with Gerald, Plan, Tororo is used instead of both Kamuli and Tororo

\textbf{Week 7: February 23rd: Number of interactions for iteration \#1 cut down}
Interactions canceled for week 7, the day before Wednesday-Friday, because of local elections.

"Det var tråkigt att höra att det inte blev lika många interaktioner som planerat.
MEN jag tänker: Det här är verkligen en del av lärdomarna att jobba med tjänstedesign i andra kulturer (som jag även tar med mig från vårt projekt i Kenya). Det går bara att planera till en viss grad, och det blir aldrig riktigt som man tänkt sig :) Man får vara beredd på att ändra planen i sista sekund, mycket mer än vad man behöver i sin egen kultur. Bra lärdom!

Så utifrån dina fåtal interaktioner i början på nästa vecka kommer du iallafall ha en hypotes, även om den kanske är lite vagare än vad vi tänkt från början. Jag kan skicka dig nästa kapitel i Coaching Handbook som handlar om Analys senare i veckan så kan du börja fundera på hur du bäst gör analysen utifrån det material du har. " - Susanna, Expedition Mondial

\textbf{Week 7: Friday, February 26th}
Ringer Gerald 26 fredag februari, som meddelar att nya tidschemat jag hade är omöjligt. Han har bara bokat alla inblandade kl. 8-17, då Plan inte tillåter field trips p.g.a. local elections

Krismöte med Josefina, som föreslår att gå bakom kulisserna och engagera Christine och Patrick, utan Plans inblandning. Kanske till och med kan besöka coachgrupp
Sammanfattning: interaktionerna har gått från 3 dagar, till 2 dagar, till 1 dag

Varje gång har jag behövt anpassa mig, och hitta ett nytt koncept
Nu kanske det blir 1 dag i Plans regi, och jag ändå är i Tororo måndag-onsdag.

Interviews less than planned (4 instead of 3 or 8)

\subsection{Iteration 2}
It was shown already at Iteration \#2, that if I would have created the app myself, I would have assumed more functionality was necessary and requested.

\subsubsection{Interaction findings}
\begin{itemize}
\item Short iterations are very effective, however not perfect
\item Field hackathon, designing and developing together with the users, is fantastic
\item I would never have come this far without the short iterations
\end{itemize}

\subsubsection{Replacing the teacher}
This is something is evident that Josefina (the teacher) does not want or think is valid. However, there would be many benefits to YoungDrive.

How it could be done in practise: \todo{Add note how it could be done in practise}.

\subsection{Consequences of involving end users and stakeholders throughout the whole process}

\subsubsection{Product Benefit from involving users and stakeholders}
Design thinking, human-centered design and service design, has been proven to be crucial for the success of this project. Service design thinking and methods, gave a framework to have all of these perspectives in balance and consideration, always with the end user as the most important person.

\subsubsection{Support Benefit from inolving users and stakeholders}
The fact that the end users and stakeholders has been involved from the start, made them feel ownership of the product. This has many benefits, among others that they \textit{everyone} involved is satisfied with the \textit{final} app, since they think that their opinions and expertise has been taken into consideration and implemented. This further increases trust, and the the likelihood of them supporting future work. Even more so, the end users are more likely to use the app, as they have been co-creators of the product.

\subsubsection{Complications}
I was not a designer. I was a computer expert with social skills, now needing to design and develop an app for a cultural and socio-economic context very different from my own.

In this regard, the technical aspect was but one. I \textit{did} need to learn how to develop hybrid apps in JavaScript that worked offline, and had an online back-end. Those was the technical demands.

But more so, I needed to quickly become a good designer. Not mainly from a perspective of graphic design or interaction design, but \textit{how} to explore, design, and implements what the user needs from the requirements "fun, user friendly, and good for learning". The approach to learn design from these perspective was to read extensive literature, consult a diverse set of experts, and be very humble and curious in interactions with the end-users and stakeholders.

This took me a long way, to the point where research, experiments, and constant improvements could lead to increasingly well-informed decisions.

I now have new-found skills in:
\begin{itemize}
\item etnologicy (getting to know and learn from people in a different culture)
\item human-centered design
\item design thinking
\item service design thinking
\item interaction design
\item digital learning
\item data analysis
\end{itemize}

It has placed high psychological pressure and leadership demands on me as a new designer, to:
\begin{itemize}
\item always be in charge of balancing all the different perspectives, with the end user's best in mind
\item be able to change the planned process when new learnings or opportunities emerge (leading an agile design process)
\item always implement new functionality from customer needs instead of designer or engineer bias
\item continually design and run workshops and tests suitable for the target groups
\end{itemize}

The reason why this has been especially hard, is that simultaneously to learning design and technological skills, I have been in a different cultural setting than the designer is used to. This has also been extremely rewarding, at the same time exhausting.

\section{Discussion of result}
In three months time, an app was developed with precision to the needs and context of the end users. The design has been heavily influenced by the end users, from day 1 of the project, in conjunction with relevant research, and in balance to stakeholder goals and considerations, and supervisor advice.

The results shows that the ideal coach, according to the quiz app, would be a woman, since she has better knowledge in spite of having less formal education. She prepares more, is more aware of her own knowledge and has a better study technique, respecting the app feedback for meta-cognition and meta-memory. This can be seen by higher quiz results, faster learning, and more honesty in "Are you sure?".

It could be that first-time smartphone users have a disadvantage with the app, since they will not learn as fast as experienced users. The interactions shows however, that at the second session, almost all of the coaches felt intermediate instead of beginners, using the smartphone and the YoungDrive app. The quiz data verifies this, with no direct correlation between technical skills and quiz results.

The final version of the app shows users can get 100\% on quiz results much faster that the previous version, where the score board had been improved. Since the target group in Zambia and Uganda was different, it is hard to say if it went faster getting 100\% with the possibility of repeating only the wrong questions, asking "Are you sure?", and providing individual feedback. The qualitative study does show however, that 100\% thought the feedback was good for learning, and that they appreciate the app.

\section{Future work}

\subsection{Iteration 3}

\subsubsection{Ideation}

\text{Self-reflection after a youth session}
Josefina talks about a different need: doing self-reflection after a youth session. \todo{Add to Future work}. She says that this is \texit{at least} as important as the coach training, especially in cases where Josefina or other project leaders don't have the resources to visit the coaches physically.

It is determined that while physical follow-up meetings are essential, the app can be used to help the coach in a smart self-assessment and self-reflection. Also, on encounters with the teacher, it can guide the coach-teacher discussion.

This does not need to be a new app. Questions can be asked in a way that they are indeed meta-cognitive, encouraging \textit{learning by reflection}.

Josefina mentions that when she is there to give feedback, it is very clear to the coach that he or she lacks knowledge and has not prepared enough. Asking: "What happens if you say X (giving the wrong information, e.g. what a cost is)?". "Why is it important that you answer correct on this question?".

An app with self-evaluation and monitoring, would help keep the coach thoughtful and give the coach important insights. They are described to sometimes over-estimate their own knowledge.

\textbf{Including coach guides in the app}
She also points out a problem with the training: it feels like some of the coaches forgets the coach guide, even if it has been improved and better integrated with the participant manual. Some of them, don't even use the coach guide.

This speaks for that the app should include quizzes for all coach guides as well. however, the test showed that coach guide should not be designed as a quiz, but better suggested as a drag-and-drop exercice.

When asked if the coach guide quiz are more important than the topic quizzes, she answers that the correct knowledge is more important, because that is the one that needs to be explained correctly to the youth.

Therefore, it should be moved into Future work. \todo{Add to Future work}.

She also says, that while it would be great if the training did prepare the youth more actively for holding youth session, it would not be something to implement in the first day. One idea would be to start with topic quizzes during the first days, and then introducing coach guide quizzes and similar themes. She mentions the challenge with time: Friday, the last day, should be dedicated to preparing a session. But the time has never been there.
