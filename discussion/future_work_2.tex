\subsubsection{\#2 Replacing the teacher}
In iteration 2, Josefina (the teacher) mentions that she does not want to be replaced by the app. However, there would be many benefits to YoungDrive if the coach training could happen 100\% digitally, for example access in locations where YoungDrive currently has no funding.

How could it be done in practise? Also, when Josefina does not want to be replaced? In practice, a freemium model could be proposed, where it is possible to take the coach training for free digitally, but pay for a physical training. Currently, this contextual constraint has affected the app in the way that it should complement the physical training, and ease the burden for the teacher.

\subsubsection{\#2 Scaffolding the coach guides}
Josefina says, after iteration 2, that while it would be great if the training did prepare the youth more actively for holding youth session, it would not be something to implement in the first day. One idea would be to utilize scaffolding: start with topic quizzes during the first days, and then introducing coach guide quizzes and similar themes. She mentions the challenge with time: Friday, the last day, should be dedicated to preparing a session. But the time has never been there.

\subsubsection{\#2 Showing the match or mismatch between correctness and confidence more clearly}

\todo{Lägg till mer om att designa för empowerment}

One idea during the ideation of iteration 3, was to give the student two scores: one showing how correct they were, and one with how confident they were. You can be correct, but still not be empowered (you are unsure but correct). Similarly, you can be incorrect but confident (not empowered).

The reason for not having a "empowerment bar" (combining correct and sure, giving a summative score), was given by Josefina from YoungDrive: "The coaches might want to game the system to get a better score, or be confused by how they got their score". For this reason, the coaches have stated they are OK with getting minus points if the are sure but incorrect, which is easily understood by them and feels fair.

In the app now, there is still a struggle with coaches answering positive on "Are you sure?", even if they are not. Reasons are among others that they say they are \textit{partly} sure, and sometimes that they think they will be more punished for not being sure \textit{than} being sure and incorrect. As this is not the view of the teacher, this needs to become much more clear in the app.

\subsubsection{\#2 Designing the app for different Need groups}
Already since iteration 1, different need groups have been identified. It is shown from the tests that the idealistic and realistic coach might be more probable to have a growth mindset, where challenged coaches might have a more fixed mindset. Research could inform how to design for these different mindsets.
