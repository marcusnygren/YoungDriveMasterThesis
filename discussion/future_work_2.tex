\section{How is the Design Affected by the Contextual Constraints?}

There were many things to consider when designing for the contextual constraints. Below, aspects that should be taken into consideration are presented.

\subsection{Replacing the Teacher}
In iteration 2, Josefina (the teacher) mentions that she does not want to be replaced by the app. However, there would be many benefits to YoungDrive if the coach training could happen 100\% digitally. When the app was tested with refugee innovators in iteration 2, several of them asked if it was possible for them to use the app exclusively to train themselves in entrepreneurship or starting a YoungDrive group.

How could it be done in practice? Also, when the teacher does not want to be replaced? In practice, a freemium model could be proposed, where it is possible to take the coach training for free digitally, but pay for a physical training. Currently, this contextual constraint has affected the app in the way that it should complement the physical training, and ease the burden for the teacher.

\subsection{Scaffolding the Coach Guides}
Josefina says after iteration 2 that it is indeed valuable if the training prepares the youth more actively for holding youth sessions, an insight that was discovered in iteration 1. She mentions two challenges with introducing coach guides for each topic. In the end, a solution is suggested:

\begin{itemize}
\item The coaches are not ready the first day, as they have not gotten used to the app yet. \textit{As such, they should be introduced in the middle of the training.}
\item A recurring issue is that the Friday, the last day of training, should be dedicated to preparing a session, but the time has never been there. If so, the coach guides will not be used. \textit{One idea could be to make the topic quizzes smaller, and mix topic questions with coach guide questions.}
\end{itemize}

\subsection{More Time Designing the App for Different Need Groups}
Already since iteration 1, different need groups have been identified. It is shown from the tests that the idealistic and realistic coach might be more probable to have a growth mindset, where challenged coaches might have a more fixed mindset. Continuing to use growth mindset feedback in the app is crucial according to \cite{dweck}, who found her method made lower and medium achievers also played until the end, while the lack of such feedback only kept high achievers.

Also, there are tendencies that different need groups are more present in different countries. While a lot of research was done about the cultural dimensions of Uganda, the research done on the Zambia and local Kabwe population was more sparse. It was known that the socio-economic differences were large, but not much more. For future work, it is recommended to put more work into how the local and national culture in a country affects the mentality of the coaches and the design. It is not possible to assume that the Zambia and Uganda culture should be similar, and it will be similar to other developing countries, or elsewhere.

Since doing app development together with the coaches has been so beneficial to discovering different needs, a wish is to have done more so with the coaches in Tororo. While in Zambia, the development was done in the real-world training, which was superb. In Uganda, more of the trigger material could have been created in Tororo instead of Kampala. Even if it would have been more costly and internet is slower, it would have been valuable being closer to the coaches.

\subsection{Training the Coaches with Using a Smartphone}
An additional insight from the smartphone test in iteration 1 was that using a smartphone operation system like iOS or Android needs to be made as easy as possible. This to avoid confusion with things like not finding the YoungDrive icon, or accidentally hitting a button, or click the power button: all of which relates to ease of using the operating system, and not to the app in itself. A lot of training is needed to avoid errors,  and should be taken into consideration from a service design standpoint.

In iteration 3, such a co-creation workshop was held after the app test. This resulted in that for iteration 4, all of the devices had the YoungDrive icon on the home screen of the device, and was the only noticeable app. This lowered confusion a lot with finding the app, and realising where to click. If a coach by accident clicked the home button, they immediately found their way back.

In the future, for new coaches, training how to use a smartphone is needed, before they are handed the device. While the YoungDrive is now simple to use to maybe not need an introduction, it would still help how to act in the app (for example encouraging to be honest with answering "Are you sure?").
