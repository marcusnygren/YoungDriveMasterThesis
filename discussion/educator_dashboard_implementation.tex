\textbf{Implementation of Educator Dashboard}

HighCharts var påtänkt som verktyg för att visualisera datan. Tanken var att den vanliga appen skulle kunna ha super-användare som är admins, och kommer till ett särskilt gränssnitt där de ser data om användarna. Detta kunde göras direkt i Meteor.

Stefan frågade vad jag tänkt om detta, och frågade om jag funderat över integration med deras verktyg Podio, och om det var möjligt. Det sade jag att det var framöver. Podio har ett API som bl.a. stödjer JSON, vilket jag använder. Då frågade jag om Podio har bra visual dash-board -verktyg, vilket han skulle kolla upp. Inom exjobbet behöver jag än så länge inte bry mig om detta. Problemet är att Stefan ser ett värde i att lagra datan i Podio, men de har inte i sig själv bra visualiseringsverktyg.

9 april hade jag då ett möte med SolarSisters COO Dave, som är en social enterprise med 1 huvudansvarig (Dave), 70 field staff och 2000 entreprenörer, så ganska likt YoungDrive i Uganda när Josefina var där.

Dave byggde 2013 upp deras backend i SalesForce för databas, och sedan TaroWorks för datainsamling. TaroWorks är en plugin till SalesForce, med offline-app anpassad för fält. De har sedan utrustat alla field staff med tablets, då det var för dyrt att ge till alla 2 000 entreprenörer. Field staff träffar entreprenörer varje dag, och hjälper entreprenörerna att knappa in t.ex. kvitton, utvärderingar och undersökningar (t.ex. från finansiärer) via appen.

Det tog 3 veckor att bara sätta upp systemet, och det var snabbt. För Dave har det varit en 100\%-tjänst i början, och fortfarande 20\%. Men fördelarna är att de nu är 100\% datadrivna, och de kan följa exakt hur det går för field staff och entreprenörerna - detta guidar även vilka som får promotions och vilka som blir avskedade.

Det mest intressanta är kanske att SolarSister inte bara avnänder datan för internt bruk, utan även för dess partners. De gör undersökningar via TaroWorks som inte är direkt kopplade till SolarSister, för att få in pengar. Men framför allt, sticker de ut gentemot andra social enterprises, då de enkelt kan ge partners all data de önskar, och det gör dem väldigt framgångsrika med grants. De är sannerligen en datadriven organisation. Datan, ger SolarSisters story ett trovärdigt narrativ, vilket Dave beskriver som en extrem framgångsfaktor.

Idag står 2/3 av finaniseringen från grants (fördel med socail entrepriser, du kan få pengar både från finansiärer och kunder), 1/3 från entreprenörerna. De vill bli mer self-sustainable för varje år som går, och detta är storyn som datan måste berätta - vilket är varför de t.ex. avskedar människor som inte presenterar. Datan måste stämma med storyn de vill berätta, för den storyn är vad som avgör att de får in pengar.

Detta gav mig insikter på hur mycket min datainsamling från coacherna kan spela roll för organisationen. Det var något jag inte tänkt på innan, och som jag vill vara medveten kring. För om det är något jag lärt mig denna iteration, är det hur "kunskap är makt", och hur mycket vettig kunskap jag, coacherna och Josefina och YoungDrive kan få ut av att helt enkelt lägga till frågan "Är du säker?" till varje quiz-fråga.
