
\subsubsection{\#4 Assessing coach guide knowledge before the youth session}
When asked about the Zambia coach rollout, Josefina points out several challenge. "It feels like some of the coaches forgets the coach guide, even if it has been improved and better integrated with the participant manual. Some of them, don't even use the coach guide.""

This speaks for that the app should include quizzes for all coach guides as well. When asked if the coach guide quiz are more important than the topic quizzes, she answers that the correct knowledge is more important, because that is the one that needs to be explained correctly to the youth. Therefore, it should be moved into Future work.

\subsubsection{\#4 Using a flipcards approach}
In the ideation for iteration 2, flip cards are discussed again, with Henrik Marklund.

In iteration 3, this was tested as a lo-fi material with successful results, but more work should be done.

In the ideation of iteration 4, a proposal was given that did not have time to implement. Therefore, the idea is described here:

At the coach's second quiz try (having assessed and reflected on the knowledge), flipcards could be introduced to assist the coach in retrieving from memory, before getting the multiple-choice.

For future work, when in Training after the first quiz try, The question should be shown \textit{before} the answers are shown, and prompt the coach to think aloud about what they think the answer is, before recieving the alternatives. The coach might be hindered from progressing to the multiple-choice answers until the app has understood the coach has thought hard about their answer to the question.

This is a good use of scaffolding, slowly introducing complicated app features. The hypothesis is inspired from Bjork \cite{bjork}, that knowledge is strengthened if the coach retrieves from memory, versus looking up the answer or choosing the most likely answer.

\subsubsection{\#4 Förbättringar Träningsläge}
From iteration 4, it was clear that the app was designed for learning and self-reflection, and not for effectiveness.
\todo{Översätt}
Problemet nu, var att de tar certifikations-läget och inte får 100\%, vilket är mödosamt och väldigt tidsslösande, då coachen igen måste gå tillbaka till Träning och nå 100\% igen.

Tränings-läget behöver förbättras, och vara säker på att coachen verkligen är redo för Certification.

Ett problem är att "Improve" endast upprepade frågor som varit inkorrekta, och inte upprepade gissningar som varit rätt. Det gjorde att en coach kunde få fel på Certification quiz, för att kunskapen inte var befäst. Så vill vi inte ha det. Därför föreslår jag följande förbättringar i Future Work: Frågor i "Can't do", är frågor som coachen ej vet svar på ännu (t.ex. om svarat fel).
Frågor i "Can do with effort", har coachen ett hum om (gissat rätt, eller gått från fel till rätt).
Frågor i "Can do effortlessly", har coachen rätt och den vet att den har rätt.

Låt coachen välja vilken typ av frågor de vill upprepa.

\subsubsection{\#4 Förbättringar Certifikationsläge}

From the end results of iteration 4, we can learn that notably the intriscic motivation is high, deliberate practice is precent, and the coach can feel the instrinsic reward of having pushed herself and learned the material. This is very positive.

This reaction, could and should be even more amplified. It is discussable if this should be done by simple gamification, but an opinion by a coach was that medals earned should be more visible and that sounds could strenghten the feeling of achievement. Also, the quiz list could show these results, increasing motivation to take other quizzes that you have not yet mastered, or to better your score in a topic where you had not become certified.

\subsubsection{\#4 Improvements training Correct Structure and Time Management}
During all app tests (iteration 2-4), it has been shown that since Correct Structure and Time Management are both ordinal, the Training mode for such topics would be more suitable as interactive exercises than multiple-choice. The proposal is to first use drag-and-drop to place each activity of a youth session in the correct order, and then selecting the right time for the each activity. This assists the coach in creating a mental model, which can be used to retrieve from memory during the assessment.

\subsubsection{\#4 Scaffolding with Flashcards}
After the coach's first new try, Flashcards could be introduced to assist the coach in retrieving in memory, before getting the multiple-choice. To do this after the first assessment, is partly because of technology scaffolding (introduce new concepts in steps), partly because the knowledge is strengthened if the coach retrieves from memory versus looking up the answer or choosing the most likely answer \cite{bjork}.

\subsubsection{\#4 Memory design}
For the ideation of iteration 4, Henrik Marklund pointed out that if knowledge is to be memorized, memory techniques could be used. One such e-learning tool is Memorize \cite{Memorize}. The tool has interactive learning modes, aiming to learn facts and terms with speed. This was underprioritized because of time constraints working with technical features that were not essential. Also, the idea was never proposed by users, only by experts. Moreover, the teacher opposed the idea of remembering answers that were not in the factual remember category. To do so, would oppose the learning objectives, which score higher on Bloom's revised taxonomy. However, to study how the coaches can remember better via an app, and learn memory techniques via the app, could be a future work which is advisable.


\subsubsection{\#4 Sharing with one another}

In future versions of the app, mechanics of sharing content and co-creation would add value connected to Bloom's, reaching Create and Apply. Adding these game elements goes in line with Clark's research, which showed a positive correlation with learning and games that required multi-player collaboration \citep{gates}.
