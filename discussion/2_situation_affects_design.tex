%\section{Study development}

\section{How is the Design Affected by the Contextual Constraints?} % such as X, Y, Z, \Å?

  An insight is how quickly the coaches have increased their fluency with using the smartphone and the app. Even though the design at first needs to be very simple, as long as features are introduced slowly and as intuitive, the app could become more complex over time, to the point where no compromises needs to be made.

  When it comes to design constraints in regards to overcoming cultural differences, learning from the expertise of local partners and technology companies can not be overestimated. They have been very willing to share previous mistakes, learnings and successes. This has saved a lot of time, and made the sparse amount of interactions and development time so much more efficient. Also, the value of getting to know the coaches on a first-hand basis has been greatly beneficial. The app has been designed together with them as co-creators, with a developed mutual interest and understanding, having a common goal of creating an app that works for their needs.

  The design was heavily influenced by starting the project with an iteration to truly understand the target group and context, and see the needs before starting thinking about ideas. It was good that this iteration did not have a digital focus, but even questioned if the best way to solve the needs would be an app altogether. Thanks to service design, everything in the app is informed by needs, which is an example of being a thoughtful designer \citep{lowgren} \citep{stickdorn}).

  Service design methodology allowed the design to be embraced by the contextual constraints. Since service design involves looking at the whole context, both the digital context and the physical context, it was possible to understand in what situation the app can be used, and the situation of the people using it.

  When it comes to designing for entrepreneurship, the focus for this thesis was on the educators: both the teacher, and the YoungDrive coaches. This goes in line with the insights from \cite{ruskovaara}, where the teacher seemed to be the main factor for qualitative entrepreneurship education. However, contrary to \cite{ruskovaara}, women did perform better than the male counterparts in the quiz results. Similar results have been found in other developing countries as well: from the pre-test, it is noticed that women puts more work into preparation, even if they have a lower school level. This could be one reason why their results where better.

  More so, the interviews with coaches shows that the app has been fun to use, which was a recommendation from both YoungDrive and the research from \cite{dickson} on recommendations for entrepreneurship education. Secondly, the app has focused on another best practice from \cite{dickson}, which is that the app should develop the competences as a mentor, enabler or coach. This can be noticable by interview answers such as "open's up ones mind" and that it "evaluates my performance" with "how I've understood", "how I'm performing" and that "the feedback is excellent".

  \subsubsection{From Computer Expert into a Digital Service Designer}
  The study method took the project a long way, to the point where research, experiments, and constant improvements could lead to increasingly well-informed decisions.

  New-found skills are acquired within:
  \begin{itemize}
  \item ethnology (getting to know and learn from people in a different culture)
  \item human-centered design
  \item design thinking
  \item service design thinking
  \item interaction design
  \item digital learning
  \item data analysis
  \end{itemize}

  It has placed high psychological pressure and leadership demands as a new designer, to:
  \begin{itemize}
  \item always be in charge of balancing all the different perspectives, with the end user's best in mind
  \item be able to change the planned process when new learnings or opportunities emerge (leading an agile design process)
  \item always implement new functionality from customer needs instead of designer or engineer bias
  \item continually design and run workshops and tests suitable for the target groups
  \end{itemize}

  The reason why this has been especially hard, is that simultaneously to learning design and technological skills, it has been in a different cultural setting than the designer is used to. This has also been extremely rewarding, at the same time exhausting.

  \subsubsection{Having Many People Involved in the Design Process to Adapt to the Contextual Constraints}

  The contextual constraints affected the design in the way that more expertise and guidance was needed than otherwise, from a diverse set of people. Service design research proposes to have a diverse team to build with a holistic perspective \citep{stickdorn}, and this has been followed.

  In the project, working with service design in another culture has been one of the hardest and most exciting challenges. When the interactions for iteration 1 were cancelled the day before the trip because of local elections, experts could empower and affirm that this was not out of the ordinary. A learning was that it is only possible to plan to a certain extent, but then changing the plan in the last second, is needed more than when one's own culture. This support has been very helpful, having several people familiar with working in a different cultural context before.


  To seek out people and situations that were not obvious, gave new insights into the work.  The project partner Plan International, did not only back up the interactions and provide their own expertise, but also allowed interviews with experienced consultants from Grameen Foundation and Designers without Borders. Another examples includes involving Expedition Mondial in the design process, testing the app on university students and refugee innovators and at startup hubs, and involving the local YoungDrive project leaders more than originally thought.

  It was very valuable to combine having a diverse team with the designer having a clear direction and caring for the vision of the product. Otherwise, the product might have ended up in many different directions

  The fact that the end users and stakeholders has been involved from the start, made them feel and have actual ownership of the product. This has many benefits, among others that \textit{everyone} involved is satisfied with the \textit{final} app, since their opinions and expertise has been taken into consideration and implemented. The fact that they can notice this further increases trust, and the the likelihood of them supporting future work, which goes in line with \cite{stickdorn}. To conclude, the design has been affected heavily by the contextual constraints, to the point where the end users are more likely to use the app as they have contributed to making a tailor-made product for themselves.

  \subsubsection{Spending Time in the Real-World Training}
  The original time plan stated that the interactions for Iteration \#2 would have been in Tororo, and that it would not be possible to test the app during coach training whatsoever. However, during a Skype meeting with YoungDrive project leader Josefina, it was announced that it would be possible to participate in the coach training in Zambia during Iteration \#3. A new work plan was created, which would allow travel to Zambia and to develop the app and participate in the YoungDrive coach training together with the coaches.

  %Therefore, Iliana Björling from YoungDrive did questions during Iteration \#2 initially for only two sessions, guided by the YoungDrive manuals and Bloom's Taxonomy educational objectives.

  Now, it was shown already at Iteration \#2, that if the app would have been created solely by the designer, it would have been assumed necessary with more functionality. The field hackathon, designing and developing together with the users, was fantastic - already for iteration 1, the purpose "Validate the coaches' level of knowledge during their education" could be fulfilled. The two following iterations could not focus on "Train the coaches on distance", and "Certify all staff". Also, without the 5 days of the training, questions for each topic would not have needed to be created, and this would then have been a must-have for Future work. %Thus, Josefina gradually created questions for all the days, but because of the lack of time, she did not use Bloom's Taxonomy to analyze the questions against educational objectives.

  The intense training in Zambia gave a lot of time to discuss and interact with the trainer, Josefina Lönn. One important contextual constraint that was noticed, was that Josefina did not want to be replaced, but appreciated having the YoungDrive to the point where the app should not replace her, even if it in the future would benefit YoungDrive in terms of for example monetary reasons. This might not only be personal preference, as \cite{ruskovaara} claims how important the role of the teacher is for effective entrepreneurship education.

  Without visiting the Zambia training, it would have been much harder to focus on the coach training purpose of the app, since the Uganda coach training was already over. A consequence might have been that the app would have been more focused solely on "Train the coaches on distance". This shows that spending time in the real-world context of the situation you are developing for, is very important. - especially when it is unfamiliar to you. This goes in line with the service design principles \citep{stickdorn}.

  \subsection{Having Low Scores on the Quizzes Might Lower Motivation}
  Interestingly from iteration 4, the coaches scoring from 0-53\% on quizzes included two coaches being top performers in Iteration 4, but also the two coaches that did not show up for Iteration 4. Further research could be done, if the difference was on motivation or fixed versus growth mindset. It could be for some, that low quiz results in iteration 3 led to not wanting to use the app any more. The app was designed to be more empowering in iteration 4, but naturally, the app could not be tested on the two coaches that did not show up.
