%\section{Study development}

\section{How is the design affected by the contextual constraints, e.g. young entrepreneurs, entrepreneurship education, and culture?} % such as X, Y, Z, \Å?
  \todo{Lägg till svar: how \textit{should} the design be affected...}

  \subsubsection{From computer expert into a digital service designer}
  The study method used in this research took me a long way, to the point where research, experiments, and constant improvements could lead to increasingly well-informed decisions.

  I now have new-found skills in:
  \begin{itemize}
  \item etnologicy (getting to know and learn from people in a different culture)
  \item human-centered design
  \item design thinking
  \item service design thinking
  \item interaction design
  \item digital learning
  \item data analysis
  \end{itemize}

  It has placed high psychological pressure and leadership demands on me as a new designer, to:
  \begin{itemize}
  \item always be in charge of balancing all the different perspectives, with the end user's best in mind
  \item be able to change the planned process when new learnings or opportunities emerge (leading an agile design process)
  \item always implement new functionality from customer needs instead of designer or engineer bias
  \item continually design and run workshops and tests suitable for the target groups
  \end{itemize}

  The reason why this has been especially hard, is that simultaneously to learning design and technological skills, I have been in a different cultural setting than the designer is used to. This has also been extremely rewarding, at the same time exhausting.

  The fact that the end users and stakeholders has been involved from the start, made them feel and have actual ownership of the product. This has many benefits, among others that \textit{everyone} involved is satisfied with the \textit{final} app, since their opinions and expertise has been taken into consideration and implemented. The fact that they can notice this further increases trust, and the the likelihood of them supporting future work. To conclude, the design has been affected heavily by the contextual constraints, to the point where the end users are more likely to use the app as they have contributed to making a tailor-made product for themselves.

  \subsubsection{Involving consultants to support the design process}

    \todo{Should I address this, what I would have done without them?}

  \textbf{When in Iteration 1: Week 7: February 23rd: Number of interactions for iteration \#1 were cut down}

  Interactions canceled for week 7, the day before Wednesday-Friday, because of local elections.

  "Det var tråkigt att höra att det inte blev lika många interaktioner som planerat.

  MEN jag tänker: Det här är verkligen en del av lärdomarna att jobba med tjänstedesign i andra kulturer (som jag även tar med mig från vårt projekt i Kenya). Det går bara att planera till en viss grad, och det blir aldrig riktigt som man tänkt sig :) Man får vara beredd på att ändra planen i sista sekund, mycket mer än vad man behöver i sin egen kultur. Bra lärdom!" - Susanna, Expedition Mondial

  This support has been on very much help, having a person familiar with working in a different cultural context before.

  \subsubsection{Iteration 2: The benefit of going for the YoungDrive training in Zambia}

  The original time plan stated that the interactions for Iteration \#2 would have been in Tororo, and that it would not be possible to test the app during coach training whatsoever.

  %Therefore, Iliana Björling from YoungDrive did questions during Iteration \#2 initially for only two sessions, guided by the YoungDrive manuals and Bloom's Taxonomy educational objectives.

  However, during a Skype meeting with YoungDrive project leader Josefina, it was announced that it would be possible to participate in the coach training in Zambia during Iteration \#3.

  A new work plan was created, which would allow travel to Zambia and to develop the app and participate in the YoungDrive coach training together with the coaches.

  Now, it was shown already at Iteration \#2, that if I would have created the app myself, I would have assumed more functionality was necessary and requested.

  \begin{itemize}
  \item Short iterations are very effective, however not perfect
  \item Field hackathon, designing and developing together with the users, is fantastic
  \item I would never have come this far without the short iterations
  \end{itemize}

  Also, without the 5 days of the training, questions for each topic would not have needed to be created, and this would then have been a must-have for Future work. %Thus, Josefina gradually created questions for all the days, but because of the lack of time, she did not use Bloom's Taxonomy to analyze the questions against educational objectives.

  The intense training in Zambia gave a lot of time to discuss and interact with the trainer, Josefina Lönn. One important contextual constraint that was noticed, was that Josefina did not want to be replaced, but appreciated having the YoungDrive to the point where the app should not replace her, even if it in the future would benefit YoungDrive in terms of for example monetary reasons.

