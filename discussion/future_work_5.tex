\section{How can Users' Feedback be Used to Inform Modifications of the App?}

The users themselves has been essential in creating a valuable app. Users include coaches, but also the teachers and stakeholders. Below, one of these aspects are presented together with future work to answer the research question.

\subsection{Educator Dashboard}
Josefina has no means of accessing live quiz results today, as there is no educator dashboard developed. Instead, quiz results today needs to be transferred from a database into Google Sheets, which is cumbersome and not user-friendly.

There was not enough time to develop an educator dashboard, even if this had been a goal. Instead, low-fi trigger material was created, and co-creation stakeholder workshops were held, both for iteration 3 and 4.

In the future, this will be a must-have, and it ties well into YoungDrive's future wish of strengthening its quality assurance via monitoring and evaluation. Exactly how powerful the educator dashboard should be, might turn into a ethical discussion. One could argue that measures needs to be taken so that the app focus is to keep supporting the coaches to become better, and not to punish those coaches that doesn't have the same quiz results as others. As the combination of "Are you sure?" and correctness can give insight into the attitudes and care of the coaches, carefulness must be taken.

From an empowerment perspective, the Educator Dashboard can guide both the teacher in her teaching, allow her to have better coach feedback sessions, and doing future improvements of the YoungDrive program.

\subsection{To Answer the Research Question More Elegantly}
Working with user feedback to inform modifications of the app should be as central as has been during the master thesis. To further answer the research question, new situations and methodology could be tested. The new methodology Digital Service Design has been a success, and could be shared to other organizations than YoungDrive, especially (but not exclusively to) when working with technology in other developing countries.

Future work should also include to look at research suggesting how to work with balancing user feedback, stakeholder opinions and research. Such findings would be especially relevant when prioritizing and working the the backlog and doing sprint planning.

When it comes to user feedback, it comes in various ways: both qualitative data like interviews, and quantitative data such as quiz results, could be seen as a form of feedback. This is evident by the large number of methods used in the master thesis, see section \ref{dataanalysisframework} Data Analysis Framework. As such, it would be interesting to measure how the app is used after this finished master thesis, and if the usage of the app suggests what additional work needs to be done. The main output of the app should be better-performing coaches, and as long as the coaches are part of and feels ownership of the product development, the app will continue to give great results for the youth sessions.

 %då Expedition Mondial ifrågasatte "Visst är väl även Christine och Patrick?" målgrupp för detta? Och vad har de för utrustning? Christine har mobil, Patrick ingen. Så detta talade för att Educator Dashboard skulle behöva fungera på mobil, och inte bara dator som jag tänkt, iom att Josefina har dator.

%Då bestämdes med Expedition Mondial att jag skulle ha workshop med dem på onsdag. Med samtal med Josefina, sade hon att de garanterat borde utrustas med tablets då de samlar in data digitalt, så jag kan tänka mig att de får en tablet framöver. Skönt! Detta stämmer även med vad Stefan FalkBoman hade tänkt sig, och de iPads han köpt in till mig. Så då kunde jag ha dessa som tanke att utforma educator-app-dashboarden ifrån.
