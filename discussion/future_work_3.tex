\subsubsection{\#3 Self-reflection after a youth session}
When discussing the goals for iteration 3, Josefina talks about a need she has noticed during the coaches' rollout in Zambia, where the app could help: doing self-reflection after a youth session. She says that this is at least as important as the coach training, especially in cases where Josefina or other project leaders don't have the resources to visit the coaches physically.

It is determined that while physical follow-up meetings are essential, the app can be used to help the coach in a smart self-assessment and self-reflection. Also, on encounters with the teacher, it can guide the coach-teacher discussion.

This does not need to be a new app. Questions can be asked in a way that they are indeed meta-cognitive, encouraging learning by reflection.

Josefina mentions that when she is there to give feedback, it is very clear to the coach that he or she lacks knowledge and has not prepared enough.
%Asking: "What happens if you say X (giving the wrong information, e.g. what a cost is)?". "Why is it important that you answer correct on this question?".

An app with self-evaluation and monitoring, would help keep the coach thoughtful and give the coach important insights. They are described to sometimes over-estimate their own knowledge.
