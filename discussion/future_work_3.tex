\section{How can test questions be developed to support entrepreneurship learning?}

\subsection{Designing for honesty with "Are you sure?"}

\todo{Lägg till mer om att designa för empowerment}

One idea during the ideation of iteration 3, was to give the student two scores: one showing how correct they were, and one with how confident they were. You can be correct, but still not be empowered (you are unsure but correct). Similarly, you can be incorrect but confident (not empowered).

The reason for not having a "empowerment bar" (combining correct and sure, giving a summative score), was given by Josefina from YoungDrive: "The coaches might want to game the system to get a better score, or be confused by how they got their score". For this reason, the coaches have stated they accept getting minus points if the are sure but incorrect, which is easily understood by them and feels fair.

In the app now, there is still a struggle with coaches answering positive on "Are you sure?", even if they are not. Reasons are among others that they say they are \textit{partly} sure, and sometimes that they think they will be more punished for not being sure \textit{than} being sure and incorrect. As this is not the view of the teacher, this needs to become much more clear in the app. One idea is to ask "How sure are you?", but keep the binary scale of only two answers. This would mean, that the coach needs to think about \textit{how} sure she is about the answer, instead of \textit{if} she is sure, which could have metacognitive benefits.

\subsection{Self-reflection after a youth session}
When discussing the goals for iteration 3, Josefina talks about a need she has noticed during the coaches' rollout in Zambia, where the app could help: doing self-reflection after a youth session. She says that this is at least as important as the coach training, especially in cases where Josefina or other project leaders don't have the resources to visit the coaches physically.

It is determined that while physical follow-up meetings are essential, the app can be used to help the coach in a smart self-assessment and self-reflection. Also, on encounters with the teacher, it can guide the coach-teacher discussion.

This does not need to be a new app. Questions can be asked in a way that they are indeed meta-cognitive, encouraging learning by reflection.

Josefina mentions that when she is there to give feedback, it is very clear to the coach that he or she lacks knowledge and has not prepared enough.
%Asking: "What happens if you say X (giving the wrong information, e.g. what a cost is)?". "Why is it important that you answer correct on this question?".

An app with self-evaluation and monitoring, would help keep the coach thoughtful and give the coach important insights. They are described to sometimes over-estimate their own knowledge.

\subsection{Avoiding memorization}
To avoid memorization, the alternatives should be randomized in the future. While it is unlikely that a coach has an easier time remembering the correct answer by order instead of content, since they only repeats the wrong answers until the certification test, it is an extra measure.

\subsection{Improving the questions}
Data analysis of results on specific questions could give a lot of insight, both into coach behavior, and misunderstandings of questions, in the future. Here, a lot of data is already collected to be able to guide conclusions. Not only are questions recorded with correct and if answered confidently, but also number of tries per coach, and if it is a training quiz or certification quiz.

Simple analysis could be for example mean seeing what are the most difficult questions, where most people have answered wrong repeatadly. From the interviews in iteration 4, it is explained that some answers might be answered wrong becuase of for example difficult wording of questions, not neccesarily because of lack of knowledge. To avoid this, data analysis could be effective, together with getting input from the teacher and coaches.

Improving the questions today has mostly been made from direct feedback from the coaches, or comparing quiz question formulation with current and desired level on Bloom's revised taxonomy. Regarding mapping educational objectives, it needs to be made sure that there are questions for each educational objective of the topic, which has not been done today. In \cite{yengin} questions were designed to support gradual knowledge building with an alignment to Bloom’s Taxonomy, which could also be a viable alternative for this app, where questions are currently formulated as information appears in the participant and coach manual respectively.
