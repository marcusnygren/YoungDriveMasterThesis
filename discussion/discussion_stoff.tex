\subsection{Iteration 1}
Skype interview with Gerald, Plan, Tororo is used instead of both Kamuli and Tororo

\textbf{Week 7: February 23rd: Number of interactions for iteration \#1 cut down}
Interactions canceled for week 7, the day before Wednesday-Friday, because of local elections.

"Det var tråkigt att höra att det inte blev lika många interaktioner som planerat.
MEN jag tänker: Det här är verkligen en del av lärdomarna att jobba med tjänstedesign i andra kulturer (som jag även tar med mig från vårt projekt i Kenya). Det går bara att planera till en viss grad, och det blir aldrig riktigt som man tänkt sig :) Man får vara beredd på att ändra planen i sista sekund, mycket mer än vad man behöver i sin egen kultur. Bra lärdom!

Så utifrån dina fåtal interaktioner i början på nästa vecka kommer du iallafall ha en hypotes, även om den kanske är lite vagare än vad vi tänkt från början. Jag kan skicka dig nästa kapitel i Coaching Handbook som handlar om Analys senare i veckan så kan du börja fundera på hur du bäst gör analysen utifrån det material du har. " - Susanna, Expedition Mondial

\textbf{Week 7: Friday, February 26th}
Ringer Gerald 26 fredag februari, som meddelar att nya tidschemat jag hade är omöjligt. Han har bara bokat alla inblandade kl. 8-17, då Plan inte tillåter field trips p.g.a. local elections

Krismöte med Josefina, som föreslår att gå bakom kulisserna och engagera Christine och Patrick, utan Plans inblandning. Kanske till och med kan besöka coachgrupp
Sammanfattning: interaktionerna har gått från 3 dagar, till 2 dagar, till 1 dag

Varje gång har jag behövt anpassa mig, och hitta ett nytt koncept
Nu kanske det blir 1 dag i Plans regi, och jag ändå är i Tororo måndag-onsdag.

Interviews less than planned (4 instead of 3 or 8)

\subsubsection{Interactions}

As Plan International staff are not allowed to support visiting coaches in the field during local elections, the co-project leaders in Tororo were consulted to carry out the field trips, so that it was still possible to attend the youth group meetings.

\subsection{Iteration 2}
It was shown already at Iteration \#2, that if I would have created the app myself, I would have assumed more functionality was necessary and requested.

\subsubsection{Interaction findings}
\begin{itemize}
\item Short iterations are very effective, however not perfect
\item Field hackathon, designing and developing together with the users, is fantastic
\item I would never have come this far without the short iterations
\end{itemize}

\subsubsection{Replacing the teacher}
This is something is evident that Josefina (the teacher) does not want or think is valid. However, there would be many benefits to YoungDrive.

How it could be done in practise: \todo{Add note how it could be done in practise}.

\subsection{Iteration 3}

There were four ideas originally, for the pedagogical model:

\begin{enumerate}
\item The coach result from Iteration 2: "Try again"-button. When clicked, your wrong answers are repeated.
\item If 100\% on the 1st try, gold. On 2nd try: silver. On 3rd try: bronze.
\item Ask meta-cognitive questions, e.g. "How sure are you?", at the end of each question.
\item Record your answer to the question before you are shown alternatives.
\end{enumerate}

Option 1, 2 and 3 were determined good after the interviews, while item 4 had too many challenges (difficult to use, difficult to implement, cumbersome).

\subsection{Implementation}
Could have benefitted from CI, passing tests before ready for production. Solved this by having a stage environment (since April 19th) where stage is YoungDrive-beta (branch Iteration 4), and YoungDrive is master.

