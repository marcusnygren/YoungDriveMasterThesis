\subsection{Iteration 1}
Skype interview with Gerald, Plan, Tororo is used instead of both Kamuli and Tororo

\textbf{Week 7: February 23rd: Number of interactions for iteration \#1 cut down}
Interactions canceled for week 7, the day before Wednesday-Friday, because of local elections.

"Det var tråkigt att höra att det inte blev lika många interaktioner som planerat.
MEN jag tänker: Det här är verkligen en del av lärdomarna att jobba med tjänstedesign i andra kulturer (som jag även tar med mig från vårt projekt i Kenya). Det går bara att planera till en viss grad, och det blir aldrig riktigt som man tänkt sig :) Man får vara beredd på att ändra planen i sista sekund, mycket mer än vad man behöver i sin egen kultur. Bra lärdom!

Så utifrån dina fåtal interaktioner i början på nästa vecka kommer du iallafall ha en hypotes, även om den kanske är lite vagare än vad vi tänkt från början. Jag kan skicka dig nästa kapitel i Coaching Handbook som handlar om Analys senare i veckan så kan du börja fundera på hur du bäst gör analysen utifrån det material du har. " - Susanna, Expedition Mondial

\textbf{Week 7: Friday, February 26th}
Ringer Gerald 26 fredag februari, som meddelar att nya tidschemat jag hade är omöjligt. Han har bara bokat alla inblandade kl. 8-17, då Plan inte tillåter field trips p.g.a. local elections

Krismöte med Josefina, som föreslår att gå bakom kulisserna och engagera Christine och Patrick, utan Plans inblandning. Kanske till och med kan besöka coachgrupp
Sammanfattning: interaktionerna har gått från 3 dagar, till 2 dagar, till 1 dag

Varje gång har jag behövt anpassa mig, och hitta ett nytt koncept
Nu kanske det blir 1 dag i Plans regi, och jag ändå är i Tororo måndag-onsdag.

Interviews less than planned (4 instead of 3 or 8)

\subsection{Iteration 2}
It was shown already at Iteration \#2, that if I would have created the app myself, I would have assumed more functionality was necessary and requested.

\subsubsection{Interaction findings}
\begin{itemize}
\item Short iterations are very effective, however not perfect
\item Field hackathon, designing and developing together with the users, is fantastic
\item I would never have come this far without the short iterations
\end{itemize}

\subsubsection{Replacing the teacher}
This is something is evident that Josefina (the teacher) does not want or think is valid. However, there would be many benefits to YoungDrive.

How it could be done in practise: \todo{Add note how it could be done in practise}.
