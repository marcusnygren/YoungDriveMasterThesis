\section{Future Work}

\subsection{Bloom's Revised Taxonomy}

From a question assessment, it is shown that all of Bloom's levels can be reached via the app, but two: create, and apply. This is because users can not create anything in the app, and they can not match knowledge towards what suits them. In future versions of the app, mechanics of sharing content and co-creation would add value. This goes in line with Clark's research, which showed a positive correlation with learning and games that required multi-player collaboration \citep{gates}.

\subsection{Iteration 3}

\textbf{Self-reflection after a youth session}
Josefina talks about a different need: doing self-reflection after a youth session. She says that this is at least as important as the coach training, especially in cases where Josefina or other project leaders don't have the resources to visit the coaches physically.

It is determined that while physical follow-up meetings are essential, the app can be used to help the coach in a smart self-assessment and self-reflection. Also, on encounters with the teacher, it can guide the coach-teacher discussion.

This does not need to be a new app. Questions can be asked in a way that they are indeed meta-cognitive, encouraging learning by reflection.

Josefina mentions that when she is there to give feedback, it is very clear to the coach that he or she lacks knowledge and has not prepared enough.
%Asking: "What happens if you say X (giving the wrong information, e.g. what a cost is)?". "Why is it important that you answer correct on this question?".

An app with self-evaluation and monitoring, would help keep the coach thoughtful and give the coach important insights. They are described to sometimes over-estimate their own knowledge.

\textbf{Including coach guides in the app}
She also points out a problem with the training: it feels like some of the coaches forgets the coach guide, even if it has been improved and better integrated with the participant manual. Some of them, don't even use the coach guide.

This speaks for that the app should include quizzes for all coach guides as well. however, the test showed that coach guide should not be designed as a quiz, but better suggested as a drag-and-drop exercise.

When asked if the coach guide quiz are more important than the topic quizzes, she answers that the correct knowledge is more important, because that is the one that needs to be explained correctly to the youth.

Therefore, it should be moved into Future work.

She also says, that while it would be great if the training did prepare the youth more actively for holding youth session, it would not be something to implement in the first day. One idea would be to start with topic quizzes during the first days, and then introducing coach guide quizzes and similar themes. She mentions the challenge with time: Friday, the last day, should be dedicated to preparing a session. But the time has never been there.
