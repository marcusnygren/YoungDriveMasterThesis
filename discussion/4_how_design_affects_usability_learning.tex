%\section{Learning development}

\section{How does design affect usability and learning done via the app?}

  \subsubsection{Benefits of "Are you sure?"}
  %"Det här kan motverka %traditioner och ”så här vi alltid gjort det” genom att %tvinga en att reflektera över varför inte ens rätta svar %korrelerar med hur empowered du känner dig. Bryter normer, %sätter sig emot lathet och agerar proaktivt för en skarpare %utveckling tillsammans."

  The YoungDrive app is the first known application which uses this approach for the student's own sake, and not only assessment (reaching meta-cognitive on Bloom's).

  %Lena frågades också om vilken skala jag ska ha på 5, 4 eller %vad jag inte tänkt på, 2-gradig skala. 5 eller 4 är vilket %som enligt literatur, det finns två olika skolor. 2-gradiga %skalan bedömde jag vara bäst, p.g.a. användarvänlighet, tydligt för coacherna

  \textbf{Using the "Are you sure?" data for the coach}
  If a coach is wrong and sure about a lot of questions, it might be the indication that the coach is teaching the wrong information to the youth, which might potentially hurt hundreds of their youth's businesses. Also, the usage of quizzes (e.g. does the coach wants to improve via the app) can indicate the motivation of the coach.

  %Fanns extremt många fördelar med denna, och kom bara fram %till ännu fler efter diskussioner med människor och Lena %Tibell, framför allt hur denna kan förbättra utbildningen %och 1-on-1 coachning, och bli väldigt bra självreflektion %för coachen.

  \subsubsection{From Iteration 3:  Deciding on learning methods}
  There was a lot of work behind choosing the learning design methods in Iteration \#3. The way to progress, was to brainstorm various solution, discuss them with experts, and then create trigger material and test some of these approaches.

  Retaking questions that were wrong ("Improve") was inspired by deliberate practice \cite{sierra}, and is common in e-learning driver license softwares to learn traffic signs or how to act in various situations.

  Showing the coach how many quiz tries they have done, was inspired by Linköping University's work with the e-learning tool NTA Digital, where they reward students with badges for getting 100\% in few tries. Their goal with this kind of "gamification", is to reward students for studying before taking the test.

  %In their application, this is so subtle to the user that only performance-driven students might take notice, whereas students that are not motivated by this are not discouraged.

  "Are you sure?" was inspired by a Swedish teacher, adding it to his multiple-choice quizzes at university, to determine if a right answer should award a point or not. The pedagogical expert for this thesis, Henrik Marklund, suggested that the teacher had overlooked a learning benefit of this approach: the student reflecting on their own knowledge, which is proved great for learning.

  The positive effects of flipcards are known, however, its disadvantage is that it would prove difficult to analyze the coach answers.

  It was discussed if multiple-choice answers should be completely abandoned, replaced with flipcards. Challenges were that the coaches had no previous knowledge of typing on a keyboard, and analyzing recorded answers would be too technically demanding. The integration of flipcards was discussed in various ways, see Future work.

  \subsubsection{From Iteration 3: Benefits with Feedback for Self-Reflection}

  Genom att på varje fråga besvara "Are you sure?": Yes/No, så stärker vi inte bara coachens meta-kognitiva förmåga, utan vi kan vi även ge personliserad feedback i resultattavlan, istället för att bara visa Question 1: 1 point. Question 2: 0 points, som i Iteration 2.

    Detta gör att coachen kan reflektera över sitt lärande på t.ex. följande sätt:
    - få en självförtroende-boost (via feedback "You were correct, and you were sure")
    - gå från gissning till självsäkerhet (via feedback "You guessed, but you were correct")
    - ändra uppfattning snabbare (via feedback "You were incorrect, but you were sure")
    - uppmuntra coachen att läsa i manualen (via feedback "You were incorrect, and you were not sure")
