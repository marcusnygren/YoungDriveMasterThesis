%\section{Study development}

\section{How Can Users' Feedback be Used to Inform Modifications of the App?}\label{sec:user-feedback}

  %That the design has been heavily influenced by the end users from day 1 of the project, in conjunction with relevant research, and in balance to stakeholder goals and considerations, and supervisor advice, is believed to have been success factors.

  The number one time waster, has been spending time building ideas that is not informed by the users' feedback, and not immediately realizing the mistake. Such ideas will not be used, are often not supported by research, or does not fit the context. This is the same finding that \cite{stickdorn} has made, and is a violation from the service design principle of making the process user-centered. Luckily, almost all of the work with the master thesis has indeed been user-centered.

  Getting enough feedback to evaluate each component of the app has been important. Enough feedback, means that modifications to the app can be made with confidence. The faster this can happen, the faster the app gets better. This is why the field hackathon, and service mini-sprints, were so beneficial, see \ref{digital-service-design}. It ensured that fast iterations could happen with enough feedback to provide effective modifications and additions. The use of service mini-sprints was so effective that the interactions in Tororo could be expanded with even more days, as long as this is properly planned for.

  As long as the users understand the purpose of the app, and how it should benefit them, they can be part of the creation process, not only the testing. This is an extraordinary opportunity, that should not be underestimated, as shown in case studies in \cite{stickdorn}, although these projects did not have a digital focus. It is found that deeply involving the users and getting to know them is essential to understand the true needs of the coaches, and can greatly inform modifications of the app.

  The app evaluation shows that not only should users' feedback inform modifications of the app, they should be the \textit{basis} of the app. To do this, this report has provided but one approach, via Digital Service Design. It is also a service design principle, making the process co-creative \cite{stickdorn}.

  The presence of different perspectives (from users, stakeholders, experts, and the designers care for the vision) lead to a holistic view when designing, in line with \cite{stickdorn}. The designer can balance these, as long as the views are present during the whole development process, and not only in the end of the project. For the questionnaire made for iteration 1, stakeholder views were very important. From their views, the questions could be improved and put into context, and lessons were learned which otherwise would not have happened. In iteration 1 it was felt that maybe too much time was spent with stakeholders, almost to the point where the process where no longer user-centered.

  By asking Why-why-why continually, the true needs of the coaches could be gradually exposed. Similarly, by hosting co-creation workshops, the coaches can be part of designing lo-fi app prototypes that address these needs. The role of the designer then, becomes putting these ideas into a context, and comparing with best practices and relevant research (in this case, most notably learning design), modifying, developing, and testing. As \cite{lowgren} says, the output will not be better than the designer carrying out the process. To embrace the role of a socio-technical expert, more so than the computer expert or political agent, has been one of the enablers for co-creation. Service design thinking and methods, gave a framework to have all of these perspectives in balance and consideration, always with the end user as the most important person.

  %Discuss how user's feedback should be used to inform modifications of the app, compared to what I have done

  %Do this by comparing the results with research, and analyzing the method

  %What research was used to let users' feedback inform modifications of the app?
  %Research recommends when developing an app, that the app is regularly tested with the thought users. Both qualitative and quantitative data can inform modifications.

  %To evaluate usability, the interaction design principles of desirability, utility, usability and pleasurability can be used.

  %To evaluate learning, interviews and comparing pre-test and test results (and ideally also a post-test).


  %When evaluating the app, there are many different frameworks that can be used.

  %Service design takes this one step further, putting the users in the center of the product development, and letting the users be co-creators.



  %Users feedback was analysed in terms of interaction design principles (evaluate desirability, utility, usability and pleasurability), service design methodology (qualitative data analysis) and

