\section{Service Design Methodology}

Below, brief descriptions of five principles of service design are described according to \cite{stickdorn}, together with how the work is divided into iterations, and examples of tools that can be applied.

\subsection{Principles}
\cite{stickdorn} describes five principles that constitute service design thinking, and how to follow these.

He describes how to follow these principles, by making the process user-centered (e.g. via \textit{design ethnography}), co-creative (involve all stakeholders) and holistic (keep the big picture). Sequencing (visualize the service, and make iterations) and evidencing (make the service tangible) are the two last important principles.

\subsection{Sequencing}
Sequencing the process means splitting the design process into iterations, which consists of a number of steps, which are repeated for each iteration. This is a common denominator with the agile methodology SCRUM, which is often applied in software development.

While service design literature and practice refer to various frameworks, regardless of number of steps, every service design project includes: exploration, creation, reflection and implementation \citep{stickdorn}. \cite{expedition-mondial} suggests a model where one iteration consists of insights, ideation, trigger material, and interactions. See figure \ref{fig:iteration}.

\begin{figure}[h]
    \centering
    \includegraphics[width=0.7\textwidth]{Iteration.png}
    \caption{In the model by \cite{nissar} an iteration consists of Interactions, Insights, Ideation and Trigger material.}
    \label{fig:iteration}
\end{figure}

\begin{enumerate}
\item Interactions, where you are listening, the \textit{Explorative phase}.
\item Insights, which is where you use the Interactions in order to try to understand, the \textit{Understanding phase}. % better word+
\item Ideation, where you find possible ideas and when creation of new version of the app is done, the \textit{Design phase}.
\item Trigger material, where material is developed to test the outcome of our evaluation in the next round, the \textit{Trigger development}.
\end{enumerate}

The iterations should come closer and closer to a desired outcome. It is not always obvious what this outcome is. For each iteration, the process takes the project closer, from Why? to What? to How?, often with overlaps \citep{expedition-mondial}. See figure \ref{fig:iterationprocess}.

%\begin{wrapfigure}{r}{0.25\textwidth} %this figure will be at the right
%    \centering
%    \includegraphics[width=0.25\textwidth]{IterationProcess.png}
%    \caption{Iteration process}
%    \label{fig:iterationprocess}
%\end{wrapfigure}

\begin{figure}[h]
    \centering
    \includegraphics[width=0.8\textwidth]{projectLoop.png}
    %IterationProcess.png
    \caption{The iteration process consists of a number of iterations with different focus, starting with broad strokes, and narrowing down into a concrete product. Between iterations, there is an overlap in "Why?" and "How?", "How?" and "What?", which signals that there is a learning process which means conclusions may need to be quickly questioned as new insights emerge. This is especially important in projects where you work with an unfamiliar target group and there are several uncertainties and constraints.}
    \label{fig:iterationprocess}
\end{figure}

\subsection{Service Design Tools}

There are a number of popular service design tools that follows the five principles, e.g. how to make it user-centered. One is Customer Journey Map, in which an activity (like hosting a youth session) is broken into Before, During and After. Another method is Personas, which \textit{exemplifies} thought users of the app into people with names, having realistic character traits and opinions. The persona's needs can then be thought of when designing. An alternative to Personas are Need groups, where thought users are broken down by their different needs. Instead of designing for a specific person, you design for a person with a specific need. The advantage of Need groups, are that it accepts the view that the same person (Persona) might have different needs, depending on situation. The 5 Why's is a simple method used to dig deep into understanding the interviewee. Variants of the question "Why?"" is repeated five times as a rule of thumb, to understand underlying motives. This method is called "Why-why-why" within interaction design \citep{lowgren}).

Tools to create and reflect can be done via certain work methodology. When you structure and inspire brainstorms, you can ask "What if...?" and do Co-Creation, meaning doing ideation together with stakeholders or users. To create, agile development can be used, which is often suitable for software engineering. The manifesto for agile development is \citep{agile-manifesto}:

\begin{itemize}
\item Individuals and interactions over processes and tools
\item Working software over comprehensive documentation
\item Customer collaboration over contract negotiation
\item Responding to change over following a plan
\end{itemize}

An example of an agile methodology is \textit{SCRUM}, where a project is divided into several iterations similar to a service design approach of sequencing, but also introducing consepts like retrospectives (reflecting on one's work) and sprint demo (demonstrating the results of the iteration to stakeholders) \citep{kniberg}.

There are also some service design best practices: interviews are often done via open questions (encouraging stories) and dialogue can be facilitated with a questionnaire guide. In workshops, post-its are often used, and followed up with specific questions. Service design methodology encourages taking pictures, filming and recording audio, benefiting the analysis done afterwards \citep{expedition-mondial}.

%\subsubsection{Relevancy within Social Innovation}
%\citep{socialinnovation-ehn}

%\subsection{Service Design Thinking}

%\subsubsection{Methodology}

\textbf{The iteration process}

The time in Uganda is divided into three iterations. For each iteration, the result becomes more and more clear. In iteration 1, there is a very broad scope, without digital focus whatsoever, where iteration 2 and 3 gradually introduces the digital solution. See figure \ref{fig:iterationprocess}.

%\begin{wrapfigure}{r}{0.25\textwidth} %this figure will be at the right
%    \centering
%    \includegraphics[width=0.25\textwidth]{IterationProcess.png}
%    \caption{Iteration process}
%    \label{fig:iterationprocess}
%\end{wrapfigure}

\begin{figure}[h]
    \centering
    \includegraphics[width=0.8\textwidth]{IterationProcess.png}
    \caption{The iteration process consists of a number of iterations with different focus, starting with broad strokes, and narrowing down into a concrete product. Between iterations, the overlap between "Why?" and "How?", "How?" and "What?", signals that there is a learning process which means conclusions may need to be quickly questioned as new insights emerge. This is especially important in projects where you work with an unfamiliar target group and there are several uncertainties and constraints.}
    \label{fig:iterationprocess}
\end{figure}

\textbf{One iteration} \\
In the way of reasoning around development and design for learning, the steps for each iteration, see figure \ref{fig:iteration}, might be translated into:

\begin{enumerate}
\item Interactions, where you are listening, the \textit{Explorative phase}. 
\item Insights, which is where you use the Interactions in order to try to understand, the \textit{Understanding phase}. % better word+
\item Ideation, where you find possible ideas and when creation of new version of the app is done, the \textit{Design phase}.
\item Trigger material, where material is developed to test the outcome of our evaluation in the next round, the \textit{Trigger development}.
\end{enumerate}

%\input{theory/design/service_design_stoff}
