\subsubsection{Vad är ett bra digitalt läromedel? Björn Sjödén, 2013}

Istället för att upprätta en ny lista med bedömningskriterier vill jag ta ett större grepp och föreslå hur man kan rama in digitala mervärdesfunktioner i tre kontexter, som i ökad grad involverar eleven och dennes omgivning. I varje kontext blir olika bedömningar relevanta.

För det första kan man studera hur läromedlet representerar information – grafik, ljud, film, animationer etc, som konkret visas eller spelas upp på datorskärmen vid ett visst tillfälle. För det andra kan man studera interaktionen med den enskilde eleven – elevens egna handlingar och möjligheter att påverka representationen, exempelvis hur man klickar sig fram i en virtuell miljö, och den återkoppling som systemet ger. För det tredje kan man studera läromedlets sociala positionering av eleven – hur representation och interaktion tillsammans öppnar upp för sociala rolltaganden, framför allt i samverkan med digitala karaktärer, som påverkar hur man som elev angriper uppgifter i läromedlet.

Gemensamt för dessa tre kontexter – representation, interaktion och social positionering – är att de tar utgångspunkt i just de egenskaper som skiljer digitala läromedel från andra läromedel. Som användare är det lätt att bara tänka på datorns unika egenskaper i termer av representationen på ytan, exempelvis multimedia-funktioner och att man kan ”klicka” fram information. Men en dator är i grunden en beräkningsmaskin, vars kanske mest kraftfulla (men osynliga) funktion i digitala läromedel är att på basis av olika variabler beräkna när, hur och vilken information som ska presenteras för eleven (t.ex. ”Uppgift 2 visas när eleven klarat av uppgift 1, 2 och markerat C”). Man kan då se det ”egentliga” digitala läromedlet som ett underliggande styr- och kontrollsystem som enligt programmets variabler för in en struktur i lärprocessen. Effekten på lärandet blir mer eller mindre kraftfull beroende på hur detta system ser ut.

--

Min poäng är att digitala mervärden skapas genom hur representation, interaktion och social positionering samverkar, genom pedagogiska funktioner som aktiveras under det att läromedlet används.

\textbf{4.1. Representation}

%Tes 1: Ett bra digitalt läromedel främjar förståelse genom att utnyttja mediets egenskaper att representera information på flera olika sätt (visuellt och auditivt, statiskt och dynamiskt, verbalt och i bild, som narrativ och instruktioner, samtidigt och i sekvenser, etc).

%Jag vill börja med att förtydliga två vanliga missuppfattningar. Den första har att göra med läromedlets visuella representation. Producenter av digitala läromedel lägger ofta stor omsorg vid det grafiska gränssnittet och en tilltalande visuell miljö.

%Man kan dock skilja mellan det initiala intresset att engagera sig i en uppgift för att där finns något (vad som helst) som verkar lockande, och det intresse som byggs upp för att övningarna i sig är motiverande.

%Den andra vanliga missuppfattningen har att göra med vad en anpassning till olika lärstilar innebär. Det finns en spridd populär föreställning om att individer är exempelvis visuellt, auditivt eller kinetiskt orienterade och därför skulle lära sig bättre genom att öva material på motsvarande sätt.

%Utgår man från detta så skulle eleven gynnas av att kunna välja om t.ex. en text ska läsas upp eller visas på skärmen. Men även om individer uttrycker preferens för ett bestämt presentationssätt, finns inget vetenskapligt stöd för att sådan anpassning av material har någon betydelse för hur väl man lär sig (Pashler et al, 2008). Snarare är det så att multimodal presentation, vanligen ljud och bild i kombination, gynnar lärandet för de allra flesta (Mayer, 1989).

%Att digitala läromedel ska ”representera information på flera olika sätt” bör alltså förstås i relation till kognitiva stilar (och inte estetiska preferenser). Jag väljer att sammanfatta vilka sätt att representera information som då blir intressanta, med att relatera kognitionsforskaren Stellan Ohlssons teoribildning för hur människan uppnår ”djupinlärning” (Ohlsson, 2011). Ohlsson kopplar själv sin modell till riktlinjer för hur digitala läromedel (eg. ”educational software”) bör konstrueras (Ohlsson, 2008).

%Enligt Ohlsson finns det nio distinkta typer av information, som svarar mot lika många sätt att lära och förklara.

%En rimlig hypotes är att en elev som är ny inför en uppgift och exponeras för olika sätt att representera den, har större utsikter att hitta ett sätt som underlättar lärandet, just därför att fler kognitiva stilar går att tillämpa. I Tabell 1 redovisar jag de informationstyper som Ohlsson identifierat och ger exempel på hur varje typ kan gestaltas i olika digitala läromedel enligt kategorierna i föregående avsnitt.

%(lägg in tabell här, finns på sista sidan av artiklen) Säger t.ex. att om informationstypen är Demonstrationer och förklarande exempel, är exempel i digitala läromedel Simuleringar.

%Resultat av ”trial and error” (uteslutning)
%Lärspel, drillning- och övningsprogram: När A och B inte fungerar provas C.

%Positiv återkoppling
%(förutsätter interaktion)
%Lärspel, simuleringar:
%”Detta gjorde du rätt!”; simulering visar önskat förlopp

%Negativ återkoppling
%(förutsätter interaktion)
%Lärspel, simuleringar:
%”Detta gjorde du fel!”; simulering visar oönskat förlopp
%
%Erfarenhet och minne
%(förutsätter att man utfört tidigare uppgifter)
%Lärspel, problemlösning:
%Uppgifter utförda på en tidigare svårighetsnivå hjälper eleven att lösa fler %uppgifter.
%
%Igenkänning av mönster/ regelbundenheter i omgivningen
%Drillning-och övningsprogram, vägledning:
%Efter att ha läst/bearbetat många engelska meningar observerar eleven att alla %regelbundna verb i tredjeperson singular slutar på -s.

%Den vägledande principen bör enligt Ohlsson vara att representera material ”good enough”, det vill säga tillräckligt väl för att aktivera olika lärmekanismer, men inte på bekostnad av antalet informationstyper som representeras. Motivet är att flera informationstyper i kombination ger synergieffekter. Exempelvis har kombinationen av negativ och positiv återkoppling visat sig mer än halvera lösningstiden för en uppgift (Barrow et al, 2008). Att bistå med förklarande exempel, verbala instruktioner och låta eleven prova på själv torde likaså effektivisera lärandet (Ohlsson, 2011).

\textbf{4.2 Interaktion}

%Tes 2: Ett bra digitalt läromedel tillåter eleven att interagera med materialet med tydlig och omedelbar återkoppling från systemet som möjliggör kunskapsutveckling.

%Jag kommenterade tidigare att digitala läromedel är som mest effektiva när de anpassar innehållet till elevens svar och beteenden i systemet – alltså med avseende på hur interaktionen går till. Interaktionsanalys i dessa sammanhang handlar mindre om konventionell ”användbarhet” och mer om ”lärbarhet”.

%En särskild effektiv funktion för lärande är det digitala läromedlets möjligheter att framkalla återkopplingsloopar, det vill säga anpassa feedback efter elevens svar, så att eleven korrigerar sitt beteende och får ett nytt gensvar från systemet osv, tills han/hon uppnått tillräcklig färdighet eller demonstrerat godtagbar kunskap för uppgiften. Forskning kring motivation pekar på att återkopplingsloopar är särskilt viktiga för elevens förmåga att själv reglera sitt lärande och därmed upprätthålla sitt intresse för ämnet. En elev som lär sig att själv anpassa sitt lärande, framstår även mer motiverad att ta sig an nytt material utanför det digitala systemet (Zimmerman & Cleary, 2009; Chin et al, 2010).

%För att interaktionen ska leda till kunskapsutveckling krävs också en väl avvägd ökningstakt i svårighetsgraden – en anpassad progression – för de uppgifter man utför. Medan denna princip länge varit väl inarbetad i kommersiella datorspel, så är det långt från alla digitala läromedel som visar en tydlig progression; kanske för att många är begränsade till drillnings- och övningsmoment av enskilda färdigheter (som att namnge geografiska platser i Webbmagistern). Ett läromedel med ofullständig återkoppling riskerar att försätta eleven i ett ”trial-and-error”-beteende som inte gagnar förståelsen, utan bara förmågan att beskriva något som rätt eller fel.

\textbf{4.3 Social positionering}

%Tes 3: Ett bra digitalt läromedel utnyttjar teknikens möjligheter att (radikalt) påverka hur eleven angriper en uppgift genom att aktivera positiva attityder och beteendemönster förknippade med sociala roller.

%Digitala läromedel skapar inte sociala roller; däremot kan funktioner i det digitala läromedlet aktivera olika attityder och förhållningssätt som eleven lärt sig från sin sociala miljö. Gulz och Haake (denna volym) lyfter särskilt fram den potential som finns i att tillämpa Learning-by- Teaching-pedagogik i digitala läromedel genom att gestalta ”digitala elever”. Möjligheten att bryta invanda tankemönster och påverka hur man angriper en uppgift genom att byta kön eller ta en främmande social roll som avatar utgör också unika digitala funktioner, som poängterats tidigare.

%Jag går därför inte närmare in på dessa exempel här, utan fokuserar på den sociala positioneringens yttersta mål i lärandesammanhang: att återge eleven makt och kontroll över sitt lärande. Att kunna försätta eleven i en kontrollposition är en funktion av digitala läromedel som inte får underskattas, speciellt som den kan vara svår att omsätta för en lärare som – alla goda avsikter till trots – själv utgör en auktoritet i klassrummet. Människan tycks vara programmerad att söka social bekräftelse för att befästa nya kunskaper – man vill kort sagt att det man lär sig har betydelse i relation till andra, om så för att utöva mer kontroll eller för at kunna hjälpa och förklara nya sammanhang. Detta har fått vissa forskare att hävda ”the social purpose hypothesis”, att allt lärande tjänar på att ha ett socialt syfte (Zimmerman & Cleary, 2009).

%Effekterna på ansvarstagande och känslan av att kunna kontrollera förlopp vore onekligen värdefulla för att stötta elevers lärande – likväl lyser de med sin frånvaro i det digitala läromedelsutbudet.


%Den viktigaste funktionen av att positionera eleven som ”ledare” över materialet torde vara den tydliga återkopplingen på att även svåra och kanske tråkiga moment i lärandet har en betydelse i ett större sammanhang – det är stor skillnad på att lära sig beståndsdelarna i en kroppscell för sakens skull, och att kunna sätta samman delarna för att cellen ska fungera i en människokropp (om än digital och simulerad). Väsentligen handlar det om att gestalta den enskilt viktigaste faktorn för att man ska fortsätta engagera sig i en uppgift, nämligen bekräftelse på ökad kompetens (Bransford et al, 1999; Zimmerman & Cleary, 2009). Det är alltså inte maktpositionen i sig som är viktig (att vara härskare över ett rike), utan den positionering som får eleven att se sin påverkan på omgivningen (hur man härskar) – med insikten att man alltid kan göra saker bättre.

\textbf{Slutord}

%Ett bra digitalt läromedel är ett läromedel som tar tillvara den digitala teknikens möjligheter att framställa och organisera kunskap på nya sätt, som stödjer en variation av sätt att lära. Jag har beskrivit hur man kan identifiera digitala mervärden i olika kontexter, som utgår från vad datorn gör (representation) vad eleven gör (interaktion) och hur eleven förhåller sig till läromedlet (social positionering). Varje kontext är förknippad med vetenskapligt påvisade lärmekanismer. \textbf{För att kunna utvärdera effekten av enskilda faktorer i respektive kontext krävs omfattande empiriska studier.}

%För läraren innebär digitala mervärden ett stöd i de normala undervisningsuppgifterna, exempelvis i att tydliggöra material, arbetsprocesser och struktur i lärandet. En konsekvens av att tekniken gör "mer" i termer av att tillhandahålla och presentera lärmaterial blir att undervisningsprocessen förändras och att mänskliga resurser frigörs för andra ändamål. Läraren kan fokusera mer på de större sammanhangen inom vilka de digitala läromedlen används och ägna mer tid åt att vägleda eleverna i deras kunskapsutveckling, istället för att administrera, konstruera och rätta rutinuppgifter.

%Avslutningsvis finns ingen anledning att befara att ”tekniken tar över” klassrummen, eller att läraren ska belastas med den dubbla bördan av att hantera både pedagogik och teknik. De två hör ihop, men kräver förändringar i såväl utbudet som användningen av digitala läromedel. Hattie (2012) sammanfattar läget apropå att nästan alla insatser för att förbättra skolundervisningen – inklusive skoldatorer – tycks ge positiva resultat: ”we need not more, but different”.