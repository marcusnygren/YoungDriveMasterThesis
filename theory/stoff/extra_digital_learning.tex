Digital Learning}

\subsection{"Design factors for educationally effective animations and simulations"}

\textbf{Cognitive load theory: icons and pictures vs. symbols and words}
The effect of the representation type of the information can be explained by \textbf{cognitive load theory}, which predicts that processing depictive information requires less mental effort than processing descriptive information as depictive information (i.e., icons and pictures) by definition relate directly to their referent, whereas descriptive information (i.e., symbols and words) need to be interpreted before they can be integrated with other information.

\textbf{Graphical feedback vs. textual feedback}
Rieber’s findings suggest that graphical feedback led to higher performance than textual feedback on tests of implicit learning through in computer-game-like tasks, but only in some cases on tests of explicit learning through traditional text-based questions.

\textbf{Interaction design principles}
These principles are important if the learner can interact with models, e.g. simulations. For my work, they are probably not as usable. The exception, is the Additional principles they describe:

\textit{Learner control-pacing}
In summary, learning from dynamic representations is improved when learners are able to control the pacing of the presentation because new information can be integrated into existing knowledge structures at a rate that reflects the capabilities and needs of the learner (Betrancourt 2005; Hasler et al. 2007; Mayer and Chandler 2001; Swaak and de Jong 2001; Tabbers et al. 2004). In addition, allowing a learner to speed through or skip parts of a presentation that he or she perceives as easy, and to focus on the more difficult parts (Schwan and Riempp 2004), can avoid a redundancy effect through learner control. In contrast to the segmenting principle, which applies to the control of advancing from one segment to the next, the benefits of learner control of pacing refer to the finer level of granularity of control that is possible in continuous dynamic media, such as animation and video, for which learners are provided with functionality to start, pause, and stop the dynamically presented content, and to change the speed and direction of the presentation.

\textit{Task appropriateness}

The emerging task appropriateness principle states that the efficacy of a simulation depends on the degree to which it is in line with learning objectives. Research suggests that visualizations must be task-appropriate in order to improve learning outcomes, i.e., they need to prepare learners for future tasks to be performed (Levin 1989; Schnotz and Bannert 2003).

\textit{Manipulation of content}

The emerging principle of manipulation of content suggests that learning from visualizations is improved when learners are able to manipulate the content of a dynamic visualization compared to when they are not able to do so.

\subsection{"Simulations for STEM Learning: Systematic Review and Meta-Analysis"}

Many different modifications or enhancements were used in the 31 studies analyzed. The types of modifications did cluster in a few general areas, specifically, scaffolding, representations, haptic feedback (feedback involving touch), and cooperative learning. A preliminary finding based on these categories of modifications showed that scaffolding within a simulation had a strong positive effect, as did cooperative learning, whereas representations had a more mixed overall effect. We found no significant differences across age groups, simulation type, or group size.

\subsection{"Digital Games, Design, and Learning: A Systematic Review and Meta-Analysis - Executive Summary" - Bill \& Melinda Gates Foundation}

These are their conclusions:
\begin{itemize}
    \item Motivating why a digital tool/game is a good thing
    \item Learning outcomes
    \item Single-player vs. multi-player
    \item Spending time correlation to learning
    \item Spaced learning vs. massed learning
    \item Effektivitet av att ge coacher instruktioner inför att de stänger appen
    \item The value of gamification
    \item The value of scaffolding
    \item Competitive vs. collaborative
    \item Design rather than medium alone predicts learning outcomes
    \item Schematic games more effective than cartoon-like or realistic serious-games
    \item Game with narrative vs. non-narrative
\end{itemize}

\textbf{\textit{Overarching Media-Comparison and Value-Added Findings}}

\textbf{Motivating why a digital tool/game is a good thing}
Fifty-seven studies included comparisons of digital game interventions versus other non-game instructional conditions (i.e., media comparisons). Overall, results indicated that digital games were associated with a .33 standard deviation improvement relative to control conditions, even after adjusting for baseline differences in achievement between groups.

\textbf{Learning outcomes}
These findings generally confirm and parallel the overall findings of Vogel (2006), Sitzmann (2011), and Wouters et al. (2013) in terms of learning outcomes. Meta-analysis, motivation was incorporated within the intrapersonal learning outcome domain (as outlined in the NRC report on education for life and work in the 21st century), but intrapersonal outcomes also included intellectual openness, work ethic and conscientiousness, and positive core self-evaluation.

do demonstrate that game conditions support improvements in intrapersonal learning outcomes relative to non-game instructional conditions.

\textbf{\textit{General Study Characteristics in Media Comparisons}}
\textbf{Single-player vs. multi-player}
we found that (a) game conditions involving multiple game-play sessions demonstrated significantly better learning outcomes than non-game control conditions and (b) game conditions involving single game-play sessions did not demonstrate significantly different learning outcomes than non-game control conditions.

\textbf{Spending time vs. learning}
we found no evidence of a consistent correlation between total duration and effects on learning outcomes.

\textbf{Spaced learning vs. massed learning}
Taken together, these findings may reflect a memory benefit of spaced learning as compared to massed learning in game contexts.

\textbf{Ang. att ge coacher instruktioner inför att de stänger appen}
This might suggest that additional teaching or activities specifically designed to supplement game content as part of an integrated experience can increase learning, but that unintegrated supplemental teaching on a topic is unlikely to contribute to larger gains.

\textbf{\textit{Game Mechanic Characteristics in Media Comparisons}}
\textbf{The value of gamification}
The comparison of broad design sophistication (Hypothesis 4a) demonstrated that simple gamification as well as more sophisticated game mechanics can prove effective. Future research and analyses should explore whether or not the “simple gamification” studies (e.g., games that simply add contingent points and badges to learning activities) more frequently focus on lower-order learning outcomes as compared to studies with more sophisticated game mechanics. Regardless, these results support the proposal that simple gamification can prove effective for improving certain types of learning outcomes (as suggested by Charles, Bustard, \& Black, 2011; Lee \& Hammer, 2011; Sheldon, 2011).

\textbf{Scaffolding}
Regarding the nature of scaffolding (Hypothesis 4d), each category of scaffolding demonstrated significant effects on learning relative to non-game control conditions, and higher levels of scaffolding were associated with higher relative learning outcomes than lower levels of scaffolding.

\textbf{Competitive vs. collaborative}
When controlling for game characteristics, gains from single-player games without competition and gains from collaborative team competition games exceeded those from single-player games with competition. This explanation would align with research on motivation. The motivational support of self-efficacy for certain students in a single-player competitive structure is necessarily a failure to support other students (because one student’s gain necessitates another student’s loss).

\textbf{\textit{Visual and Narrative Game Characteristics in Media Comparisons}}
\textbf{Design rather than medium alone predicts learning outcomes}
Much of the research to date on digital games has focused on proof-of-concept studies and media comparisons. The present meta-analysis highlights the importance of questions that ask not if but how games can support learning. More specifically, the results of the present meta-analysis parallel those of the recent NRC report on laboratory and inquiry activities (Singer, Hilton, \& Schweingruber, 2005). Design, rather than medium alone, predicts learning outcomes. Research on games and game-based learning should thus shift emphasis from proof-of concept-studies (“can games support learning?”) and media-comparison analyses (“are games better or worse than other media for learning?”) to value-added comparisons and cognitive-consequences studies exploring how theoretically-driven design decisions influence learning outcomes for the broad diversity of learners within and beyond our classrooms.

\textbf{Schematic games more effective than cartoon-like or realistic serious-games}
schematic games were more effective than cartoon-like or realistic serious games

\textbf{Game with narrative vs. non-narrative}
suggesting that games with no narrative might be more effective than games with narratives. More tangentially, our findings also suggest a possible parallel to findings from Sitzmann (2011) showing that the entertainment value of the simulations and games did not significantly affect learning outcomes.