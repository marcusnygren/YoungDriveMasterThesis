\textbf{Från Why gamers don't learn more: An ecological approach to games as learning environements}

Games can be designed to facilitate both the exploratory and performatory mode of action.

This means that progressing in a game, being able to take actions and reach built-in game goals, is not solely a matter of learning. Since affordances can be shown in a game, the player does not always have to learn to differentiate between the available information in the gaming domain.

Games can also be designed to facilitate performatory actions, perhaps the most obvious example being micro-transaction systems where players can buy advantages that speed up game progress. Compared to performatory actions in other domains like playing an instrument, performing surgery, playing a sport, dancing, writing a novel or acting on a stage, such tools are not introduced systematically.

It might be correct that games have unique properties as learning envi- ronments. But with no detailed analysis of either gaming practices or game design, this must be seen as an open question.

This article has attempted to illustrate that there are ways to design games so that the player can progress through a system with very little learning. Thus, games have some system features that can hardly be used in schools where children need to master other domains. These domains will differ from games since progression here demands learning and skill development. Gee states that part of the pleasure of gaming is to learn the game: ‘Good video games offer pleasure from continuous learning and problem solving. They are hard and complex and their difficulty ramps up as the game proceeds. If no one could learn them, the companies that make them would go broke’ (2007b: xi). But as suggested here it might just as well be the undemanding nature of some games, the fact that they are not as complex as we think at first glance, that makes them pleasurable and motivating. Games can give us the sensation of progress and empower us without demanding that we develop the kind of skills that many other domains require. Thus, perhaps some good video games offer a pleasure that comes from a continuous illusion of learning.

%Achievements in Gaming Practices. Game Studies 11(1). %Oika personligheter (Jacobsson, 2011) - alla gillar inte att spela spel (problem: de som jobbar i spelbranchen gillar spel) - alla gillar inte att vinna (en del gillar att samarbeta)