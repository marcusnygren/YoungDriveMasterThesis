\subsubsection{2012 Luckin: Decoding Learning: The Proof, Promise and Potential of Digital Education}

\textbf{Chapter 2: Learning with Technology}

\textbf{Chapter 2.7: Learning from Assessment}

Technology can be used to support assessment in a variety of ways. it can be used to compile learning activities and enable both teachers and learners to reflect upon them; and to track the progress of learning and to present that information in rich and interactive ways.
interest in formative e-assessment is increasing. There are numerous examples of developments in e-assessment using mobile and immersive environments as well as social and collaborative networks.106 a large amount of development has also taken place on diagnostic testing environments that allow teachers and learners to assess present performance against prior performance.107

in most examples learners used a web-based or virtual environment via desktop or laptop computers. exceptions included the subtle stone (described above)113 and a mobile phone app designed to support self-assessment by learners at secondary school and university.114