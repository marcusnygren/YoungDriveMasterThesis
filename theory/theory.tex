\chapter{Theory}\label{cha:Theory}
%

Below, the findings of literature and interviews is presented, for each relevant subject the master thesis has involved.

The literature and interviews aims to give a theoretical answer for research question \#1, and guidance for how the method can answer the research question. Some of the literature also suggests how to answer research question \#2.

\section{Methods}

Experts and literature has been used to get a theoretical understanding of the relevant research areas.

In the first section, experts in the project are described with their name, role and professional area. In the second section, the literature basis is presented.

\subsection{Experts}
I want to thank the following experts, who either have already helped me with finding research material, or will contribute with knowledge during the project.

\begin{itemize}
    \item Service design: Susanna Nissar and Erik Widmark, Expedition Mondial
	\item Social innovation: Peter Gahnström, LiU Innovation
    \item Technical support: Daniel Marklund and Stefan FalkBoman, YoungDrive
    \item Entrepreneurship education: Konrad Schönborn, Linköping University, Joachim Svärdh, Thoren Innovation School, Iliana Björling, YoungDrive, Josefina Lönn, YoungDrive
	\item E-learning: Henrik Marklund, Pedagogic Development at Lurn AB % Annika Silvervarg, Educational Technology Group at Linköping University.
    %\item Others
\end{itemize}

\subsection{Literature}

Below, the planned literature basis is presented, for each relevant subject the master thesis will involved. The topics are: \\

\begin{itemize}
	\item Service Design 
    \citep{servicedesign-ruth},
    \citep{servicedesign-balis},
    \citep{servicedesign-ruth}
    \citep{servicedesign-ideo}
    \citep{servicedesign-stickdorn}
    
    \item Social Innovation 	
    \citep{socialinnovation-ehn}

	\item Interaction Design, Product Development, Usability 
    \citep{interactiondesign-sierra-youtube}, 
    \citep{interactiondesign-badass}
    \citep{interactiondesign-lowgren}
    \citep{interactiondesign-beyer}
    
    \item Entrepreneurship \& Entrepreneurship Education
    \citep{entrepreneurship-pihkala}
    
    \item Learning % formative assessment (given to you, for your own sake) instead of summative assessment (given to the employee, for the employee's sake). You have to secure that it's a process. You have to see that there is an effect! "Assessing for Learning" :) - much debated, has drawbacks. Feedback is one of the most effective ways for learning.
    \citep{effectivelearning-robert}
    \citep{learning-krathwohl}
    \citep{learning-ucla}

	\item Motivation
    \citep{motivation-pink}
    
    \item Edtech \& E-learning \& Gamification
    \citep{edtech-clark} 
    \citep{edtech-sjoden} 
    \citep{edtech-dangelo}
\end{itemize}

%\subsection{Entrepreneurship}

\input{theory/learning/entrepreneurship/definition}

\subsubsection{Entrepreneurship Education}

"Entrepreneurship Education in Schools: Empirical Evidence on the Teacher’s Role" says that "The findings indicate that the training teachers have received in entrepreneurship seems to be the main factor determining the observable entrepreneurship education provided by the teachers."

%\citep{entrepreneurship-pihkala}

In this study, we aimed to bring empirical data into the discussion on entrepreneurship education, as there are still few empirical studies available on the topic area (Dickson et al., 2008),

Also assessment practices that include peer and self-assessment have brought new depth into assignments and their completion. Similarly, activity outside of the classroom (Fayolle \& Gailly, 2008; Kickul et al., 2010; Shepherd, 2004; Solomon, 2007) is stated to have widened learners’ perceptions of their possibilities to be active citi- zens, and to also have clarified the role of different actors in society. In addition to the above, Rae and Carswell (2001) utilize entrepreneurship cases to analyze how the self-confidence and self-awareness of learners have grown.
Fiet (2001a) presented a group of methods and argues that both teachers and learners may become bored in the classroom if the teaching is predictable and the learners encounter no surprises.

Finally, the teacher is the central actor in entrepreneurship education and the teachers’ role in defining the time, frequency, contents and methods of entrepreneurship education is decisive. (Fiet, 2001a; Jones, 2010; Lobler, 2006; Seikkula-Leino, Rusko- vaara, Ikavalko, Mattila, \& Rytkola, 2010; Ruskovaara \& Pihkala, 2013).

% Teachers’ gender and entrepreneurship education practices.
Even though we have found studies with a feminist approach to entrepreneurship edu- cation and studies concerning women entrepreneurs (Komulainen, Keskitalo-Foley, Korhonen, \& Lappalainen, 2010; Korhonen, 2012), their findings do not show any indications of differences or similarities between women and men. According to Bennett (2006), a lecturer’s gender does not play a significant role in inclining entrepreneur- ship education. On this basis, we formulate the following proposition:

Proposition 1: Entrepreneurship education practices do not differ between male and female teachers.

% Teachers’ business enterprise background’s positive effect on entrepreneurship education.

Proposition 2: The stronger the teacher’s business back- ground is, the more he/she is bound to execute entre- preneurship education.

Proposition 3: The more the teacher has work experience, the more he/she is inclined to conduct entrepreneurship education.

Proposition 4: Entrepreneurship education differs between education levels.

Proposition 5: Enterprise-related teacher training positively affects teachers’ entrepreneurship education practices.

% Viktigt för YoungDrive, såklart!
Furthermore, the teacher’s professional teaching experience has no sig- nificance in terms of entrepreneurship education. These findings suggest that as a competence area, entrepreneur- ship education is not dependent on the teacher’s experi- ence as a teacher.

% Elena Ruskovaara is working as a researcher, lecturer, and project manager at Lappeenranta University of Tech- nology. For the past 15 years, she has worked in the field of further education for teachers and has been involved in many national and international entrepreneurship proj- ects. Her main interests are entrepreneurship education and especially the challenges of measuring and evaluating entrepreneurship education.

A large number of useful methods and practices have been discov- ered (Seikkula-Leino, 2007), and training concerning dif- ferent pedagogical solutions could be of great value. For example, the playful side of teaching and learning (Solo- mon, 2007) as well as teacher training that develops the competences of a mentor, enabler, or coach should enhance entrepreneurship education practices. When shifting the focus from Gibb’s (2002) idea of developing students’ understanding of entrepreneurship to the teacher, what are the ways for a teacher to see, feel, do, think, and learn entrepreneurship? How to provide teachers with the skills to cope with, create, and perhaps enjoy uncertainty and complexity?

Dickson et al. (2008) found that entrepreneurship education correlates positively with entrepreneurial activity, but admit the challenges of the long time span between the educational experience and the actual entrepreneurial behavior that follows. This, together with other findings, shows a great need for longi- tudinal research.

% Refer to entrepreneurship definition}

\input{theory/learning}

\section{Design}

\textbf{What contextual technical constraints need to be overcome, and what compromises need to be made, in design of the app?}

The goal is to make the app as sophisticated as possible with as little menace as possible.

The overall question is how much the designer needs to know about the present situation in order to have a good foundation for the design work. In order to decide, she needs to \citep{interactiondesign-lowgren}:

\begin{itemize}
    \item She has to decide which dimensions of a design situation to examine
    \item Other issues include how much information is needed, what techniques and methods are suitable, and how much time is available.
\end{itemize}

There are three roles an interaction designer can take \citep{interactiondesign-lowgren}: computer expert, socio-technical expert, and political agent.

It is very important to understand the constraints when planning a proposed method. Therefore, these are delved deeper into in this section.\\

\subsection{Digital Learning}

%\citep{edtech-clark}
%\citep{edtech-sjoden}
%\citep{edtech-dangelo}

\subsubsection{Mobile Learning}

\textbf{The use of deliberates practices on a mobile learning environment}

TODO, superbra artikel

\textbf{An experiment for improving students performance in secondary and tertiary education by means of m-learning auto-assessment}

Luis de-Marcos

% Proved successfull, everyone improved their knowledge. I can use a similar methology, and compare to the development context.

--

Huang et al. (Huang et al., 2008) indicated the common problems encountered in m-learning applications: (1) software integration, (2)
limitations of the web browser, (3) interface usability, (4) reduced size of the screen, and (5) limitation of the battery life. Such limitations are
of particular relevance when the application is intended to run on students’ personal phones; in this case, decisions need to be taken in an
attempt to alleviate the impact such issues may have. Of the problems listed above, item 2 can be mitigated by developing a mobile
application that does not run on the web browser. Items 3 and 4 can be alleviated by designing an interface that minimizes the amount of
information displayed and the input required from the user. This was the main reason for preferring multiple-choice questions to other
kinds of questions, since these questions can usually be stated in a few lines and require the selection of one or more choices. A few mobile
phone buttons can then be programmed to select/unselect each option. Moreover, various experimental studies (Chen, 2010; Ventouras,
Triantis, Tsiakas, \& Stergiopoulos, 2010) support the validity of this assessment method. Solving the problem represented by item 5 was
beyond the scope of this study; however, students were advised to charge their devices before taking the tests and teachers were advised to
design tests of no more than approximately 10 questions, in order to reduce connection times to a maximum of 20 min. Finally, item 1 was
especially difficult to tackle. When the technological framework was set up, our decision was to define the minimal software requirements
that handheld devices would have to meet in order to run the application.

% NTA Digital, Om Digitalt Lärande, Att lära med digitala verktyg
% http://ntadigital.se/teacher/tutorings/2


\subsection{Service design methodology}

Below, brief descriptions of the five principles of service design is described, together with how the work is divided into iterations, and examples of tools that can be applied.

\subsubsection{The five principles}
Stickdorn \cite{stickdorn} describes five principles that constitute service design thinking, and how to follow these.

The book describes how to follow these principles, by making the process user-centered (e.g. via design ethnography), co-creative (involve all stakeholders) and holistic (keep the big picture). Sequencing (visualize the service, and make iterations) evidencing (make the service tangible) are the two last important principles.

\subsubsection{Sequencing: The iterative process}
While literature and practice refer to various frameworks, with different number of steps, every service design project includes: exploration, creation, reflection and implementation \cite{stickdorn}.

Nissar \cite{expedition-mondial} suggests a model where one iteration consists of insights, ideation, trigger material, and interactions.

The iterations should come closer and closer to a desired outcome. It is not always obvious what this outcome is. For each iteration, the process takes the project closer, from Why? to What? to How?, often with overlaps \cite{expedition-mondial}.

\subsubsection{Tools}

There are a number of popular service design tools that follows the five principles, e.g. how to make it user-centered.

Explorative tools are e.g. Shadowing, Customer Journey Map, Contextual Interviews, The 5 Why's (same as "Why-why-why" within interaction design \cite{thoughtful}), Cultural Probes, Mobile Ethnography and Personas.

Tools to create and reflect can be done via a certain work methodology, e.g. agile development, and structuring and inspiring brainstorms, e.g. via "What if...?" and Co-Creation, inviting stakeholders in the creation process.

%\subsubsection{Relevancy within Social Innovation}
%\citep{socialinnovation-ehn}

%\subsubsection{Service Design Thinking}

%\subsubsection{Methodology}

\textbf{The iteration process}

The time in Uganda is divided into three iterations. For each iteration, the result becomes more and more clear. In iteration 1, there is a very broad scope, without digital focus whatsoever, where iteration 2 and 3 gradually introduces the digital solution. See figure \ref{fig:iterationprocess}.

%\begin{wrapfigure}{r}{0.25\textwidth} %this figure will be at the right
%    \centering
%    \includegraphics[width=0.25\textwidth]{IterationProcess.png}
%    \caption{Iteration process}
%    \label{fig:iterationprocess}
%\end{wrapfigure}

\begin{figure}[h]
    \centering
    \includegraphics[width=0.8\textwidth]{IterationProcess.png}
    \caption{The iteration process consists of a number of iterations with different focus, starting with broad strokes, and narrowing down into a concrete product. Between iterations, the overlap between "Why?" and "How?", "How?" and "What?", signals that there is a learning process which means conclusions may need to be quickly questioned as new insights emerge. This is especially important in projects where you work with an unfamiliar target group and there are several uncertainties and constraints.}
    \label{fig:iterationprocess}
\end{figure}

\textbf{One iteration} \\
In the way of reasoning around development and design for learning, the steps for each iteration, see figure \ref{fig:iteration}, might be translated into:

\begin{enumerate}
\item Interactions, where you are listening, the \textit{Explorative phase}. 
\item Insights, which is where you use the Interactions in order to try to understand, the \textit{Understanding phase}. % better word+
\item Ideation, where you find possible ideas and when creation of new version of the app is done, the \textit{Design phase}.
\item Trigger material, where material is developed to test the outcome of our evaluation in the next round, the \textit{Trigger development}.
\end{enumerate}

%\input{theory/design/service_design_stoff}


\subsection{Interaction Design}

	\subsubsection{Definition}

\section{Technology}

Rapid App Development
    
  \subsection{The Full-Stack Developer}

  \subsection{Full-Stack using Meteor}

  \subsection{Front-End using React}
  
  \subsection{Staging environment using Heroku}
  Needed when the Meteor free tier was removed. Connected to deploy from GitHub branches automatically. Could have benefitted from CI, passing tests before ready for production. Solved this by having a stage environment (since April 19th) where stage is YoungDrive-beta (branch Iteration 4), and YoungDrive is master.
  
  \subsection{Hosting using GitHub}
  Using Issues, Releases, Commits, Collaborator