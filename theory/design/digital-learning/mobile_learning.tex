\subsubsection{Mobile Learning}

\textbf{The use of deliberates practices on a mobile learning environment}

TODO, superbra artikel

\textbf{An experiment for improving students performance in secondary and tertiary education by means of m-learning auto-assessment}

Luis de-Marcos

% Proved successfull, everyone improved their knowledge. I can use a similar methology, and compare to the development context.

--

Huang et al. (Huang et al., 2008) indicated the common problems encountered in m-learning applications: (1) software integration, (2)
limitations of the web browser, (3) interface usability, (4) reduced size of the screen, and (5) limitation of the battery life. Such limitations are
of particular relevance when the application is intended to run on students’ personal phones; in this case, decisions need to be taken in an
attempt to alleviate the impact such issues may have. Of the problems listed above, item 2 can be mitigated by developing a mobile
application that does not run on the web browser. Items 3 and 4 can be alleviated by designing an interface that minimizes the amount of
information displayed and the input required from the user. This was the main reason for preferring multiple-choice questions to other
kinds of questions, since these questions can usually be stated in a few lines and require the selection of one or more choices. A few mobile
phone buttons can then be programmed to select/unselect each option. Moreover, various experimental studies (Chen, 2010; Ventouras,
Triantis, Tsiakas, \& Stergiopoulos, 2010) support the validity of this assessment method. Solving the problem represented by item 5 was
beyond the scope of this study; however, students were advised to charge their devices before taking the tests and teachers were advised to
design tests of no more than approximately 10 questions, in order to reduce connection times to a maximum of 20 min. Finally, item 1 was
especially difficult to tackle. When the technological framework was set up, our decision was to define the minimal software requirements
that handheld devices would have to meet in order to run the application.