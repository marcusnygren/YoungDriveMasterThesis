\subsection{Service design methodology}

Below, brief descriptions of the five principles of service design is described, together with how the work is divided into iterations, and examples of tools that can be applied.

\subsubsection{The five principles}
Stickdorn \cite{stickdorn} describes five principles that constitute service design thinking, and how to follow these.

The book describes how to follow these principles, by making the process user-centered (e.g. via design ethnography), co-creative (involve all stakeholders) and holistic (keep the big picture). Sequencing (visualize the service, and make iterations) evidencing (make the service tangible) are the two last important principles.

\subsubsection{Sequencing: The iterative process}
While literature and practice refer to various frameworks, with different number of steps, every service design project includes: exploration, creation, reflection and implementation \cite{stickdorn}.

Nissar \cite{expedition-mondial} suggests a model where one iteration consists of insights, ideation, trigger material, and interactions.

The iterations should come closer and closer to a desired outcome. It is not always obvious what this outcome is. For each iteration, the process takes the project closer, from Why? to What? to How?, often with overlaps \cite{expedition-mondial}.

\subsubsection{Tools}

There are a number of popular service design tools that follows the five principles, e.g. how to make it user-centered.

Explorative tools are e.g. Shadowing, Customer Journey Map, Contextual Interviews, The 5 Why's (same as "Why-why-why" within interaction design \cite{thoughtful}), Cultural Probes, Mobile Ethnography and Personas.

Tools to create and reflect can be done via a certain work methodology, e.g. agile development, and structuring and inspiring brainstorms, e.g. via "What if...?" and Co-Creation, inviting stakeholders in the creation process.

%\subsubsection{Relevancy within Social Innovation}
%\citep{socialinnovation-ehn}

%\subsubsection{Service Design Thinking}

%\subsubsection{Methodology}

\textbf{The iteration process}

The time in Uganda is divided into three iterations. For each iteration, the result becomes more and more clear. In iteration 1, there is a very broad scope, without digital focus whatsoever, where iteration 2 and 3 gradually introduces the digital solution. See figure \ref{fig:iterationprocess}.

%\begin{wrapfigure}{r}{0.25\textwidth} %this figure will be at the right
%    \centering
%    \includegraphics[width=0.25\textwidth]{IterationProcess.png}
%    \caption{Iteration process}
%    \label{fig:iterationprocess}
%\end{wrapfigure}

\begin{figure}[h]
    \centering
    \includegraphics[width=0.8\textwidth]{IterationProcess.png}
    \caption{The iteration process consists of a number of iterations with different focus, starting with broad strokes, and narrowing down into a concrete product. Between iterations, the overlap between "Why?" and "How?", "How?" and "What?", signals that there is a learning process which means conclusions may need to be quickly questioned as new insights emerge. This is especially important in projects where you work with an unfamiliar target group and there are several uncertainties and constraints.}
    \label{fig:iterationprocess}
\end{figure}

\textbf{One iteration} \\
In the way of reasoning around development and design for learning, the steps for each iteration, see figure \ref{fig:iteration}, might be translated into:

\begin{enumerate}
\item Interactions, where you are listening, the \textit{Explorative phase}. 
\item Insights, which is where you use the Interactions in order to try to understand, the \textit{Understanding phase}. % better word+
\item Ideation, where you find possible ideas and when creation of new version of the app is done, the \textit{Design phase}.
\item Trigger material, where material is developed to test the outcome of our evaluation in the next round, the \textit{Trigger development}.
\end{enumerate}

%\input{theory/design/service_design_stoff}
