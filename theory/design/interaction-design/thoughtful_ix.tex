% About Thoughtful Interaction Design
%The authors of this book do not teach ready-to-use methods and techniques for creating design visions. They have another purpose—namely, to write a book prompting thoughtful reflection on what it means to be a good interaction designer and how to be prepared to act professionally in the design situations. What kind of knowledge is this? What kind of design ability is needed to address these concerns?
% the "reflective practicioner" (Donald Schön), design "talking back" to them, how they reflect-in-action, and how they develop theories-in-use - to the new context of information technology.

%we argue that the discipline of developing digital artifacts requires a new perspective on design— thoughtful design. Thoughtful design is needed since the design challenges we face today are more complex than ever. Research and experience provides us with more and more knowledge and information. But rapid technological development prevents us from experimenting with and learning about all the new possibilities created by new technology and new knowledge. Consequently, designers today have to deal with a reality marked by complexity and change. It is essential that members of the design discipline collectively find appropriate forms for growing and nurturing design knowledge. We believe such a demand can only be met by an approach based on a foundation of design thoughtfulness.
%We aim to critically examine and challenge prevailing ideas in the IT industry and IT academia on what interaction design is and what it ought to be.
% When we write about the “how” of interaction design, we do not address how to do interaction design but rather how to thinkabout interaction design. Hence, this book is not a complete manual on interaction design. It has to be complemented with other material where necessary skills, methods, and techniques are introduced.

\subsection{Thoughtful Interaction Design}

These are information from the book "Thoughtful Interaction Design":

There are a few core concepts that require short introductions, since they are used throughout the book "Thoughful Interaction Design: A Design Perspective on Information Technology". These concepts are: interaction design, design process, design situation, and digital artifact.

Interaction design refers to the process that is arranged within existing resource constraints to create, shape, and decide all use-oriented qualities (structural, functional, ethical, and aesthetic) of a digital artifact for one or many clients.

First, it is a design discipline, which means that concepts and theories from other design disciplines and from the transdisciplinary academic field of design studies are relevant in understanding and developing interaction design.

Secondly, interaction design has a strong relation to the academic field of human-computer interaction, where the human use of digital artifacts has been studied and enhanced for over thirty years. Finally, the concentration on digital artifacts implies that all fields concerned with constructing and developing digital material contribute to the intellectual tradition of interaction design in various degrees. These fields include computer science, information systems, and software engineering.

\textbf{Design process}
The design process begins when the initial ideas concerning a possible future take shape. The process goes on all the way to a complete and final specification that can function as a basis for construction or production. In some cases, the final specification is identical to the final product. We do not distinguish between processes that lead to construction of new technology and processes that lead to the composition of an artifact by assembling readymade components or configuring an off-the-shelf product. In both cases, the work constitutes a design process.

\textbf{Design situation}
Design is always carried out in a context. The concept design situation refers to the situation that is both the reason for the design process to be initiated and the context within which the design work is carried out. One simple case is when an organization perceives the need for new information technology support. They ask someone to act as a designer and work with the people in the organization. In this scenario, the organization more or less becomes the design situation. In other cases, the limits of the design situation are not as clear-cut. For instance, when design is performed for a mass market on the Internet, the delimitation of the design situation becomes more complex. Another example is when design is carried out for products that people will use in their homes, their cars, or carry in their shirt pockets. A designer is always charged with figuring out the situation at hand, what should be considered to be part of the design situation, and what can be left out. The situation therefore becomes a core concept in interaction design. The situation is the starting point for the design, as well as the more or less malleable target for interventions through design. In other words, the design situation evolves along with the design process. The “now” that exists when the design process starts is not unaffected by the design work and its outcomes. Design amounts to standing in the “now” with the task of studying possible futures, or ways in which the design situation might evolve due to our intervention.

\textbf{Digital artifact}
he result of an interaction design process is what we choose to call a digital artifact. An artifact refers to “something made by humans.” This concept is normally used to denote physical objects, but it can be used in a broader sense as well. We use “digital artifact” in this book to refer to artifacts whose core structure and functionality are made possible by the use of information technology. Moreover, we limit our studies to digital artifacts that operate in rather close relations with humans in social contexts.

For instance, we will not address automated processes or fully embedded components unless they have a direct relation to users. This follows from our focus on interaction design and use-oriented qualities as opposed to information technology design in general.

Digital artifacts are commonly referred to by such terms as systems, programs, or products.

\textbf{Designer, client and user}
There are many roles and many people involved in design. The ones we will be mentioning most frequently are the designer, the client, and the user. We have tried to keep the meaning of these roles as simple as possible. A designer is any person who actively takes part in the shaping of the digital artifact. A client is a person or an organization contracting with the designer. The client typically pays for the design work and makes final decisions about whether the results are acceptable. A user is a person who will be using the digital artifact when it is implemented.

More elaborate schemes of roles are prevalent in professional IT practice. There, we typically find that our generic “designer” role is divided into a number of specializations, such as information architects, graphic designers, interface programmers, and so on. The intention behind our using a more simple set of roles is that the arguments we present reside on a more generic level, and are therefore applicable to several specialized roles after suitable appropriation and adaptation.

\subsubsection{Design theory}
This book can be viewed as an attempt to contribute to a design theory; that is, it contains ideas about the essence and nature of design work that are intended to support designers in becoming more proficient.

Our basic assumption about the design process is that its form, structure, and qualities are not given or ruled by laws of nature. Design work is given form and structure by designers’ own thoughts, considerations, and actions. Its character is influenced by people’s habits, traditions, and practice.

Knowledge about design concerns differences in design traditions and practices, limitations in the design process, and the nature of design thinking. % cultural

\subsubsection{Chapter 2: The design process}

\textbf{From Vision to Specification}
The vision, the operative image, and the specification

\textbf{Design as a Thought Process}

2.2.1: The solution and the problem
The problem (the designer's current understanding of the design situation) and the solution (the designer's idea on how to shape her intervention in the situation).

Motivation till varför jag åker till Uganda: authentic attention, a possibility of "approaching reality with carefulness and concern"

2.2.2. The Process and Levels of Abstraction
Jättebra bild s. 25, motiverar att jag ändrar vision/skiss hela tiden.

The erratic path of the design process might seem unrealiable, perhaps even scary to the inexperienced designer. Since the up and down movement and changes in course are not an explicit part of most design methodologies, the inexperienced designer might feel as if they are "wrong."

s. 26: perfekt exempel på Service design, Susanna?

"A good designer is therefore someone who has the ability to work in a highly restricted situation and still be able to create surprising and satisfactory solutions and designs."

2.2.5. Exploring Design Possibilities

Extremt bra mål! "Rather, a design process is driven by the will to learn as much as possible about different opportunities existing in a particular situation."

Moving forward in a design process usually means the designer has to explore as many possibilities as time and resources allow. Consequently, the early design work [...], several ideas are developed instead of focusing a single on. The aim is to explore the spaces of possible designs and problem formulations.

Divergent thinking - considering serveral ideas in parallel - has an important practical advantage. In a design process, it is not uncommon for a designer to "fall in love" with a favorite idea and defend it by refuting all criticism from other team members.

2.2.6 Capturing the Design Situation

* Understanding of what constitutes reality
* What we believe is possible to know about reality

Since a design situation can be approached from any aspect, a designer has to make a decision on what needs to be studied most carefully and which dimensions of the situation will have a real impact on the design process.

"Crudely stated, a researcher is interested in reality whereas a designer is interested in what reality could become."

2.2.7 the Final Composition
Thoughtful compositions.

"We all know of examples when a designer of a digital artifact finds it hard to understand why athe users do not want to use her "perfect" design. The users fail to find the artifact useful since it does not create a meaningful whole together with their existing reality. When this happens, it does not matter how perfect the design is by itself. In the design situation, it will be judged as a part of the whole composition."

2.3 Design as a Social Process
Viktigt: The ones involved: Core, Periphery and Context

In order to lead the project in a successful way: Caring for the vision, How to deal with relationships and roles, and How to see the process as a project.

2.3.1 Caring for the Vision

Three different strategies are outlined: facilitate informal communication across organizational borders, appoint a "super designer", or write documents of various types (such as project-specific design rules, specifications, and descriptions). % Pros and cons are outlined, especially about documentations

2.3.2 Relationships and Roles
Om att personer kanske inte vill vara med i design-processen

Three roles an interaction designer can take: computer expert, socio-technical expert, and political agent:
I'm the computer expert kind of designer. :/ :) :(

Trend: vi går mot socio-technical expert (service design)

2.3.3 Design as a Project

The client needs to have knowledge about the process in order to handle implementation planning properly. The client may also want to create decision points along the path of a project. Moreover, other stakeholders in a design project often have their own reasons for some kind of organization and management of the process. For example, managers in departments adjacent to the client’s may need to coordinate their own activities with development of the new system.

Interaction design inherits a rich history of methods, models, and methodology from fields such as human-computer interaction, systems development, and software engineering. Most of those tools are intended to support the coordination, organization, and management of what we would call “the design process.”

There is always a tension between the larger organizational context and the individual project.
This leads to the conclusion that “real” design is about finding ways to design a project within these preconditions and limitations—by accepting them or trying to change them. A design project is itself designed and depends on creative and innovative thinking for its success.

2.3.4 Designing the Design Process

The issue of designing the design process, however, is not as well addressed. It is usually assumed that the solution is to use a predesigned model or method. For purposes of managing the design process, this solution may be adequate. But if we assume that the design process has to be created, invented, and designed, then other aspects appear as crucial.

%Thoughtful design has to be based on a realization and understanding of the fundamental aspects of the design process described in this chapter.

A thoughtful designer knows that almost nothing is given or true when it comes to what and how to design. It is also obvious that the complexity of the process demands conceptual clarity from the designer. The thoughtful position is to view the whole situation as a design task: to design the design process.

\subsubsection{Chapter 3: The designer}
% What constitutes a thoughtful designer? Some ways for a designer to develop her design ability.

%What is a designer and what does it take to be a good one? Is it possible to learn to be a good designer? The image of the designer that we present in this chapter is based on research findings from various sources, as well as our own experience as designers and design teachers.

To us, anyone participating in design work that includes the use-oriented shaping of digital artifacts is in principle an interaction designer.

\subsubsection{Chapter 4: Methods and techniques for interaction design}
Knowledge about design concerns differences in design traditions and practices, limitations in the design process, and the nature of design thinking.

4.1.2 Why-why-why?
In order to keep a broad perspective during inquiry-intensive phases of a design process, it is essential to question and move beyond the problem as it is currently perceived. One way of doing this is to ask a series of why-questions and build a chain of reasoning backward from the original formulation.

%Assume, for instance, that we are involved in a project with a local hospital where the handling of X-ray images seems problematic. There is a manual X-ray archive, which works like a library for interlibrary loans, where all the images are stored. Medical staff and others authorized to access X-ray images do so by f illing out request forms and, after a day or two, the requested images arrive via internal mail unless they are already on loan to someone else. Let us further assume that the project we are engaged in was initiated by a dissatisfied physician:

%Physician (P): I am not happy with the way X-ray images are currently handled at this hospital.
%Designer (D): Why?
%P: Because I have to order my images several days in advance.
%D: Why?
%P: I never know in advance how long it will take to get them.
%D: Why?
%P: If the images I want are on loan to someone else, then I have to wait until they return them. D: Why?
%P: There is only one copy of each image.

Note that the why-chains can branch off in different directions, depending on how the why-question is interpreted and whose perspective is assumed in answering:

%P: I am not happy with the way X-ray images are handled today.
%D: Why?
%P: Because I get the wrong images sometimes.
%D: Why?
%P: Because the X-ray archive staff makes mistakes.
%D: Why?
%P: I have no idea. Too much work, perhaps.

It is important to notice that the why-why-why method is a way of probing the problem formulation and not a systematic method that infallibly leads to good results. In order to create a useful why-chain, a designer has to be sensible to which paths are promising and which ones are dead ends:

%P: I am not happy with the way X-ray images are currently handled at the hospital.
%D: Why?
%P: Because I am generally grumpy and frustrated.

%This why-why-why example illustrates what we discussed earlier—namely, that

the result of a method is never better than the people involved in carrying out the process.

There are an infinite number of why-question chains leading to completely different conclusions.

Therefore, setting up a useful why-chain is a bit like the whole design process in a micro-perspective: it involves designing the problem and the solution in parallel.

In inquiring about a design situation, the designer has to reflect on what kind of information she really needs. Inquiry without intention—that is, without an idea of purpose and outcome—easily becomes a randomly executed examination of the situation. The design of an inquiry, and choice of methods and techniques for doing it, is therefore one of the most important aspects of design.
% Diskussion: ang. att vilja, men inte borde, styra svaren

\subsubsection{Chapter 7: Thoughtful Design}

7.6The Material without Qualities
Interaction design is about shaping digital artifacts.

It is about giving structure and form to human environments and activities.

Interaction designers create spaces for action in which parts of people’s working lives and private lives take place.

7.7 Being Thoughtful

What is needed to deal with the complexities of design, however, is not necessarily more information, but rather a bit more conceptual clarity from the designer.

A thoughtful designer, equipped with appropriate tools for reasoning, will be more able to sort out what is important, make necessary judgment calls, distinguish true needs for more information from better-safe-than-sorry approaches, and identify fruitful directions in the exploration of possible futures that is called design.

%The ideas we have presented in this book are intended to serve as such tools for reasoning. The responsibility of using these tools skillfully will always rest with the designer.
