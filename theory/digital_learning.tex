\section{Learning with Mobile Technology}

    YoungDrive has asked for a multiple-choice question learning game used for assessment and improvement of the YoungDrive learning objectives during and after the training. In recent times, digital learning (e-learning) has had a tremendous impact both outside and inside the classroom. With a growing teacher interest, research so far shows that digital education is difficult and risky, but potentially rewarding \citep{luckin}. Thus, digital education simultaneously shows great potential and vulnerability. Much of the research on digital learning to date on digital games has focused on proof-of-concept studies and media comparisons \citep{luckin}. A study by \cite{gates} motivates why a digital tool or game is a good thing by showing an increase in learning outcomes, relative to non-game instructional conditions.

    %Luckin \citep{luckin} emphasises the need to care for the context when designing for learning. Stickdorn \citep{stickdorn} exemplifies how the design process should be altered when the context is social innovation. Service design in a social innovation context is called "social design", and is a new field. \citep{stickdorn}. No longer is service design solely focused on creating and promoting consumer goods, but to offer services to society. The design process should be designed to tackle a social issue, or with the intent to improve human lives. The focus is on delivering positive impact.

    A large amount of development has taken place on diagnostic testing environments, that allow teachers and learners to assess present performance against prior performance \citep{luckin}. There are numerous examples of developments in \textit{e-assessment} using mobile environments, as well as immersive environments and social and collaborative environments. One such example is the educational app platform iSchool, developed by iSchool Zambia \citep{ischool}. The app has been praised and made popular as it was designed to fit the Zambia school curriculum to the point, accessible as a home edition, pupil edition and teacher edition. Interest in formative e-assessment is increasing.  It has been shown that multiple-choice tests in e-assessment can be used to good effect \citep{nicol}.

    Two studies within electronic assessment (e-assessment) or mobile learning (m-learning) have been done that this master thesis is inspired by. One uses deliberate practices on a mobile learning environment \citep{yengin}. The other focused on and further validated the research of various experimental studies, that multiple-choice can be a viable auto-assessment method to improving student learning, especially for m-learning \citep{de-marcos}.

    %Nicol, D. (2007). E‐assessment by design: using multiple‐choice tests to good effect. Journal of Further and higher Education, 31(1), 53-64.

    \cite{luckin} says that further consideration should be given to how technology can be used to enable the assessment of knowledge and skills not usually distinguished within current curricula. \cite{gates} encourages a focus on how theoretically-driven decisions influence learning outcomes: for the broad diversity of learners, within and beyond the classroom. Combining these two, introducing e-assessment of entrepreneurship in a developing country context is a contribution to existing research.
