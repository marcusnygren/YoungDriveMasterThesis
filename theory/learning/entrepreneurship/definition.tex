\subsubsection{Definition}

Från artiklen "The role of Service Design in the Effectual Journey of Social Entrepreneurs" % http://www.servdes.org/conference-2014-lancaster/:

Social Entrepreneurship Background
Entrepreneurship research aims at a better understanding of the highly heterogeneous process phenomenon of entrepreneurship. The term entrepreneur evolved from a French term meaning “one who undertakes or manages” and was used in the 1800s by a French economist to capture the activity of someone who creates value by “shifting economic resources out of an area of lower and into an area of higher productive and greater yield” (Martin \& Osberg, 2009, p. 31). Although the field has been established as a distinct domain of research, there is still no consensus about the object of study in the field with the concept of entrepreneurship being reinterpreted constantly (Cornelius et al., 2006). Some persisting perspectives include a focus on facing uncertainty (Knight, 1921), on introducing new processes and products by innovating (Schumpeter, 1934) and recognizing opportunities (Kirzner, 1978).

Recently, the phenomenon of entrepreneurship is conceived as more multifaceted than in the past (Bruyat \& Julien, 2004) with researchers looking into its role in society and its social dimensions challenging the economic discourse that is dominating the field (Steyaert \& Katz, 2004). Some of the assumptions that stem from the association of the field with economics, for example the fact that motivation of entrepreneurs is mainly wealth accumulation do not appear appropriate (Mitchel et al., 2007) as entrepreneurship is increasingly identified as an activity that contributes to society in other significant ways that are not captured by the commercial entrepreneurship literature (Steyaert \& Katz, 2004).