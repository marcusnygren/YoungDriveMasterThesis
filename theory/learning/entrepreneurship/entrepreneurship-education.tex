\subsubsection{Entrepreneurship Education}

"Entrepreneurship Education in Schools: Empirical Evidence on the Teacher’s Role" says that "The findings indicate that the training teachers have received in entrepreneurship seems to be the main factor determining the observable entrepreneurship education provided by the teachers."

%\citep{entrepreneurship-pihkala}

In this study, we aimed to bring empirical data into the discussion on entrepreneurship education, as there are still few empirical studies available on the topic area (Dickson et al., 2008),

Also assessment practices that include peer and self-assessment have brought new depth into assignments and their completion. Similarly, activity outside of the classroom (Fayolle \& Gailly, 2008; Kickul et al., 2010; Shepherd, 2004; Solomon, 2007) is stated to have widened learners’ perceptions of their possibilities to be active citi- zens, and to also have clarified the role of different actors in society. In addition to the above, Rae and Carswell (2001) utilize entrepreneurship cases to analyze how the self-confidence and self-awareness of learners have grown.
Fiet (2001a) presented a group of methods and argues that both teachers and learners may become bored in the classroom if the teaching is predictable and the learners encounter no surprises.

Finally, the teacher is the central actor in entrepreneurship education and the teachers’ role in defining the time, frequency, contents and methods of entrepreneurship education is decisive. (Fiet, 2001a; Jones, 2010; Lobler, 2006; Seikkula-Leino, Rusko- vaara, Ikavalko, Mattila, \& Rytkola, 2010; Ruskovaara \& Pihkala, 2013).

% Teachers’ gender and entrepreneurship education practices.
Even though we have found studies with a feminist approach to entrepreneurship edu- cation and studies concerning women entrepreneurs (Komulainen, Keskitalo-Foley, Korhonen, \& Lappalainen, 2010; Korhonen, 2012), their findings do not show any indications of differences or similarities between women and men. According to Bennett (2006), a lecturer’s gender does not play a significant role in inclining entrepreneur- ship education. On this basis, we formulate the following proposition:

Proposition 1: Entrepreneurship education practices do not differ between male and female teachers.

% Teachers’ business enterprise background’s positive effect on entrepreneurship education.

Proposition 2: The stronger the teacher’s business back- ground is, the more he/she is bound to execute entre- preneurship education.

Proposition 3: The more the teacher has work experience, the more he/she is inclined to conduct entrepreneurship education.

Proposition 4: Entrepreneurship education differs between education levels.

Proposition 5: Enterprise-related teacher training positively affects teachers’ entrepreneurship education practices.

% Viktigt för YoungDrive, såklart!
Furthermore, the teacher’s professional teaching experience has no sig- nificance in terms of entrepreneurship education. These findings suggest that as a competence area, entrepreneur- ship education is not dependent on the teacher’s experi- ence as a teacher.

% Elena Ruskovaara is working as a researcher, lecturer, and project manager at Lappeenranta University of Tech- nology. For the past 15 years, she has worked in the field of further education for teachers and has been involved in many national and international entrepreneurship proj- ects. Her main interests are entrepreneurship education and especially the challenges of measuring and evaluating entrepreneurship education.

A large number of useful methods and practices have been discov- ered (Seikkula-Leino, 2007), and training concerning dif- ferent pedagogical solutions could be of great value. For example, the playful side of teaching and learning (Solo- mon, 2007) as well as teacher training that develops the competences of a mentor, enabler, or coach should enhance entrepreneurship education practices. When shifting the focus from Gibb’s (2002) idea of developing students’ understanding of entrepreneurship to the teacher, what are the ways for a teacher to see, feel, do, think, and learn entrepreneurship? How to provide teachers with the skills to cope with, create, and perhaps enjoy uncertainty and complexity?

Dickson et al. (2008) found that entrepreneurship education correlates positively with entrepreneurial activity, but admit the challenges of the long time span between the educational experience and the actual entrepreneurial behavior that follows. This, together with other findings, shows a great need for longi- tudinal research.

% Refer to entrepreneurship definition}