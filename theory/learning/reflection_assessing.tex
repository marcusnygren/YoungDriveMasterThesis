\subsubsection{Reflection \& Assessing}

Luckin: "Knowing what learners know, and don’t know, is crucial to effective learning. if learners attempt tasks that are too complex, they are likely to fail; if they attempt tasks that are too easy they may not progress as they should. accurate information about learners’ current understanding can help us to offer appropriate feedback and increase learners’ own awareness of their learning needs. accurate assessment and analysis also allows learning to be tailored. learners differ physically, emotionally and cognitively, and in their ability to understand what they know and how they can progress. recognising these differences can help to ensure that everyone achieves their full potential.

Two important processes underpin how we identify what learners know and understand. \textit{Reflection} involves learners considering their own learning activity. by reflecting learners develop the skills and self-awareness they need to refine their own learning activities. \textit{Assessing} involves teachers considering the learners’ learning activity. effective assessing provides feedback and feed–forward advice to a learner about their learning activity: learners must be able to respond to a critical voice. self–assessment requires the learner to provide that critical voice, which links back to the process of reflection. We must
also recognise that at times teachers are also learners, for example, when taking part in professional development activities. The processes of reflection and assessment are no less important for them as they are for any other learner."

It's not only for the teacher though, that reflection is crucial. Learning by Thinking, % https://drive.google.com/file/d/0BzlK1PD8EE75THdnRnAzWDl0VDg/view?usp=sharing
suggests that reflection has been a overlooked area of research for a long time, but that it is something that can be more effective than additional learning.

"In particular, we theorize that, once a person has accumulated a certain amount of experience with a task, the benefit of additional practice is inferior to the benefit of reflecting upon the accumulated experience. In other words, the intentional attempt to synthesize, abstract, and articulate the key lessons learned from experience generates higher learning outcomes as compared to those generated by the accumulation of additional experience."

We propose that the link between learning by thinking and greater performance is
explained by self-efficacy, or a personal evaluation of one’s capabilities to organize and execute
courses of action to attain designated goals (Bandura, 1977).
2 In particular, we claim that
reflecting on one’s past experience on the same or similar tasks allows individuals to reduce
uncertainty about their ability to perform such tasks successfully going forward. This reduced
uncertainty will translate into greater effort (Rosen, Mickler, \& Collins, 1987), which in turn will
drive the observed increase in the ability to perform well with the task at hand.

\textbf{Conclusion}
Research on learning has primarily focused on the role of doing (experience) in fostering
progress over time. Drawing on literature in cognitive psychology and neurosciences, we
propose that one of the critical components of learning is reflection, or the intentional attempt
to synthesize, abstract, and articulate the key lessons taught by experience. Using a mixedmethod
design that combines laboratory experiments with a field experiment conducted in a
large business-process outsourcing company in India, we find that individuals who are given time
to reflect on a task outperform those who are given the same amount of time to practice with
the same task. Interestingly, we show evidence that if individuals themselves are given the
choice, they prefer to allocate their time to practicing on the task rather than reflecting on it. Our
findings, however, suggest that this preference is irrational given the higher benefits associated to
reflection. Together, our results reveal reflection to be a powerful mechanism behind learning,
confirming the words of American philosopher, psychologist, and educational reformer John
Dewey (1933: 78): “We do not learn from experience...we learn from reflecting on experience.”