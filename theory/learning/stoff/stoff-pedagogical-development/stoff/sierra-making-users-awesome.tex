\subsection{Making Users Awesome}
\subsubsection{The Challenge: Prologue: Think Badass}
"We want to build products, service, and support in ways that inspire users to talk about themselves."

The brand goal is often to be perceived as awesome, while the user goal is to "Be awesome".

Too often, the goals of a company and the goals of its users aren’t just different but mutually exclusive. (se s. 25, bra bild)

If the answer lives not in the product but in the successful user, what are the common attributes of a successful user?

"Users don’t evangelize to their friends because they like the product, they evangelize to their friends because they like
their friends." (inspired by a quote from Mike Arauz)

"User Badass = User Results Badass users are better users. Much better.

It’s about what they can do or be as a result of what our product, service, experience enables.
Sustained bestsellers help their users get badass results."

"What happens if we don’t make snowboards, video editing software, or anything else that someone can become badass at?"

Regardless of how unlikely it may seem, anything can be linked to the potential for badass results.The key lies in answering this:
“What are you a subset of?” What’s your bigger compelling context?

Most products and services support a bigger, compelling, motivating context.

Tools matter. But being a master of the tool is rarely our user’s ultimate goal. Most tools (products, services) enable and support the user’s true–and more motivating–goal.

"the deeper we get into photography, the more likely we are to recognize and appreciate the benefits of higher-end cameras.
In other words, it’s the context– not the tool–that builds desirability for more/better tools.

\textbf{Being better is Better}
Being better means better results.
But being better is about more than just results. Being more skillful, more knowledgeable, more advanced is itself an intrinsically rewarding experience.

Badass is about more than just results. Badass means deeper, richer experiences.

The better you get at [x], the more detail you perceive in [x]. Greater detail is one of the most overlooked benefits of being better.To the human (and animal) brain, the ability to “pick-up” more from the environment is deeply rewarding.

While the ability to gain more resolution in music is an obvious benefit, having a higher-res experience of nearly anything makes that experience more pleasurable.

% Higher resolution means higher-end products
The better you get at [x], the more distinctions you perceive in [x]. Enhanced perception means the ability to appreciate the value of higher-end and/or more advanced versions of products. %An audiophile, for example, might perceive a substantial difference between two speaker systems, while a non-audiophile swears the speakers are identical.

"Don’t just upgrade your product, upgrade your users."

"Badass users talk.They’re our best source of authentic, unincentivized word of mouth.The word of mouth our “formula” depends on."

"Badass users talk.They can’t help it.
But sometimes they don’t need to talk about their amazing capabilities and results.
Sometimes it’s obvious." (because others talk for them, or because it's obvious"
- Did you see Roy’s latest video? The color is amazing.
- He said he used a new film-look app. I want it!

Word OF Obvious (WOFO) is even better than Word Of Mouth (WoM)

\textbf{ beware of faux-badass}
\textbf{Gamification} awards for purchases, visits to a website, comments, etc. typically reward behavior the company wants, not what the user wants.

It’s not about helping people feel badass. It’s about helping them be badass.
Have they gained new higher resolution at the bigger context they care about?
Are they now more skilled at the bigger context they care about?
Do they now know more, and can they use their new knowledge in ways they find useful or meaningful?
Are they getting better results?

Competing on Customer Service Excellence doesn’t necessarily mean User Excellence

What customers want
World-class Customer
whereas company mean World-class (Customer) Service

Competing on out-caring the competition is fragile unless “caring” means “caring about user results.”

“Feeling loved by a brand” does not mean badass. If we love
our users more than the com- petition loves theirs, the proof doesn’t live in what we do, but in what our users do as a result.

The Better User POV: (s. 50)
Don’t just make a better [X] (coach training app),
make a better User of [X] (coach training material)

\textbf{Summary so far}
Where we’re at now:
Sustained desirability drives success.
Honest, non-bribed word of mouth drives desirability. User badass drives word of mouth.
Badass not at the tool, but at what the tool enables. Badass at the bigger context.
Badass means higher resolution.
Badass means user results.
(But not faux badass)
Don’t make a better [x], make a better [user of X].

And yet we STILL haven’t answered the REAL question...

How did \textit{these} make badass users?

How do you make users badass?
How do you help them want to be badass?
How do you get them to be users before they’re badass?

We're almost there.

\subsubsection{Think Badass: Design for the Post-UX UX (s. 55)}
“Point of view is worth 80 IQ points”
To create sustained success, create high-resolution, badass users.

UX: "I  can switch between shooting stills and video with just one button"
Post-UX UX: "Wow. The comments on my video are inspiring." User Experience after using the “tool”.
The UX of results.

\textbf{All that matters is what happens when the clicking’s done
When our results are tied to the results of our users, all that matters is what happens when the clicking, swiping, interacting, using is done.
What did that experience enable? What can they now do?
What can they now show others? What will they say to others?
How are they now more powerful?
What happens after the UX drives our success. It’s what drives our users to talk about and recommend us, and it’s what leaves our users so obviously better that they might not need to.}
% Detta går att koppla jättebra till service design, tror jag - ett bra komplement

Designing for the post-UX UX means designing not just for your users but for your users’ users.
It’s not so much what our user thinks of us but what our user’s friends, family, peers think of our user.

The best place to begin is to help a few people become noticeably better at the bigger context, and let word of mouth and word of obvious emerge as natural side effects.

\subsubsection{The User Journey}
The ideal (user) curve continues up and to the right (between ability and time). Users don’t drop out or plateau.They keep going.They keep making progress. They keep getting better.

Along the way they’ll hit many milestones, some more crucial than others. But of all points on the curve, the two most important are crossing the Suck Threshold and crossing the Badass Threshold.

Later we’ll look at strategies for helping our users push through these thresholds to keep moving forward.

The most vulnerable time for new users is in The Suck Zone. If we lose them here, we won’t get them back.  ("Too painful, not worth it")

Your users clawed their way over the Suck Threshold. Finally they can do something.They’re competent.
Along comes the NEW AWESOME IMPROVED version. “Upgrade!” we said.“You’ll love it!” we said.
And just like that, they’re back in the Suck Zone.Whether it’s a new version of a product or a new process, change is most painful when we lose previously hard-fought ability. It now takes more effort and frustration to do what we used to do.

Returning to the Suck Zone rarely seems worth it, despite promises that the upgrade will makes us more powerful.

\textbf{Crossing the Suck Threshold and leveling off} (s. 70)
They crossed the Suck Threshold and moved firmly into intermediate range. And stayed. They’re still your user.This might look good on a spreadsheet for user retention.


But if they’re no longer moving up and to the right, they aren’t increasing resolution, gaining new skills, or becoming more powerful.Their enthusiasm for their new abilities and results will slowly fade.

The benefits of badass depend on users steadily moving
up the curve, not leveling off at “competent.” Consistent, competent users who level off might be reasonably happy and comfortably loyal, but for our sustainable bestseller, comfortably competent might not be enough.We need those dinner party conversations.

The Stuck Zone looks good, but isn’t

Auto mode is seductive. It’s easy, comfortable, safe. But if our users don’t break free, they might never develop a deeper interest in photography. If they don’t push past the Stuck Zone, they aren’t painting with light, they’re just... taking snapshots.

\textbf{What if entry-level/beginner tools are all we offer?}
Entry-level \textit{products} don’t have to mean entry-level \textit{context}.

An entry-level camera maker can help users become badass at the subset of photography skills that don’t depend on manual controls. Instead of helping them master full exposure control with shutter speed and aperture, what about gear-independent artistic and technical skills like composition and lighting?

If you make only entry-level tools, think about the context subset you can enable

"What if my users don’t want to be a total badass? What if they just want to be reasonably good at it?"
"And that whole path-to-master is intimidating and feels like overkill when they’re thinking, “Hey, I just want to get something DONE, not be a ninja."

Our users don’t have to go all the way to world-class to experience the intrinsic rewards of high resolution knowledge, skills, and results.
Simply getting past the Suck Zone can feel badass.
In many challenging domains, just going from zero to any capability is motivating.
% People first learning computer programming often start by writing code that prints “Hello World” to the screen.That’s it. Just “Hello World.” But if you’ve never written a line of code before, seeing “Hello World” on the display can seem like a superpower.

If you’re worried users will be intimidated by a path-to-badass when they “just want to be good enough to get something done,” the solution is simple: just tell them. Let them know that the early steps are the same no matter how far up the expertise curve they want to go.

Things change when you start to improve

Day 1: "I just want to have fun on the slopes and not die. It’s not like I’m ever gonna do the back-country..."
Day 20: "So... what does it take to be good enough to do the back- country? Asking for a friend."

And here we are STILL without a formula. We know what to do, but... how? How, exactly, do we help users become badass?

\subsubsection{What Experts Do: Science of Badass}
\textbf{Moving up and to the right}
How do we help our users keep moving up and to the right on the user journey? We must do at least two things: help them continue building skills/resolution/abilities, and help them continue wanting to.

Kathy Sierra builds her knowledge on top of "The Cambridge Handbook of Expertise and Expert Performance" and "The Oxford Handbook of Human Motivation".

We won’t find much academic research on becoming badass. But there is a robust, practical, science of expert performance and the development of expertise. In addition to the “formal” science of expertise, many other areas of scientific research give us tools and techniques for knowledge and skill acquisition.
From this point forward in the book we’ll use “badass,” “expertise,” “expert performance,”and “mastery” interchangeably. People often use “expert” to mean well-known in a domain, regardless of measurable performance. But in the science of expertise, the word “expert” has a more precise definition based on demonstrated ability and results.
Expertise is not about popularity, or years of experience, or even depth of knowledge.
\textbf{Experts are not what they know but what they do.}
\textbf{\textit{Repeatedly.}}

% The science of expertise applies to everything
Technical, creative, intuitive, logical, analytical, physical, mental, emotional—the science of expertise applies to any domain.

Technical definition of badass: "Given a \textbf{representative} task in the domain, a badass performs in a \textbf{superior} way, more \textbf{reliably}."

Experts are what they do.
It’s not just what they know.
It’s what they do with what they know.
And it’s their ability to do it again and again and again.
Expert/badass performance is both superior and more consistent than the performance of those who are knowledgeable and experienced but not producing expert results.
Experts make better choices than others.

If she’s badass, she picks better (chess) moves, more often. (s. 84)

Experts make superior choices
(And they do it more reliably than experienced non-experts.)

It’s up to us to find or create a useful definition of what “better” performance means for our context.

\textbf{Examples of expert “creativity”}
Creative work might seem too subjective for defining expertise, but remember, this is about defining results.

to meet the definition of expertise, an expert must also \textit{sustain} that popularity.They must be able to “be popular” more reliably. In other words, not simply a one-hit-wonder.

"Given a representative task in the domain, an expert performs in a superior way, more reliably." (Kathy Sierra) She then gives a template for creating a definition of expertise (or for badass) for the context and the tool. (s. 89 och 90)

\textbf{The myth of "natural talent"}
Evidence has been mounting for decades that for most non-sport domains and for most people,“natural talent” is not an absolute requirement for reaching high levels of expertise.

But even without a natural talent for focused practice, we can all learn the meta-skill of building skill. People often find that after they’ve developed a high level of skill in one domain, it becomes easier to develop high skill in an entirely different domain.They’ve become badass at becoming badass.And so will our users.

% THE best vs. ONE OF the best (s. 93)
The big difference (when there is one) between the “naturally talented” and those who aren’t typically shows up only at the edges of expertise.

But again, most of us will “settle” for being simply really really good
Even just a little better can be obvious.

\textbf{Support and encouragement}
"What about support and encouragement? World-class athletes and musicians often had parents that made sure their kid had the best coaches, training, etc."

Yes, support and encouragement can be crucial.
That’s what you’re going to do.

\textbf{\textit{Learning environments}}
Where you find high expertise, you often do find an environment centered around building that expertise.
That usually means access to a good training program (school, coach, etc.) and the time, space, and tools for practice.

\todo{Add into method}
\textbf{We can’t necessarily give our users extra time and space for practice, but as we’ll see in the rest of this book, we can give them the kind of support and tools that make every learning and practice moment as effective as possible.}

\subsubsection{What Experts Do: Science of Badass -  Building Skill}
% Här har jag mina tidigare anteckningar från YouTube-föreläsningen också

Skills left automatic/ unconscious for too long can slowly degrade no matter how often we use them.

Experts (i.e. those who perform in a superior way, more reliably, in a representative task in the domain) build skills from Can’t do to Mastered.

• Experts never stop adding new skills.
• Experts build skills both consciously and unconsciously.
• Experts refine existing skills.

STILL doesn’t answer the question... ANYONE can follow this framework, but not everyone becomes badass.

Now we have the right question:
Given this framework, what did experts do differently from experienced non-experts who followed the same framework? (And how can we help our users do this?)

Those who became experts did two key things differently than experienced non-experts.

The first one won’t be a big surprise...

\textbf{Experts \textit{practiced} better}
Those who became experts practiced more effectively than experienced non-experts with the same amount of practice hours.

It’s not harder practice It’s not longer practice It’s better practice

In the science of expertise, the form of explicit practice that’s
known to work effectively is referred to as \todo{Läs på mer om detta} \textbf{Deliberate Practice}.

“Deliberate Practice” does not mean “Practice, deliberately”

Deliberate Practice is usually the best (sometimes only) way to consciously* build skills from Can’t do to Mastered

Deliberate Practice Moves skills from A to B (from "can’t do" to "with effort") Moves skills from B to C (from "with effort" to "mastered")

Deliberate Practice fixes the single biggest problem most people have when trying to build expertise...

\textbf{The single biggest problem for most people on most expertise curves is having too many things on the B board}

We try to learn and practice too many things simultaneously instead of nailing one thing at a time.

Practice activities that are not Deliberate Practice can be riskier
than just not practicing at all.

Mastering one tiny useless-on-its-own sub-skill at a time
is nearly always a more effective, efficient way to move explicitly-practiced skills from A all the way to the C board.

Too big, too many simultaneous sub-skills
Divide it into fine- grained sub-skills
Work on THESE, one at a time

“Half beats Half-Assed” inspired by Jason Fried and David Heinemeier Hansson from their book, Rework.

Exception to Half-a-Skill beats Half-Assed Skills:

\textbf{For some domains, beginners need a starter set of a few half-assed skills}

Minimum viable skills threshold

\textbf{Simplified rules for Deliberate Practise}
Help them practice right. Goal: design practice exercises that will take a fine-grained task from unreliable to 95\% reliability, within one to three 45-90-minute sessions

Pick a small sub-skill/task that you can’t do reliably (or at all), and get it to 95\% reliability within three sessions. (Getting to 95\% in a single session is often better).

If you can’t get to 95\% reliability, stop trying! You need to redesign the sub-skill
The longer we practice an explicit skill that we can’t do reliably, the less likely we are to ever reach mastery. If you can’t get 95\% reliability in one to three practice sessions, change the exercise! Either split the task into a smaller sub-task, or reduce the performance criteria.

If the task/skill is too complex (too many unmastered skills), break it into finer-grained sub-tasks/sub-skills

If it’s not complex but it’s too difficult, make the performance criteria easier

\textbf{Not all practice is Deliberate Practice}
Any practice activity for which you do not become significantly more reliable in a task or skill within one to three sessions, is not Deliberate Practice. Common practice-related activities that don’t qualify as Deliberate Practice:

* Work on a project. For example, create a small game in a new programming language.
* Work through a step-by-step tutorial.
* Listen to a lecture or presentation.
* Play a complete piece of music, at a speed that you already perform reliably.
* A practice exercise for a skill that will take more than one session to become reliable AND there’s too long a gap between sessions. (Too much time between practice sessions means each subsequent session is virtually a repeat of the first session.)

You CAN’T be saying that tutorials and projects don’t work...

\textbf{Practice that is NOT Deliberate Practice can still be valuable and necessary}

Playing/Performing is not Deliberate Practice
Play enough games of chess against a stronger opponent and you will improve but it could take forever compared to an approach that involves both game play and Deliberate Practice. Playing is not practicing, or at least not the kind of practicing proven to effectively and efficiently build skills.

Step-by-step tutorials are not Deliberate Practice
Step-by-step procedural instructions can be the key to figuring out which Deliberate Practice exercises to do and how to do them.

Projects are an excellent learning tool, but they’re more about discovery and problem-solving than reliable skill-building.

\textbf{That 10,000 hour thing}
And there are other factors; Deliberate Practice is not the complete answer to expertise. But many many many hours of some form of Deliberate Practice are always required to build deep advanced expertise in a complex domain.

Why wouldn’t every teacher/trainer/coach/ mentor/employer—everyone with a stake in our improvement—use textbook Deliberate Practice?

It's complicated...

\textbf{The wrong ways to practice feel right}
Most of us were taught that to practice means doing more of it more often.We’re told it’s about putting in the time and working hard. Most practice does not focus on building individual skills and sub-skills to 95\% reliability within one to three practice sessions.

\textbf{The right ways to practice feel wrong}
Deliberate Practice is always just beyond our current ability/comfort zone.

\textbf{There's more to they story than Deliberate Practise}
Yes, some Deliberate Practice is necessary to build skills. And no, Deliberate Practice is generally not fun. (Later in the book we’ll look at ways to make it easier and more likely for our users to do the hard work of Deliberate Practice.)

1. Experts practiced better

\subsubsection{2. Experts were around better (perceptual exposure): Experts develop deep perceptual knowledge and skills through high-quantity, high-quality exposure with feedback}
The second attribute of those who became experts is this: \textbf{they were exposed to high quantity, high quality examples of expertise.}

(versus His parents are pro musicians, he obviously inherited his talent)

Where you find deep expertise, you find a person who was \textit{surrounded} by expertise.The more you watch (or listen) to expert examples, the better you can become.The less exposure you have to experts or results of expert work, the less likely you are to develop expert skills.

To understand perceptual exposure, we’ll begin with the most extreme example... (exemplet om att bestämma kön på kyckligarna)

the key is in how the existing experts “trained” new experts.

With a little ingenuity, the British finally figured out how to successfully train new spotters: by trial-and-error feedback. A novice would hazard a guess and an expert would say yes or no. Eventually the novices became, like their mentors, vessels of the mysterious, ineffable expertise.

Your brain learns things you don’t
It’s not magic.
It’s perceptual knowledge.

Perceptual knowledge includes what we think of as expert intuition.

\textbf{What is the criteria for perceptual exposure that leads to perceptual knowledge?}

Both of these exercises were carefully designed to help the brain “discover” the deeper underlying patterns and structure.

What “large” is depends on many factors, but always err on the side of more examples. Many, many, many more.Without a high quantity of diverse examples, the brain can’t separate signal from noise... it doesn’t have enough information to be certain that this and ONLY this is the invariant attribute/pattern.

To design a good perceptual exposure activity that can help the brain find a deep accurate pattern, use a high quantity of high quality examples that seem different on the surface, but actually aren’t.

\textit{Good Perceptual Exposure exercises don’t explain. They create a context that lets the learner’s brain “discover” the pattern}

(s. 142, bild)

\subsubsection{Review Summary: What Experts Do}
s. 150, jättebra

\subsection{Help them keep wanting to (be badass): Help them move forward - Remove Blocks}
The key question is not,“What pulls them forward?” (they already want to get forward) It’s “What makes them stop?”

\textbf{Working on what stops people matters more than working on what entices them}


What pulled them off the forward path?

What’s more compelling than the forward-pulling magnet?

What’s more powerful than whatever motivated them to start on this path?

\textbf{Once they’ve started on the path, the “secret” to helping them move forward is to focus on reducing what slows or stops them.}

\textbf{The Gap of Disconnect (e.g. The Gap of Suck)}
Before they buy/join we’re all about the Context. This is where motivation lives.
After they buy/join we’re all about the Tool. This is where motivation dies.

Solution: To think of it another way, what can’t they tell us because we’re not right there with them?

The secret for keeping them going when things get tough is this: acknowledge it.

Some things are just hard.

Why he \textbf{did} snowboard a second day: "Dude. Everyone sucks the first day. Everyone. But the second or third day, everyone suddenly gets so much better. Google it. It’s totally a thing."

The main reason people stop when they’re struggling is not because they’re struggling.
It’s because they don’t know that struggling is appropriate.
It’s because they don’t know that they’re exactly where they should be.
It’s because they don’t know that everybody struggles at this point.

\textbf{Just tell them}
They stop not because of the struggle.They stop because they don’t realize the struggle is \textit{typical} and \textit{temporary}.

"If you tell them “here’s the problem you’re going to struggle with and seriously, it’s
not you, it’s us, but here’s what’s gonna happen...” their confidence and trust in both you and themselves goes up.Way up."

They don’t need you to be perfect. They need you to be honest.

\textbf{Derailer solution:
Anticipate and Compensate}

\textit{Anticipate} the most likely faces they might make and questions they might ask if you were next to them when they use your product and/or work at the bigger context.

\textit{Compensate} for their inability to show and tell you what they’re experiencing, and more importantly for your inability to notice and respond.

Just tell them.
“What our marketing promised you’d be able to do? We’ll help you do that. It’s going to be harder than our website makes it sound.”
Be brave.

\subsubsection{Help them move forward: Remove Blocks - Progress + Payoffs}
To help users stay motiveted, give them:
•  A description of the path with guidelines to help them know where they are at each step.
•  Ideas and tools to help them use their current skills early and often.

\textbf{Performance Path Map: a key to motivation and progress}
A Performance Path Map is about what you do, not what you learn

A learning/mastery path that tells you what course to take next, book to read, or topic to study is still better than nothing, but it’s far more effective if we map topics-learned directly to mastered-skills.

\textbf{To create a path:}
Make a list of key skills ordered from beginner to expert, then slice them into groups to make ranks/levels. For motivation, the earlier, lower levels should be achievable in far less time and effort than the later, advanced levels. One possibility is to have each new level take roughly double the time and effort of the previous level.

%Each rank above black belt in the ancient game of Go is thought to take twice the knowledge/skill/ effort of the previous level

That experts cannot agree on The One True Path proves there’s more than just one “right” way to get there. That’s encouraging.

most experts teach (and argue over) that which is easiest to \textit{represent} rather than that which is most \textit{valuable} for improving performance.

\textbf{The secret to a motivating Path Map}
It’s not about the belt.
It’s about the progress.

It is never about the belt.
It is about what the belt reflects and enables. It is about meaningful progress.
(source: The Progress Principles, Teresea M. Amabile, Steven J. Kramer)

% “Of all the things that can boost emotions, motivation, and perceptions during a workday, the single most important is making progress in meaningful work.And the more frequently people experience that sense of progress, the more likely they are to be creatively productive in the long run.Whether they are trying to solve a major scientific mystery or simply produce a high-quality product or service, everyday progress—even a small win—can make all the difference in how they feel and perform.”
%– Teresa M. Amabile, Steven J. Kramer

\textbf{Making progress is necessary, but it’s not enough: What payoffs are they getting along the way?}

Helping them believe they’ll get better matters. Helping them actually get better matters.
Helping them realize they’re getting better matters.
But none of that matters if they don’t benefit from getting better.
Which brings us to...

\textbf{What can they do within the first 30 minutes?}
Do: User surprised/delighted/ impressed by what HE does

Not: User surprised/delighted/impressed by what the PRODUCT does

\textbf{Remove fear from the start}
If we want them to feel powerful early, we must anticipate and compensate for anything that keeps them from experimenting.

"DON’T PANIC
Recovery is easy.
Be brave. (see inside for super simple reset steps"

\textbf{It doesn't need to be practical to be meaningful}

Remember,“meaningful-in-30-minutes” is about exceeding their expectation of what they’d be able to do at first.

What’s the smallest step they can take that leaves them feeling more creative, smart, powerful, capable?
What’s the smallest step that gives a hint of future power?

\textbf{After the 30 minutes: Design with a motivating payoff loop}
Payoff (Ideas/activities to benefit from current “superpowers”) ->
Motivating "next superpower" (New goal for compelling capabilities/results)
-> Practice + Exposure (Activities to build skills and knowledge)

\textbf{The best payoff of all: intrinsically rewarding experiences}

EXtrinsically rewarding: Rewarding because of an external motivator: reward, status, peer pressure, etcetera. These are NOT the payoffs that lead to robust, long-term success.

INtrinsically rewarding: Rewarding for its own sake.

\textbf{Powerful Intrinsic Motivation:
High Resolution and Flow}
High Resolution: Deeper, richer experience
Appreciation for increasinely more subtle details mere mortals can’t perceive.

Flow: "“In the zone”
“In your element” (se mer s. 208)

High-resolution: badass users "talk different"
Learning to understand and especially converse in the high-res technical jargon of a domain is both intrinsically rewarding and extremely useful.

Beginners don’t necessarily hate jargon in a domain they want to join.They’d love to understand and use it ASAP. So that’s our job: help beginners come up to speed on jargon, and the more they’re around it, the faster that will happen.

\textbf{Flow: The psychology of optimal experience}
"FLOW - The psychology of Optimal Experience" is one of the most influential books for user-experience design.

For Flow to be possible, there must be a balance between perceived challenge and current ability to meet that challenge.

\textbf{Give your users high-payoff tips}
What are the simplest high-payoff tips and tricks in your domain? The earlier you help your users learn those, the quicker they’ll have high-res, high-payoff, high-WOFO opportunities.

"Giving “tips and tricks” is not Deliberate Practice and it’s not Perceptual Exposure, so why are we doing it? Doesn’t this just reinforce “mechanics without feel”?"

Remember when we said half-a-skill beats half-assed skills?
We also said that when people are first starting, they need a Minimum Viable Half-Assed Skillset to be able to do something rewarding.Tips and tricks can make a motivating half-assed skill at the beginning of a curve, but they can also work at advanced levels.

This is not about giving people shortcuts; it’s about helping them bypass the unnecessarily long way.We don’t want our users to spend much time reinforcing (locking-in) beginner or mediocre skills.Tips and tricks are one way to help let users practice at being better even if they don’t yet understand how and why the shortcut works.

\subsection{Support Cognitive Resources}
\subsubsection{Design}
7-digit memorizers were nearly 50\% more likely to choose cake than the 2-digit memorizers. (s. 217)

Insight: \textbf{Willpower and cognitive processing draw from the same pool of resources.}

The seven-digit memorizers weren’t choosing cake simply because their tired brain needed more calories; they chose cake because the memorization task \textit{depleted their willpower to resist the cake}

If a product you use every day is poorly- designed and hard to use, it drains your cognitive resources.That means it also drains your willpower.

NOTE:There are things you can do to replenish cognitive resources, but the big two are a good night’s sleep and good nutrition.

\textbf{There’s only one resource pool for both willpower and cognitive tasks}

\textit{Make sure your users spend their scarce, easily drained cognitive resources on the right things: e.g. with the compelling context, instead of on the tool}

Think of cognitive resources as a single bank account, and every cognitive task or use of willpower as a withdrawal from that one account.

Always be asking,“where do my users want to spend their precious cognitive resources? What can we do to help? What are we doing that hurts?”

Every moment spent struggling with a confusing UI, frustrating customer service, poor documentation, or anything requiring patience, self-control, or intense concentration on the tool could be stealing resources from learning, practice, and becoming badass at the thing they really care about: the context.

\textbf{Cognitive resource management for your users is mainly about reducing leaks}

Becoming badass is hard.
There will be cognitive resource drain.
You do want your users to use cognitive resources. You don’t want your users to waste them.
Don’t make them think about the wrong things.

Your brain won’t need to spend resources “worrying” about an unfinished cognitive task if it “believes” something or someone has a trustworthy plan for handling it.

\textbf{Death by a thousand cognitive microleaks}
A common cognitive microleak comes from that subtle feeling of uncertainty about whether some small action you took did exactly what you intended. (by Designer Dan Saffer, author of Microinteractions)

\textbf{When you’re considering adding a new feature...}
“is this a cake feature or a fruit feature?”

Is this feature worth the drain?
Does this feature directly contribute to user expertise and
results for the meaningful bigger context?
Is there a way to hide or minimize the feature until the user is ready for it?

\subsubsection{Support Cognitive Resources: Reduce Cognitive Leaks}
You want the Knowledge for using it is in the world (the device itself) versus in the user's head.

\textbf{Don’t make them memorize}
\textbf{Knowledge in the head vs. world is a trade-off}
Learning and memorizing drains cognitive resources, but after you’ve learned/memorized it, using the now-automatic/ memorized knowledge and skill is fast and effortless. Knowledge in the head trades slower learning/using time now for faster using time later.
For both your bigger context and your tool, tell your users which facts and procedures are worth spending the effort to memorize, and when they should do it.

“It’s not necessary or worth it to memorize those right now, so here’s where to look them up when you need them. But when you start doing [more advanced thing], it’ll make life easier if you spend a few days memorizing these....”

The concept of “knowledge in the head vs. knowledge in the world” was popularized by Donald Norman, in The Design of Everyday Things. If you can choose just one book about design, UI, usability, saving cognitive resources, and caring about users— this is that book.

\textbf{The power of \textit{affordances}}
to reduce their cognitive leaks, make the right thing to do the most likely thing to do (this, "percieved affordances", also comes from The Design of Everyday Things)

\textbf{To reduce their cognitive leaks, don’t make them choose}
Choices are cognitively expensive

There’s a world of difference between having choices and having to make choices.

In the perfect scenario, we give our users as many options as they could want or need, but we also give them trusted defaults, pre-sets, and recommendations. Especially in the beginning, we make decisions so our users don’t have to.
Be the expert, the mentor, the guide.

\textbf{Again: To reduce their cognitive leaks, help them automate skills}
Skills in “B” drain cognitive resources. “C” skills are cognitively cheap.
Remember, the single biggest problem for most people learning a complex skill is working on too many sub-skills simultaneously instead of nailing one sub-skill at a time.

\textbf{To reduce their cognitive leaks, give them practice hacks}
Deliberate Practice drains cognitive resources. Help your users make everything else around practice easier to do.

\textbf{To reduce their cognitive leaks, help with the top-of-mind problem}
what if the problem
is not lack of skill but lack of constant reminders? Fortunately there’s a simple, powerful tool for solving the massively-draining top-of-mind problem: give reminders (e.g. a wearable is called MotivAider, that doens't drain cognitive resources).

Trying to keep thing A at top-of-mind while simultaneously working on thing B is a background drain.Worse, it doesn’t work.You can’t keep reminding yourself over and over while working on tasks that demand cognitive attention.
Tell your users about the MotivAider! Give them tips for ways to use it to get better at your context. Encourage users to share their top-of-mind problems. It’s like a superpower.

\textbf{To reduce their cognitive leaks, reduce the need for willpower}
Self-control / willpower is cognitively expensive

\textbf{The secret to willpower is... assume it doesn’t exist}
Take a moment to imagine what changes you would make to your life if you could no longer depend on willpower.
Those things you would do if your willpower was surgically removed?
Do them for your users.

To reduce the need for willpower, help them build automatic habits

(You can learn to help users build habits by reading Charles Duhigg’s book,The Power of Habit.
(And get your users to read it, too))

\textbf{To reduce the need for willpower help them have intrinsically rewarding experiences}
Big problem: Deliberate Practice is NOT intrinsically rewarding. They’ll still need willpower for doing the hard work.

Actually, there is a way that some of the hard, non-intrinsically-rewarding activities can be nearly as motivating as the ones we do enjoy.

"I run because I am a rock climber.” (the tool (running) is part of a compelling context (I am a rock climber)

\textbf{When the bigger context is not just something you do but something you are, motivation for the non-enjoyable parts takes less willpower}

When that happens, your relationship to the required
but non-intrinsically rewarding work changes. In terms of willpower demand, some of the non-pleasurable work can be indistinguishable from the deeply rewarding experiences that take little or no willpower.The work is still just as hard, painful, frustrating, and non-rewarding—the activities don’t magically become pleasurable—but they do become less dependent on willpower.
\todo{Detta ska jag lätt använda till min app!}
When the bigger context is part of your identity, the hard- but-necessary work becomes nearly just as motivating as the intrinsically rewarding experiences.

"Aren’t we missing the TOTALLY OBVIOUS? Why not just give them EXtrinsic rewards for the parts that are NOT INtrinsically rewarding?"

Obvious? Yes.
Intuitively good idea? Yes.
Actually good idea? No.

Giving extrinsic rewards for anything we
hope to sustain long-term can do more harm than good. Extrinsic rewards are especially dangerous because they often are motivating in the beginning. But it’s motivation around the reward, not motivation for the activity we hope will be rewarding for its own sake.

(book recommendation: Drive)

If we want to reward users without the long-term negative side effects, we can make the rewards completely unexpected, not directly tied to the users’ behavior. For example, a special thank-you gift is a powerful “reward” without the harmful side effects.

\textbf{To reduce the need for willpower, help their brain pay attention}
Staying focused takes self-control - so Make paying attention easier.

\textbf{Get past the brain’s spam filter}
.To help our users pay attention and stay focused, help their brain realize, “This matters! Not spam! Let it through!”

How do you get past the brain’s spam filter?
We must give the brain a reason to “care”

\subsection{My points}
Kathy Sierra makes the point for that gamifaction might make a user look good on a spreadsheet for user retention, but that creating an expert user is much more rewarding for getting people to talk about your product.


\subsubsection{Support Cognitive Resources: Espace the Brain's Spam Filter}
Wherever staying focused is something our user wants and needs but struggles to do, we must help convince their brain to agree.
We must inject the thing our user wants to do with something our user’s brain cares about.
Then what do brains care about?

(s. 252)

Brains pay attention to things that are odd, surprising, unexpected.

The Zeigarnik Effect drains cognitive resources, but we can also use the Zeigarnik Effect strategically, like a filmmaker, to keep the brain interested.

\textbf{Make it visceral (s. 261}
Brains don’t want to waste scarce resources making those “little” leaps
Remember, this isn’t about your users’ intelligence, it’s about their brain’s spam filter. Yes, they could make the leap... if their brain were already riveted.
You’ve experienced this while reading a textbook or user manual: you read clear, straightforward paragraph after paragraph, page after page, thinking OK, makes sense, yes, whatever, and then at the end BAM! Suddenly there’s this sentence,“If you forget to do this, here’s the [catastrophic thing] that will happen...”
That warning is way too late.

The best way to deal with the brain’s spam filter is to reduce the amount of things that need to get past it
Making content easier for their brain to pay attention to is good. Making less content for their brain to pay attention to is much better.

\textbf{What about the facts they absolutely positively must know?}

They can’t just practice and build skills. They do need to learn, you know, ACTUAL FACTS.

But the best time for explicit knowledge is only when \textit{absoutely neeed}.

\textbf{Brains prefer Just-in-Time over Just-in-Case}

Tring to earn nowege before ou nee to use it
ustinase means fighting the spam fiter
ust-In-Time means learning something only when/because you actually need to use it. But ust-In-Case is the predominant model for most forms of education (and most user manuals). ust-in-Case knowledge is easier to present, but harder to learn, understand, and remember. Much harder.

To the brain, ust-in-Case can seem useless.

But if there is ust-in-Case knowledge your users absolutely must learn before they need to use it, minimize the damage:

\textbf{1. Validate the need for this knowledge}
Are you really really really certain they must know this right now?

The first step is to narrow down the topics.We’ll use two approaches to validate the knowledge our users are expected to learn: topics on trial, and mapping to skills.

\textbf{2. Convince their brain}
Now that you’re sure users must learn this ust-in-Case knowledge, you have to “sell” it to their brain.

\textbf{The problem with “intellectual curiosity”}

Cognitive resources. Scarce. Limited. Zero sum. The problem is not if they’re intellectually curious, it’s when.

(In books or user guides, an appendix is a useful compromise for including non-necessary-but-interesting knowledge. sers who want it will find it, but overwhelmed users won’t feel pressured to learn it.)

\textbf{Summary}
Speaking of limited cognitive resources (mine), we’ve come a looooon way in main “the formula help users uild epertise help them stay motivated to mae proress and especially help conserve their conitive resources y reducin conitive leas

Lägg in total-bild här

\textbf{Badass at life}

You have the chance to help people become more skillful, more knowledgeable, more capable.
You have the chance to help make the world a little more high-res.
You have the chance to help people become better learners and better managers of their own limited cognitive resources.
You have the chance to help people spend more of their scarce, precious cognitive resources on the people they care about. (And don’t forget the dog.)
You have the chance to raise the bar on what it means to care about users as people with lives. Complicated, resource-draining lives.
You have the chance to help people become more badass not only at using your tool within a meaningful context, but badass at life.