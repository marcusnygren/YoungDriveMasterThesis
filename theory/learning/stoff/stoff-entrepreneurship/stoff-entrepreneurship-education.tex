"Entrepreneurship Education in Schools: Empirical Evidence on the Teacher’s Role" says that

"The findings indicate that the training teachers have received in entrepreneurship seems to be the main factor determining the observable entrepreneurship education provided by the teachers."

%\citep{entrepreneurship-pihkala}

% "The teacher is the central actor in entrepreneurship education and the teachers’ role in defining the time, frequency, contents and methods of entrepreneurship education is decisive. (Fiet, 2001a; Jones, 2010; Lobler, 2006; Seikkula-Leino, Rusko- vaara, Ikavalko, Mattila, \& Rytkola, 2010; Ruskovaara \& Pihkala, 2013)."

\textbf{Considerations for digital learning}
The playful side of teaching and learning (Solomon, 2007) as well as teacher training that develops the competences of a mentor, enabler, or coach should enhance entrepreneurship education practices.
\todo{Svara på i Genomförande. Relevancy of "fun"}

When shifting the focus from Gibb’s (2002) idea of developing students’ understanding of entrepreneurship to the teacher, what are the ways for a teacher to see, feel, do, think, and learn entrepreneurship? How to provide teachers with the skills to cope with, create, and perhaps enjoy uncertainty and complexity?
\todo{Svara på i Analys}
\todo{Refer to entrepreneurship definition}

%\subsection{Proposition}
%On this basis, we formulate the following proposition:

%Proposition 1: Entrepreneurship education practices do not differ between male and female teachers.

% Teachers’ business enterprise background’s positive effect on entrepreneurship education.

%Proposition 2: The stronger the teacher’s business background is, the more he/she is bound to execute entrepreneurship education.

%Proposition 3: The more the teacher has work experience, the more he/she is inclined to conduct entrepreneurship education.

%Proposition 4: Entrepreneurship education differs between education levels.

%Proposition 5: Enterprise-related teacher training positively affects teachers’ entrepreneurship education practices.

\textbf{Result}

\textbf{Teachers’ gender and entrepreneurship education practices}
Even though we have found studies with a feminist approach to entrepreneurship education and studies concerning women entrepreneurs (Komulainen, Keskitalo-Foley, Korhonen, \& Lappalainen, 2010; Korhonen, 2012), their findings do not show any indications of differences or similarities between women and men. According to Bennett (2006), a lecturer’s gender does not play a significant role in inclining entrepreneurship education.

% Viktigt för YoungDrive, såklart!
\textbf{The teacher’s professional teaching experience has no significance in terms of entrepreneurship education.} These findings suggest that as a competence area, entrepreneurship education is not dependent on the teacher’s experience as a teacher.

\textbf{Entrepreneurship education correlates positively with entrepreneurial activity} according to Dickson et al. (2008).
\todo{Kommentera i Analys}

% Elena Ruskovaara is working as a researcher, lecturer, and project manager at Lappeenranta University of Tech- nology. For the past 15 years, she has worked in the field of further education for teachers and has been involved in many national and international entrepreneurship proj- ects. Her main interests are entrepreneurship education and especially the challenges of measuring and evaluating entrepreneurship education.

% A large number of useful methods and practices have been discovered (Seikkula-Leino, 2007), and training concerning different pedagogical solutions could be of great value.
