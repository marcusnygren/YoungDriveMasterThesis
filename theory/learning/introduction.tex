\subsection{Pedagogical Development}

On January 28th, 2016, Henrik Marklund\ref{effectivelearning-expert} at the educautional technology startup Knowly was interviewed about Pedagogic Development. He means there are two main areas of research, and an additional one. 

The first one is connected to memory, cognitive psychology, where the second one is about motivation theory, social psychology. Finally, he mentions a third area, training transfer, which is the research on how to make sure a course gives effect in everyday life.

The first area of research asks "How do you get people to learn things?". Often school is studied, where learning is about being taught a subject, and then to pass a test. E-learning tools are often designed to do similar things to what schools does.

The second area of research, asks "How do you get people to behave differently?". One area of research is about building habits. This is highly relevant in e-learning, where behavior change may be necessary to build the habit of using an app or a digital tool repeatedly.

The third area is training transfer, and asks "How do you make sure a course gives effect in your everyday life?". For YoungDrive, the wish is that the coach training gives effect in the coaches' everyday life. The master thesis aims to be able to assess and encourage this.