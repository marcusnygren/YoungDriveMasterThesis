%\subsection{Pedagogical development}

%The first section describes "How do you get people to learn things?", cognitive psychology. Often school is studied, where learning is about being taught a subject, and then to pass a test. E-learning tools are often designed to do similar things to what schools does.

%The second section describes "How do you get people to behave differently?", social psychology. One area of research is about building habits. This is highly relevant in e-learning, where behavior change may be necessary to build the habit of using an app or a digital tool repeatedly.

%\include{theory/learning/pedagogical-development/cognitive_psychology}

\subsection{Designing learning using Cognitive psychology}

Cognitive psychology deals with how our brain works in regards to our memory.

This section presents strategies and techniques to design learning for the mind, and what needs to be considered.

Two aspects are especially relevant when it comes to education: how humans learn (the first four sections), and how humans forget (the two last sections).

In how humans learn, the purpose is to find the most powerful strategies and techniques to design effective learning (mapping educational objectives, how to build skills, pattern-matching techniques, and the power of reflection and assessing).

In how people forget, UCLA Bjork's Learning and Forgetting Lab \cite{ucla} researches how people forget, and how to design so that people do not forget (see Retrieval practice and Spaced practice).

%\subsubsubsection{Learning}

  \subsubsection{Learning the Right Things: Mapping educational objectives with Bloom's Revised Taxonomy}

  What to learn should be determined by learning objectives. Depending on the learning objective, and where it fits in the Knowledge dimension and Cognitive Process dimension of Bloom's Revised Taxonomy, the design of the learning needs to be different. \cite{bloom}

  \subsubsection{Building skills: by Spaced practice, Deliberate practice and Perceptual exposure}

  Spaced practice deals with spreading out learning, with the purpose of not forgetting.

  Designing for this, could mean making the user apparent on the person's meta-cognitive ability (personal insight into what you'll remember), and meta-memory (when you need to repeat information in order not to forget).

  Moreover, dividing learning into 45-90-minute chunks, getting to 95\% reliable within three sessions, has been proven highly effective. This is called deliberate practice.

  Sierra presents a number of strategies, most notably research within deliberate practice \cite{yengin} \cite{sierra}. Deliberate practice has been proven to be an effective way to build skills. It has also been tested before for mobile learning environments. \cite{yengin}

  Sierra \cite{sierra} suggests skills to be divided into three buckets: can't do (but need to do), can do with effort, and mastered (reliable/automatic). The goal then is to move skills from can't do into mastered, in the best way possible.

  Desirable difficulties applies: with deliberate practice, it may feel as if learning gets harder and harder, but in the long term the user is actually learning more. As a result, less people does true deliberate practice, but they do not get the same reward in return. This needs to be designed for, e.g. using social psychology.

  A way to build skills quickly, is to utilize that the brain is brilliant at pattern-matching, by the method "perceptual exposure". \cite{sierra}

  By exposing users to very high-quality samples during a very limited time, experts can learn intuitively.

  \subsubsection{Learning from Assessment}

  Knowing what learners know, and don't know, is crucial to effective learning, Luckin \cite{luckin} says.

  Assessment can partly help to design for flow, matching challenge and ability \cite{bruhlmann}, which is effective for intrinsic motivation (see next chapter).

  Moreover, it also has cognitive benefits. It can help to offer appropiate feedback, increase learners' awareness of their learning needs, and give accurate assessment and analysis, and allows learning to be tailored.

  By recognizing differences of students, in their ability to understand what they know and how they can progress, it is possible to ensure that everyone achieves their full potential.

  Effective assessment by a teacher or agent includes individual feedback (task-oriented and informal) and appropiate feed-forward advice.

  \subsubsection{Learning by Thinking: Reflection \& Retrieval Practice}

  When reflecting, the student develops neccessary skills and self-awareness to refine their own learning activities. This surely applies to the teacher as well, Luckin says. \cite{luckin}

  Stefano \cite{stefano} suggests that that reflection has been an overlooked area of research for a long time.

  They found that individuals who are given time to reflect on a task, outperforms students who are given the same amount of time to practice with the same task.

  His results suggests that reflection as an activity that can be more effective than additional learning.

  Similar to deliberate practice, it is a desirable difficulty. Individuals in the test themselves, had a tendancy to allocate time to practice on the task rather than reflecting on it.

  %\subsubsection{Retrieval practice}

  Bjork \cite{bjork} shows that retrieval from memory is more effective than people who repeat reading the same thing to remember.

  They also showed, that the more effective students, retrieves from memory.

  E.g. "What was in that article?", instead of immediately reading the article, is an example of memory retrieval that is extremely effective for learning, their research shows.

  One design method to encourage this, would be flip cards, where the question is on one side, the answer is on the other, versus giving the person a multiple-choice question.

%\subsubsection{Not forgetting}

%UCLA Bjork's Learning and Forgetting Lab researches how people forget, and how to design so that people do not forget.

%\include{theory/learning/pedagogical-development/social_psychology}

\subsection{Designing learning using Social psychology}

Social psychology can guide the design, when there is a wish to make people behave differently. A big research area is motivation.

With a compelling context, the users are already motivated. Their motivation, is to become better. Kathy Sierra \cite{kathy}, instead suggests the focus to be how to help users progress.

To do so, she suggests two factors: what pulls users off (derailers), and what pulls users forward.

First, Sierra argues working on what stops people matters more than working on what entices them. Here, cognitive load theory is highly relevant. Even though the area is part of cognitive psychology (see previous section), in this section cognitive load theory is described as a way to remove blocks.

Second, Sierra aruges that to pull users forward, to stay motivated, progress and payoffs are essential. Both of these, are investigated in terms of motivational psychology.

\subsubsection{What pulls them off: Cognitive load theory}

Sierra \cite{sierra} describes how humans have scarce cognitive resources.

CBT is divided into three areas: intrinsic CBT, extrinsic CBT, and germane CBT.

Intrinsic CBT, needs to be dealt with if the effort is too high. Sierra \cite{sierra}describes two strategies. She first says that according to deliberate practice, if you can not get to 95\% reliability within three 45-90 minute sessions, split skills that can be done with effort into sub-skills. The purpose is to reduce time spent practising being mediocre.

Extrinsic CBT, the way presented to a learner, should be handled via designing to support cognitive resources, Sierra says \cite{sierra}.

Scaffolding is a technique to step by step remove the support wheels for the user, e.g. present information in different ways.

Also, reduce cognitive leaks by e.g. don't make them memorise, and make the thing you want the user to do, the most likely thing to do (affordances). Everything that takes willpower, reduces cognitive leaks.

Germane CBT, is the work put into creating a permanent store of knowledge. To support cognitive resources, escape the brain's spam filter by making the information essential. Either by designing for the compelling context, or desining for just-in-time learning versus just-in-case, Sierra says. \cite{sierra}

\subsubsection{What pulls them forward: Motivational psychology}

Progress and payoffs, are methods to stay motivated. The feeling of progress can be emphasised by a path with guidelines to help the user know where they are at each step, e.g. for a training.

The best payoff, is a intrinsically rewarding experiences. It is superb to gamification, says Sierra \cite{sierra}. This goes in line with self-determination theory, where e.g. Pink \cite{pink} says that the surprising truth about what motivates us is that drive is fostered by autonomy, mastery and purpose.

The most efficient way is therefore to design for having intrinsically rewarding experiences.

Caring for the compelling context, why the user wants to learn the skill, are helpful strategies. Other strategies are flow, mentioned before, or to give high pay-off tips, helping the user progress in a fair way.

