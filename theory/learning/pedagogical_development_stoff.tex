\subsection{Cognitive Psychology}
"How do you get people to learn things?" leads to asking "How do people forget? How do you make sure that people do not forget?". There are two principles: Bjork's Learning and Forgetting Lab at UCLA\citep{learning-ucla} suggests that there are Retrieval practice vs. Spaces practice. Whereas Retrieval practice is about how retrieval of information from memory is more effective than those who read the same thing again and again to remember. The more effective students, retrieve from memory, e.g. they ask themselves "What was in that article?" instead of immediately reading the article. This kind of memory retrieval, is extremely effective for learning, research shows.

Spaces practice is about spreading out the learning with the purpose of not forgetting. Connected to this is e.g. the notion of meta-cognitive ability, and meta-memory, meaning how much personal insight you have into what you'll remember, and when you need to repeat information in order not to forget.

These describe research advances within cognitive psychology, and are connected to how our brain works in regards to our memory. Connected to the area of learning, is also "Desirable difficulties", "Scaffolding" and Cognitive load theory \citep{effectivelearning-lab}.

A desirable difficulty, is when it feels like learning gets harder and harder, but in the long term you are actually learning more. Less people do it, as you do not get the same reward in return.

Scaffolding, is when step by step, you remove the support wheels for the user. E.g. you may start with easy questions using multiple choice, but eventually graduates into using flashcards as principle \#1, if the user can master it.

Cognitive load theory asks "How do you design for the mind?", and "How much information can we store in our work memory?". For example, research shows that visual input and audio input (voice and audio) are stored in separate memory channels, meaning that use both mediums are often a good choice. There are three types of cognitive load, "intrinsic", "extrinsic" and "germane".

\subsection{Taxonomy}

Table 2 i Krathwohl (2002) är otroligt bra, detta får frågor utformas ifrån. I sin tur, går denna kanske att koppla till tabellen i What makes good educational software?.

(lägg in Table 2 här, Krathwohl 2002)

Table 2: Structure of the Knowledge Dimension of the Revised Taxonomy
Factual Knowledge - The basic elements that stu- dents must know to be acquainted with a discipline or solve problems in it.
  Aa. Knowledge of terminology
  Ab. Knowledge of specific details and elements
Conceptual Knowledge - The interrelationships among the basic elements within a larger structure that enable them to function together.
Procedural Knowledge
Metacognitive Knowledge

Table 3: Structure of the Cognitive Process Dimension of the Revised Taxonomy
1. Remember
  Recognizing
  Recalling
2. Understand
  2.1 Interpreting
  2.2 Exemplifying
  2.3 Classifying
  2.4 Summarizing
  2.5 Inferring
  2.6 Comparing
  2.7 Explaining
3. Apply
4. Analyze
5. Evaluate
6. Create

Sedan kan Figure 1 användas för att placera in vart lärandet i appen ska ske i Figure 1: The Cognitive Process Dimension.


\subsection{Social Psychology}

"How do you get people to behave differently?" looks at motivation theory. Kathy Sierra, author of Badass: Making Users Awesome\citep{interactiondesign-badass}, researches how \textit{learning is tied to product development}. She says that our product development should be led by a focus on building great users, instead of great products. She takes a stance in "How do you build expertise?".

While emphasizing e.g. cognitive load theory, Sierra looks a lot at motivation. How to get people to continue to build expertise or using an app, self-determination theory is a good area of research. The book Drive: The Surprising Truth About What Motivates Us\ref{motivation-drive} is a good basis, which pushes that if Autonomy, Mastery and Purpose is present, this fosters drive.

Another insight Sierra shares is that we should start taking about Post-UX-UX: what happens after your product has been used? How do you design \textit{that} experience? Here, service design is a step towards actively working with how your app fits in a larger context \citep{servicedesign-ideo}.

In physical as well as digital training, smart individual feedback is of high importance. Feedback should be task-oriented, always \textit{informational}. An example might be: "I see that you're ranking number \#1 in the subject of X, may I suggest Y?". The carrot-stick method (e.g. giving achievement points or rewards) is sub-par with motivating learning in itself for the user. Therefore, feedback should not be in a motivational form, e.g. saying "You're good!". Triggering the enthusiasm for learning, is a more effective strategy.

\subsubsection{Building Expertise}
Sierra % https://www.youtube.com/watch?v=FKTxC9pl-WM
divides expertise into three areas:
(A) "Can't do (but need to do)"
(B) "Can do with efford"
(C) "Mastered (reliable/automatic).
The goal is to always moving this across this board.

But there are three main problems:
1 Pile-up on B (there are too many cognitive resources)
2 The Intermediate Blue (something has made it see it's not high quality)
3 Too slow

Fixing the problems:
1. Fixing the pile-up on B - Keep more on A (this is common, but not the best solution).

2. Split B into subskills to move to C (How progress happens, getting it right 95 out of 100): Maximum 3 sessions, maximum 45-90 minutes/session - according to 50 years of cognitive research) - half-a-skill beats a half-assed skill.
Practice makes perfect < Practice makes permanent
- Crucial: reduce time spent practicing being mediocre

3. Since people revisit old knowledge, that's unneccesray check. We want to Bypass B. Perceptual learning is utilizing how the brain is brilliant at pattern-matching. This is why there are experts who can't teach you how to do it, but they don't know and thus can't teach the pattern-matching.

This is what we need:
- High quality (during a very limited time)
- Very high-quantity examples (200-300 examples)

Care about one another's scarce cognitive resources. We're not humanoids. We're humans.

\subsection{Training Transfer}

Finally, \textbf{"How do you make sure a course gives effect in your everyday life?"} looks at Training transfer. For a long time, the focus on learning effectiveness in leadership trainings has been to maximize the quality of the lectures themselves. Recently, the focus has shifted into researching how the before and after of the training affects the learning. If the leader is involved with the participants before the training, and communicates expectations, and gives the participants support afterwards, it leads to increased success. On the other hand, the leader can kill the learning of a whole education program, by simply saying the wrong words before an education is held. As a lot of trainings are now based on research, and quality of trainings thus has increased dramatically, this is where a lot of learning potential is. This is connected to both concepts included in "How do you get people to learn things?" and "How do you get people to behave differently?". \\

\subsubsection{The Importance of Before and After}

\textbf{Robert O. Brinkerhoff - Making L\&D Matter : Learning Technologies 2013}
% https://www.youtube.com/watch?v=__wUZzJV8lM

De visar att före/efter är lika viktigt som själva träningen

%http://blog.dronahq.com/dr-robert-o-brinkerhoff-making-learning-development-ld-matter/

\subsubsection{Reflection \& Assessing}

Luckin: "Knowing what learners know, and don’t know, is crucial to effective learning. if learners attempt tasks that are too complex, they are likely to fail; if they attempt tasks that are too easy they may not progress as they should. accurate information about learners’ current understanding can help us to offer appropriate feedback and increase learners’ own awareness of their learning needs. accurate assessment and analysis also allows learning to be tailored. learners differ physically, emotionally and cognitively, and in their ability to understand what they know and how they can progress. recognising these differences can help to ensure that everyone achieves their full potential.

Two important processes underpin how we identify what learners know and understand. \textit{Reflection} involves learners considering their own learning activity. by reflecting learners develop the skills and self-awareness they need to refine their own learning activities. \textit{Assessing} involves teachers considering the learners’ learning activity. effective assessing provides feedback and feed–forward advice to a learner about their learning activity: learners must be able to respond to a critical voice. self–assessment requires the learner to provide that critical voice, which links back to the process of reflection. We must
also recognise that at times teachers are also learners, for example, when taking part in professional development activities. The processes of reflection and assessment are no less important for them as they are for any other learner."

It's not only for the teacher though, that reflection is crucial. Learning by Thinking, % https://drive.google.com/file/d/0BzlK1PD8EE75THdnRnAzWDl0VDg/view?usp=sharing
suggests that reflection has been a overlooked area of research for a long time, but that it is something that can be more effective than additional learning.

"In particular, we theorize that, once a person has accumulated a certain amount of experience with a task, the benefit of additional practice is inferior to the benefit of reflecting upon the accumulated experience. In other words, the intentional attempt to synthesize, abstract, and articulate the key lessons learned from experience generates higher learning outcomes as compared to those generated by the accumulation of additional experience."

We propose that the link between learning by thinking and greater performance is
explained by self-efficacy, or a personal evaluation of one’s capabilities to organize and execute
courses of action to attain designated goals (Bandura, 1977).
2 In particular, we claim that
reflecting on one’s past experience on the same or similar tasks allows individuals to reduce
uncertainty about their ability to perform such tasks successfully going forward. This reduced
uncertainty will translate into greater effort (Rosen, Mickler, \& Collins, 1987), which in turn will
drive the observed increase in the ability to perform well with the task at hand.

\textbf{Conclusion}
Research on learning has primarily focused on the role of doing (experience) in fostering
progress over time. Drawing on literature in cognitive psychology and neurosciences, we
propose that one of the critical components of learning is reflection, or the intentional attempt
to synthesize, abstract, and articulate the key lessons taught by experience. Using a mixedmethod
design that combines laboratory experiments with a field experiment conducted in a
large business-process outsourcing company in India, we find that individuals who are given time
to reflect on a task outperform those who are given the same amount of time to practice with
the same task. Interestingly, we show evidence that if individuals themselves are given the
choice, they prefer to allocate their time to practicing on the task rather than reflecting on it. Our
findings, however, suggest that this preference is irrational given the higher benefits associated to
reflection. Together, our results reveal reflection to be a powerful mechanism behind learning,
confirming the words of American philosopher, psychologist, and educational reformer John
Dewey (1933: 78): “We do not learn from experience...we learn from reflecting on experience.”

% Growth mindset or Mastery orientation
% http://www.strivetogether.org/growth-mindset-or-mastery-orientation
