\section{Findings in Multi-Variable Data}

% Important to be objective
% En diskussion om hur resultaten kan användas i praktiken är också i de flesta fall belysande och relevant i rapporter

% https://liu.se/ias/kontakta-oss?l=en

In this section, the conclusions from the different group characteristics are presented.

First-hand observations (before the parallell coordinates) were that there was a strong corrolation between pre-quiz results and quiz 9 try 1 (slightly visible also in quiz 3 try 1, but with more outliers). Also, with manuals there was a higher probability to finishing quiz 9 training + certification.

\subsection{Correlations}

\textbf{Youth mentor (brun), 6 st vs. CBT (blå), 14 st}
* Youth mentors has higher school level than CBT's
* 1/6 Youth mentors had brought manual, compared with 8/14 CBT's
* Only 1 CBT has above 2 (1 st 3) on School, while YM have (2st 0, 1 2st, 2 2st)
* Inverse correlation: CBT old, YM young
* There are no female Youth Mentors (i.e. 100\% male Youth Mentors)
* All of the YM's run their own businesses, compared with 5/10 for CBT's
* Only CBT's said they didn't feel comfortable with smartphone (2 st) - because of age?
* Seemingly no difference CBT vs. YM in when prepares for session
* All YM prepares 2 times for session, while CBT can train also 3 or 1 time)
* 13/14 CBT's gjorde quiz 3 try 1, 6/6 YM's
* YM's och CBT's presterar lika på quiz 3 try 1
* CBT 6/14 st certifierade, YM 4/6 st

Quiz 9 (rött=CBT:
* 6 CBT's gör ej Q9 try 1, 2 YM gör ej
* YM är top performers på Q9 try 1 jämfört med CBT's
* endast 1 YM klarade däremot träningen, medan

* YM's är bättre på quiz 9 try 1 än CBT's
* Det är endast 1/7 som klarade quiz 9 training som är Youth Mentor
* Antal försök man gjorde är likvärdigt, förutom en YM som hade 12 försök (och klarade quiz)
* Det var endast 2 st som klarade certifieringen, och båda dessa var YM och kvinnliga

\subsection{Women}

It is clear from the data that women:
* Have lower education level than the men
* Spread results on the pre-test (probably because of school level)
* Half of them are around 25 years old, half spread out (up until 45 years old)
* 2/6 har eget företag
* Det är bara 1 som ej preppar alls
* 1 st som endast preppar 1 gång, alla andra preppar 2 gånger (3 st) eller 3 gånger (2 st)
* Alla hade max 1 fel på quiz 3 på första försöket!
* De som ej hade rätt, tog det bara 2:a (1 person) eller 3:e försöket (1 person)

* 3/6 gjorde certifieringen - kolla upp: började de?

* Alla förutom 1 tjej gjorde svåraste quizet. De hade minst 42\%! Varav 2 st hade 67\%, 1 hade 50, 1 hade 42, och 1 hade 83

* Quiz 9 tjejerna hade mycket högre lägstanivå () än killarna (, och mycket högre högstanivå än killarna () - tiden är jämförbar, med svag tendens snabbare tjejer
* Av de som hade 50\% på 1:a försöket, gick det betydligt snabbare för tjejerna än killarna att jobba igenom quizet - tyder på att tjejerna är säkrare på materialet än tjejerna - dessutom är det bara 2 killar som fick över 50\% på första försöket
* Om du kollar tvärtom, så är det bara 1 tjej som fick under 50\% på första försöket, medan det var 8 killar
* Quiz 3 syns ej lika tydlig skillnad (OBS: kolla vilken fråga de flesta hade fel på, och kolla om det skiljer sig mellan killar och tjejer)
* Skolnivå verkar oberende på hur quizen blir, om man kollar quiz 9
* Tjejer, antal försök quiz 9 hade de 2 (2 st), 5 (2st, 12 (1st) försök innan de klarade - bland killarna var det 5 (1st) och 7 (1st). Men sedan så var det 0 av killarna som blev certified, men 2 tjejer (de som gjorde på 12 försök och 2 försäk). Att antal försök skiljde sig mellan 2 och 12, men ändå klarar det, berättar att antal försök kanske ej korrelerar. Den på 12 försök hade 70\% på försök, och jobbade igen de 12 försöken väldigt snabbt.
* Den andra tjejen som klarade certifikation quiz 9 klarade 83\% försök 1, (hade tillgång till hjälp), klarade träningen sedan på 2 försök.

Slutsats:
* Anställ bara tjejer. De har högre kunskap och förbereder sig mer, trots lägre skolutbildning.

\subsection{Use of participant and coach manuals}

Användande av appen:
* Vi hittade ingen korrelation quiz-resultat 9 första försöket om man fick hjälp eller inte, antagligen pga att man ej använder manualen före

\subsection{Certified quiz 9}
Only two people were fast enough to get certified on the final quiz before the app evaluation ended.

Characteristics were:
* Both of them used the manual
* Both of them were CBT's, not youth mentors
* Both were women
* They were 24 or 26 years old
* They had a good pre-test score (57\% or 71\%)
* They had top scores (1st place and 2nd place (shared with one other)) on quiz 9 try 1
* They had high scores on quiz 3 try 1 (100\% and 92\%
* They prepared many times per youth session (2 or 3 times)

What didn't seem to matter:
* Number of tries quiz 9 (12 vs 2 on Q9)
* Time to pass training quiz 9 (35.5, slowest vs 12 minutes, below average)
* When day trained (1 trained same day, 1 trained the day before)
* One had a business, one didn't
* School level (1 S?, one S lower)

Other:
* They were medium skilled on using a smartphone
