\chapter{Methods \& Implementation}\label{cha:Method}

% längsta avsnittet i rapporten. Den består av en redogörelse av ditt arbete och den visar hur du kommer fram till dina resultat.

% Undvik egna synpunkter. Dessa framförs i Inledningskapitlet och Diskussions-kapitlet

% Ange källa till figurer i slutet. T.ex.  Source: Expedition Mondial.

% Metoddelen beskriver tillvägagångssättet – intervjuer, observationer, litteraturstudier, laborationer och så vidare. Motivera varför en viss metod valdes och vilka eventuella svårigheter som har förekommit. Metoden ska vara replikerbar, vilket innebär att en annan skribent ska kunna göra om studien med hjälp av informationen i metoddelen. Det finns en mängd böcker om olika vetenskapliga metoder. Till exempel kan en intervju utföras på en mängd olika sätt. I rapporter inom humaniora brukar metoddelen vara mer utförlig än i en teknisk rapport.

Using novel methods like \textit{service design} when developing the app according to research question \#1 and \textit{data-driven design} and interviews for understanding interaction according to research question \#2.

\input{method/learning}

\input{method/design_process}

\input{method/development}
