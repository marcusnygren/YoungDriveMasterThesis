\section{Answering research question \#2}

\subsubsection{Consequences}

To be able to answer research question \#2, evaluation needs to be done. \textbf{\textit{How the Evaluation affects the development process: }} If no evaluation, there would be no need to write code, instead of working with a hi-fi prototype by using existing design tools. Now, we want to use a data-driven approach to measure, and therefore an app needs to be developed.\\

\subsubsection{How to measure}

Answering research question \#2 is a matter of choosing how to measure effectiveness. After choosing what should be evaluated, there needs to be a careful balance between what should be understood via interviews with the target group, and data collection via the app. There are three main aspects that are interesting:

These ways of measuring the questions are subject to change.

%Effectiveness measurements.

\begin{itemize}
    \item
    \textbf{How do users interact with the app?} (Usability) Give the users a mandatory task (e.g. use the app once a week), and see if they use it more on a voluntary basis, in order to determine if they use the app
    \textit{(Measure)} and determine why \textit{(Interview and Measurement)}. "Have you during the latest week felt that you've needed any support? Have you then used the app? Did the app help? Or have you searched for support elsewhere?"
\item \textbf{What usability aspects are associated with using the app?} (Usability) Do they like it? Ask: Are they stimulated? If not, why? What didn't they like? \textit{(Interview)} When can they use the app, and when are they not able to?

    \item \textbf{What learning outcomes are associated with using the app for the coaches?}  (Learning of Entrepreneurial Knowledge))
    How good are they at answering the questions? 1.
    \textit{(Measure)} What percentage of the answers were correct/inorrect?  2. \textit{(Interview)} Were the answers were correct/incorrect because of lack of knowledge or wrong formulated?

    Ask the teacher/country manager/project leader if they got valuable information. Ask: did it help them become a better teacher? Were the results trustworthy? \textit{(Interview)}

    Do they want to improve their knowledge via the app? This can be measured via how many times they repeat a test, what material they are studying for (e.g. measuring active time spent reading each section).
\end{itemize}
