\subsection{Iteration \#3}

\subsubsection*{Insights}

Findings:
\begin{itemize}
	\item Josefina does not want to be replaced
    \item The app would be great and could actually work outside the physical coach training - with revision, be stand-alone, even being able to distribute online
    \item Refugee innovators has a great need for such an app
    \item Test with university student scored 100\% correct, means that common sense can go a long way, and that the results can't be 100\% trustworthy, and that multiple-choice questions has serious issues - this, we already knew during and before the coach training - but it needs to be taken care of
\end{itemize}

Check with findings iteration \#1:

\begin{itemize}
\item The app is only working on assessment now, not for learning
\item The need for a field app still feels relevant (especially for sessions long since the coach training)
\item The potential for YoungDrive having online coach training is huge
\end{itemize}

Determine:
\begin{itemize}
	\item Focus for the next iteration: design quiz app for learning, focus on field app (CI, CS, TM, FA), or design app that works stand-alone from the YD coach training.
\end{itemize}

\textbf{After-talk with Josefina, on Skype in Uganda, 2016-04-02:}
Nu har vi praktiska bevis för vad jag upptäckte under Iteration 1:

* Utbildningen ska faktiskt träna dem i förbereda quiz-session
* Därför borde även quiz testa detta
* Vad det inenbär vara bra coach, är hålla bra ungdomssessioner
* Det finns vissa coacher, Josefina hade velat stoppa från att hålla ungdomssession, om de inte har 90-100\% correct information
* Men hon kan inte asessa detta
* Här, är quiz en väldigt bra möjlighet
* Quiz-app under utbildning och ute i fält går därmed ihop


\subsubsection*{Ideation}

Josefina: “Jag kan ju inte kontrollera dem på något sätt hur de förberett sig”

Förbereda session:
Antingen inför du börjar, sedan har du ett hum vad du behöver träna mest
Eller (bättre/säkrare iom quiz inte täcker allt), så bättre förebereder sig, och sedan gör quiz när de tycker de är färdiga

Har du fått 9/10 rätt, då är du förberedd! (9-10 rätt)
Det de har fel på, det kommer de ju lära ut ungdomarna fel på.

Om du har 8 eller under, då ligger du i röd zon
Om du har fel på t.ex. “Vad är en entreprenör?”, då kan du ju inte förklara det för ungdomarna!

Därför borde de ha alla rätt.

Där kan man ju också jobba mycket mer med feedback via appen, och skapa en annan typ av quiz för förberedande

Därför får jag Josefina att ta fram en ny quiz, som är "Förebered session X"

%https://docs.google.com/document/d/1DaVj3sOBxsnBJvQYM1UbvHcJrowaUBUnlXPYznc6csE/edit?usp=sharing

Syftet är att komma högre upp på Bloom's revised taxonomy

Dessutom, pratade vi om utvärderings-appen: varför behöver den vara en app? Det är extremt tydligt att det fysiska ändå måste finnas. Den behövs för att ge coachen smart själv-utvärdering.

Lärdom: det passar faktiskt inom ramen för quiz-appen
Frågor blir automatiskt meta-cognitive
Passar jättebra med Learning by Reflection

\subsubsection{Monitoring \& Evaluation app can converge with Quiz App}
När någon är där och ger dem feedback, så blir det extremt svart på vitt, att de inte är så himla förberedda som de tror

Och: “Vad händer om du ger fel information här?” Vad händer om du säger cost är X?

\textbf{Varför behövs det en app för M\&E?}
För coacherna, ska få insikt på hur de håller en session
De är inte ärliga med sig själva egentligen

Appen skulle kunna ge självutvärderings-frågor?

Redan nu, så skulle man ju kunna fråga: “Vad hade hänt om du svarat X på Y?”
“Varför är det viktigt du svarar rätt på detta?”

\subsubsection{Appens fråge-struktur}

För att förbättra multiple-choice och komma högre på Bloom's taxonomy

Fyra idéer:
1. Coachernas idé från Zambia: "Gör om"-knapp (ger dig nytt quiz med bara de frågor du hade fel
2. Första idén för att lösa: håll in knapp för att spela in svar "Vad är en entreprenör?". När du klickar "Nästa", så syns multiple-choice och du väljer det alternativ som var närmast vad du svarat. Feedback experter: utmaningar med användarvänlighet (liknar Snapchat, ingen vana), du kan fortfarande välja det mest sannolika alternativet (coachen gör en självbedömning, kanske tycker A var närmast när B var närmast, och coachen kan ljuga). Fördel: YoungDrive kan använda de inspelade svaren på bra sätt. Nackdel för mig: tar tid att implementera.

3. Från Lena: gör som i NTA Digital, du får medalj guld, silver, brons baserat på hur många gånger du försökt få 100\% rätt

4. Henrik Marklund kom med följande förslag istället, inspirerat av lärare han kände: "Är du säker?" efter varje fråga
4a) Först var idén: gör som läraren, Ta fram ett gemensamt betyg, MVG, VG, G, genom att vikta Korrekthet med Empowerment
4b) Fick feedback från Josefina: ha två separata staplar Korrekthet och Empowerment. Coacherna kan 1) vilja gamea systemet, och 2) undra hur de fick sin score. Kan då bli svårt att förklara.

Fanns extremt många fördelar med denna, och kom bara fram till ännu fler efter diskussioner med människor och Lena Tibell, framför allt hur denna kan förbättra utbildningen och 1-on-1 coachning, och bli väldigt bra självreflektion för coachen.

\textbf{Sammanfattning, de 4 idéerna}

App för lärande, inte bara utvärdering.
4 vägar att gå vad gäller hur frågor ska ställas:
1. Fältmetod: Efter du har fått slutresultatet så kan du trycka på improve för att få alla felaktiga svar. Klassiskt inom körkortsproven i Sverige.

2. LiUmetod: För varje försök sänks resultaten från guld, silver och brons. Motiverar till att studera innan = gamification.

3. Pedagogikmetod: Teknologi och förenkla livet. Efter varje fråga lades till ”hur säker är du på ditt svar?”. Kräver studenten att reflektera över sitt svar, metakognitivt tänkande.

Två staplar:
Så här mycket rätt har du / Så här empowered är du!

4. Innan du får svarsalternativen så får du spela in dina svar, sen välja ett alternativ som de tycker är närmast. Det är bra för de som utbildar coacherna.

\textbf{Beslut av approach}

Idé för interaktioner blev först att A/B-testa Idé 1+3 vs. idé 4 under en workshop, och sedan testa Idé 2 ute i fält för att mäta användbarhet, i.o.m. den metoden gav mycket kvalitativ data, och var bra feedback till utbildarna.

Feedback kom först från Expedition Mondial (som hälsade på under veckan) att under min workshop med coacherna kommer säkert idé 1+3 och 4 blandas ihop.

Inför mitt möte med Grameen, pratade jag med innovationsrådgivare Peter Gahnström om hans analys av de 4 alternativen. Han gillade alternativ 4 mest, och gav följande nya insikt till mig: "Det här kan motverka traditioner och ”så här vi alltid gjort det” genom att tvinga en att reflektera över varför inte ens rätta svar korrelerar med hur empowered du känner dig. Bryter normer, sätter sig emot lathet och agerar proaktivt för en skarpare utveckling tillsammans."

Jag träffade Juliet och en utvecklare på Grameen Foundation (som gjort LedgerLink), och gick igenom de 4 alternativen. Svaret gavs att idé 2 definitivt är för okänd för användarna. När indikationer kom från även Grameen Foundation att 1+3 och 4 säkert skulle blandas, och var grymma alternativ, fråga jag "Hur?".

Svaret blev en diskussion med att i ett resultat efter ett quiz få två scores Korrekthet och Empowerment. Sedan på Improve, så får du medaljer/score baserat på Antal försök. (Grameen trodde ej det skulle bli problem med gamification på idé 3. ) Du ska nå t.ex. 90\% rätt på båda staplarna.

Hon (Juliet) föreslog också att du kanske inte måste ha chans på guldstjärna bara på första försöket. T.ex. att om du gör quizet gång 1, så måste du få 90\% för guld, men på ditt andra försök måste du få 95\% för guld. Detta då vi ju vill att coacherna ska ha läst på innan.

Lena Tibell menade vid förslaget att "Belöna inte hur snabbt en elev går från att kunna till att inte kunna, för olika människor lär sig olika snabbt". "Vad vi ville åstadkomma med Antal försök var endast att undvika gissningarna".

Lena frågades också om vilken skala jag ska ha på 5, 4 eller vad jag inte tänkt på, 2-gradig skala. 5 eller 4 är vilket som enligt literatur, det finns två olika skolor. 2-gradiga skalan bedömde jag vara bäst, p.g.a.
* användarvänlighet, tydligt för coacherna
* behöver inte vara krångligare än så till en början, en bra test
* blir enkelt att mäta empowerment, rätt svar + säker = pluspoäng, rätt svar + osäker = du gissade men hade rätt, gör om och var säker -> empowerment, fel svar + säker = måste ge feedback (väldigt intressant för Josefina), och fel svar + osäker = du ahde rätt, det var ett annat alternativ, gör om -> empowerment + korrekt information.

\subsubsection{Appens datainsamling}
Denna gång behövde appen samla in data av sig själv, istället för att Josefina manuellt skrev ner resultat-tavlan efter varje quiz.

Kravet kom dels från Josefina (det kommer inte gå om det är mer än 10 coacher, vi har oftast 20-30), dels från att jag i mina interaktioner i Tororo skulle testa på 2 olika kontrollgrupper med 10 personer vardera, och jag visste baserad på Interactions 1 att jag inte skulle ha tid att både skriva ner resultat och observera hur de beter sig med appen.

\textbf{Inloggning}
Att samla in data för användare, skulle kräva inloggning. Men det är ett användbarhets-problem för de flesta. Om de skapar en användare med lösenord, hur ska de 1) tycka det är intutivt och 2) komma ihåg sina användarnamn och lösenord till Interactions 4 om 2 veckor?

Jag pratade med flera om detta, Expedition Mondial och Grameen. Från EM lärde jag mig att de trodde min idé med en färdiggjort lista med coachernas namn (vi vet ju vilka som är i Tororo) skulle fungera, och från Grameen fick jag höra om dera erfarenhet att de validerat använda samma approach, med en PIN (längre än 4 siffror dock), men att de inte nailat konceptet ännu, och att de också itererar på sin approach för nästa uppdatering av LedgerLink.

Tyvärr har också Meteor begränsningar med deras auto-login-modul. Den tvingar både användarnamn och lösenord, och har automatiskt registrering. Går det att stänga av? Jag kan skapa användare och lösenord åt alla, och funderade på hur jag skulle generera lösenord. Ett förslag blev att bara registrera deras förnamn, och sedan skapa lösenordet baserat på T9 med de 6 första bokstäverna utan att berätta det för dem. Sedan tänkte jag på det kulturella, att det kan vara oartigt med förnamn, och bestämde mig för efternamn istället. Hela namnet skulle bli för långt och krångligt.

Helst skulle jag behöva gå runt Meteors standard-inloggning, och istället ha en enkel login-rullista som den ovan beskrivet, istället för att använda deras standard-lösning.

\textbf{Meteor Collections}
En annan problematik var att om data ska skickas till en server, måste det finnas en server med Collections. I version 1 av appen sparades inga resultat i huvud taget.

Jag gjorde en exempel-app med Meteor Collections under veckan, och det är ganska coolt med DDP, och appen kändes snabb oavsett ej internet-connection. Det var däremot svårt simulera samma internet-problem som ute i fält. Det är en risk jag tar, att appen kanske inte kommer skicka in rätt resultat.

Därför ville jag även ha offline-databas, och då fanns det en plugin som hette GroundDB.

Detta var tidsödande, och vi får se om det fungerar bra på måndag.

Ett annat problem är, hur ska detta visualiseras pedagogiskt för Josefina och de andra utbildarna?

\subsubsection{Educator Dashboard}
Detta hanns inte med i Iteration 3 även om det var ett mål. Istället gjordes trigger-material och workshop-upplägg till Tororo, då Expedition Mondial ifrågasatte "Visst är väl även Christine och Patrick?" målgrupp för detta? Och vad har de för utrustning? Christine har mobil, Patrick ingen. Så detta talade för att Educator Dashboard skulle behöva fungera på mobil, och inte bara dator som jag tänkt, iom att Josefina har dator.

Då bestämdes med Expedition Mondial att jag skulle ha workshop med dem på onsdag. Med samtal med Josefina, sade hon att de garanterat borde utrustas med tablets då de samlar in data digitalt, så jag kan tänka mig att de får en tablet framöver. Skönt! Detta stämmer även med vad Stefan FalkBoman hade tänkt sig, och de iPads han köpt in till mig. Så då kunde jag ha dessa som tanke att utforma educator-app-dashboarden ifrån.

\textbf{Tekniskt}

HighCharts var påtänkt som verktyg för att visualisera datan. Tanken var att den vanliga appen skulle kunna ha super-användare som är admins, och kommer till ett särskilt gränssnitt där de ser data om användarna. Detta kunde göras direkt i Meteor.

Stefan frågade vad jag tänkt om detta, och frågade om jag funderat över integration med deras verktyg Podio, och om det var möjligt. Det sade jag att det var framöver. Podio har ett API som bl.a. stödjer JSON, vilket jag använder. Då frågade jag om Podio har bra visual dash-board -verktyg, vilket han skulle kolla upp. Inom exjobbet behöver jag än så länge inte bry mig om detta. Problemet är att Stefan ser ett värde i att lagra datan i Podio, men de har inte i sig själv bra visualiseringsverktyg.

9 april hade jag då ett möte med SolarSisters COO Dave, som är en social enterprise med 1 huvudansvarig (Dave), 70 field staff och 2000 entreprenörer, så ganska likt YoungDrive i Uganda när Josefina var där.

Dave byggde 2013 upp deras backend i SalesForce för databas, och sedan TaroWorks för datainsamling. TaroWorks är en plugin till SalesForce, med offline-app anpassad för fält. De har sedan utrustat alla field staff med tablets, då det var för dyrt att ge till alla 2 000 entreprenörer. Field staff träffar entreprenörer varje dag, och hjälper entreprenörerna att knappa in t.ex. kvitton, utvärderingar och undersökningar (t.ex. från finansiärer) via appen.

Det tog 3 veckor att bara sätta upp systemet, och det var snabbt. För Dave har det varit en 100\%-tjänst i början, och fortfarande 20\%. Men fördelarna är att de nu är 100\% datadrivna, och de kan följa exakt hur det går för field staff och entreprenörerna - detta guidar även vilka som får promotions och vilka som blir avskedade.

Det mest intressanta är kanske att SolarSister inte bara avnänder datan för internt bruk, utan även för dess partners. De gör undersökningar via TaroWorks som inte är direkt kopplade till SolarSister, för att få in pengar. Men framför allt, sticker de ut gentemot andra social enterprises, då de enkelt kan ge partners all data de önskar, och det gör dem väldigt framgångsrika med grants. De är sannerligen en datadriven organisation. Datan, ger SolarSisters story ett trovärdigt narrativ, vilket Dave beskriver som en extrem framgångsfaktor.

Idag står 2/3 av finaniseringen från grants (fördel med socail entrepriser, du kan få pengar både från finansiärer och kunder), 1/3 från entreprenörerna. De vill bli mer self-sustainable för varje år som går, och detta är storyn som datan måste berätta - vilket är varför de t.ex. avskedar människor som inte presenterar. Datan måste stämma med storyn de vill berätta, för den storyn är vad som avgör att de får in pengar.

Detta gav mig insikter på hur mycket min datainsamling från coacherna kan spela roll för organisationen. Det var något jag inte tänkt på innan, och som jag vill vara medveten kring. För om det är något jag lärt mig denna iteration, är det hur "kunskap är makt", och hur mycket vettig kunskap jag, coacherna och Josefina och YoungDrive kan få ut av att helt enkelt lägga till frågan "Är du säker?" till varje quiz-fråga.

\subsubsection{Teknisk utveckling: från Meteor 1.2 till 1.3}
Branchade ut projektet och uppdaterade till Meteor 1.3 från 1.2, vilket gav bättre utvecklarupplevelse och många sådana fördelar (till exempel kan använda NPM), men fanns inte längre bakåtkompatibilitet till mobilerna som coacherna använder, samt att buildpack för Meteor till Heroku inte hade uppdaterats, så vid en push (även om det fungerade på localhost) så krashade hemsidan youngdrive.herokuapp.com, vilket fick feedback från handledare Lena som behövt accessa sidan.

Detta är ett bra exempel på hur tekniska begränsningar påverkar projektet. I slutändan, tog det ganska mycket tid under veckan "i onödan", och jag fick ta igen tiden genom att jobba fredag kväll och lördag inför mina interaktioner.

\subsubsection*{Trigger material}

Vad som guidade trigger material \#3, litteratur:
In particular, we theorize that, once a person has accumulated a certain amount of experience with a task, the benefit of additional practice is inferior to the benefit of reflecting upon the accumulated experience. In other words, the intentional attempt to synthesize, abstract, and articulate the key lessons learned from experience generates higher learning outcomes as compared to those generated by the accumulation of additional experience.
% https://drive.google.com/file/d/0BzlK1PD8EE75THdnRnAzWDl0VDg/view?usp=sharing
% Koppla till Bloom's Taxonomy, self-conginition

\subsubsection*{Interactions}
