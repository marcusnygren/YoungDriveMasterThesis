\section{Iteration \#2}
Here, the work and result from iteration \#2 is presented.

The interactions for this iteration were planned to be in Tororo. However, during a meeting during the first week with YoungDrive project leader Josefina, I was invited to participate in the coach training in Zambia. A new work plan was created, so that I could travel to Zambia and develop the app and participate in the YoungDrive coach training together with the coaches.

\subsection*{Insights}

There were two main insights from iteration \#1.

1. The aim is for the coach to feel self-confidence for its youth session
2. The skill to be trained is having a youth session

During the evaluation meeting with Linköping University and YoungDrive, it was the determined that Iteration \#1, is answering the research question \#1, \#1a, and \#1b.

The iteration has provided a good basis for answering research question \#2.

It was concluded during the partner meeting, that iteration \#2 should:

1. Allow me to test the validity of my insights from iteration \#1.
2. Be carried out in a way that I can compare usability and learning done via the app, between iteration \#2 and \#3.

\subsection*{Ideation}

This was the start of the quiz app. The focus was on assessment. For example, it was decided with Iliana, that no facts would be presented before the quiz.

It was discussed, how the correct information about YoungDrive would be presented. Thus, for this iteration, questions were created by the YoungDrive team, and I developed the app.

The ideation started with me creating a quide how to write questions according to Bloom's revised taxonomy, which was shared to Iliana and Josefina.

The initial plan was that the team would only produce questions for two sessions, not all 10.

Iliana did questions initially for the two sessions, mapping each question to the Bloom taxonomy.

Then, when it was decided that the app would be developed and used during an actual coach training in Zambia, it was decided that questions would be created for all sessions.

\subsection*{Trigger material}
Project leader Josefina in Zambia refined the first question sets, prepared for my visit in Zambia. Josefina created question sets with Bloom at the back of her head, also taking into account the structure and the order of the coach manuals, what it means being a coach within the topic, and lastly scenarios.

On the week before the trip, and on the airplane, I did a working prototype, a very basic quiz app, keeping it as simple as possible. I brought with be devices (tablets and smartphones).

When I arrived at the evening before the coach training started, I added the questions to the app, and installed the app to all of the devices. This process was repeated for all the days, Sunday-Friday.

\subsection*{Interactions}

\subsubsection{Design workshop \#1}
The coach training started with me having a design workshop with the coaches, not showing them the app that I had created.

Since the knowledge about smartphones and apps were low, I started by introducing these topics.

All were familiar with Facebook, so thus I showed the Facebook app. Me wanting to know what the app would look like if the coaches would have designed the app, I first needed to train them how to design an app via drawing wireframes.

Using postits, they started with during limited time drawing the start view from the Facebook app.

Then, they were asked to draw what they thought happened on the friend icon click, drawing the view on another postit.

Then, the mission of the YoungDrive app was described. They were then divided into two teams, having limited time to draw the best imaginable YoungDrive coach quiz app they could. First, they designed the app from the top of their heads. They then pitched their results to each other.

On the next iteration, they were to suggest and design improvements how the app should be designed to improve learning, not only assessment. They then again pitched their results to each other.

The result was fantastic, in the sense that it gave me an unbiased look at what the coaches expected from the app, what functionality wasn't important, and into their technical preferences.

The designs and insights gained were used throughout the week to further improve the app I had actually started creating, and gave great insights to who the coaches were and their thinking.

\subsubsection{Assessment via quiz}
At the end of each day, the app was used to test the coaches' knowledge. Each coach got either a smartphone, tablet or computer. The coach first took the quiz for the most recent session, and could then choose what to do next.

As there were no back-end developed, Josefina by hand documented the scores of each coach, writing the name of the coach, the session, number of correct answers, and what questions had been answered wrong.

Josefina then, when planning the next day, looked at the statistics, looking for trends that would inform the sessions for the following day.

She also evaluated the quality of the questions, before creating the new question sets for the next day.

\subsubsection{Experimenting with quiz before or after the session}
Since the coaches appreciated the app so much, we felt tempted to try what would happen with fun and learning if we tried using the app \textit{before} a session instead of only after. During the rest of the week, we continued, finally finding preferences and tendencies from the coaches, via observation, interviews, and survey.

\subsubsection{Experimenting with design of questions}
Number of questions
Multiple-choice questions
Framing of questions
Challenge level of questions
Determining what made a question hard

\subsubsection{Testing the app outside the YoungDrive context}
Test with refugee innovators was surprisingly successful, Humanitarian Innovation Jam
Test with university student from Makarere University scored 100\% correct

\subsection*{Result}
Using the quiz before the session increases learning, slightly decreases fun of the session

The app works for assessment!

Results from the coaches:

Trends from the coaches:

\subsection*{Analysis}
Bugs
Simpler design than I thought (KISS)

\subsection*{Discussion}
If I would have created myself, would have assumed more functionality was necessary and requested

\subsection*{Conclusion}
\begin{itemize}
\item Short iterations are very effective, however not perfect
\item Field hackathon, designing and developing together with the users, is fantastic
\item I would never have come this far without the short iterations
\item The app works for assessment!
\item Fun and encouraging and good for learning for the coaches
\item A good indicator for Josefina
\item A great way to scale the YoungDrive training in the future, both for online coach-training and the physical training
\end{itemize}

\subsection{Staging environment using Heroku}
Needed when the Meteor free tier was removed. Connected to deploy from GitHub branches automatically. Could have benefitted from CI, passing tests before ready for production. Solved this by having a stage environment (since April 19th) where stage is YoungDrive-beta (branch Iteration 4), and YoungDrive is master.

