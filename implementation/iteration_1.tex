\section{Iteration \#1}
Here, the work and result from iteration \#1 is presented.

\subsection{App/Web Development}
Early in the project, it was thought that existing tools could be used, instead of building the app from scratch. E.g. using existing tools like Knowly or Typeform\footnote{examples include https://showroom.typeform.com/to/ggBJPd and https://showroom.typeform.com/report/njdbt5/dIzi} during the first iterations for understanding users, and during development e.g. the Typeform API (http://typeform.io/). The Typeform API allows developers to create surveys from within their own applications or systems.

\subsection*{Insights}
\textbf{Week 6-7}
Learn about the YoungDrive organization

\subsubsection{Start-up meeting with partners}
On February 10th, a first stakeholder meeting was held between me and the country managers for YoungDrive (Iliana Björling) and Plan Interational (Shifteh Malithano).

\textbf{Plan: Learn from previous work}
Led to visits and interviews with Designers without Borders and Grameen Foundation (carried out on February 26th).

Skype interview with Gerald, Plan, Tororo is used instead of both Kamuli and Tororo
Entebbe, literature review \& research
Write on the report

\subsection*{Ideation}
\textbf{Week 6}
Outbox workshop with Mango Tree
Create Workshop \#1 and Workshop \#2 with Expedition Mondial

\textbf{Week 8}
Create questionairre guide with Expedition Mondial and YoungDrive

Designers without Borders
Grameen Foundation

\subsection*{Trigger material}
Preperation
Quizical
Duolingo

\subsection*{Interactions}
\textbf{Week 7: February 23rd: Number of interactions for iteration \#1 cut down}
Interactions canceled for week 7, the day before Wednesday-Friday, because of local elections.

"Det var tråkigt att höra att det inte blev lika många interaktioner som planerat.
MEN jag tänker: Det här är verkligen en del av lärdomarna att jobba med tjänstedesign i andra kulturer (som jag även tar med mig från vårt projekt i Kenya). Det går bara att planera till en viss grad, och det blir aldrig riktigt som man tänkt sig :) Man får vara beredd på att ändra planen i sista sekund, mycket mer än vad man behöver i sin egen kultur. Bra lärdom!

Så utifrån dina fåtal interaktioner i början på nästa vecka kommer du iallafall ha en hypotes, även om den kanske är lite vagare än vad vi tänkt från början. Jag kan skicka dig nästa kapitel i Coaching Handbook som handlar om Analys senare i veckan så kan du börja fundera på hur du bäst gör analysen utifrån det material du har. " - Susanna, Expedition Mondial

\textbf{Week 7, Thursday, February 25th}
I realize what I’m actually doing is “Designing and Developing Mobile Learning for Entrepreneurship Coaches in Uganda”. The master thesis title is changed to this.

\textbf{Week 7: Friday, February 26th}
Ringer Gerald 26 fredag februari, som meddelar att nya tidschemat jag hade är omöjligt. Han har bara bokat alla inblandade kl. 8-17, då Plan inte tillåter field trips p.g.a. local elections

Krismöte med Josefina, som föreslår att gå bakom kulisserna och engagera Christine och Patrick, utan Plans inblandning. Kanske till och med kan besöka coachgrupp
Sammanfattning: interaktionerna har gått från 3 dagar, till 2 dagar, till 1 dag

Varje gång har jag behövt anpassa mig, och hitta ett nytt koncept
Nu kanske det blir 1 dag i Plans regi, och jag ändå är i Tororo måndag-onsdag.

\textbf{Week 8, Sunday-Wednesday in Tororo}
Sunday, travel to Tororo
4 (not 3 or 8) face-to-face-interviews
1 meeting with Plan, 1 with local partners
Workshop \#1: Customer Journey Map: A day as a coach
Workshop \#2: Quizical and Duolingo
2 field visits
Stay over with Patrick

\textbf{Week 8, Thursday \& Friday}
Thursday: Expert meeting with Expedition Mondial and LiU Innovation
Friday: Partner meeting with Linköping University and YoungDrive
