\subsection{Conclusions}

\todo{Add everything from the mindmap}

In three months time, an app was developed with precision to the needs and context of the end users. The design has been heavily influenced by the end users, from day 1 of the project, in conjunction with relevant research, and in balance to stakeholder goals and considerations, and supervisor advice.

The results shows that the ideal coach, according to the quiz app, would be a woman, since she has better knowledge in spite of having less formal education. She prepares more, is more aware of her own knowledge and has a better study technique, respecting the app feedback for meta-cognition and meta-memory. This can be seen by higher quiz results, faster learning, and more honesty in "Are you sure?".

It could be that first-time smartphone users have a disadvantage with the app, since they will not learn as fast as experienced users. The interactions shows however, that at the second session, almost all of the coaches felt intermediate instead of beginners, using the smartphone and the YoungDrive app. The quiz data verifies this, with no direct correlation between technical skills and quiz results.

The final version of the app shows users can get 100\% on quiz results much faster that the previous version, where the score board had been improved. Since the target group in Zambia and Uganda was different, it is hard to say if it went faster getting 100\% with the possibility of repeating only the wrong questions, asking "Are you sure?", and providing individual feedback. The qualitative study does show however, that 100\% thought the feedback was good for learning, and that they appreciate the app.
