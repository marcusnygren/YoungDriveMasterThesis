\subsection{Iteration \#2}



\subsubsection{Bonus results: Testing the app outside the YoungDrive context}
Back in Uganda, a test was done with refugee innovators, at the Humanitarian Innovation Jam.

Also, the app was tested on a university student from Makarere University

The test with the refugee innovatiors were surprisingly intriguing and successful.

It was found that refugee innovators says they would have a great need for such an app.

The university student from Makarere University scored 100\% correct, in spite of not having any entrepreneurship training. This showed that guessing was possible, or that the quizzes were too easy.

\subsubsection{Findings}

Test with university student scored 100\% correct, means that common sense can go a long way, and that the results can't be 100\% trustworthy, and that multiple-choice questions has serious issues - this, we already knew during and before the coach training - but it needs to be taken care of

The app would be great and could actually work outside the physical coach training - with revision, be stand-alone, even being able to distribute online.

Now there are observable evidence for what the interactions from Iteration 1 showed:

\begin{itemize}
\item The purpose of the coach training should be to prepare the coach in having great youth sessions
\item Therefore, this is what the quizzess should assess
\item What it really means being a good YoungDrive coach, is having good youth sessions
\item Josefina would have liked to be able to stop coaches from having taught, if they do not have 90-100 \% correct information on the subject
\item Today, Josefina can not assess this. This means that some coaches, are teaching incorrect information to hundreds of youth.
\item Here, the quiz has a very good need to fill.
\end{itemize}

With all of these findings in summary, it can be concluded that an app for coach training, and an app to use before a youth session, could be the same app, since the purpose of preparing the coach to be great with its youth session is the same.

From the interviews, it was learned that while it \textit{may} be technically possible, the teacher desires the app support her, not replace her.
