\section{Example}\label{sec:research:history}
%
Liksom \citep{Duck:2005} har vi kommit fram till att glass smakar bäst på sommaren \citep{Khalil02NonlinearSystemsBook}.

\marginpar{Kommer att tänka på en liten anekdot\ldots}

\Warning[TODO]{Ta bort den löjliga anekdoten!}

\begin{figure}[tbp]
  \centering
  \subfloat[Alldeles för tidigt.][\label{fig:times:very-early}Det här är väl tidigt — din glass hinner smälta innan ditt sällskap dyker upp.]{\includegraphics[page=1]{clocks}}
  \qquad
  \subfloat[Med marginal.][\label{fig:times:early}Kiosken stänger snart, men inte nu — perfekt!]{\includegraphics[page=2]{clocks}}
  \\
  \subfloat[I grevens tid.][\label{fig:times:on-time}Precis i tid — du får in ett finger i luckan just när kiosken ska stänga.  Han som jobbar blir sur, och det blir smolk i bägaren.]{\includegraphics[page=3]{clocks}}
  \qquad
  \subfloat[Försent.][\label{fig:times:late}Du är sen — kiosken är stängd.]{\includegraphics[page=4]{clocks}}
  \caption{\label{fig:times}%
    Illustration av \emph{subfloats}.  Den så kallade \emph{bounding box}en visas i \protect\subref{fig:times:late}.  Lägg märke till att bounding boxen har satts så att alla bilder har samma storlek, med enhetlig placering av själva innehållet i förhållande till bounding boxen.  Antag att du ska träffa en kompis för att äta glass just när kiosken stänger för dagen vid 08:30.  När dyker du upp?}
\end{figure}