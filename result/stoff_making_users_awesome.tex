Making Users Awesome: "My compelling context: I want to help you become an even better coach. (2016-02-24)

The Better User POV:
Don’t just make a better [X] (coach training app),
make a better User of [X] (coach training material)

Assignent:
Write a detailed description of what you see and hear, for the time-frame that makes sense for your bigger context.
You can do variations of this exercise for users at different levels from first-time newbie to expert.The key is to start thinking—hard—about what most products and services don’t: the post-UX UX. (s. 57)

Create a definition of badass for your context (s. 89)
"Given a teaching situation among the youth group, a great coach can teach an entrepreneurship topic more consistent with what the coach material said."

Create a definition of badass for your context (s. 90)
"Given a question in the app, a badass user of the coach training material will get the right answer more often, and leverage the correct answer to their coach situation."

\textbf{Simplified rules for Deliberate Practise}
Help them practice right. Goal: design practice exercises that will take a fine-grained task from unreliable to 95\% reliability, within one to three 45-90-minute sessions

If you can’t get to 95\% reliability, stop trying! You need to redesign the sub-skill

\textbf{Perceptual exposure}
With a little ingenuity, the British finally figured out how to successfully train new spotters: by trial-and-error feedback. A novice would hazard a guess and an expert would say yes or no. Eventually the novices became, like their mentors, vessels of the mysterious, ineffable expertise.

Perceptual knowledge includes what we think of as expert intuition. (like being a good coach) Thus, we should expose him/her to trail and error via the coach training app.

\textbf{Create a path (s. 193)}
Make a list of key skills ordered from beginner to expert, then slice them into groups to make ranks/levels. For motivation, the earlier, lower levels should be achievable in far less time and effort than the later, advanced levels. One possibility is to have each new level take roughly double the time and effort of the previous level.

\textbf{Exercise: design a “belt” path for your context (s. 197)}

\textbf{Exercise: What’s the first “superpower” for your context? (s. 205)}
Like going up a rank?
A hint of becoming a Certified coach

\textbf{Just-in-Time learning (vs. Just-in-Case)}
To bypass the brain's spam filter.

\textbf{Validate the need for this knowledge - What knowledge goes on the board?}
topics on trial, and mapping to skills.

s. 271 jätteviktig

1. Compile the cards of initial knowledge and skills

2. Map the knowledge to a skill and verify that you \textit{can’t} possibly do this skill without this knowledge

3. Remove orphaned (unmapped) knowledge. Never forget cognitive resources are scarce and limited.