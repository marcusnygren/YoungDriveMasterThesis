\subsection{Quantitative data}

    %What the data said

    \subsubsection{Observable trends from the coaches}

     \textbf{Usability: } The most notable thing from the app test, was that the app was not user friendly at all for 1st-time smartphone users. There were a lot of bugs, e.g. resizing of the font size for each new question. This forced some coaches to try to zoom on the devices, even if they did not know how. This could in turn cause refresh of the web page, and sometimes there was no Internet available. Thus, the data can not be fully reliable.

  This was the first time true frustration was shown. Out of 23 respondents, 7 rated the app easy, 11 medium, and 6 hard. This was not viable.

    \textbf{Related to assessing Correct Structure and Time Management}
    Using a "Are you ready?"-quiz the multiple-choice-structure was tested to assess and train Correct Structure och Time Management. This was not shown very effective. While it does test Factual Remember, e.g. "How many minutes should you spend on X?", since the answer to a lot of questions are retrieved from memory, instead of analyzing, questions often does not score higher on Bloom.

    The data and observations shows that learning Correct Structure and Time Management via multiple-choice is not effective for learning. To score well on such a test, the coach would retrieve from memory using a clear mental image.

    \textbf{Related to mobile experience}
    Before the quiz started, the coaches were asked to raise their hand if they felt proficient with using a smart phone. 8 out of 23 said yes.

    After the quiz, 16 said they were proficient (25\% increase), while 5 said low proficiency, and 2 said no (we don't yet feel proficient, still fear).

    \textbf{Field tests}
    During field tests, 3 CBTs were visited, to further observe usage of the app after having prepared a youth session.

    Some things were notable from the interactions with John:
    \begin{itemize}
    \item "Are you sure?" is understood intuitively (you can't progress without answering), but some coaches deliberately answer "Yes" even if they are not sure.
    \item Idea to highlight different words of similar answers, to increase speed
    \item In summary, if wrong, show the other alternatives either way, not only the wrong answer
    \end{itemize}

    It was also here, that it was first observed that this is a light version of both deliberate practice and perceptual exposure. It is just that the app as of now is quite inefficient, especially in terms of speed.

    \todo{Add idea for future work: Show how a persons correctness level has increased over time}

    From the interactions with Juliet, this was discovered:

    \begin{itemize}
    \item Idea for future work: "Go to participant manual" within the app
    \item If correct and unsure, she says "I still feel good". "Include it in wrong, because maybe I was still guessing". (This later informed the Certification quiz-insight)
    \item Change button to "Become certified", to increase likelihood to press the button. As of now, it was not obvious.
    \end{itemize}

    When she did get certified, she said "I feel good". When asked why, she said: "They have appreciated what I have done".

    \textbf{"Are you ready?" app test}
    The next day, the three CBTs gathered at the Plan International office to do an app test on the hardest quiz.

    \todo{Add results}
