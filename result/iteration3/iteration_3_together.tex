\subsection{Sprint Demo}

  A sprint demo concluded the findings from iteration 3. Now the coaches could not only assess, but also \textit{learn} Correct Information, which was successful, but needs to be done more effectively. It took an unacceptable amount of time to reach 100\% proficiency on all the quizzes. This was especially evident, on the quiz on Correct Structure and Time Management, "Week 9: Are you ready?", when it took a coach 102 minutes to reach 100\% without errors. In iteration 2, when "Improve" did not exist, it probably would have taken even longer.

  For the first time both signs of learning via perceptual exposure (many questions during a limited time, by trial and error) and deliberate practice (via learning via reflection) could be identified from the app. It is just that the app as of now is quite inefficient, especially in terms of speed, so while the ideas are there, the criteria are not fulfilled.

  The focus had been on "I am well prepared", but also including "I am certified.". It was shown that most coaches do not care about the simple gamification aspect of "I am certified" (which the workshop already had shown) but that they do care about their learning progress and learning results. The app could further embrace this.

  If there is one thing additional learned during the iteration, it is the insight that data is knowledge, and knowledge is powerful. A realization is that both the developer, the coaches, the teacher, and the project partners can gain important insights from the data. Adding "Are you sure?" to each quiz question, coach understanding was amplified, because now, also the coach's attitude towards learning can be evaluated. See more about this in the Discussion chapter \ref{cha:discussion}.

  \subsubsection{Next Iteration}
  To improve effectiveness for the next iteration, a couple of goals were chosen. While the app would work well for the Ugandan coach training, the use case of a youth session was not good enough yet. Mostly, this is in regards to that it takes too long time to improve via the app, and that the feedback is not sufficient. This leads to introducing these forthcoming goals, with the associated recommendations:

  \begin{itemize}
    \item Improve Deliberate practice. The criteria for Deliberate practice is not fulfilled today.
    \begin{itemize}
      \item  Follow the recommendation to design so that knowledge in a topic can go from unreliable to 95\% reliability within one to three 45-90-minute sessions.
      \item If this is not possible from changing the learning tactic, don't continue trying: split the quizzes into smaller pieces \citep{sierra}.
    \end{itemize}
    \item Improve Perceptual expose
    \begin{itemize}
      \item Divide the learner's expertise according to \cite{sierra}, "Can't do", "Can do with effort", and "Can do effortlessly".
    \end{itemize}
    \item Increase the use of questions to prompt self-monitoring  and self-evaluation \citep{sitzmann}.
    \begin{itemize}
      \item Using "learning by thinking" and encouraging a growth mindset, can benefit reaching metacognitive skills on Bloom's Revised Taxonomy.
      \item Help the coach to analyze and evaluate its own learning, possibly improving faster in the app.
    \end{itemize}
    \item Improve feedback to reflect that the teacher does not want to encourage coaches to have their youth session before they are 100\% confident with the material
    \item Data collection should be online and needs to be individual, so that the data is increasingly without faults and can be more easily analysed
    \begin{itemize}
      \item To do it online means that there needs to be a database, but also a login, so individuals are traceable.
    \end{itemize}
  \end{itemize}
