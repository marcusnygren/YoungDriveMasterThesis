\subsection{Conclusions}

  For iteration 3, the coaches could not only assess, but also \textit{learn} Correct Information, which was successful, but needs to be done more effectively.

  It took an unacceptable amount of time to reach 100\% proficiency on all the quizzes. This was especially evident, on the quiz on Correct Structure och Time Management, "Week 9: Are you ready?", when it took a coach 2.5 hours to reach 100\% without errors. In iteration 2, when "Improve" did not exist, it probably would have taken even longer.

  The focus had been on "I am well prepared", but also including "I am certified.". It was shown that most coaches does not care about "I am certified" (which the workshop already had shown), but that they do care about their learning progress and learning results.

  \subsubsection{Pedagogical model}
  There were four ideas originally, for the pedagogical model:

  \begin{enumerate}
  \item The coach result from Iteration 2: "Try again"-button. When clicked, your wrong answers are repeated.
  \item If 100\% on the 1st try, gold. On 2nd try: silver. On 3rd try: bronze.
  \item Ask meta-cognitive questions, e.g. "How sure are you?", at the end of each question.
  \item Record your answer to the question before you are shown alternatives.
  \end{enumerate}

  Option 1, 2 and 3 were determined good after the interviews, while item 4 had too many challenges (difficult to use, difficult to implement, cumbersome).

  \textit{To improve Deliberate practice: not satisfied.}
  Goal: design practice exercises that will take a fine-grained task from unreliable to 95\% reliability, within one to three 45-90-minute sessions.

  If not possible, don't continue trying: split into smaller tasks, Sierra says. This could be reinforced in various ways.

  To improve effectiveness, it was determined that "learning by thinking", regarding metacognitive skills, could be one of the most beneficial methods. This would help the coach to analyze and evaluate its own learning, possibly improving faster in the app. Using the theory from \cite{sitzmann}, questions can prompt self-monitoring and self-evaluation.

  \textit{Youth session}
  While there was now an MVP for the coach training, there was not yet a MVP for the youth session; only an MP (minimum product, but not viable yet).

  \subsubsection{Bonus results}
  The Kampala test showed how well the app works for learning entrepreneurship also outside of the YoungDrive context. Some modifications would greatly improve this further.

  \subsubsection{Data collection needs to be online}

  Data collection manually was not viable with coaches more than people, it got to hectic. To do it online means that there needs to be a database, but also a login, so individuals are traceable.

  \subsubsection{Collecting the data on Are you sure?}
  If there is one thing learned during the iteration, it is the notion of "data is knowledge, and knowledge is power". A realization is that both the developer, the coaches, the teacher, and the project partners can gain important insights. Adding "Are you sure?" to each quiz question, this was amplified, because now, also the coach's attitude can be evaluated. See more about this in the Discussion chapter. \todo{Add reference}

  \subsubsection{For the next iteration}
  \begin{itemize}
  \item Divide the learner's expertise according to \cite{sierra}, "Can't do", "Can do with effort", and "Can do effortlessly".
  \item Increase the use of questions to prompt self-monitoring and self-evaluation
  \item Implement login and a database in a suitable way, to store quiz results online
  \end{itemize}

