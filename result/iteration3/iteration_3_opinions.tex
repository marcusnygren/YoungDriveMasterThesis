%\subsection{Real-World App Tests}

  %These are results and observations from the real-world app tests done in Iteration 3.

  %What the observations said

  \subsection{Initial Evaluation of the New Version of the App}
  Before going to Kampala, because of the major changes to the app, the concept was tested with an entrepreneurship student in Kampala and the Zambia teacher, Josefina. The two tests informed that the app was now ready to be tested with the coaches in Tororo.

  The entrepreneurship student's overall opinion on the app was: "Can you give me the link, because I'd love to do more of this. I think it's amazing.". There were some issues found with phrasing: "Improve" should be renamed, because it is not intuitive what the button would do. The student was also surprised that the certification did not include something substantial (meaning it felt hollow). The student would have preferred unlocking a business challenge (showing self-determination), or something where he could get a learning reward instead of a "well done" and a badge (showing the student was not motivated by achievement in itself), see figure \ref{fig:iteration-map} I-3). This test was very valuable, and gave early insight to how the Uganda coaches might act within the app.

  The teacher in Zambia, Josefina, was consulted to comment on the app. When asked for an opinion, Josefina answered: "I like the idea that when the coaches have answered all of the questions correctly, they can consilidate the knowledge by the certification test, when the coach should get 100\% correct on their first try." This verified the relevancy of the taken approach of separating Training and Certification, see figure \ref{fig:iteration-map}.
