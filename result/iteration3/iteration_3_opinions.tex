\subsection{Real-World App Tests}

  These are results and observations from the real-world app tests done in Iteration 3.

  %What the observations said

  \subsubsection{App Test in Kampala}
  Before going to Kampala, because of the major changes to the app, the concept was tested with an entrepreneurship student in Kampala and the Zambia teacher, Josefina. The two tests informed that the app was now ready to be tested with the coaches in Tororo.

  The entrepreneurship student's overall opinion on the app was: "Can you give me the link, because I'd love to do more of this. I think it's amazing.". There were some issues found with phrasing: "Improve" should be renamed, because it is not intuitive what the button would do. The coach was also surprised that the certification did not include something substantial (meaning it felt hollow). The student would have preferred unlocking a business challenge (showing self-determination), or something where he could get a learning reward instead of a "well done" and a badge (showing the student was not motivated by achievement in itself), see figure \ref{fig:iteration-map} I-3). This test was very valuable, and gave early insight to how the Uganda coaches might act within the app.

  The teacher in Zambia, Josefina, was consulted to comment on the app. When asked for an opinion, Josefina answered: "I like the idea that when the coaches have answered all of the questions correctly, they can consilidate the knowledge by the certification test, when the coach should get 100\% correct on their first try." This verified the relevancy of the taken approach of separating Training and Certification.

  \subsubsection{App Tests during Field Visits}

  During field visits, 3 Community Bases Trainers were tested with the app, to observe usage of the app immediately after having prepared a youth session.

  Some things were notable from the interactions: %with John:
  \begin{itemize}
    \item "Are you sure?" is understood intuitively (you can't progress without answering), but some coaches deliberately answer "Yes" even if they are not sure.
    \item Idea to highlight different words of similar answers, to increase speed
    \item In summary, if wrong, show the other alternatives either way, not only the wrong answer
    \item Idea for future work: "Go to participant manual" within the app % Juliet, this was discovered:
    \item If correct and unsure, she says "I still feel good". "Include it in wrong, because maybe I was still guessing".
    \item Change button to "Become certified", to increase likelihood to press the button. As of now, it was not obvious.
  \end{itemize}

  When she did get certified, she said "I feel good". When asked why, she said: "They have appreciated what I have done". The next day, the same three CBTs gathered at the Plan International office to do an app test on the hardest quiz.

  Having a service mini-sprint after the field visit, quick iterations could be made to the app. One such example comes from the field visits. Originally, it was believed best to use Gold/Silver/Bronze in the Training mode, and "Are you sure?" in the Certification mode. User tests showed that the other way around was better, and this was changed for the next meeting with the coaches. This example shows the relevancy of testing the app with the intended users, as it had not been evident from the tests with the Kampala student or the teacher.

  A service design approach was used, first observing how preparations was made without the app, and \textit{then} introducing assessment via the app, followed by interview. What was the most valuable feedback from the field visits, was to see that the app had indeed been a perfect fit for use in the field before a youth session. However, it was not possible due to time limitations to follow the coach to their youth session afterwards, to see the actual effect of preparing via the app.
