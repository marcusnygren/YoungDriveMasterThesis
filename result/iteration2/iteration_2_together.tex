\subsection{Summary}

The app works for assessing Correct Information! Since the coach training app was said to most importantly test Correct Information, secondly Correct Structure and Time Management, the iteration was considered successful.

Regarding the workshop \# 2, for iteration 2 the focus had been to assess "I am well prepared", for the very purpose of building confidence, in regards to Correct Information.

The review from Josefina was: "The (YoungDrive coach training) app is a great tool to measure how much the coaches learned and understood from the daily training; it provides a clear overview of what the coaches truly understood and what they actually still don’t completely understand. Based on that information I as a tutor can adjust the training for the following day to make sure that the coaches understand everything correctly. The app also works as a motivator for the coaches; it´s clearly reflect their own daily performances. If they score high they become very happy and satisfied, if they score low they are eager to check their wrong answers.".

\subsubsection{Insights}

  \begin{itemize}
    \item The app works for assessment!
    \item Good for learning for the coaches
    \item A good indicator for Josefina
    \item A great way to scale the YoungDrive training in the future, both for online coach-training and the physical training
  \end{itemize}

After the meeting with the partner and expert group, the following was concluded from iteration \#2:

\begin{itemize}
\item The app is only working on assessment now, not for learning
\item The need for a field app still feels relevant (especially for sessions long since the coach training)
\item The potential for YoungDrive having online coach training is huge
\item Multiple-choice is flawed in its current form
\end{itemize}

The insights on learning needed to be considered:
\begin{itemize}
  \item Are coaches really learning via the app, especially learning to be better coaches?
  \begin{itemize}
    \item How can questions be formulated in a way that teaches entrepreneurship, which is so practical?
  \end{itemize}
  \item How can the current multiple-choice quiz app be improved, to:
  \begin{itemize}
  \item reduce guessing
  \item improve confidence
  \item encourage learning
  \end{itemize}
\end{itemize}

Discussing the importance of self-reflection after a youth session with Josefina, led to asking more of such questions in coach quizzes.

Josefina: “I have a problem: there is no way I can control them how they have prepared themselves for a youth session."

An app could be used, either before you start planning (to guide what you need to study the most on), or after you think you are ready (so you can assess and improve).

Focus for the next iteration:
\begin{itemize}
  \item Score higher on Bloom's revised taxonomy, while still including multiple-choice questions in the app.
  \item Design quiz app for learning, focus on field app (CI, CS, TM, FA), and design having an app that works stand-alone from the YD coach training in mind.
  \item Try the Power of Yet approach in the app ("growth mindset" approach of "Not yet", versus fixed mindset and assessment)
  \item Test if the app created in Zambia could work also in Uganda
  \item All the quiz questions would need to be converted from the new (Zambia) manual to the old (Uganda) manual, since both structure and content had changed.
  \item Josefina was given a task to create a quiz "Are you ready for Session 9?".
  \begin{itemize}
    \item partly to score higher on \textit{Bloom's revised taxonomy}
    \item partly to test if Correct Structure and Time Management could be assessed using multiple-choice
  \end{itemize}
\end{itemize}

\subsubsection{Findings}

Test with university student scored 100\% correct, means that common sense can go a long way, and that the results can't be 100\% trustworthy, and that multiple-choice questions has serious issues - this, we already knew during and before the coach training - but it needs to be taken care of.

The app would be great and could actually work outside the physical coach training - with revision, be stand-alone, even being able to distribute online.

Now there are observable evidence for what the interactions from Iteration 1 showed:

\begin{itemize}
\item The purpose of the coach training should be to prepare the coach in having great youth sessions
\item Therefore, this is what the quizzess should assess
\item What it really means being a good YoungDrive coach, is having good youth sessions
\item Josefina would have liked to be able to stop coaches from having taught, if they do not have 90-100 \% correct information on the subject
\item Today, Josefina can not assess this. This means that some coaches, are teaching incorrect information to hundreds of youth.
\item Here, the quiz has a very good need to fill.
\end{itemize}

With all of these findings in summary, it can be concluded that an app for coach training, and an app to use before a youth session, could be the same app, since the purpose of preparing the coach to be great with its youth session is the same.

From the interviews, it was learned that while it \textit{may} be technically possible, the teacher desires the app support her, not replace her.

To get an app suitable for learning, it was determined that the pedagogical model behind the app needed to change, emphasising feedback.
