\subsection{Qualitative data}

\subsubsection{Lessons learned from app test observations: motivation}
Regarding motivation from the coaches, one coach wished the app to be available on the Google Play store immediately, so that "The app could be used on my spare time". Another coach, without a smartphone, said "I'll buy one", because the utility of the app seemed so high. There were also suggestions for improvements, like "The app should have notes, not only quesitions". Regarding usability, low resolution screens made the text be barely visible. This showed, that the app needed to be tested on a lot of different devices. This is particularly true, as on day 1, the coaches did not know how to zoom, which could cause accident refreshs, frustration or confusion. Even more importantly, the app needs to work offline! To be online on the phone is too expensive for the coaches, and too unrealiable to give a satisfactory experience. Also, during testing, relying on internet can cause a lot of problems, especially if the teacher is alone.

When asked about what they thought about doing one quiz ("Graduation") as a pre-quiz (before the session), 10/10 said they liked doing the quiz before, and that it benefited their learning during the session. When asked why it helped, 10/10 said agreed on the statement "During session, it is easier to follow" and that "Giving the paper manuals before, scanning headings and pictures etc, would not help". So, using the quiz before the session increases learning, slightly decreases fun of the session, according to coaches. One of the coaches described it as "Fun and encouraging".

    It was also tested to work in group or individually. The ones who answered, said that you learned more individually (3/3), and more fun doing it together (3/3). Doing it together, was enjoyable as it was "Very easy because of using different minds" and "We can collaborate to do better". It can be argued that the quiz being easier is not a valid motivation, but describes the learning in the app as a desirable difficulty.

    DWhen doing a post-quiz ("Goal setting") immediately afterwards versus at the end of the day (doing spaced versus massed learning), quotes were "I thought it was fun and challenging to do the quiz immediately afterwards", with another coach commenting "The mind was still fresh". After a discussion with the teacher, these were the results:

    %When asked on timing preference, 10/10 said it would be more fun to do the quiz immediately afterwards, not at the end of the day. The motivation, seemed to be that it was easier.

    %9/10, said they wanted to do the quiz afterwards. The outlier, said it would be better for learning doing it later.

    %After this comment, this was the distribution:

    \begin{itemize}
    \item 3/10 wants to do the quiz both before and after a session
    \item 1/10 wants to do the quiz before and at the end of the day
    \item 7/10 wants to do the quiz only immediately afterwards
    \item 10/10 wants to do the quiz immediately afterwards, and then again at the end of the day
    \item 7/10 wants to do the quiz immediately afterwards, and then a joined quiz with other topics at the end of the day
    \end{itemize}

    The high scores on using the app a lot indicates that they like the app. The teacher wants to listen to coach opinions, at the same time not spending more time than necessary on assessment.

Regarding motivation from the teacher, asking Josefina what would hinder her from using the app, she says: "Not doing data collection digitally works whenever they are 10 - but not with bigger numbers than that." Also, according to the final interview with Josefina, she does not wish the app to replace her. She enjoys teaching, thinks she has an important role, and suggests the app to be designed to support her and the coaches, not replace her. She accnolodges that bugs in the app was a hindrance to functionality, and that a lot of testing (both high-dose, and high-scale) is very important.

\subsubsection{Lessons learned from app test observations: learning}
Regarding assessing knowledge, coaches had surprisingly high quiz results, and at day 3 they wished harder questions when asked. The response was to give harder questions the other days, for example by introducing similar answers, and testing 4 alternatives instead of 3. This was appreciated. The app was later tested on a university student in Uganda after the Zambia training, both on early and later quizzes. The university student from Makarere University scored 100\% correct, in spite of not having any entrepreneurship training. This showed that guessing was possible, or that the quizzes were too easy.

The teacher Josefina commented that this might not be a problem, as the YoungDrive coaches are not as skilled with using a process of elimination, and had indeed scored lower results on average with the later quizzes. When testing the app with refugee innovators during Humanitarian Innovation Jam in Uganda, similar results to the YoungDrive coaches were found. She explains this by that the cultural context is different, and that thanks to coaches in rural areas not being equally educated and skilled with reasoning, the problem is not as big as could be. Josefina is very happy with the app, and reviewed the app in the following way to Plan International after the training:

"The (YoungDrive coach training) app is a great tool to measure how much the coaches learned and understood from the daily training; it provides a clear overview of what the coaches truly understood and what they actually still don’t completely understand. Based on that information I as a tutor can adjust the training for the following day to make sure that the coaches understand everything correctly. The app also works as a motivator for the coaches; it´s clearly reflect their own daily performances. If they score high they become very happy and satisfied, if they score low they are eager to check their wrong answers.".

A coach scoring only 9/19 showed the relevancy of the quiz, as Josefina did not think she would have discovered that the coach was lagging behind otherwise. In the data, it was observable that the coach had done well together with others, but 3/7 when done individually. Josefina said about the 9/19: "This is where a control group would be beneficial". "He is often passive during open questions, but active during the team exercises."

According to Josefina, if you have a high score, you are ready. If not, you need to redo the quiz. If you are 8/10 or lower, you are in the red zone. If lower than 10/10, they are not ready, the motivation being that what they don't know, they will teach in an improper way: affecting hundreds of youth. This is why Josefina thinks they should need all of the answers correct.

Up until now, merely Correct Information has been assessed, not the other three factors. The fact that the app already is appreciated with assessing Correct Information, makes starting to assess the other factors interesting. Josefina informs that Correct Structure, Time Management and Fun Atmosphere would be the most viable to test \textit{after} a youth session, not before. She notes, that \textit{some} aseessment could be made via the app before a session. This could to be further investigated.
