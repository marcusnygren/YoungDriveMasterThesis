\subsection{Quantative data}

    %What the data said

    Two workshops were held, which together would inform the future development of an application.

    There were the findings from those two workshops:

    \subsubsection{Workshop \#1: Customer Journey Map: A day as a coach}

    After lunch, I held two workshops with the coaches. The first one continued from where the interviews, “A day as a coach”, using a customized Customer Journey Map. First, three personas were created based on the interviews: an “ideal coach”, “realistic coach”, and “poor coach” were named. I had created a timeline with “Before”, “During [the youth sesssion]”, and “After”, and each post-it color represented one persona: John, Joan, and Suzan. They understood the timeline and personas they created very effortlessly.

    %After the 15 minute introduction, they started with 5 minutes + 10 minutes discussion mapping out Before, During, After for the ideal coach, John. Green postits were used.

    %Since it took too much time to do this for the other two personas, they were given 5 minutes to either use pink post-its for steps that a realistic coach would skip, and yellow notes that the coach would do differently. This worked well, and the results were discussed within the 5 minutes between the coaches, and explained during 5 minutes witth audio recording.

    %Since they were understandingly tired, they were given a 10 minute break, during which time they were asked to think about things that could go wrong for the sad/angry persona, Suzan. When I got back from the toilet, they had already started working! I took time of 5 minutes, and they walked through the concerns and it’s effects, just like they had did with the 2nd persona.

    The first workshop was finished with many important insights. \todo{Add image of CJM and the insights}

    \subsubsection{Workshop \#2: Quizical and Duolingo}

    Quizoid and Duolingo were tested to understand the technical possibilites of the coaches.

    The result was that the app can place itself somewhere in the middle of the two, regarding difficulty level.

    Patrick [från YoungDrive] undrade om han kunde låna en smartphone under tiden jag var här.

    Efter workshopen, berättade han:

    “Vet du vad Marcus, idag har en av mina drömmar gått i uppfyllelse.”
    Det var första gången han använde en smartphone

    Even for coaches that had never touched a smartphone before, some concepts were easily understood (like using the camera and Quizical).

    Other concepts were harder (e.g. accidently getting to the settings menu, unlocking the device, understanding Cut the Rope 2, or training languages using Duolingo with advanced interactions). Point and click is easily understood, whereas sliding is much more unnatural.

    Also, it’s important that the app is fail-safe - but how do I avoid errors with the Android OS and iOS? A lot of training is needed to avoid errors, or I need to find another solution.
