\subsection{Conclusions}

\subsubsection{Motivation of app}
The scope of the app is to examine and strengthen the entrepreneurship the student already has. One important goal is to give good feedback.

\textbf{Assessment of knowledge and skills is today mostly based on how good they feel can answer the youth’s question, and how the audience reacts}
Is this a good way of assessing? I see two problems. First, the feedback comes only after a session is carried out. Second, it is a very subjective approach. Third, this feedback is not sent to Patrick and Christine, unless they’re visiting.

\textbf{Technology could increase accuracy and accountability}
Most coaches plan their next session during the morning, or immediately after a session with their group. Since a coach has somewhere between 7-10 groups (some even more), and the youth groups are at different modules, there is a lot of knowledge for the coach to handle - not only theoretical knowledge, but also the struggles of the youth, assignment presentations, workshops to be facilitated, etc.

It is easy for a coach not to do everything as planned or as specified in the manual. By an app, they could keep a record of the module content, and see when and if they do need to refresh their skills.

\textbf{The users are already motivated to become a better coach.} Thus, I can follow Sierra's advise designing for the compelling context.

My compelling context is that I want to help you become an even better coach.

The better user point of view: don’t just make a better coach training app - make a better user of coach training material.

For me, this means:

"Given a teaching situation among the youth group, a great coach can teach an entrepreneurship topic more consistent with what the coach material said."

"Given a question in the app, a great coach will get the right answer more often, and increasingly leverage the correct answer to their coach situation."

\subsubsection{Insights}

\textbf{What's it like being a coach?} \todo{Översätt till svenska}
I iteration 1 fanns ingen digital ansats alls. Jag var i Tororo för att besvara "What's it like being a coach?". Upptäckte att vad det innebär att vara en bra YoungDrive-coach, är att kunna ha bra ungdoms-sessioner. För att ha bra ungdoms-sessioner, är din självkänsla och självförtroende enormt viktigt. Och det är inte alla coacher som har detta, och därför skiljer sig kvaliteten mycket, vilket Josefina upplever som en utmaning.

Jag började leta efter hur och var en coach-app kan underlätta. En aktivitet som alla coacher har gemensamt för lärande och avgörande för coachens framgång, är (1) coach-träningen (som jag redan visste var viktig), men framför allt (2) förberedelserna av en ungdomssession. Jag övertygade Josefina att vi skulle ha ett mycket fokus på (2) än hon tänkt. Medan Josefina kan vara inblandad i (1), kan en app vara extremt viktig i (2), upptäckte jag under mina fält-besök på ungdomssessioner och intervjuer med coacher och projektledare.

Enligt Iteration 1: Självförtroende = empowerment
Enligt iteration 1 kom självförtroende ifrån att under ungdomstillfället kunna ha: Correct Information, Correct Structure, Time Management, och Fun Atmosphere. Det är alltså detta appen borde testa och träna.

For iteration 2, there were two main insights to consider from iteration \#1:

\begin{itemize}
\item The aim is for the coach to feel self-confidence for its youth session
\item The skill to be trained is having a youth session
\end{itemize}

During the evaluation meeting with Linköping University and YoungDrive, it was the determined that Iteration \#1, provided answers for research question \#1, \#2, and \#3.

The iteration had provided a good basis for answering research question \#5.
