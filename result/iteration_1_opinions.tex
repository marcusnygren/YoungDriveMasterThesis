\subsection{Qualitative data}

\subsubsection{Entrepreneurship education considerations}
Throught early interviews with YoungDrive staff, it is clear that YoungDrive's entrepreneurship education methodology goes hand in hand with the presented theory. It's mottos are: "Dream big, start small", "Learning by doing" and "We have fun!" \citep{youngdrive}.

Both in regards to designing for the users and for the above reason, the app should be a complement to YoundDrive's existing training material and the structure of the program.

A challenging part of the work is that YoungDrive consists of both the practical skills of the entrepreneur, theoretical material of running a business, and an entrepreneurial mindset. Therefore, both how to assess knowledge, and build habits, needs to be examined.

\subsubsection{Understanding the coach situation}

A CBT can be responsible for from 7 up to 12 different youth groups in different programs, and such a high number places huge demands on the CBT.

Even if there are only 7 groups, being behind on schedule or not being confident, can be very demanding.

\textbf{Stayover at Patrick:} På morgonen visade Patrick mig hur han jobbar med deras tomt, var det odlas ris, och andra råvaror, deras djur, deras story från Syd-Sudan, till Kampala, till hyddan här i Tororo, och hans värderingar.

Efter en ungdomssession nästföljande dag besökte vi och hälsade kort på en av de 2 CBT:s vi har session med idag. Sedan hade jag och Patrick den obligatoriska review av ungdoms-sessionen, och jag bjöd honom på middag. Kl. 19 ringde hans fru (som har börjat få tecken av malaria) och skyndade hem.

Nästa ungdomssession fick jag besöka en annan CBT. Vi var tidiga. Sedan började jag prata med henne, och fick bra tillfälle att intervjua henne och även förklara för henne vad jag gjorde där. Det blev underligt att förklara för henne: Patrick påminde, när jag tabbat mig, att “Marcus, du måste förklara för X vad en app är”. Så hon fick låna min mobil, och jag förklarade att app var kort för applikation, och att för varje applikation har ett eget syfte, t.ex. ta foton. Jag bad henne klicka, svårt, råkade klicka åt henne, men sedan lät jag henne göra. Efter hon sett att det hon såg i skärmen var det hon såg på riktigt, blev hon jätteglad och började fundera vad hon skulle fota. Hon reste sig och gick runt hörnet, och jag följde efter. Hon fotade, efter att noggrannt tänkt igenom, att hon fotade buskarna med frukt. Sedan sade jag hon kunde fortsätta fota, och då tog hon ett litet runt hus utanför.


\subsubsection{Different kind of coaches}

The interviews with CBTs, PLs and stakeholders led to the realization that different coaches handles this differently well.

Depending on the situation, e.g. you are not confident, or you are falling behind with the schedule, you can be in one of these need groups.

\begin{itemize}
  \item The ideal coach
  \item The realistic coach
  \item The challenged coach
\end{itemize}

It was discovered, that coach confidence comes largely from being able to have good youth sessions.

\subsubsection{A good youth session}

For having a good youth session, these are the most important attributes:

\begin{itemize}
  \item Correct information
  \item Correct structure
  \item Time management
  \item Fun atmosphere
\end{itemize}

The fact that the coach knows they have these qualities, leads to self confidence from the coach. This in turn, leads to better meetings with the youth.

\subsubsection{The room for a digital solution}

It is definitely a problem that so extremely few have smartphones.

This needs to be designed for. Either, I build only for the use case of having an app tailored for the coach training, where the donated devices can be used.

Alternatively, I design only for the users that does have a smartphone, and count that more will get smartphones in the future.

Thirdly, I can use a SMS tool, not building an app but an SMS-based service, which also could be an app. Such tools exists, and are compatible with multiple-choice questions, like VOTO Mobile.
