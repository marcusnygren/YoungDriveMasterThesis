\subsection{Iteration \#1}

\subsubsection{Understanding the coach situation}

A CBT can be responsible for from 7 up to 12 different youth groups in different programs, and such a high number places huge demands on the CBT.

Even if there are only 7 groups, being behind on schedule or not being confident, can be very demanding.

\subsubsection{Different kind of coaches}

The interviews with CBTs, PLs and stakeholders led to the realization that different coaches handles this differently well.

Depending on the situation, e.g. you are not confident, or you are falling behind with the schedule, you can be in one of these need groups.

\begin{itemize}
  \item The ideal coach
  \item The realistic coach
  \item The challenged coach
\end{itemize}

It was discovered, that coach confidence comes largely from being able to have good youth sessions.

\subsubsection{A good youth session}

For having a good youth session, these are the most important attributes:

\begin{itemize}
  \item Correct information
  \item Correct structure
  \item Time management
  \item Fun atmosphere
\end{itemize}

The fact that the coach knows they have these qualities, leads to self confidence from the coach. This in turn, leads to better meetings with the youth.

\subsubsection{The room for a digital solution}

It is definitely a problem that so extremely few have smartphones.

This needs to be designed for. Either, I build only for the use case of having an app tailored for the coach training, where the donated devices can be used.

Alternatively, I design only for the users that does have a smartphone, and count that more will get smartphones in the future.

Thirdly, I can use a SMS tool, not building an app but an SMS-based service, which also could be an app. Such tools exists, and are compatible with multiple-choice questions, like VOTO Mobile.
