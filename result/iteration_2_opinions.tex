\subsection{Iteration \#2}

\subsubsection{The desire perspective}

Insight: "The app could be used on my spare time". This is particularly true, about the bonus quizzes that were produced during the week.

Insight: \textit{Coach:} "I'll buy one" (a smartphone), \textit{Response from other coaches: } "Whoa!"

\subsubsection{The utility perspective}

Ideation: "The app should have notes, not only quesitions"

\subsubsection{The usability perspective}

Low resolution screens, made the text be barely visible. This showed, that the app needed to be tested on a lot of different devices. This is particularly true, as on day 1, the coaches did not know how to zoom, which could cause accident refreshs, frustration or confusion.

The app needs to work offline! To be online on the phone is too expensive for the coaches, and too unrealiable to give a satisfactory experience. Also, during testing, relying on internet can cause a lot of problems, especially if the teacher is alone.

\subsubsection{The learning perspective}

    The coaches had surprisingly high results, and at day 3 they wished harder questions when asked.

    We responded by having harder questions, e.g. by introducing similar answers, and testing 4 alternatives instead of 3. This was appreciated.

    The following analysis could be done:

    \subsubsection{Doing a pre-quiz: good for learning}
    When asked about what they thought about doing "Graduation" as a pre-quiz (before the session), 10/10 said they liked doing the quiz before, and that it benefited their learning during the session.

    When asked why it helped, these were the results:

    \begin{itemize}
    \item "During session, it is easier to follow" - 10 (100\%)
    \item "Giving the paper manuals before, scanning headings and pictures etc, would not help" - 10 (100\%)
    \end{itemize}

    It was also tested to work in group or individually. The ones who answered, said that you learned more individually (3/3), and more fun doing it together (3/3). Doing it together, was enjoyable as it was "Very easy because of using different minds" and "We can collaborate to do better".

    \subsubsection{Doing a post-quiz: Spaced versus massed learning}
    In "Goal setting", quotes were "I thought it was fun and challenging to do the quiz immediately afterwards", with another coach commenting "The mind was still fresh".

    When asked on timing preference, 10/10 said it would be more fun to do the quiz immediately afterwards, not at the end of the day. The motivation, seemed to be that it was easier.

    9/10, said they wanted to do the quiz afterwards. The outlier, said it would be better for learning doing it later.

    After this comment, this was the distribution:

    \begin{itemize}
    \item 3/10 wants to do the quiz both before and after a session
    \item 1/10 wants to do the quiz before and at the end of the day
    \item 7/10 wants to do the quiz only immediately afterwards
    \item 10/10 wants to do the quiz immediately afterwards, and then again at the end of the day
    \item 7/10 wants to do the quiz immediately afterwards, and then a joined quiz with other topics at the end of the day
    \end{itemize}

\subsubsection{The teacher perspective}

  \textbf{Low scores}
  The 9/19 shows the relevancy of the quiz, as Josefina did not think she would have discovered that the coach was lagging behind otherwise.

  In the data, it was observable that the coach had done well together with others, but 3/7 when done individually.

  Josefina said about the 9/19: "This is where a control group would be beneficial". "He is often passive during open questions, but active during the team exercises."

  The question we needed to ask ourselves, was: "Does this imply he is a good or bad coach?".

  \textbf{What would hinder Josefina from using the app}

  Josefina says: "Not doing data collection digitally works whenever they are 10 - but not with bigger numbers than that."

  According to the final interview with Josefina, she does not wish the app to replace her. She enjoys teaching, thinks she has an important role, and suggests the app to be designed to support her and the coaches, not replace her.

  \textbf{Acceptance criteria}

  If you have a high score, you are ready. If not, you need to redo the quiz.

  If you are 8/10 or lower, you are in the red zone. If lower than 10/10, they are not ready, the motivation being that what they don't know, they will teach in an improper way: affecting hundreds of youth. This is why Josefina thinks they should need all of the answers correct.

\subsubsection{The developer perspective}

  \begin{itemize}
    \item Bugs was a big hindrance to functionality, and a lot (both high-dose, and high-scale) of testing is very important
    \item Simpler design than I thought (KISS) was sufficient
  \end{itemize}
