
  For iteration 3, thanks to recording both if they were correct and confident, the app can give very precise learning feedback (e.g. showing that the coach answered alternative B with confidence, but showing that A was actually the correct alternative). After giving feedback, the coach can train, and improve on incorrect answers and guesses, and when being 100\% correct and confident, ideally take the whole quiz without faults.

  \subsubsection{Result}

  In iteration 3-4, the coach answers "Are you sure?" (yes/no) for each question.

  In the score board, they can read their answers, the correct ones, and how sure they were of their answers.

  After observing their incorrect ones, and learning the correct answers, they try again.

  Number of quiz tries is shown to the coach sduring the training to the coach for self-evaluation, but it is not a performance measure.

  %Lena Tibell menade vid förslaget att "Belöna inte hur snabbt %en elev går från att kunna till att inte kunna, för olika %människor lär sig olika snabbt". "Vad vi ville åstadkomma %med Antal försök var endast att undvika gissningarna".

  At 100\% correct answers, they can get certified, by getting the whole quiz correct without faults. If they do anything wrong, they are put back into training mode, or select another quiz.

  In Certification, the coach is encouraged to get 100\% on their first, second or third try (recieving Gold, Silver or Bronze), to reward the coach for truly having learned the questions, and not provided guesses. They are then encouraged to select a new topic if they want to.

  \subsubsection{ScoreBoard} \todo{Översätt till engelska}
  For iteration 4, it is realized that the combination between correct/incorrect and sure/not sure can be used for empowerment.

  \begin{itemize}
  \item rätt svar + säker = pluspoäng
  \item rätt svar + osäker = du gissade men hade rätt, var säker nästa gång
  \item fel svar + säker = måste ge feedback (väldigt intressant för Josefina)
  \item fel svar + osäker = du ahde rätt, det var ett annat alternativ, gör om
  \end{itemize}

  \subsubsection{Pedagogisk modell}
  \todo{Översätt till engelska}
  Träningsläge: ta ett quiz, med "Are you sure?". Baserat på svar, låt frågor hamna i tre olika lådor: "Can't do", "Can do with effort" och "Can do effortlessly". Genom att fel frågor (Can't do) igen, och reflektera över Correct + Unsure, kommer till slut alla frågor vara på "Can do effortlessly". Då låser du upp "Certification".

Om du har fel på en "Can do effortlessly"-fråga, flyttas du tilllbaka till träningen, och får igen omfördela "Can't do" eller "Can do with effort" och "Can do effortlessly".

Frågor i "Certified", är frågor som coachen befäst genom att upprepat korrekt från "Can do effortlessly". Så coachen blir Certified i ett helt quiz, när den har rätt på alla frågor i Certification test. Då är den klar, och redo att lära ut ämnet!

Men coachen kan också välja att lämna quizet när som helst, och komma tillbaka i ett senare tillfälle. Detta handlar om att coachen ska kunna bestämma själv hur den vill bli bättre.

Målet är alltså att i coachens egna tempo, flytta över frågor från "Can't do" till "Can do effortlessly" till "Certified". Så planerar jag bygga expertis som YoungDrive-coach.
