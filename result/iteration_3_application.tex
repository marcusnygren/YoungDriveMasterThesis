\subsection{Iteration 3}

  \subsubsection{Result}
  Quiz-flöde 2.0: designat för learning och självreflektion, men ej för effektivitet
  vid varje fråga besvarar du det alternativ du tror är rätt samt "Are you sure?" Yes/No
  vid färdigt quiz, få en resultattavla med personliserad feedback
  läsa igenom dina felaktiga svar och hur säker du varit på dem
  observerat de korrekta svaren
  klicka "Improve" för att bara få dina felaktiga svar igen
  upprepa tills inga felaktiga svar var kvar (det står "quiz try: 3", om det är försök 3)
  vid 100%, låsa upp "I can get 100%" (som är hela quizet igen)
  innan dess, uppmuntrades du läsa igenom coach/deltagar-manualen
  om du då fick något fel, fick du gå tillbaka till träning igen
  om alla rätt på, blev du Certified coach. Om du klarade det på första försöket, fick du även en guldstjärna (andra försöket = silver, tredje försöket = brons)
  sedan kunde du ta ett annat quiz
  Kommentar, fördelar med feedback-läge:
  Genom att på varje fråga besvara "Are you sure?": Yes/No, så stärker vi inte bara coachens meta-kognitiva förmåga, utan vi kan vi även ge personliserad feedback i resultattavlan, istället för att bara visa Question 1: 1 point. Question 2: 0 points, som i Iteration 2.

  Detta gör att coachen kan reflektera över sitt lärande på t.ex. följande sätt:
  - få en självförtroende-boost (via feedback "You were correct, and you were sure")
  - gå från gissning till självsäkerhet (via feedback "You guessed, but you were correct")
  - ändra uppfattning snabbare (via feedback "You were incorrect, but you were sure")
  - uppmuntra coachen att läsa i manualen (via feedback "You were incorrect, and you were not sure")

  Fördelar med tränings-läge, och certifikations-läge:
  Jag gillar idén att när coachen har kunnat svara rätt på alla frågor, kunna befästa kunskapen med hjälp av certifikations-läget, då coachen ska kunna få 100\% rätt på 1 försök.
