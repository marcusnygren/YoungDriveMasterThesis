\subsection{Discussion}

During a evaluation meeting with Linköping University and YoungDrive, it was determined that Iteration \#1 did provided answers for research question \#1, \#2, and \#3, and partly \#5. Thus, the iteration could be considered very successful, and now, the development of the app could begin.

It is clear from the data that the motivation of the app should be to assess and strengthen the entrepreneurship knowledge and skills of the coach. For coach quality to improve was a desire from the stakeholders as well as the coaches themselves, even if they were also satisfied with the current results. This lead to a challenging situation how an app can address becoming a better-performing coach.

An app could increase accuracy of correct information. With an app, the coach could keep a record of the module content, and see when and if they do need to refresh their skills. It was discovered that a coach app can benefit not only the coach training, but also in a surprisingly precise way, what was called "distance learning" in section \ref{purpose}. Accountability if the coach is ready for a session by automatic assessment. A very important aspect to increase learning and confidence will be to give good feedback (see section \ref{learning-assessment}).

With all the possible benefits of an app, it is definitely a problem that so few coaches have smartphones. Either continued development could be guided solely by the use case of having an app tailored for the coach training (where donated devices are available). But this would be to ignore that an app helping the coach to prepare for a session would be extremely beneficial, which discovered during the field visits to youth sessions and during interviews.

The motivation for using technology is very high, so one way forward would be that the app for distance training will reach only the users that can be given access to a smartphone, counting that more coaches will get smartphones in the future. Not using smartphones but feature phones (which all coaches possess), would mean building an SMS-based service (see \ref{rq1}.

As most coaches are already motivated to become a better coach and using technology, the advise from \cite{sierra} of designing for their compelling context can be followed. From a YoungDrive perspective, this might mean "Given a teaching situation among the youth group, a great coach can teach an entrepreneurship topic more consistent with what the coach material said". Their performance in the YoungDrive app could translate into: "Given a question in the app, a great coach will get the right answer more often, and increasingly leverage the correct answer to their coach situation".

\subsection{Next Iteration}
It is agreed with Josefina that the most important found skill of a YoungDrive coach is having great youth sessions. It is a challenge that the coach surely needs to feel, but does not always possess, self-confidence for its youth session. This partly stems from the lack of practical experience being put into realistic situations during the coach training.

If self-confidence comes from being able to deliver Correct Information, Correct Structure, Time Management and Fun Atmosphere, an app strengthening these will surely improve youth session quality. According to Josefina, assessing and increasing Correct Information is the parameter she values the most highly, and this will be the continued primary focus of the master thesis.

It is agreed with Josefina that preparing for a youth session can have an increased focus. It is a worry that designing for both the coach training and preparing for a session might be too ambitious within the given time frame. If so, designing for the coach training is deemed more important.
