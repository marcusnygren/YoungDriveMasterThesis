\subsection{Iteration \#3}

  %What the observations said

  \subsubsection{Observations from Kampala test}
  The entrepreneurship student in Kampala, informed the following changes:

  Instead of "Become certified", he would be more motivated by unlocking the opportunity to apply the skills.

  "Improve" should be renamed "Try again", because it is more intuitive.

  His overall opinion on the app was:

  "Can you give me the link, because I'd love to do more of this. I think it's amazing."

  \subsubsection{Field visits}

  \subsubsection{Big app test}
  \textbf{Learning: } The app test simulated the app being used to assess preparedness for a youth session. They clearly showed evidence between the difference between designing for Assessment and Learning:

  Given a coach having prepared for their youth session on week 9, and then only scoring 5/10, what should happen? In a similar way, what should happen if 9/10 correct answers?

  For the coach training, the assessment was okay, since Josefina could pick up and give feedback.

  Before a youth session, leaving the coach there is not viable. If the coach has 9/10, that coach should not only be let be, and especially if the score has been 5/10.

  Feedback was that one user did not want to press "Improve", until having read the manual. The motivation was: "Not because that is what the info says, but because I can learn more from the manual, about more than what the questions says."

  This is indeed the preferred behaviour from Josefina, and the app should continue to encourage only using the app training or certification mode after having prepared via the manual. This way, the app is still assessment, but it is "learning by thinking", with feedback.
