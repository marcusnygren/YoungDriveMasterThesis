\subsection{Iteration 2}

The interactions for this iteration were planned to be in Tororo. However, during a meeting during the first week with YoungDrive project leader Josefina, I was invited to participate in the coach training in Zambia. A new work plan was created, so that I could travel to Zambia and develop the app and participate in the YoungDrive coach training together with the coaches.

\subsubsection*{Insights}

There were two main insights to consider from iteration \#1:

\begin{itemize}
\item The aim is for the coach to feel self-confidence for its youth session
\item The skill to be trained is having a youth session
\end{itemize}

During the evaluation meeting with Linköping University and YoungDrive, it was the determined that Iteration \#1, is answering the research question \#1, \#2, and \#3.

The iteration had provided a good basis for answering research question \#5.

It was concluded during the partner meeting, that iteration \#2 should:

\begin{itemize}
\item Allow to test the validity of the insights from iteration \#1.
\item Be carried out in a way that allows comparison of usability and learning done via the app, between iteration \#2 and \#3.
\end{itemize}

\subsubsection*{Ideation}

This was the start of the quiz app. The focus was on assessment. For example, it was decided with Iliana, that no facts would be presented before the quiz. The app would solely ask questions, not give the information beforehand. For that purpose, previous knowledge or the physical manuals could be consulted, meaning, it was not a top priority.

It was discussed, how the correct information about YoungDrive would be presented:

It was determined that questions would be created by YoungDrive, using the thesis' relevant theory and recommendations on designing for learning.

Thus, the ideation started with me creating a guide how to write questions according to Bloom's revised taxonomy, which was shared to the YoungDrive team.

The initial plan was that the team would only produce questions for two sessions, not all 10.

Iliana Björling from YoungDrive did questions initially for the two sessions, mapping each question to the Bloom taxonomy using the guide. Then, it was decided that the app would be developed and used during an actual coach training in Zambia, for each day.

\subsubsection*{Trigger material}
Project leader Josefina in Zambia refined the first question sets, prepared for my visit in Zambia. Josefina created question sets with Bloom at the back of her head, also taking into account the structure and the order of the coach manuals, what it means being a coach within the topic, and lastly scenarios.

A hi-fi trigger material was done, a very basic quiz app, keeping it as simple as possible (see Application Implementation, Iteration 2). All of the devices (tablets and smartphones) that I had available were brought to Zambia.

I added the questions to the app, and installed the app to all of the devices. This process was repeated for all the days, Sunday-Friday.

\subsubsection*{Interactions}

\subsubsection{Design workshop \#1}
The coach training started with me having a design workshop with the coaches, not showing them the app that I had created.

Since the knowledge about smartphones and apps were low, I started by introducing these topics.

All were familiar with Facebook, so thus I showed the Facebook app. Me wanting to know what the app would look like if the coaches would have designed the app, I first needed to train them how to design an app via drawing wireframes.

Using postits, they started with during limited time drawing the start view from the Facebook app.

Then, they were asked to draw what they thought happened on the friend icon click, drawing the view on another postit.

Then, the mission of the YoungDrive app was described. They were then divided into two teams, having limited time to draw the best imaginable YoungDrive coach quiz app they could. First, they designed the app from the top of their heads. They then pitched their results to each other.

On the next iteration, they were to suggest and design improvements how the app should be designed to improve learning, not only assessment. They then again pitched their results to each other.

The result was fantastic, in the sense that it gave me an unbiased look at what the coaches expected from the app, what functionality wasn't important, and into their technical preferences.

The designs and insights gained were used throughout the week to further improve the app I had actually started creating, and gave great insights to who the coaches were and their thinking.

\subsubsection{Assessment via quiz}
At the end of each day, the app was used to test the coaches' knowledge. Each coach got either a smartphone, tablet or computer. The coach first took the quiz for the most recent session, and could then choose what to do next.

As there were no back-end developed, Josefina by hand documented the scores of each coach, writing the name of the coach, the session, number of correct answers, and what questions had been answered wrong.

Josefina then, when planning the next day, looked at the statistics, looking for trends that would inform the sessions for the following day.

She also evaluated the quality of the questions, before creating the new question sets for the next day.

\subsubsection{Experimenting with quiz before or after the session}
Since the coaches appreciated the app so much, we felt tempted to try what would happen with fun and learning if we tried using the app \textit{before} a session instead of only after. During the rest of the week, we continued, finally finding preferences and tendencies from the coaches, via observation, interviews, and survey.

\subsubsection{Experimenting with design of questions}
During the week, extra tests were done to test the following:

\begin{itemize}
\item Number of questions per quiz
\item Single-answer questions or multiple-answer questions
\item Framing of questions
\item Challenge level of questions
\item Determining what made a question hard
\end{itemize}

\subsubsection{Interviews with Josefina}
At the end of each day, an evaluation interview was held with Josefina. At the end of the week, a final interview was held.
