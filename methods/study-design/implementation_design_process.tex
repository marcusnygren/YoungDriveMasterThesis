%Har gått igenom planeringsrapporten lite noggrannare idag och ser två saker som vi kanske ska borde fånga upp under arbetets gång.

% Under 2 Purpose står det ett upplevelsemål från Young Drive. Bör vi mäta detta upplevelsemål om det stämmer med deltagarnas faktiska upplevelse, d v s ska vi försöka få in det under 3 Research Questions?

% På våra avstämningsmöten borde vi också följa upp dina Research Questions så att kundinteraktionerna och servicedesignmetoden tyligt leder dig framåt mot dessa mål.
%* Reflektioner på vilka designprinciper som bör väljas? (utifrån kundinteraktioner)
%* Reflektioner angående tekniska begränsningar?
%* Reflektioner på processen?

I'm the computer expert kind of designer, but aspiring to be a socio-technical expert (which e.g. Expedition Modial are, as service design experts).

Expedition Mondial helped with a method for creating a MVP of the digital support for the coaches, so that the app was developed from the perspective of the end users and the education and a "learning by doing" mentality. The suggested design process was designed with them after a start-up meeting on Skype, and an Education day in Stockholm. During that day a crash course in service design was given, then creating a common plan for the future work based on my needs.

The result is that the design and development phase in Uganda is an iterative process with the human in focus. The process is built on top of service design process and methodology. There were four iterations.

Expedition Mondial gave support in each iteration, helping with the design of each iteration, and they were able to educate me during the different stages with methodologies. They also recommended service design literature.

Each Interactions phase consisted of a meeting with thought users, the entrepreneurship coaches.

Expedition Mondial's competence was valuable to me when formulating questionnaires.

How to frame the questions is an art.  Therefore in preparation for meetings with the target group it was discussed exactly what was wanted to know and they helped to phrase the questions.

Finally, it was concluded if the master thesis work is going in the right direction. In most of the meeting, the next iteration was the biggest focus. % "Kan vara ganska vänskapligt, håll det så enkelt som möjligt så jag får tid till annat."

Before moving to Uganda, a time plan for the design and development phase was developed, see Appendix. \todo{Add appendix}. The four iterations are presented below: \\

\textbf{The iterations} \\
The time in Uganda is divided into four iterations. For each iteration, the result becomes more and more clear. In iteration 1, there was a very broad scope, without digital focus whatsoever, where iteration 2, 3 and 4 introduced a digital solution. This were the methods used in each iteration:\\

%resultat: design proposal, Thoughful Interaction Design. "This [iteration 1] is where the designer gets involved in design work, establishes a preliminary understanding of the situation, navigates through available information, and initiates all neccessary relationships with clients, users, decision makers, and so forth. Based on all this, she creates a design proposal.

\textit{\textbf{Iteration 1}}\\

   \textit{\textbf{Iteration 2}}\\
   This time, the iteration has a more detailed scope, with a hypothesis on what needs the app should meet in the end, and create lo-fi and hi-fi trigger material to meet those needs.

   A co-creation workshop started the interactions, followed by repeated app tests at minimum one session per day, followed by a feedback round. At the end of the week, there was a co-refinement workshop of the current hi-fi material, and also lo-fi material for the new version of the app.

   The co-creation workshop was made to identify important functionality in the minds of the coaches.

   After the week, there was again a summarizing meeting with experts and partners to determine the way forward. A second trigger material would be created in iteration 3.\\

   \textit{\textbf{Iteration 3}}\\
   Iteration 3 had an even more detailed scope. An app test was held, and also a co-creation workshop was held.

   Before the workshop, the wished functionality and goals were well formulated. It was also discussed beforehand how to best design the workshop, together with Linköping University and Expedition Mondial.

   Questionnaire 3 was created. In conjunction with the workshop the coaches tested with the app, then interviewed, divided into co-creation workshop groups, and their interactions studied.

    In the end of iteration 3, an analysis was done, and a summarizing meeting with experts and partners determined the way forward.\\

    \textit{\textbf{Iteration 4}}\\
    The focus of iteration 4 was a summative test. \\
