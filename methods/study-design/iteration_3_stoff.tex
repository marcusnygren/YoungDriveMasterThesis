Then, choose the most similar answer. The answers are then analyzed.
Flip card method: ask question (e.g. "What is an entrepreneur?"), record your open answer by voice, click next, be shown the multiple-choice answers, and choose the one that was \textit{the most similar to your open answer}.

App för lärande, inte bara utvärdering.
4 vägar att gå vad gäller hur frågor ska ställas:
1. Fältmetod: Efter du har fått slutresultatet så kan du trycka på improve för att få alla felaktiga svar. Klassiskt inom körkortsproven i Sverige.

2. LiUmetod: För varje försök sänks resultaten från guld, silver och brons. Motiverar till att studera innan = gamification.

3. Pedagogikmetod: Teknologi och förenkla livet. Efter varje fråga lades till ”hur säker är du på ditt svar?”. Kräver studenten att reflektera över sitt svar, metakognitivt tänkande.

Två staplar:
Så här mycket rätt har du / Så här empowered är du!

4. Innan du får svarsalternativen så får du spela in dina svar, sen välja ett alternativ som de tycker är närmast. Det är bra för de som utbildar coacherna.

Feedback from experts on idea number 2 (e.g. Grameen Foundation)_

Challenges with usability. E.g. Snapchat, a popular application in Europe that has the same record behaviour, is still not made popular in Uganda.

Also, it is still \textit{possible} to be dishonest, and choose the most likely alternative instead of the one that was closest to your answers.

The advantage, would be that YoungDrive could use the recorded answers to improve the questions and answers, and get insights about the coaches.

Another disadvantage, is that this method would take a lot of time to implement.

3. Från Lena: gör som i NTA Digital, du får medalj guld, silver, brons baserat på hur många gånger du försökt få 100\% rätt

4. Henrik Marklund kom med följande förslag istället, inspirerat av lärare han kände: "Är du säker?" efter varje fråga
4a) Först var idén: gör som läraren, Ta fram ett gemensamt betyg, MVG, VG, G, genom att vikta Korrekthet med Empowerment
4b) Fick feedback från Josefina: ha två separata staplar Korrekthet och Empowerment. Coacherna kan 1) vilja gamea systemet, och 2) undra hur de fick sin score. Kan då bli svårt att förklara.

Fanns extremt många fördelar med denna, och kom bara fram till ännu fler efter diskussioner med människor och Lena Tibell, framför allt hur denna kan förbättra utbildningen och 1-on-1 coachning, och bli väldigt bra självreflektion för coachen.

\textbf{Beslut av approach}

Idé för interaktioner blev först att A/B-testa Idé 1+3 vs. idé 4 under en workshop, och sedan testa Idé 2 ute i fält för att mäta användbarhet, i.o.m. den metoden gav mycket kvalitativ data, och var bra feedback till utbildarna.

Feedback kom först från Expedition Mondial (som hälsade på under veckan) att under min workshop med coacherna kommer säkert idé 1+3 och 4 blandas ihop.

Inför mitt möte med Grameen, pratade jag med innovationsrådgivare Peter Gahnström om hans analys av de 4 alternativen. Han gillade alternativ 4 mest, och gav följande nya insikt till mig: "Det här kan motverka traditioner och ”så här vi alltid gjort det” genom att tvinga en att reflektera över varför inte ens rätta svar korrelerar med hur empowered du känner dig. Bryter normer, sätter sig emot lathet och agerar proaktivt för en skarpare utveckling tillsammans."

Jag träffade Juliet och en utvecklare på Grameen Foundation (som gjort LedgerLink), och gick igenom de 4 alternativen. Svaret gavs att idé 2 definitivt är för okänd för användarna. När indikationer kom från även Grameen Foundation att 1+3 och 4 säkert skulle blandas, och var grymma alternativ, fråga jag "Hur?".

Svaret blev en diskussion med att i ett resultat efter ett quiz få två scores Korrekthet och Empowerment. Sedan på Improve, så får du medaljer/score baserat på Antal försök. (Grameen trodde ej det skulle bli problem med gamification på idé 3. ) Du ska nå t.ex. 90\% rätt på båda staplarna.

Hon (Juliet) föreslog också att du kanske inte måste ha chans på guldstjärna bara på första försöket. T.ex. att om du gör quizet gång 1, så måste du få 90\% för guld, men på ditt andra försök måste du få 95\% för guld. Detta då vi ju vill att coacherna ska ha läst på innan.

Lena Tibell menade vid förslaget att "Belöna inte hur snabbt en elev går från att kunna till att inte kunna, för olika människor lär sig olika snabbt". "Vad vi ville åstadkomma med Antal försök var endast att undvika gissningarna".

Lena frågades också om vilken skala jag ska ha på 5, 4 eller vad jag inte tänkt på, 2-gradig skala. 5 eller 4 är vilket som enligt literatur, det finns två olika skolor. 2-gradiga skalan bedömde jag vara bäst, p.g.a.
* användarvänlighet, tydligt för coacherna
* behöver inte vara krångligare än så till en början, en bra test
* blir enkelt att mäta empowerment, rätt svar + säker = pluspoäng, rätt svar + osäker = du gissade men hade rätt, gör om och var säker -> empowerment, fel svar + säker = måste ge feedback (väldigt intressant för Josefina), och fel svar + osäker = du ahde rätt, det var ett annat alternativ, gör om -> empowerment + korrekt information.
