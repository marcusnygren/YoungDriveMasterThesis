\subsection{Iteration 1}

% How was this iteration designed?

Following the service design sequencing, the first iteration had a very broad scope and truly is a service design iteration: "From your perspective, what is it like being a coach?". \footnote{A coach meaning either a Community Based Trainer (carrying out all the trainings), or a YoungDrive coach, depending on who was asked the question.}

Lowgren's though about how to start the project was used, meaning that the purpose was to get a preliminary understanding of all important aspects, and build relationships with all stakeholders.

Insights depended heavily on interviews with all the stakeholders  (2 with Plan International, 3 with YoungDrive), and local experts (1 visit each at Grameen Foundation and Designers without Borders, 1 workshop with Mango Tree), since no Interactions with users had been made yet. Also, I immersed myself with the Uganda tech scene as possible, from the new home and office in Kampala, working at the tech hub and co-working space Hive Colab.

Ideation were about creating a questionnaire guide for the interviews, a co-creation workshop using "Customer Journey Map", and identifying how the app test should be designed to test their existing knowledge (and be informed of the design preferences of the YoungDrive app).

Trigger material was the finished questionnaire guide (constructed with Expedition Mondial) a written plan for the co-creation workshop ("A day as a coach"), and a written plan for testing the quiz app Quizoid and the language learning app Duolingo, and a schedule for the interactions.

The interactions were focused on design ethnology, getting to know and learn from people in a different culture, namely the coaches. The focus was on the their needs, motivations, and context.

To accomplish these, four days were spent in Tororo, with one day of travel. There were four face-to-face-interviews,
one meeting with Plan, one meeting with the local partners, two workshops, one coach stay-over, and two youth session visits.
