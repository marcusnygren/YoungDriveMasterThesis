\subsection{Iteration 1}

% How was this iteration designed?

The first iteration had a very broad scope. The focus was on the coaches' needs, motivations, and context.  Creation of questions for questionnaire 1 was done. Interviews were done with coaches and other involved parties. Whenever coaches met in-group, open questions and dialogue was facilitated, using post-its and following up with specific questions. These sessions were all recorded.

Afterwards, a first analysis was done to summarize the coaches' needs, motivations, and expectations. Then, a summarizing meeting was held with the expert and the partner group to determine possible ways of going forward. These insights were the basis for iteration 2.\\

% Nedslag

The goal of iteration 1 was to answer "From your perspective, what is it like being a coach?". A coach could have two meanings, a YoungDrive coach, or a Community Based Trainer (having responsibility for several trainings, one of them being YoungDrive).

\subsubsection{Insights}

Lowgren's though about how to start the project was used, meaning that the purpose was to get a preliminary understanding of all important aspects, and build relationships with all stakeholders.

Four interviews were held with stakeholders to answer "From your perspective, what is it like being a coach."

Two interviews were held with Plan International, to understand their needs. They recommended learning from previous work with Designers without Borders and Grameen Foundation.

Also, several interviews were done with Iliana Björling and Josefina Lönn from YoungDrive in Uganda and Zambia respectively, to learn more about the organization and its needs.

\subsubsection{Ideation}
An external workshop was visited at co-working space Outbox with Mango Tree.

Interviews were held with Designers without Borders and Grameen Foundation.

A questionnaire guide was created, with feedback from Expedition Mondial, Linköping University and YoungDrive. Expedition Mondial helped planning the interactions.

\subsubsection{Trigger material}
For the interactions, one co-creation workshop was created based on Customer Journey Map. Also, an app test was planned where the existing apps Quzical and Duolingo would be tested.

\subsubsection{Interactions}

As Plan International staff are not allowed to support visiting coaches in the field during local elections, the co-project leaders in Tororo are consulted to carry out the field trips, where it will be possible to attend youth group meetings.

Four days were spent in Tororo, with one day of travel. There were four face-to-face-interviews,
one meeting with Plan, one meeting with the local partners.

Two workshops were conducted:
\begin{itemize}
\item Workshop \#1: Customer Journey Map: A day as a coach
\item Workshop \#2: Quizical and Duolingo
\end{itemize}

Additionally, there were two field visits to attend youth sessions by coaches. One of the nights were spent with one of the co-project leaders, which supported understanding the target group.

Back in Kampala, an Expert meeting was held with Expedition Mondial and LiU Innovation, and the following day the Partner meeting with Linköping University and YoungDrive, which led to the ultimate conclusion of findings.
