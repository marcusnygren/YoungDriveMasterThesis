\subsection{Iteration 4}

Efter prat med Henrik:
https://memorize.com/

Growht mindset vs Performance mindset
Goal-mastery-mindset vill vi få dem hamna i

Flashcards vid Improve

\textbf{Self-monitoring, vad du vill åstadkomma}
% http://digitalcommons.ilr.cornell.edu/cgi/viewcontent.cgi?article=1492&context=cahrswp

Questions Used to Prompt Self-Monitoring and Self-Evaluation
Self-Monitoring
1. Am I concentrating on learning the training material?
2. Do I have thoughts unrelated to training that interfere with my ability to focus on training?
3. Are the study tactics I have been using effective for learning the training material?
4. Am I setting learning goals to help me perform better on the final exam?
5. Am I setting learning goals to ensure that I will be ready to take the post test?
6. Have I developed a strategy for increasing my knowledge of the training material?
7. Am I setting learning goals to ensure I have a thorough understanding of the training
material?
8. Are the study strategies I'm using helping me learn the training material?
9. Am I distracted during training?
10. Am I focusing my mental effort on the training material?
Self-Evaluation
1. Do I know more about the training material than when training began?
2. Would I do better on the final exam if I studied more?
3. Do I know enough about the training material to answer at least 80% of the questions
correct on the post test?
4. Have I forgotten some of the terms introduced in previous training material?
5. Are there areas of training I am going to have a difficult time remembering for the final
exam?
6. Do I understand all of the key points of the training material?
7. Have I spent enough time reviewing to remember the information for the final exam?
8. Have I reviewed the training material as much as necessary to perform the skills on the
final exam?
9. Do I need to continue to review before taking the final exam?
10. Am I making progress towards answering at least 80% of the questions correct on the
post test?

\subsubsection{FÖRSLAG ITERATION 4}

Designat för learning och självreflektion, och effektivitet
Problemet nu, var att de tog certifikations-läget och inte fick 100\%, vilket är mödosamt och väldigt tidsslösande, då coachen igen måste gå tillbaka till Träning och nå 100% igen.

Tränings-läget behöver alltså förbättras, och vara säker på att coachen verkligen är redo för Certification.

Ett problem är att "Improve" endast upprepade frågor som varit inkorrekta, och inte upprepade gissningar som varit rätt. Det gjorde att en coach kunde få fel på Certification quiz, för att kunskapen inte var befäst. Så vill vi inte ha det. Därför föreslår jag följande förbättringar i Träningsläge:

Förbättringar Träningsläge: ta ett quiz, med "Are you sure?". Baserat på svar, låt frågor hamna i tre olika lådor: "Can't do", "Can do with effort" och "Can do effortlessly".

Låt coachen välja vilken typ av frågor de vill upprepa.

Frågor i "Can't do", är frågor som coachen ej vet svar på ännu (t.ex. om svarat fel).
Frågor i "Can do with effort", har coachen ett hum om (gissat rätt, eller gått från fel till rätt).
Frågor i "Can do effortlessly", har coachen rätt och den vet att den har rätt

Genom att ta en hög med frågor igen, flyttas de om till andra högar. Om du har fel på en "Can do effortlessly"-fråga, flyttas den tilllbaka till "Can't do" eller "Can do with effort". Om coachen igen har rätt på en "Can do effortlessly", blir coachen certified i just den frågan.

Frågor i "Certified", är frågor som coachen befäst genom att upprepat korrekt från "Can do effortlessly". De behöver inte upprepas. Coachen kan bli Certified i ett helt quiz, när den tar alla frågor som ligger i Certified. Då är den klar, och 100\% expert i ämnet! Men coachen kan också välja att lämna quizet när som helst, och komma tillbaka i ett senare tillfälle. Detta handlar om glädjen i att lära sig, bli en bättre coach, och att visuellt se hur man blir bättre hela tiden.

Målet är alltså att i coachens egna tempo, flytta över frågor från "Can't do" till "Can do effortlessly" till "Certified". Så planerar jag bygga expertis som YoungDrive-coach.

Förbättringar Certifikationsläge: om coachen klarar det, ska coachen bli enormt glad. Guld, silver och brons-medaljer ska vara tydliga, och ljud kan förstärka storheten i att ha klarat det. Det ska synas på startskärmen, att du har fått stjärnor och blivit certifierad i ett topic.

\subsubsection{Service design-insikter}

SERVICE DESIGN
Detta kapitel visar vilka insikter som har guidat mitt arbete med iteration 1, 2, 3 och 4.

ITERATION 1 \& 3: What's it like being a coach?
I iteration 1 fanns ingen digital ansats alls. Jag var i Tororo för att besvara "What's it like being a coach?". Upptäckte att vad det innebär att vara en bra YoungDrive-coach, är att kunna ha bra ungdoms-sessioner. För att ha bra ungdoms-sessioner, är din självkänsla och självförtroende enormt viktigt. Och det är inte alla coacher som har detta, och därför skiljer sig kvaliteten mycket, vilket Josefina upplever som en utmaning.

Jag började leta efter hur och var en coach-app kan underlätta. En aktivitet som alla coacher har gemensamt för lärande och avgörande för coachens framgång, är (1) coach-träningen (som jag redan visste var viktig), men framför allt (2) förberedelserna av en ungdomssession. Jag övertygade Josefina att vi skulle ha ett mycket fokus på (2) än hon tänkt. Medan Josefina kan vara inblandad i (1), kan en app vara extremt viktig i (2), upptäckte jag under mina fält-besök på ungdomssessioner och intervjuer med coacher och projektledare.

I Tororo iteration 1 kunde jag observera ungdomsbesöken, i Zambia iteration 2 kunde jag observera coach-träningen, och i iteration 3 i Tororo kunde jag observera förberedande av ungdoms-sessioner.

Därför fick app-utvecklingen för dessa iterationer ha dessa fokus. I iteration 1 fanns ingen digital ansats, men apparna Quizical och Duolingo testades för att få koll på coachernas tekniska förutsättningar. Resultatet blev att min app kan placera sig någonstans emellan i svårighetsgrad.

Iteration 2 gjordes en coach assessment quiz app, och iteration 3 utvecklades den till en coach learning quiz app. Dessa insikter guidade:

Iteration 1: Självförtroende = empowerment
Enligt iteration 1 kom självförtroende ifrån att under ungdomstillfället kunna ha: Correct Information, Correct Structure, Time Management, och Fun Atmosphere. Det är alltså detta appen borde testa och träna.

Lösning: en coach-träningss-app hade störst behov av att fokusera på Correct Information, i andra hand Correct Structure och Time Management. Till iteration 2 kunde Josefina assessa Correct Information (lyckat), och till iteration 3 kunde coacherna lära sig CI (lyckat, men behöver göras mer effektivt). Till iteration 3 hade hon via ett "Are you ready?"-quiz även försökt använda multiple-choice-strukturen till att även assessa och träna Correct Structure och Time Management (ej särskilt effektivt sätt, testar Factual Remember, men ej högre Bloom).

Det finns en medvetenhet kring att CS, TM och Fun Atmosphere är lämpligast att testa efter en ungdomssession, men att vissa förberedelser kan göras i appen innan en session. Dessa är därför sekundära.

Iteration 2 och 3: Självkänsla = kunskap om dig själv, meta-kognition
Under Iteration 2 i Zambia, passade jag på att fråga vad som byggde självkänsla. Följande kluster fanns: "I believe in myself" (3 personer), "I believe in God" (2 personer), men också "I am well prepared" (4 personer) och "I am certified" (1 person).

Till iteration 2, hade jag fokuserat på att assessa "I am well prepared" och då stärka självförtroende, med hänsyn till Correct Information.

I iteration 3 i Tororo, hade jag fokuserat på att bli "I am well prepared", och även byggt in "I am certified.". Det visar sig att de flesta inte bryr sig om "I am certified" (vilket ju undersökningen redan visade), men de bryr sig om lärande-resultaten.

Under iteration 3, lärde jag mig att det tog för lång tid för coacher att nå 100\% säkerhet. Detta blev tydligt särskilt på den svåraste quizen om Correct Structure och Time Management, "Week 9: Are you ready?", då det tog en coach 2.5 timme att nå 100\% i ett försök. I iteration 2, då "Improve" inte fanns, hade detta nog tagit ännu längre tid.

Iteration 4: Effektivitet = en förutsättning för att coacherna ska ha nytta av appen
Anledningen till misslyckandet i iteration 3: dels för att CS och TM tydligen inte lämpar sig för multiple-choice (gör sådana övningar drag-and-drop-istället), men framför allt för att feedback-systemet och tränings-läget behöver vara mer medvetet i när en coach verkligen kan sitt ämne och är redo för sin ungdomssession. Du vill inte testa 100\% rätt utan fel på 13 frågor, förrän du är helt säker på att en coach är färdig med sin träning. Till iteration 4, vill jag göra appen effektiv.
