\subsection{Iteration 3: Uganda Formative Test}\label{sec:sd-3}

Because of the many research and functionality needs, the study design of Iteration 3 became very important. A lot of development and ideation needed to be done. Iteration 3 had an even more detailed scope. Since the app now succeeds with the first use case, the coach training, now the focus could be on "learning at distance".

\subsubsection{Trigger Material Used}
It was chosen that "Are you sure?" + Improve would be included in the hi-fi material of the app, a flashcard approach would be tested as a low-fidelity material, and to "record answer via voice" could only be presented as an idea during a field interview (experts said there would be usability issues, and the first-time smartphone user agreed). The Gold/Silver/Bronze reward system was included into the high-fidelity material.

\subsubsection{Interaction Activities in Uganda}
Instead of only testing the app in Tororo, because of the major changes, a test was held in Kampala, to early get feedback from an entrepreneurship student. During the test in Tororo, as Plan International staff are not allowed to support visiting coaches in the field during local elections, the co-project leaders in Tororo were consulted to carry out the field trips, so that it was still possible to attend the youth group meetings.

For the interactions, a big app test was held, a group interview was held, and then they were divided into co-creation workshop groups, with a presentation in the end. There was another partner meeting, with Plan International and Community Vision present. There was an app test with all of the coaches, "Testing the YoungDrive coach app", followed up by splitting into six workshop groups based on solving different problems discovered during the test.

%Before the workshop, the wished functionality and goals were well formulated. It was also discussed beforehand how to best design the workshop, together with Linköping University and Expedition Mondial.

The following day, there were three field visits to CBTs, observing how they prepared themselves for a youth session, and then observe usage of the app immediately after having prepared a youth session for assessing and becoming prepared for a session.

The app test simulated the app being used to assess preparedness for a youth session.

After the high-fidelity app test, it was tested with a low-fidelity prototype that the coach thinks aloud about the question, \textit{before} receiving the multiple-choice answers. This test was done as a live quiz, and if the interviewee could not answer the question directly, the audience were asked and tested if they knew the answer (raised hands), and if nobody knew the answer, it was tested which of the multiple-choice alternatives they found most likely. During the afternoon, the coaches was divided into five groups focusing on improving the app experience for the coaches.

On Wednesday, the coaches from the field visits were gathered for a workshop. The purpose was to see how they acted when given the challenge: "Get 100\% correct answers in one go, on the hardest quiz". A co-creation workshop ("Educator Dashboard") was held in parallel, with 3 CBTs and 1 project leader respectively.
