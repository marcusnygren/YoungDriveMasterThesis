\subsection{Iteration 3}

Iteration 3 had an even more detailed scope. Since the app now succeeds with the first use case, the coach training, not the focus could be on "learning at distance".

A requirement from Josefina was also to test if the app created in Zambia could work also in Uganda. All the quiz questions would need to be converted from the new manual to the old manual, since both structure and content had changed.

To test on all of the coaches in Uganda, it would be preferable if data collection would happen via the app instead of manually, since there would be more than 10 test subjects, which had been the limit in Zambia.

A future requirement was that quiz responses would need be available to the teacher. This means that there needs to be a database, but also a login, so individuals are traceable.

How can login and the database be implemented in the best possible way?

The insights on learning also needed to be considered:
\begin{itemize}
  \item Are coaches really learning via the app, especially learning to be better coaches?
  \begin{itemize}
    \item How can questions be formulated in a way that teaches entrepreneurship, which is so practical?
  \end{itemize}
  \item How can the current multiple-choice quiz app be improved, to:
  \begin{itemize}
  \item reduce guessing
  \item improve confidence
  \item encourage learning
  \end{itemize}
\end{itemize}

Thus, the study design of Iteration 3 became very important. A lot of development and ideation was done.

Also, instead of only testing the app in Tororo, a test was held in Kampala, to get feedback from an entrepreneurship student.

For the interactions, a big app test was held, and also a co-creation workshop was held.

Before the workshop, the wished functionality and goals were well formulated. It was also discussed beforehand how to best design the workshop, together with Linköping University and Expedition Mondial.

Questionnaire 3 was created, used after the test. After interviewed in a big group, they were divided into co-creation workshop groups, with a presentation in the end.

\subsubsection{Aim}

To get an app suitable for learning, it was determined that the pedagogical model behind the app needed to change, emphasising feedback.

The aim was to score higher on Bloom's revised taxonomy, while still including multiple-choice questions in the app.

\subsubsection*{Trigger material}

Josefina was given a task to create a quiz "Are you ready for Session 9?". The aim of this quiz, was partly to score higher on \textit{Bloom's revised taxonomy}, partly to test if Correct Structure and Time Management could be assessed using multiple-choice.

Also, the questions were translated from the new manual into the old manual, which is used in Uganda.

\subsubsection*{Interactions}

There was another partner meeting, with Plan International and Community Vision present. There was an app test with all of the coaches, "Testing the YoungDrive coach app", followed up by splitting into six workshop groups based on solving different problems discovered during the test.

The following day, there were three field visits to CBTs, observing how they prepared themselves for a youth session, and then testing the app for assessing and becoming prepared for a session.

The last day, there was a co-refinement workshop ("Usability Improvements") and one co-creation workshop ("Educator Dashboard") held in parallel, with 3 CBTs and 1 project leader respectively.
