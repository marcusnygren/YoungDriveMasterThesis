\subsection{Iteration 3: App for Preparing: Learning}

\subsubsection{Insights}

After the meeting with the partner and expert group, the following was concluded from iteration \#2:

\begin{itemize}
\item The app is only working on assessment now, not for learning
\item The need for a field app still feels relevant (especially for sessions long since the coach training)
\item The potential for YoungDrive having online coach training is huge
\end{itemize}

Determine:
\begin{itemize}
  \item Focus for the next iteration: design quiz app for learning, focus on field app (CI, CS, TM, FA), and design having an app that works stand-alone from the YD coach training in mind.
\end{itemize}

Discussing the importance of self-reflection after a youth session with Josefina, led to asking more of such questions in coach quizzes.

Josefina: “I have a problem: there is no way I can control them how they have prepared themselves for a youth session."

An app could be used, either before you start planning (to guide what you need to study the most on), or after you think you are ready (so you can assess and improve).

\subsubsection{Ideation}

To get an app suitable for learning, it was determined that the pedagogical model behind the app needed to change.

\textbf{Improving learning when answering questions}

The aim is to score higher on Bloom's revised taxonomy, while still including multiple-choice questions in the app.

There are four ideas:

\begin{enumerate}
\item The coach result from Iteration 2: "Try again"-button. When clicked, your wrong answers are repeated.
\item If 100\% on the 1st try, gold. On 2nd try: silver. On 3rd try: bronze.
\item Ask meta-cognitive questions, e.g. "How sure are you?", at the end of each question.
\item Record your answer to the question before you are shown alternatives.
\end{enumerate}

Option 1, 2 and 3 were determined good after the interviews, while item 4 had too many challenges (difficult to use, difficult to implement, cumbersome).

\subsubsection*{Trigger material}

The idea is to work a lot with feedback via the app, and create a different type of quiz for preparing. Josefina is given a new task, so create a quiz "Are you ready for Session 9?". The aim of this quiz, is also to score higher on \textit{Bloom's revised taxonomy}.

\subsubsection*{Interactions}


