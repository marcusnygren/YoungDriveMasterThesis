2016-04-05: Schema måndag: Hur det blev
Problem med internet
Svårt med internet i början, kom tillslut på att jag kunde sätta upp mobilt nätverk från min mobil, utan internet, och sedan starta igång localhost med samma wifi från min dator.

Min Vodafone-dongel fungerade inte, så skönt att alla devices var OK med att connecta.
A+B
Intro
Pre-knowledge test
Divide A (with manuals) and B group (without manuals)
Breakfast
Preparing devices (by loggin into correct IP adress)

Delade in i A + B, där A hade manualer med sig och fick börja. Några kom sent, och hamnade då i B. De gjorde före dess en intro, samt pre-knowledge-test. Sedan frukost. Under tiden förberedde jag det sista med alla enheter, ändrade om så att alla använde rätt IP-adress.
Start
Efter frukost, delade vi upp rummet i två delar, och seglade skjutdörr emellan. B-gruppen fick YoungDrive Step 1 evaluation, och fick sedan diskutera i grupp. Susan blev ledare, efter att Christine gett feedback, och det verkade bli väldigt bra diskussioner.

I A-gruppen kände de väldigt stor lyx som fick börja med plattorna. Jag recappade förra gången, särskilt då vi hade 2 som inte var med förra gången, första gången använda smartphone.

Jag bad de tänka sig att de förberedde en session, vecka 3 Financial Literacy, och inte var på TLT länge, utan där de vanligtvis förbereder sina sessioner. De har kollat igenom sina manualer, tagit anteckningar, och så kommer de ihåg att de kan kolla om de är redo med hjälp av YoungDrive Coach-appen.

Sedan förklarade jag reglerna: kort vad en smartphone och app och YD-appen är (de nya förstod inte så bra, men jag körde på, för ville inte låta de andra vänta, och jag berättade att de skulle förstå så fort de fick testa själva, för det trodde jag med YD-appen).

Sedan gav jag ut devices, och förklarade precis innan jag startade att jag skulle testa 3 saker:
Antal rätt på första försöket
Antal försök tills 100\%
Hur lång tid tills 100\% på ett försök

Samt att de skulle räcka upp handen så fort fått quiz-resultat. Patrick och Christine hade tidigare blivit invigda hur använda ett quiz-resultat-formulär jag skrivit ut på Plan-kontoret.

Christine gav ut Coach ID till varje person. De skrev ner detta, så det var aldrig något problem. Väldigt bra att appen visade “Hej X”, både p.g.a. Vi kunde vara säkra på rätt ID, plus att coachen kände sig väldigt speciell på ett väldigt positivt sätt, häftigt. :)
Startade
Efter de klarat quizen, bads de av mig, Christine eller Patrick att testa igen tills 100\%. De uppmuntrades ta quizet först när de kände de kunde svara rätt på de svar de haft fel, och till deras hjälp uppmuntrades de både använda sina manualer (jag hade medvetet känt sig känna sig stolta över att de tagit med manueler, därav grupp A) och kolla hur deras svar skiljde sig med de rätta alternativen.

När de klarade 100\%, fick de göra samma sak, men med Become certified. Kändes som ingen hade problem med vad knappen gjorde denna gång, kanske även p.g.a. Vana. Ingen anmärkte på att “Improve” blivit “Try again”.

Vissa klarade alla frågor på första försöket, och då såg det annorlunda ut - då fick de ta Become certified direkt.

Efter det, så berättade vi hur de varit så duktiga så de fick ta svåraste quizen, Week 9 Are you sure?, samma som workshops föregående gång, onsdags-workshopen som tog 2.5 h. Jag visade Christine och Patrick hur de gav feedback att förberda coachen, att det var svåraste quizen, empowerment, så de kunde undervisa likadant.

Efter en coach även klarat den quizen, så fick de välja vad de ville göra. De flesta tog ett annat ämne.

Dåligt var att två av iPadsen dog, så extraenhet behövdes. Vi behövde köpa laddare till eftermiddags-passet, särskilt då det var ännu fler denna gång.

Diskussion
I slutet av app-test, ställde jag vissa frågor, som om de tyckte appen gjorde dem självsäkra inför en session. Alla tyckte det.

En sade att lätt svara rätt pga man kunde avgöra från alternativen, så därför blev jag osäker på lärandet, och ville testa om de kunde ge svar även utan alternativ [se kväll]. Jag tog upp detta med Patrick, som bekräftade. Det formade även mina intervjuer med min grupp på eftermiddagen, vilket var jättebra.

Lunch
När klockan var 13, var vi redo för lunch. Blev lite försenade, kl. 13.10 gavs kuponger. Jag och Patrick köpte sladdar på bra IT-butik i närheten, USB + power adapter.

Tillbaka, så var både Moses och Hassan på plats. Moses (Plan) skulle varit där kl. 9, men kom till Tororo först då, och hade sedan kanske varit upptagen. Han skulle tagit med sig pennor, vilket han inte hade, men vi hade ju löst det ändå, så jag tog inte upp det. Jag bjöd in Hassan till presentationen kl. 10.

Eftermiddag
Kl. 14, drog vi igång igen. Vi bytte grupper.

Bad dem blunda denna gång, när jag berättade scenariot. Dessa hade inte manualer, och det berättade jag för dem.

Saknades enheter till alla då så många, så först bad jag 3 st vara ihop och få utmaning att istället för att ta Certification quiz skulle de kunna ge mig alla svar utantill. De antog utmaningen.

% När vi sedan fick igång fler devices, blev de 1-1 igen. En device kom ifrån att jag misstog Hassan för att vara en coach, pinsamt. :P Han gick sedan runt och observerade istället, verkade nyfiken, och allt. Jag hann inte visa honom nya varianten dock.

Under quizen så dog 1 device, och då fick han (när vi inte orkade vänta på att en iPad skulle ladda), hoppa in i den gruppen igen, så de blev 2. Jag antecknade coachens ID som haft konstiga resultat pga konstellationen, Odio Richard.
  I slutet frågade jag om han kunde tänka sig bli testad på så vis inför hela gruppen. Han ville helst ha week 3, men jag frågade om han kunde alla svar sedan innan redan, och han kunde de flesta sade han, så jag sade att han ville ha week 9. Då bad han 10 minuter. Då frågade jag om det var OK vi gjorde efter kvalitativa intervjuer istället.

Den andra gruppen, A, verkade även de ha bra diskusseiron efter att ha gjort step 1-utvärderingen.

A + B samlade
Efter lyckad app-test, så samlade jag alla i del 1 av rummet igen, så A+B samlade. Frågade om de ville ha rast, men inte. Frågade energizer, men inte. Jag tvingade dem ta 5 minuter rast, för att få tillbaka syret.

Intervjuer
Jag frågade först Christine, sen Patrick, om vilket intervju-format de ville ha. Jag föreslog att dela upp dem i 3 grupper, 8+8+8 per oss 3, Christine tyckte vi skulle köra alla på en gång, men när jag föreslog för Patrick gillade han idén och då körde vi på det. Christine var OK med det då också.

Christine gav dem nummer, 1,2,3, och sedan tog jag 3:orna ut, medan 1 och 2 stannade med Christine och Patrick. Lite före kl. 14, hade jag gått igenom frågorna jag ville ha besvarade, utifrån den delen av mitt exjobb. Superbra move, vilken tur jag hade!

Det blev jättebra frågerunda, Patrick och Christine använde de 40 minuterna jag hade gett dem, medan jag hade superintressant diskussion med mina, och hade egentligen bara avslutat #2 av #6 när en person kom och sade time up. Vi fortsatte 10 minuter till, så att vi hann med lite även på de andra frågorna. Jag kunde gå väldigt djupt! :D Fick superbra feedback.

Kväll
Kl. 16.40, så samlades vi där uppe igen, och jag började avrunda genom att tacka. Men att innan vi avslutade, skulle vi ha en tävling.

Tävling
Oddio Richard skulle bli frågad på svåraste quizet utan svarsalternativ, och jag låtsades det var en TV-show och “Vem vill bli miljonär?”, med musik och allt. På rasten innan, hade jag också skämtat lite med coacherna, rappat, och dansat, stått på händer, vilket de skrattade gott åt. Så de blivit vana med att jag var rolig och lite tramsig.

Sedan körde vi igång. På frågor han inte kunde, gav jag frågan till publiken. Visade, att quizet var både bra och dåligt. Vissa frågor kunde de inte alls, fastän 14/14 rätt på quizet, medan vissa var Richard väldigt ordagrann från quizet med.
  Det var ett väldigt bra test för mig, och blev som ett trigger material för flip cards! Och att testa att användarna verkligen lärt sig. Många applåder, dels på mitt initiativ. Ville få det att kännas lite som festlighet och samtidigt lärorikt, och jag tror det blev en bra balans nästan hela tiden - coacherna kändes engagerade och att de njöt.

Det blev ca 60/40 mellan svar som Richard kunde ge och som publiken gav. I en fråga, hade en minioritet av publiken rätt. Väldigt intressant! Det var frågan med att vara oavsiktligt försenad.

Avsked
Sedan kl. 17.04, sade jag tack för mig, och hejdå. Det blev ett bra prat. Jag förkalrade hur resultatet kommer användas i nya YD-träningar, och att det aldrig varit möjligt utan dem att komma dig vi kommit idag. Tackade alla så mycket för deras tid. Inbjöd att komma fram efteråt och ställa frågor, och att alla som kom fram jättegärna fick en kram av mig. (Vilket vissa sen gjorde.)

Tal av Hassan
Jag såg att Hassan (Community Vision) var på väg, så jag avslutade med att tacka honom och Christine Patrick och Plan. Oväntat så kom Hassan (Community Vision) fram, och tackade av mig. Han tog sedan tillfället i akt att berätta om nyttan med dagens träning, och att fråga hur Life Skills gått, och att han inte hörde något.

Patrick tog upp vad han hört och sagt även under lunchen, att fått in klagomål dålig manual. Flera instämde: manual väldigt svårförstådd, termonologi, samt att de sagt inte anteckna men nu var de alltså ändå tvungen att utgå från deras anteckningar.

Sedan tog Hassan upp varningar, att vi idag hållt på med innovation, tack vare app-testet, och att vi fått vara utvecklare och designers, oc att vi hoppas på fler träningar. Men för att få fortsätta vara med på träningar, måste man sköta sig som CBT. Jag tyckte det var ett bra prat, men när jag lyfte blicken såg jag att minst 2 coacher de närmast mig, deras ögon blänkte av nästan gråt. Det hade t.ex. Med inrapportering av siffror att göra, att inte endast ge stories. Ellre att jobba, och inte slappa. Och att alltid ta problem med Hassan, så de kunde lösas, istället för att vänta. Och att om man inte skickat in sina rapporter, fick man ingen lön.

Sedan tackade jag återigen.

Efter alla gått
Jag frågade Chirsitne och Patrick om de kunde gå igenom deras resultat från intervjuerna. De sa att det gått bra, och visade upp sina papper. Jag spelade in deras sammanfattningar, och resultaten var verkligen enormt bra och givande!! Guld värt verkligen.

Sedan ville Christine ta upp vad som händer nu. Tog mest upp vikten av smartphones, fick samhåll och hjälp att förklara och vänta mina argument fram och tillbaka med Patrick, men även att inte förlora momentum från smartphone-användande, inte tappa kunskapen, att Patrick också borde ha en smartphone, och jag lovade ta upp med Iliana, men ännu mer, viktigaste enligt dem, var att:

Appen nu är användbar i fältet!! (jag njöt inomborts av detta, vilket tecken på framgång :D)
Christine och Patrick hade tydligt sett fördelarna med appen, och lyfte även fram vikten av Educator Dashboard, att de själva skulle kunna se resultaten.

Christine om Kathy Sierra-idén
Jag bjöd då in dem itll att de 2 skulle ha sådan workshop under tiden som de 3 CBT vi bjudit in till morgondagen (via handplockning, “vassa coacher som kan sätta sig in i en challenged coach’s skor”), utformar appen för effektivt lärande. Christine frågade mer, så fastän tiden var mycket och hon ville gå visade jag min Kathy Sierra-bucket-idé.
  Hon gillade (it’s good), men tyckte den kanske blir för komplicerad för vissa av coacherna, särskilt över att de ska få välja själv. Föreslog en mycket enklare lösning, att helt enkelt ändra färg på “Question” med olika nyanser av rött och grönt. Brilliant, det såklart! :) Då blir det ju fortfarande väldigt tydligt, och går snabbare att kolla.

/* Kändes rätt idag, att kanske inte upprepa osäkra svar ändå. Det fanns nämligen ändå en extrem tendens att alltid svara “Are you sure?” Yes, och väldigt sällan “No”. */

Adjö Patrick och Christine
% Christine och Patrick frågade jag även om de ville vara med på Plan-presentation kl. 10, men de hade både jobb i trädgården de var tvunga att göra, om det ej var obligatoriskt, vilket jag tydligt berättade att det inte var. Så de är inte med imorgon. Jag försökte ringa Moses och höra efter, men han svarade inte (kanske pga efter arbetstid).
  Sedan sade vi hejdå, och ses imorgon kl.13. Kl. 10 skulle jag ha presentationen, vilket annars hade varit tiden Patrick och Christine hade föredragits att setts.

