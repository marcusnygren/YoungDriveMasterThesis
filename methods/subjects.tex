\section{Subjects (Participants)}

\subsection{Roles}

\subsubsection{Country Manager}

\subsubsection{Project Leaders}

\subsubsection{The Community Based Trainers}

The CBT's are often volunteers, receiving a small scholarship from Plan International. They are often business owners themselves.

Thus, the CBT's can be described as social entrepreneurs. As Mitchel says about entrepreneurship \cite{mitchel}, motivation does not need to be wealth accumulation anymore. The activity of entrepreneurship contributes to society, in a way that is not caputed by the commercial entrepreneurship literature.

Many of the YoungDrive participants are driven by that their business can have an impact on their community, as well as take them out of unemployment or increase their current livelihood.

\subsubsection{The Youth Mentors}

\todo{Write about the YMs}

\subsubsection{The Youth}

\todo{Write about the youth}

\subsection{Training Material}

Each youth is given a participant manual, describing each week of the 10-week YoungDrive program.

CBTs and Youth Mentors are also given a Coach guide, which describes how to carry out and teach each week's topic during the youth training.

\subsection{Data on Coaches}

\subsubsection{Tororo, Uganda}
In Tororo, there are 2 Project Leaders (PLs), 19 Community Based Trainers (CBTs) and 8 Youth Mentors (YMs).

Both (2) of the PLs, are also YMs.

14 of the CBTs, have a YM.

The subjects, lives in all parts of Tororo: 9 in East, 9 in West and 11 North, with different distances to town.

There were 26 out of 27 possible respondents among CBTs and YMs in Tororo, when in 2015, a statistics summary was carried out.

\subsubsection{Phone/power/data among CBTs and YMs}

Of these, 26 (100\%) had a cell phone, 3 (12\%) had internet on the phone, 0 (0\%) had power at home, 3 (12\%) had solar energy at home, and 4 (15\%) knew how to write on a computer.

Of the 3 that have internet on their phone, 2 (67\%) are using internet each day, and 1 (33\%) is using internet once a week. They mostly use Facebook, followed by email.

\subsubsection{Companies among CBTs and YMs}
In Tororo, 26 (100\%) are running a business, 14 (54\%) are running two businesses, and 1 (4\%) is running three or more businesses.

In Tororo, the businesses range from: ananas, water melon, onion, chili, bakery, catering, corn, beans, fabric, plastic products, bird farm, milk, fish, ground nuts, cabbage, tomato, hairdresser, sewer, shop and rice.

\subsubsection{Kabwe, Zambia}
In Kabwe, there is 1 PL, and 10 YMs.

\subsection{Project Leaders in Uganda}
In Tororo and Kamuli, there were 6 out of 6 possible respondents.

6 (100\%) have phones, 1 (17\%) have internet on phone, 2 (33\%) have power at home, 3 (50\%) have solar panels at home and 6 (100\%) are able to write on a computer.

In Tororo, there are 2 PLs. Christine's business ranges from: bakery, corn, pig farm and plastic products.Patric's business ranges from: silver fish, beans, corn, and bird farm.

In Kamuli, there are 4 PLs. Their businesses ranges from: selling office supply, boda boda, bird farm, pig farm, green pepper, corn, cabbage, tomate, aubergine, chipati ("bread"), chilli, and charging of cellphones.

\subsection{Youth in Tororo}
The youth are the ones receiving the training from the CBTs and the YMs. In the 2015 report, with 225 respondets from Tororo, these were the statistics regarding phones, power, data, language, businesses and the most popular companies:

\subsubsection{Phone/power/data}
99 (44\%) have a cell phone, 9\% have internet on their phone, 3 (1\%) have power at home, 22 (10\%) have solar panels at home, and 20 (9\%) are able to write on a computer.

The mostly use Facebook, followed by Google and WhatsApp. A few people are using it for Twitter, email, news and school information.

\subsubsection{English skills}
In Tororo, 129 (57\%) understands when someone speaks English, 116 (52\%) can speak english, 133 (59\%) can read English, and 132 (59\%) can write in English.

\subsubsection{Businesses}
In Tororo, 165 (73\%) of the youth runs a business. 60 (27\%) are not running a business.

The top 8 most popular businesses in Tororo, with 134 respondents, are corn, cassava ("potato"), saloon, fish, making of bricks, beans, brooms and rope. These range from 9 for corn (6.7\%) to 5 for rope (3.7\%).

\subsection{Coaches in Zambia}

In Zambia, the coaches have higher education, and better access to technology.

6/10 has power at home.
3/10 knows how to write on a computer.

\todo{Add more info about Zambia coaches here}
