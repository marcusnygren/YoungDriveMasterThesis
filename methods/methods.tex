\chapter{Methods and Implementation}\label{cha:Method}

% längsta avsnittet i rapporten. Den består av en redogörelse av ditt arbete och den visar hur du kommer fram till dina resultat.

% Undvik egna synpunkter. Dessa framförs i Inledningskapitlet och Diskussions-kapitlet

% Ange källa till figurer i slutet. T.ex.  Source: Expedition Mondial.

% Metoddelen beskriver tillvägagångssättet – intervjuer, observationer, litteraturstudier, laborationer och så vidare. Motivera varför en viss metod valdes och vilka eventuella svårigheter som har förekommit. Metoden ska vara replikerbar, vilket innebär att en annan skribent ska kunna göra om studien med hjälp av informationen i metoddelen. Det finns en mängd böcker om olika vetenskapliga metoder. Till exempel kan en intervju utföras på en mängd olika sätt. I rapporter inom humaniora brukar metoddelen vara mer utförlig än i en teknisk rapport.

This chapter presents the methodological framework, via presenting methods to design for learning and motivation.

Then, the setting and research context is described, together with a description of the participants.

Then, application implementation is described, followed by presenting the study design and data collection.

The final topic is data analysis theory, which presents methods to analyse quantitative and qualitative data.

\section{Methodological Framework}

  In the methodological framework, useful methods design for learning and motivation are presented, together with methods for creating a design process and data analysis.

  \subsection{Methods to Design for Learning}

%\citep{effectivelearning-robert}
%\citep{learning-krathwohl}
%\citep{learning-ucla}

The following sections, are about how to design for effective learning \cite{dirksen}, by designing for the mind, cognitive psychology. % Design for How People Learn (book, torrent)

Cognitive psychology deals with how our brain works in regards to our memory.

The section presents strategies and techniques to design learning for the mind, and what needs to be considered.

Two aspects are especially relevant when it comes to education: how humans can be supported to retaining (the first second) and retrieving (the second section) communicated information.

In how humans learn, the purpose is to find the most powerful strategies and techniques to design effective learning (mapping educational objectives, how to build skills, pattern-matching techniques, and the power of reflection and assessing).

In how people forget, UCLA Bjork's Learning and Forgetting Lab \cite{ucla} researches how people forget, and how to design so that people do not forget ( retrieval practice and spaced practice).

%On January 28th, 2016, Henrik Marklund\ref{effectivelearning-expert} at the educautional technology startup Knowly was interviewed about Pedagogic Development. He means there are two main areas of research, and an additional one.
%The third area, training transfer, is the research on how to make sure a course gives effect in everyday life.

%The third area is training transfer, and asks "How do you make sure a course gives effect in your everyday life?". For YoungDrive, the wish is that the coach training gives effect in the coaches' everyday life. The master thesis aims to be able to assess and encourage this.

%\subsection{Pedagogical development}

%The first section describes "How do you get people to learn things?", cognitive psychology. Often school is studied, where learning is about being taught a subject, and then to pass a test. E-learning tools are often designed to do similar things to what schools does.

%The second section describes "How do you get people to behave differently?", social psychology. One area of research is about building habits. This is highly relevant in e-learning, where behavior change may be necessary to build the habit of using an app or a digital tool repeatedly.

%\include{theory/learning/pedagogical-development/cognitive_psychology}

%\subsubsubsection{Learning}

  \subsubsection{Learning Entrepreneurship: Mapping Educational Objectives with Bloom's Revised Taxonomy}

  What to teach should be determined by the learning objectives of the activity.

  Learning activities often involve both lower order and higher order thinking skills as well as a mix of concrete and abstract knowledge. This needs to be designed for \todo{Konrad: specify}. Here, Bloom's revised taxonomy can provide usable insight into how to design, by the combination between lower or higher cognitive complexity, and concrete (factual or conceptual) or abstract knowledge (procedural or metacognitive). \citep{cheong} The taxonomy thus provides a framework for determining and clarifying learning objectives. See figure \ref{fig:revised-bloom} from \citep{heer}. Each colored block is an example of a learning objective matching with the two dimensions. The figure also explains the different concepts. Depending on the objective, it fits differently into the Knowledge dimension and Cognitive Process dimension of Bloom's Revised Taxonomy. \citep{krathwohl}

  \begin{figure}[h]
    \centering
    \includegraphics[width=1.0\textwidth]{RevisedBloom.png}
    \caption{Bloom's revised taxonomy visualised with examples of different learning objectives.}
    \label{fig:revised-bloom}
\end{figure}

  Bloom's revised taxonomy can be useful both to map learning objectives for entrepreneurship and as an entrepreneurship coach. To craft good multiple-choice questions could be an art, but to map the question to the learning objective makes it into more of a science:

  Entrepreneurship topic question: "What is financial literacy?" (= \textit{conceptual} and \textit{remember})

  To simulate a procedural environment, the question can be presented as a scenario:

  Entrepreneurship coach question: "It turns out that 10 youth have not carried out the business action, what should you do?" (= metacognitive and evaluating)

  There are several traps that the person formulating the question and answer alternatives can fall into, in the case of multiple-choice, where a good question might be de-amplified because of the answer alternatives.

  Consider the coach being asked to give business advice to a fictional youth named Adam: "Adam wants to start a business that is based on a product. which business should he start?". Before, the coach has been given questions on what a service and product is (factual remember), what the difference is (factual understand), and been given examples (conceptual analyze). Now, the skills are being put to a procedural test.

  If the answer alternatives are obvious (or memorized), the learning will be lower than scoring high on Bloom's revised taxonomy.

  If the answers are high-quality alternatives, all of the answers must be evaluated and considered. In such cases, multiple-choice learning can actually amplify learning, via \textit{learning by repetition} or \textit{learning by thinking}.

  In this case 3-4 valid alternatives might be: "Start a salon", "Start selling soap", "Start a bricklaying business".

  It is still hard to score high on the knowledge and cognitive dimension using techniques such as multiple-choice with entrepreneurship and coaching. This is however necessary, if the app should reach the learning objectives of YoungDrive.

  There may need to be additions to the multiple-choice design, and not only content. Such design ideas may be utilizing flip card techniques (don't see answer alternatives until you've thought of your answer), or asking "How sure are you?", both encouraging metacognitive thinking.

  More ambitious ideas, would be to simulate the entrepreneur coach environment more accurately than via text (using more channels, like audio, video, voice), or to do simulations instead of using multiple-choice. The advantage of multiple-choice, is that data can be collected easily, and that it serves the target group of first-time smartphone users, and because of ease of implementation.

  \subsubsection{Building skills: by Spaced practice, Deliberate practice and Perceptual exposure}

  Spaced practice deals with spreading out learning, with the purpose of not forgetting. E.g. Clark \citep{gates} concludes that spaced learning versus massed learning (no rest between sessions) did have a memory benefit in their study.

  Taking spaced learning into consideration, could mean making the user apparent on the person's meta-cognitive ability (your personal insight of what you'll remember and when you are likely to forget), and meta-memory (when you need to repeat information in order not to forget).

  Clark \citep{gates} found no evidence of consistent correlation between total duration and effects on learning outcomes in their study. So how do you design for optimal learning outcomes of skills, particularly if those are entrepreneurial or coaching skills?

  When building skills, Sierra suggests deliberate practice \citep{yengin} \citep{sierra}. The goal is to help users practice right, by designing practice exercises that will take a fine-grained task from unreliable to 95\% reliability, within one to three 45-90-minute sessions.

  Deliberate practice has been proven to be an effective way to build skills. It has also been tested before for mobile learning environments. \citep{yengin}

  Sierra \citep{sierra} suggests skills to be divided into three buckets: can't do (but need to do), can do with effort, and mastered (reliable/automatic). The goal then is to move skills from can't do into mastered, in the best way possible. See figure \ref{fig:sierra-practice} from Sierra \citep{sierra}. Sierra says, if you can’t get the user to 95\% reliability within this time, stop trying; you need to redesign the sub-skill. \citep{sierra}

  \begin{figure}[h]
    \centering
    \includegraphics[width=0.8\textwidth]{SierraPractice.png}
    \caption{Moving skills from A (Can't do) to B (Can do with effort) into C (Mastered) can move different ways, depending on how effective the learning is. Deliberate practices focuses on A-B-C, while perceptual expose enables A to C. Reflection allows knowledge to go backwards, to get better at the skill than previously possible. An example might be to teach "Financial literacy". Concepts and factual knowledge (like what income and profit is) might need to move A-B-C, whereas entrepreneurship skills (like taking financial decisions) can move A-C if it becomes intuitive for the user, e.g. via having been exposed to a lot of trial-and-error examples in the app.}
    \label{fig:sierra-practice}
  \end{figure}

  Desirable difficulties applies here, meaning that during deliberate practice, it may feel as if learning gets more and more difficult, but in the long term the user is actually learning more. As a result, less people does true deliberate practice, but they do not get the same reward in return. This needs to be designed for, e.g. using social psychology\todo{Konrad: how?}.

  By deliberate practice, you can practice better. The second attribute of those who became experts, were that they were exposed to high quality, high quantity examples of expertise. \citep{sierra}

  It shows that whenever a skill relies on intuition, we could try exposing the user a well-designed trail and error test. In the case of multiple-choice questions, this could be done by exposing users to very high-quality samples during a very limited time. Perceptual knowledge includes teaching what we think of as expert intuition (like being a good entrepreneurship coach).

  Sierra shows how researchers have repeatedly, by well designed tests, been able to quickly build expertise by trial-and-error feedback. A novice would hazard a guess and an expert would say yes or no. Eventually the novices became, like their mentors, masters of the expertise that could otherwise would have been intangible for long.

  \subsubsection{Learning from Assessment}\label{learning-assessment}

  Knowing what learners know, and don't know, is crucial to effective learning, Luckin \citep{luckin} says.

  Assessment can partly help to design for flow, matching challenge and ability \citep{bruhlmann}, which is effective for intrinsic motivation (see next chapter).

  Moreover, it also has cognitive benefits. It can help to offer appropiate feedback, increase learners' awareness of their learning needs, and give accurate assessment and analysis, and allows learning to be tailored.

  By recognizing differences of students, in their ability to understand what they know and how they can progress, it is possible to ensure that everyone achieves their full potential.

  Effective assessment by a teacher or agent includes individual feedback (task-oriented and informal) and appropiate feed-forward advice. Sitzmann \cite{sitzmann} has studied how questions used to prompt self-monitoring and self-evaluation benefit learning, showing gradual, positive effect on learning. Regarding multiple-choice tests, Nicol \cite{nicol} gives seven principles of good feedback practice, see figure \ref{figure:multiple-choice}.

  \begin{figure}[h]
    \centering
    \includegraphics[width=0.8\textwidth]{multipleChoice.png}
    \caption{Seven principles of good feedback practice \cite{nicol}.}
    \label{fig:sierra-practice}
\end{figure}

  Moreover, research on fixed mindset (I can't do X) versus growth mindset (I can't do X yet) talks about how mindset guides behaviour. In a math game with Dweck's research \cite{dweck-youtube} as a base, students were rewarded by the mentality of "Not yet" and effort versus getting a grade on existing knowledge. Regarding learning, those exposed to a growth mindset mentality, previously having a fixed mindset, got superior results, especially those students previously having difficulties with learning. Regarding motivation, Dweck's research showed that high achievers played to the end, but in the growth mindset version those still played to the end, but so many more lower and medium achievers also stayed until the end. \cite{dweck-youtube} \todo{Viktigt ta med detta i Resultat!}

  \subsubsection{Learning by Thinking: Reflection \& Retrieval Practice}

  Stefano \citep{stefano} suggests that that reflection has been an overlooked area of research for a long time. During the act of reflection, the student develops necessary skills and self-awareness to refine their own learning activities. His results suggests that reflection as an activity that can be more effective than additional learning. This surely applies to the teacher as well, Luckin says. \citep{luckin}

  Stefano found that individuals who are given time to reflect on a task, outperforms students who are given the same amount of time to practice with the same task. But, similar to deliberate practice, it is a desirable difficulty: individuals in the test themselves, had a tendency to believe that allocating time to practice on the task rather than reflecting on it would benefit them.

  %\subsubsection{Retrieval practice}

  When it comes to study technique, Bjork \citep{bjork} as well shows that retrieval from memory is more effective than people who repeat reading the same thing to remember: the more effective students, retrieves from memory.

  One way to use memory retrieval as a study technique, is to ask "What was in that article?", before checking the answer in the article (the flashcard principle). It is an example of memory retrieval that is extremely effective for learning, their research shows. There is a danger with multiple-choice questions, that the student is given no time to reflect on the question and their prior knowledge, before evaluating the alternatives.



%\subsubsection{Not forgetting}

%UCLA Bjork's Learning and Forgetting Lab researches how people forget, and how to design so that people do not forget.

%\include{theory/learning/pedagogical-development/social_psychology}



  \subsection{Methods to Design for Motivation}

Social psychology can guide the design, when there is a wish to make people behave differently. One of the biggest areas of research, is motivation psychology.

Motivation is commonly divided into three areas:
\begin{itemize}
\item Self-determination - the students inner motivation, genuine wishes
\item Achievement - the students motivation to achieve
\item Expectation value - the expectations on the student
\end{itemize}

Koballa \cite{koballa} and Abell \cite{goballa} gives an overview of theories developed for these three fields. Further, Fulmer \cite{fulmer} provides a review of methods to collect data which can be used to study motivation.

Deci \cite{deci} and \cite{ryan} studies self-determination theory, Elliot \cite{elliot} studies achievement theory, and Ecclies \cite{eccles} and Wigfield \cite{wigfield} studies expectancy value theory.

If you have designed for the user's compelling context, Sierra says, the users are already motivated. Their motivation, is to become better. \cite{sierra}.
With a compelling context, the users are already self-determined. Their motivation, is to become better (achieve).

In terms of effective learning and expectation value, a growing research field is training transfer \cite{brinkerhoff}. Before and after is as important as the training itself. To design for this, the leader should be involved with the participants before the training, and communicate expectations. The student should be expected implement the training in everyday life. \cite{brinkerhoff}

Sierra \cite{sierra}, suggests the focus to be how to help users progress (see "Progress and payoffs", achieve), and what pulls them off (see "Cognitive load theory").

\subsubsection{Cognitive load theory}

Sierra argues working on what stops people, matters more than working on what entices them. Thus, a focus needs to be identifying and removing blocks.

Sierra \cite{sierra} describes how humans have scarce cognitive resources, and how to design for these.

Cognitive load theory research is divided into three areas: intrinsic CBT, extrinsic CBT, and germane CBT. Below, to design for these are described.

Intrinsic CBT, needs to be dealt with if the effort is too high. Sierra \cite{sierra}describes two strategies. She first says that according to deliberate practice, if you can not get to 95\% reliability within three 45-90 minute sessions, split skills that can be done with effort into sub-skills. The purpose is to reduce time spent practising being mediocre.

Extrinsic CBT, the way presented to a learner, should be handled via designing to support cognitive resources, Sierra says \cite{sierra}.

Scaffolding is a technique to step by step remove the support wheels for the user, e.g. present information in different ways. Gates' \cite{gates} report shows that in their research, each category of scaffolding demonstrated significant effects on learning.

Also, reduce cognitive leaks by e.g. don't make them memorise, and make the thing you want the user to do, the most likely thing to do (affordances). Everything that takes willpower, reduces cognitive leaks.

Germane CBT, is the work put into creating a permanent store of knowledge. To support cognitive resources, escape the brain's spam filter by making the information essential. Either by designing for the compelling context, or desining for just-in-time learning versus just-in-case, Sierra says. \cite{sierra}

\subsubsection{Progress and payoffs}

Sierra argues that to pull users forward, to stay motivated, progress and payoffs are essential. Both of these, are investigated in terms of motivational psychology.

The feeling of progress can be emphasised by a path with guidelines to help the user know where they are at each step, e.g. for a training. To create a path, she encourages the designer to make a list of key skills ordered from beginner to expert. Then, these are sliced into groups of ranking or levels.

This way, it is possible to design a “belt” path for your context. The first level, should feel like a superpower for the user. The best payoff, is a intrinsically rewarding experiences, according to Sierra \cite{sierra}.

For motivation, the earlier, lower levels should be achievable in far less time and effort than the later, advanced levels. One practice is to try to have each new level take roughly double the time and effort of the previous level. This highly relates to flow.

Caring for the compelling context, why the user wants to learn the skill, are helpful strategies. A sometimes critiqued way of progression is to give the user high pay-off tips, but if done in a fair way, it is a good way for both learning and motivation.

This kind of path map is suberb to simple gamification, says Sierra \cite{sierra}. The statement is in-line with self-determination theory, where e.g. Pink \cite{pink} says that the surprising truth about what motivates us is that drive is fostered by autonomy, mastery and purpose. The most efficient way is therefore to design for having intrinsically rewarding experiences.

Gates \cite{sierra} says that simple gamification as well as more sophisticated game mechanics can prove effective. However, they add that it should be investigated if "simple gamification" (e.g. contingent point and badges connected to learning activities) more frequently focus on lower-order learning outcomes, compared to studies with more sophisticated game mechanics.


  \subsection{Design Thinking}

%\subsection{Digital Learning}

%\citep{edtech-clark}
%\citep{edtech-sjoden}
%\citep{edtech-dangelo}

\input{theory/design/digital-learning/mobile_learning}

% NTA Digital, Om Digitalt Lärande, Att lära med digitala verktyg
% http://ntadigital.se/teacher/tutorings/2


Interaction design talks about the creation of digital artefacts specifically. When it comes to the design process, it is influenced by related areas such as human-computer science, and more recently human-centred design. However, various disciplines suggests different design processes. For example, agile development suggest how do develop software efficiently. Whenever a project is multi-disciplinary, various design processes may need to be combined. Whenever this happens, design thinking (how to think about design) becomes a skill essential to thoughtfully design the process.

\cite{lowgren} writes about design thinking and useful techniques in general, from his interaction design perspective. Service design thinking connects various fields of activity \citep{stickdorn}, and it's methodology relies on being close to the users. While interaction design talks about the creation of digital artefacts specifically, service design talks about the creation of services. As some digital artefacts are used within a service, or can be thought of as both a product and service simultaneously, the combination of the two can be very useful. Service design could help the designer be aware of how such a artefact would need to interplay with its physical environment.

Each discipline holds efficient methods and tools, that can be modified to suit the specific situation even better. From the field of graphic design, mental models describes the perceptions of the user. From interaction design, desirability, utility, usability and pleasurability can be useful principles to evaluate a product. While none of these are a mandatory part of service design, these have been useful in service design projects previously \citep{stickdorn}. In difficult situations, combining different disciplines places demands on the designer. This is where design thinking becomes relevant. Below, relevant methods and tools are briefly described, and what it means to be a good designer.

\subsubsection{A Good Designer}\label{aGoodDesigner}

The result of a method can not be better than the people engaging in carrying out the process \citep{lowgren}. With its user-centered focus \citep{stickdorn}, service design can be said to equip the designer with tools both for reasoning and design ethnography. But it also suits to get to know and design for the learning situation. In learning, the end goal is that the student raises their level of knowledge and expertise, and the design needs to be adapted for this specifically. Central to design for learning is to dig deep into the topic being communicated. In this case, understanding entrepreneurship, understanding exactly what is being taught (the training), and adapting the design after this.

A good designer can deal with the complexities of design: a satisfactory (and surprising) solution or design can be achieved while working in a highly restricted situation \citep{lowgren}. This can be done e.g. by inventing new design techniques. One such example that would suit designing an app for entrepreneurship training in a development country, would be a \textit{field hackathon}. A field hackathon would thus allow that during the training, the topic \textit{and} the users are observed and understood. Then, the app can be tested (in this can a quiz assessment of the trained material). Then, users can be invited to give feedback, suggestions of improvements, and  ideas. For the next day, an improved version of the app is tested, and then the process is repeated.

More examples of how a service design process can be invented to deal with digital artefacts, can be desribed in the chapter \ref{digital-service-design}. However, to do such field tests (like a field hackathon), requires building trust and having an enabling environment, which is where relationships and roles becomes crucial.

\subsubsection{How to Deal with Relationships and Roles}
While a researcher is interested in reality, a designer is interested in what reality could become \citep{lowgren}. Being thoughtful means conceptual clarity from the designer, caring for the vision, and being equipped with appropriate tools of reasoning. These are all good characteristics for a successful project. According to \cite{lowgren}, "real" design is about finding ways to design a project within the existing preconditions and limitations. Being innovative and communicating well with the stakeholders becomes crucial.

There are three roles as interaction designer in particular can take: the computer expert, the socio-technical expert, and the political agent. The trend is increasingly towards socio-technical experts \citep{lowgren}, the middle ground, as human understanding and collaboration is so important. This seems to be a perfect fit with service design, where interaction design is both technical skills and design, and service design can be both design and ethnography. Even more importantly, service design suggests making the whole process co-creative, involving all stakeholders \citep{stickdorn}.

\subsubsection{Thinking of a Product as a Service}

Service design thinking is described as a process of designing, rather than to its outcome. A service's intent is to meet customer needs. If it does, it will be used frequently, and recommended \citep{stickdorn}. As this is often not the case, service design can be applicable to fields including social design, product design, graphic design and interaction design. The result can be a product service hybrid. When designed and considered well, service design shapes the value proposition and desirability of the product for the better.

\subsubsection{Starting the Project}

\cite{lowgren} writes about the beginning of a project: This is where the designer gets involved in design work, establishes a preliminary understanding of the situation, navigates through available information, and initiates all neccessary relationships with clients, users, decision makers, and so forth. Based on all this, she creates a design proposal \citep{lowgren}.


  \section{Service Design Methodology}

Below, brief descriptions of five principles of service design are described according to \cite{stickdorn}, together with how the work is divided into iterations, and examples of tools that can be applied.

\subsection{Principles}
\cite{stickdorn} describes five principles that constitute service design thinking, and how to follow these.

He describes how to follow these principles, by making the process user-centered (e.g. via \textit{design ethnography}), co-creative (involve all stakeholders) and holistic (keep the big picture). Sequencing (visualize the service, and make iterations) and evidencing (make the service tangible) are the two last important principles.

\subsection{Sequencing}
Sequencing the process means splitting the design process into iterations, which consists of a number of steps, which are repeated for each iteration. This is a common denominator with the agile methodology SCRUM, which is often applied in software development.

While service design literature and practice refer to various frameworks, regardless of number of steps, every service design project includes: exploration, creation, reflection and implementation \citep{stickdorn}. \cite{expedition-mondial} suggests a model where one iteration consists of insights, ideation, trigger material, and interactions. See figure \ref{fig:iteration}.

\begin{figure}[h]
    \centering
    \includegraphics[width=0.7\textwidth]{Iteration.png}
    \caption{In the model by \cite{nissar} an iteration consists of Interactions, Insights, Ideation and Trigger material.}
    \label{fig:iteration}
\end{figure}

\begin{enumerate}
\item Interactions, where you are listening, the \textit{Explorative phase}.
\item Insights, which is where you use the Interactions in order to try to understand, the \textit{Understanding phase}. % better word+
\item Ideation, where you find possible ideas and when creation of new version of the app is done, the \textit{Design phase}.
\item Trigger material, where material is developed to test the outcome of our evaluation in the next round, the \textit{Trigger development}.
\end{enumerate}

The iterations should come closer and closer to a desired outcome. It is not always obvious what this outcome is. For each iteration, the process takes the project closer, from Why? to What? to How?, often with overlaps \citep{expedition-mondial}. See figure \ref{fig:iterationprocess}.

%\begin{wrapfigure}{r}{0.25\textwidth} %this figure will be at the right
%    \centering
%    \includegraphics[width=0.25\textwidth]{IterationProcess.png}
%    \caption{Iteration process}
%    \label{fig:iterationprocess}
%\end{wrapfigure}

\begin{figure}[h]
    \centering
    \includegraphics[width=0.8\textwidth]{projectLoop.png}
    %IterationProcess.png
    \caption{The iteration process consists of a number of iterations with different focus, starting with broad strokes, and narrowing down into a concrete product. Between iterations, there is an overlap in "Why?" and "How?", "How?" and "What?", which signals that there is a learning process which means conclusions may need to be quickly questioned as new insights emerge. This is especially important in projects where you work with an unfamiliar target group and there are several uncertainties and constraints.}
    \label{fig:iterationprocess}
\end{figure}

\subsection{Service Design Tools}

There are a number of popular service design tools that follows the five principles, e.g. how to make it user-centered. One is Customer Journey Map, in which an activity (like hosting a youth session) is broken into Before, During and After. Another method is Personas, which \textit{exemplifies} thought users of the app into people with names, having realistic character traits and opinions. The persona's needs can then be thought of when designing. An alternative to Personas are Need groups, where thought users are broken down by their different needs. Instead of designing for a specific person, you design for a person with a specific need. The advantage of Need groups, are that it accepts the view that the same person (Persona) might have different needs, depending on situation. The 5 Why's is a simple method used to dig deep into understanding the interviewee. Variants of the question "Why?"" is repeated five times as a rule of thumb, to understand underlying motives. This method is called "Why-why-why" within interaction design \citep{lowgren}).

Tools to create and reflect can be done via certain work methodology. When you structure and inspire brainstorms, you can ask "What if...?" and do Co-Creation, meaning doing ideation together with stakeholders or users. To create, agile development can be used, which is often suitable for software engineering. The manifesto for agile development is \citep{agile-manifesto}:

\begin{itemize}
\item Individuals and interactions over processes and tools
\item Working software over comprehensive documentation
\item Customer collaboration over contract negotiation
\item Responding to change over following a plan
\end{itemize}

An example of an agile methodology is \textit{SCRUM}, where a project is divided into several iterations similar to a service design approach of sequencing, but also introducing consepts like retrospectives (reflecting on one's work) and sprint demo (demonstrating the results of the iteration to stakeholders) \citep{kniberg}.

There are also some service design best practices: interviews are often done via open questions (encouraging stories) and dialogue can be facilitated with a questionnaire guide. In workshops, post-its are often used, and followed up with specific questions. Service design methodology encourages taking pictures, filming and recording audio, benefiting the analysis done afterwards \citep{expedition-mondial}.

%\subsubsection{Relevancy within Social Innovation}
%\citep{socialinnovation-ehn}

%\subsection{Service Design Thinking}

%\subsubsection{Methodology}

\textbf{The iteration process}

The time in Uganda is divided into three iterations. For each iteration, the result becomes more and more clear. In iteration 1, there is a very broad scope, without digital focus whatsoever, where iteration 2 and 3 gradually introduces the digital solution. See figure \ref{fig:iterationprocess}.

%\begin{wrapfigure}{r}{0.25\textwidth} %this figure will be at the right
%    \centering
%    \includegraphics[width=0.25\textwidth]{IterationProcess.png}
%    \caption{Iteration process}
%    \label{fig:iterationprocess}
%\end{wrapfigure}

\begin{figure}[h]
    \centering
    \includegraphics[width=0.8\textwidth]{IterationProcess.png}
    \caption{The iteration process consists of a number of iterations with different focus, starting with broad strokes, and narrowing down into a concrete product. Between iterations, the overlap between "Why?" and "How?", "How?" and "What?", signals that there is a learning process which means conclusions may need to be quickly questioned as new insights emerge. This is especially important in projects where you work with an unfamiliar target group and there are several uncertainties and constraints.}
    \label{fig:iterationprocess}
\end{figure}

\textbf{One iteration} \\
In the way of reasoning around development and design for learning, the steps for each iteration, see figure \ref{fig:iteration}, might be translated into:

\begin{enumerate}
\item Interactions, where you are listening, the \textit{Explorative phase}. 
\item Insights, which is where you use the Interactions in order to try to understand, the \textit{Understanding phase}. % better word+
\item Ideation, where you find possible ideas and when creation of new version of the app is done, the \textit{Design phase}.
\item Trigger material, where material is developed to test the outcome of our evaluation in the next round, the \textit{Trigger development}.
\end{enumerate}

%\input{theory/design/service_design_stoff}


  \subsection{Digital Service Design}

As there was a unfamiliar target group - mostly young Ugandians with little or no experience of smartphones - service design thinking would benefit true understanding of cultural context and in-depth empathy for the end users.

Tools and methodology in service design were chosen with the help of Expedition Mondial in Stockholm, who provided education and coaching.

At the same time, the end result would be a digital artefact, which is not common in service design. While the app could be though of as a service, the tools and methodology would need to take this in mind. More suitable approaches would be Agile methodology and Interaction design. These areas, were familiar to me as a computer expert.

This led to the joined development of a Digital Service Design method, created by me and Expedition Mondial. The method combines the benefits of Service Design, Agile Methodologies (namely SCRUM) and Interaction Design. Its purpose was to contribute a holistic approach to the design solution for the specific target group. \cite{nissar-linkedin}

\subsubsection{A "Service Sprint"}
In Digital Service Design, an iteration is called a "service sprint". Similar to service design, it includes four steps: insights, ideation, trigger material and interactions. It has added methodologies from both agile development and interaction design, making the process more suitable. For example, interactions can include mini-service sprints.

\subsubsection{Insights: Analysis, Retrospective \& Stakeholder feedback}
  Insights consists of analysis (service design), but also a retrospective (SCRUM) and stakeholder meeting (service design).

    In the analysis, the app is evaluated (in terms of interaction design - pleasurability, usability, utility and desirability), and quantative data is processed (often by clustering data points) and compared with qualitative data (quiz results and questionnaires). This produces an analysis overview of the result.

    In the retrospective, the design process is evaluated ("start doing, stop doing, continue doing"), and changes to the design process are suggested for the following iteration.

    Both the result analysis and the design process analysis is then presented during two stakeholder meetings (service design), structured as "sprint demo's" (SCRUM), with the purpose of getting feedback.

    The first "Expert meeting" informs the next iteration's design process (with Expedition Mondial), while the second "Partner meeting" informs the next iteration's delivery (with Linköping University and YougnDrive).

    From the insights, a product backlog (SCRUM) is filled with needs and ideas informed by 1) user needs and 2) stakeholder needs.

\subsubsection{Ideation: planning interactions and delivery}
  Ideation consists of a sprint planning (SCRUM). There is one technical planning part, where this sprint's most important user needs from the product backlog are reformulated into stories. There is one test planning part, where interactions are determined and booked.

    \textbf{Technical planning: }

      Ideas are formulated which would satisfy the user needs. This is often a iterative process, which happens in dialogue with chosen experts and entrepreneurs in technology, design and education.

      To plan implementation of the ideas, every technical task are laid out, measured in time and prioritized. The least prioritized tasks can thus be cut or moved to the next iteration, in case it is necessary.

    \textbf{Interactions planning: }

      If the technical planning has been realistic, it is time to determine what this iteration's interactions should look like. How will this be tested?

      The interactions activities are chosen (what, how, when), so that these are communicated to Plan International, who schedules the days I will visit, and solves the needs to the best of their ability.

  \subsubsection{Trigger material}
  Trigger material is about preparing the interactions (field visits, interviews, app tests, workshops) and creating the lo-fi (pen and paper) and hi-fi prototype (developed app) to be tested with the users.

  To track the progress and plan effectively, each day starts by a daily standup, where today's targets are set, ending by reflecting if the targets were met. If they were not, either the design process needs to change, or something needs to be cut short.

  \subsubsection{Interactions: with "Service Mini-Sprints"}
  Interactions always consists of a sprint demo with the users with the lo-fi or hi-fi prototype. During the development process, these are formative tests, while for final app evaluation, this is a summative test.

    Group tests are facilitated as workshops. Often, a scenario is presented, devices are given, results are submitted, followed by an open discussion.

    Field tests are facilitated as naturally as possible (using the before, during, after technique). I observe how the coach does the job today, tests and observes if the app fits into the process, followed by an interview.

    These tests always informs what steps to be taken next, both in terms of app development and interactions. Instead of waiting for the next iteration to do these changes, I do what I call a "Service Mini-Sprint".

    \subsubsection{Service Mini-Sprints}
    The insights gathered during the day allows for last-minute adjustments of coming pre-planned workshops (co-define, co-create or co-refine) or field visits (change of interview questions), that can sometimes happen the same day.

    To take advantage of the precious time with the coaches, at the end of the day, app improvements are made and tomorrow's design process revisited.

    This means, that already the next day, an improved version of the app can be tested. If I was not satisfied with a workshop format, it has been modified.

    These mini-sprints allows for very fast iterations, which can sometimes accelerate the outcome of the visit.


  \subsection{Methods for Data Analysis}

\subsubsection{Visualizing Data}

Here, each step of the visualization pipeline is presented, allowing analysis of data.

The Visualization Pipeline describes the process of generating an image from the data: \cite{timo-ropinski-liu}

\begin{enumerate}
\item Data acquisition ($\,\to\,$data are given)
\item Data enhancement ($\,\to\,$ data are processed)
\item Visualization mapping ($\,\to\,$ data are mapped to for example a geometry)
\item Rendering ($\,\to\,$ images generated)
\end{enumerate}

% Timo Ropinski, Scientific Visualization Group, Linköping University, TNM067 - Scientific Visualization, 9/12/2014)
% https://drive.google.com/drive/u/0/folders/0BzlK1PD8EE75bHIxcXRQNWpRMm8

Data acquisition presents how data was acquired.

Data enhancement explains how the data was processed.

Visualization mapping is the process of mapping data to e.g. a geometry.

Finally, rendering allows images to be generated, presented in 2D.

\subsubsection{Calculating Correlation}

Calculate means, follow formula. Cumbersome to do with all of the axises against all the agises.

This can be done in Google Sheets as well as the R programming language.

\textbf{In Google Sheets: }

It is clear that analysis in Google Sheets can only go so far. It can be greatly helpful to sort by multiple columns (e.g. first by Manual?, then by School level, then by Quiz 3). However, it takes a long time to filter the data on multiple parameters, and the work easily becomes tedious. For some applications, it may not be viable to discover the data using this approach.

One approach is to calculate and compare means on a "control"  with a response variable.

\textbf{Psuedo-code in R would be: }

\begin{verbatim}
x1 = c(1,2,3,1,5,6)
x2 = c(2,3,4,NA,6,7)
cor(x = x1, y = x2)
cor.test(x1,x2)
\end{verbatim}

\subsubsection{Visualizing Correlation}

In Google Sheets, color scale can be used to give different column values different colors.

It is still hard to compare all of the axises towards all the axises, and it is not a scientific approach.

\todo{Include figure}

In R programming language there are more powerful tools for visualizing correlation, e.g. using a "Correlation Heatmap".

Psuedo-code in R would be:

\begin{verbatim}
random_matrix <- matrix(rnorm(100), nrow = 10, ncol = 10)
random_matrix[1,1] <- NA
colnames(random_matrix) <- paste("V",1:10)
cor_mat <- cor(random_matrix)
heatmap(cor_mat, keep.dendro = FALSE)
\end{verbatim}

The result would be:

\todo{Include figure}

\subsubsection{Calculating Logistic Regression}

A limitation with correlation is that only two dimensions can be compared with each other.

With multiple-variable data, Logistic Regression is helpful if our response variable can be a logistical dimension (e.g. women or male, used manual or not), while linear regression needs to be used if it is a linear or nominal scale (e.g. age and city respectively).

In either case, the first step is to determine a response variable: the variable I want to compare against, e.g. is there a difference between men and women? In my case, it needs to be a quantative measure of: "Have you learned anything?".

If I add more variable, e.g. also adding if a manual was used, this is called my "control". It is possible to add as many controls as possible.

If linear regression, then I need to determine a quantative measure of ("How much have you learned?").

In Google Sheets, this is not effective to do. R, however, is a very suitable tool.

First, the data is loaded, e.g. as a CSV file. Then, we tell R which the N/A values are, e.g. "N/A" or "Vet ej". We use this to filter the data.

Then, each column we want to use is converted into a factor.

When factors, a model can be created, e.g. using the General Linear Model. A different family can be selected, e.g. binomial.

Then it is possible for R to show this data, showing the coefficient, the Pr value, and others. See code below.

\begin{verbatim}
mydata <- read.csv("Development/R/quizResults.csv", na.strings = c("N/A", "Vet ej"))

mydata$y = ifelse(test = is.na(mydata$Quiz.9..y.n.1st),  yes = 0, no = 1)

mydata$y <- as.factor(mydata$y)
mydata$Help <- as.factor(mydata$Help)
mydata$Sex <- as.factor(mydata$Sex)

mymodel <- glm(formula = y ~ Pre.test.score + Sex, data = mydata, family = 'binomial')

summary(mymodel)

plot()
\end{verbatim}

For analysis, looking at the summary, coefficient (e.g. -1.0704) shows either a negative or positive correlation (in this case -7\%) for what I compare with as a response variable.

To be significantly significant, a common measure is that the Pr value ("the p-value") needs to be higher than 0.05. If the p-value is higher than 0.05, meaning it is significant with a 95\% probability.

\subsubsection{Analysing data with a Parallel Coordinates Visualization}

To learn how to analyse the data, Une-terre \cite{une-terre} was consulted. % http://une-terre.blogspot.se/2012/09/parallel-coordinates-read-out-patterns.html
He writes "||-coords are a data visualisation which allow you to "read out" the relationships and trends between your dimensions. Positive relationship (correlation), negative relationship (invert), or no relationship (random)."



\section{Involved Parties in the Project}

Involved parties in the project are the current author, supervisors, stakeholders and experts. Below, the responsibilities of these are more clearly laid out.

\subsubsection{The Current Author}
It is needed to take on several roles in the project by the current author: most notably that of a project leader, designer and developer. It is needed to balance stakeholders' different opinions and requirements, and caring for the vision (see section \ref{aGoodDesigner} A Good Designer). The motivation doing the master thesis is three-fold: learn as much as possible, create a successful project, and finish the master thesis.

There are two groups, with the current author included in both of them, which gather at the end of each sprint for a check-up meeting. The Expert group consisted of Expedition Mondial and LiU Innovation. Expedition Mondial could help with the design process, and LiU Innovation could offer input on social innovation. The meetings mostly lasted for one hour. The Partner group consisted Iliana Björling from YoungDrive, and Lena Tibell and Konrad Schönborn from Linköping University. In Partner meetings, The Insighs from each iteration was presented and discussed. Then possible decisions were laid out, followed by discussing the alternatives. Outside of these groups, these people can also give advice in certain situations. For specific areas, there are also some experts which have been beneficial during the projects. Below, the whole team is explained: % Then I tell them about which decisions has been taken and why.

\subsubsection{Supervisors}
The supervisors are from YoungDrive and Linköping University. The YoungDrive team consists of Iliana Björling, founder of YoungDrive, and Josefina Lönn, country manager in Zambia. They are both helpful in giving knowledge on the entrepreneurship education program, and giving support. The Linköping University team consists of Lena Tibell, Professor, and Konrad Schönborn, Doctor, within the Department of Visual Learning and Communication.

\subsubsection{Stakeholders}
The stakeholders are considered YoungDrive and Plan International. \textbf{YoungDrive} is the client of the work, and their needs should be satisfied. This person is mainly represented by Iliana Björling, who is part of the YoungDrive Strategic Management Team. Using service design, the project leaders in Uganda and Zambia, are also considered stakeholders: Josefina Lönn in Zambia, and the two co-project leaders in Uganda. Finally, the most important stakeholder of all according to service design, is the actual users: the coaches. They should be the main consideration of the work.

\textbf{Plan International} is the organization allowing for all the interactions with the end users in Uganda. A similar organization is operational in Zambia. They are the ones that are providing facilities, organizes transport, etcetera. They in turn, have the organization Community Vision, which organizes the coaches. If Plan International or Community Vision does not appreciate of the project and the collaboration, then the interactions with the coaches will not be possible.

%\textbf{Linköping University} is a stakeholder, as the supervisor (Lena Tibell) and examinator (Camilla Forsell) determine if the work is a valid master thesis or not. Also, LiU Innovation is interested in supporting continued work with the project, and their representative Peter Gahnström gives advice on social innovation and how this project can continue in the future during expert meetings.

\subsubsection{Experts}
Since the development country context is new to the current author, there are also specific experts advised in the project. For design process, Susanna Nissar and Erik Widmark from Expedition Mondial has supported with all of their knowledge within service design. Julien Tantege, Research Specialist at Grameen Foundation, has been kind to offer support before and during the work, sharing their insights from related work, and giving feedback during ideation. She has experience doing technical development for rural areas. For pedagogical development, Henrik Lundmark from edtech startup Knownly in Sweden has given support with regards to building skills within digital learning. For feedback for how the work relates to social innovation, Peter Gahnström at LiU Innovation has offered feedback.



\section{Subjects (Participants)}

\subsection{Roles}

\subsubsection{Country Manager}

\subsubsection{Project Leaders}

\subsubsection{The Community Based Trainers}

The CBT's are often volunteers, receiving a small scholarship from Plan International. They are often business owners themselves.

Thus, the CBT's can be described as social entrepreneurs. As Mitchel says about entrepreneurship \cite{mitchel}, motivation does not need to be wealth accumulation anymore. The activity of entrepreneurship contributes to society, in a way that is not caputed by the commercial entrepreneurship literature.

Many of the YoungDrive participants are driven by that their business can have an impact on their community, as well as take them out of unemployment or increase their current livelihood.

\subsubsection{The Youth Mentors}

\todo{Write about the YMs}

\subsubsection{The Youth}

\todo{Write about the youth}

\subsection{Training Material}

Each youth is given a participant manual, describing each week of the 10-week YoungDrive program.

CBTs and Youth Mentors are also given a Coach guide, which describes how to carry out and teach each week's topic during the youth training.

\subsection{Data on Coaches}

\subsubsection{Tororo, Uganda}
In Tororo, there are 2 Project Leaders (PLs), 19 Community Based Trainers (CBTs) and 8 Youth Mentors (YMs).

Both (2) of the PLs, are also YMs.

14 of the CBTs, have a YM.

The subjects, lives in all parts of Tororo: 9 in East, 9 in West and 11 North, with different distances to town.

There were 26 out of 27 possible respondents among CBTs and YMs in Tororo, when in 2015, a statistics summary was carried out.

\subsubsection{Phone/power/data among CBTs and YMs}

Of these, 26 (100\%) had a cell phone, 3 (12\%) had internet on the phone, 0 (0\%) had power at home, 3 (12\%) had solar energy at home, and 4 (15\%) knew how to write on a computer.

Of the 3 that have internet on their phone, 2 (67\%) are using internet each day, and 1 (33\%) is using internet once a week. They mostly use Facebook, followed by email.

\subsubsection{Companies among CBTs and YMs}
In Tororo, 26 (100\%) are running a business, 14 (54\%) are running two businesses, and 1 (4\%) is running three or more businesses.

In Tororo, the businesses range from: ananas, water melon, onion, chili, bakery, catering, corn, beans, fabric, plastic products, bird farm, milk, fish, ground nuts, cabbage, tomato, hairdresser, sewer, shop and rice.

\subsubsection{Kabwe, Zambia}
In Kabwe, there is 1 PL, and 10 YMs.

\subsection{Project Leaders in Uganda}
In Tororo and Kamuli, there were 6 out of 6 possible respondents.

6 (100\%) have phones, 1 (17\%) have internet on phone, 2 (33\%) have power at home, 3 (50\%) have solar panels at home and 6 (100\%) are able to write on a computer.

In Tororo, there are 2 PLs. Christine's business ranges from: bakery, corn, pig farm and plastic products.Patric's business ranges from: silver fish, beans, corn, and bird farm.

In Kamuli, there are 4 PLs. Their businesses ranges from: selling office supply, boda boda, bird farm, pig farm, green pepper, corn, cabbage, tomate, aubergine, chipati ("bread"), chilli, and charging of cellphones.

\subsection{Youth in Tororo}
The youth are the ones receiving the training from the CBTs and the YMs. In the 2015 report, with 225 respondets from Tororo, these were the statistics regarding phones, power, data, language, businesses and the most popular companies:

\subsubsection{Phone/power/data}
99 (44\%) have a cell phone, 9\% have internet on their phone, 3 (1\%) have power at home, 22 (10\%) have solar panels at home, and 20 (9\%) are able to write on a computer.

The mostly use Facebook, followed by Google and WhatsApp. A few people are using it for Twitter, email, news and school information.

\subsubsection{English skills}
In Tororo, 129 (57\%) understands when someone speaks English, 116 (52\%) can speak english, 133 (59\%) can read English, and 132 (59\%) can write in English.

\subsubsection{Businesses}
In Tororo, 165 (73\%) of the youth runs a business. 60 (27\%) are not running a business.

The top 8 most popular businesses in Tororo, with 134 respondents, are corn, cassava ("potato"), saloon, fish, making of bricks, beans, brooms and rope. These range from 9 for corn (6.7\%) to 5 for rope (3.7\%).

\subsection{Coaches in Zambia}

In Zambia, the coaches have higher education, and better access to technology.

6/10 has power at home.
3/10 knows how to write on a computer.

\todo{Add more info about Zambia coaches here}


\section{Implementation}

This chapter describes the implementation of methods for each iteration's interactions: in regards to data, data analysis, method, and method analysis. Figure \ref{fig:iterative-process} is made to assist the reader in which methods were used for each iteration, in terms of research and app tests.

%study design, application development, and for data analysis theory.

\begin{figure}[h]
    \centering
    \includegraphics[width=1.0\textwidth]{iterativeProcess.png}
    \caption{The iteration timeline shows the process from iteration 1-4, by different colors. Each iteration leads to a loop, with the design situation as the focus, in which interactions can happen that is related to either research ("What's ...?") and an app test ("How can we ...?"). This output from the loop, then gives force to the next iteration.}
    \label{fig:iterative-process}
\end{figure}

Methods w/ Analysis
----

5 Interviews - Notes - Questionnaire for Customer Journey
1 Customer Journey - Clustering - Personas
2 Youth Sessions - Shadowing - Needs
1 Coach Stayover - Empathy - Design Ethnography

5 Training Days - Observations - Understanding entrepreneurship training
5 Hackathon Days - Interviews - Notes and Sketches
2 App Workshops - Co-Creation/Co-Refinement - Sketches and Needs

3 Field

Acitivity
App test observations (group)

App test observation (individual)
- Affective reactions (5 Why's, think aloud)

Analysis: Interaction design evaluation (desirability, usability, utility, pleasurability)

Customer Journey Map
- Activities
- Behaviour

Written responses (individual)
- Right/wrong
- Time
- Number of tries

Interviews
- New insights

Data Collection w/ Analysis
----

Customer Journey Map w/ clustering

Pre-study w/ Quantative analysis

Written quiz responses w/ Quantative analysis

Digital quiz responses / Quantative analysis + Statistical analysis + Parallell coordinates

Quiz questions 1 w/ Bloom analysis
Quiz questions 2 w/ Bloom analysis

%\input{implementation/iteration-1}

%\input{implementation/iteration-2}

%\input{implementation/iteration-3}

%\input{implementation/iteration-4}


\section{Study Design \& Data Collection}

Using novel methods like \textit{service design} when developing the app according to research question and \textit{data-driven design} and interviews for understanding interaction according to research question.

%Har gått igenom planeringsrapporten lite noggrannare idag och ser två saker som vi kanske ska borde fånga upp under arbetets gång.

% Under 2 Purpose står det ett upplevelsemål från Young Drive. Bör vi mäta detta upplevelsemål om det stämmer med deltagarnas faktiska upplevelse, d v s ska vi försöka få in det under 3 Research Questions?

% På våra avstämningsmöten borde vi också följa upp dina Research Questions så att kundinteraktionerna och servicedesignmetoden tyligt leder dig framåt mot dessa mål.
%* Reflektioner på vilka designprinciper som bör väljas? (utifrån kundinteraktioner)
%* Reflektioner angående tekniska begränsningar?
%* Reflektioner på processen?

\subsubsection{Creation of Design Process}
As there was a unfamiliar target group - mostly young Ugandians with little or no experience of smartphones - service design thinking would benefit true understanding of cultural context and in-depth empathy for the end users.

Tools and methodology in service design were chosen with the help of Expedition Mondial in Stockholm, who provided education and coaching.

At the same time, the end result would be a digital artefact (an app), which is not common in service design.

While this product could be though of as a service, the tools and methodology would benefit to borrow from Agile methodology and Interaction design.

I'm the computer expert kind of designer \citep{lowgren}, adjusted to agile methodology and interaction design, but aspiring to be a socio-technical expert. Expedition Mondial are experienced with service design, aspiring to be more of computer experts.

This led to the joined development of a Digital Service Design method, co-created by the both.

%Expedition Mondial helped with a method for creating a MVP of the digital support for the coaches, so that the app was developed from the perspective of the end users and the education and a "learning by doing" mentality.

%The suggested design process was designed with them after a start-up meeting on Skype, and an education day in Stockholm. During that day a crash course in service design was given, then creating a common plan for the future work based on my needs (see Appendix: Original Time Plan \todo{Add reference}). They also recommended service design literature. These were the methods chosen in each iteration.

The result is that the design and development phase in Uganda is an iterative process with the human in focus. The process is built on top of service design process and methodology, while in-line with digital design practices.

\begin{figure}[h]
    \centering
    \includegraphics[width=1.0\textwidth]{iterativeProcess.png}
    \caption{The iteration timeline shows the process from iteration 1-4, by different colors. Each iteration leads to a loop, with the design situation as the focus, in which interactions can happen that is related to either research ("What's ...?") and an app test ("How can we ...?"). This output from the loop, then gives force to the next iteration.}
    \label{fig:iterative-process}
\end{figure}


\subsubsection{Implementation of Design Process}
See figure \ref{fig:iterative-process}. There were four iterations. The first iteration follows Service Design, not starting the app development, while the other three follows the new methodology, Digital Service Design.

In iteration 1, there is a very broad scope, without digital focus, where iteration 2, 3 and 4 introduces and narrows down the project into a digital solution.

Expedition Mondial gave support in each iteration, helping with refinements of each iteration as learnings happened along the way, and they were able to educate me during the different stages with methodologies whenever necessary.


\subsection{Iteration 1}

% How was this iteration designed?

Following the service design sequencing, the first iteration had a very broad scope and truly is a service design iteration: "From your perspective, what is it like being a coach?". \footnote{A coach meaning either a Community Based Trainer (carrying out all the trainings), or a YoungDrive coach, depending on who was asked the question.}

Lowgren's though about how to start the project was used, meaning that the purpose was to get a preliminary understanding of all important aspects, and build relationships with all stakeholders.

Insights depended heavily on interviews with all the stakeholders  (2 with Plan International, 3 with YoungDrive), and local experts (1 visit each at Grameen Foundation and Designers without Borders, 1 workshop with Mango Tree), since no Interactions with users had been made yet. Also, I immersed myself with the Uganda tech scene as possible, from the new home and office in Kampala, working at the tech hub and co-working space Hive Colab.

Ideation were about creating a questionnaire guide for the interviews, a co-creation workshop using "Customer Journey Map", and identifying how the app test should be designed to test their existing knowledge (and be informed of the design preferences of the YoungDrive app).

Trigger material was the finished questionnaire guide (constructed with Expedition Mondial) a written plan for the co-creation workshop ("A day as a coach"), and a written plan for testing the quiz app Quizoid and the language learning app Duolingo, and a schedule for the interactions.

The interactions were focused on design ethnology, getting to know and learn from people in a different culture, namely the coaches. The focus was on the their needs, motivations, and context.

To accomplish these, four days were spent in Tororo, with one day of travel. There were four face-to-face-interviews,
one meeting with Plan, one meeting with the local partners, two workshops, one coach stay-over, and two youth session visits.


\subsection{Iteration 2}

This time, the iteration has a more detailed scope, with a hypothesis on what needs the app should meet in the end, and create lo-fi and hi-fi trigger material to meet those needs.

A co-creation workshop started the interactions, followed by repeated app tests at minimum one session per day, always followed by a feedback round, so the app and the tomorrow's question set creation could be improved for the next day. At the end of the week, there was a co-refinement workshop of the current hi-fi material, and also lo-fi material for the new version of the app.

\subsubsection*{Creation of questions}
Project leader Josefina in Zambia refined Iliana's first question sets, prepared for my visit in Zambia. Josefina created question sets with Bloom at the back of her head, also taking into account the structure and the order of the coach manuals, what it means being a coach within the topic, and lastly scenarios.

\subsubsection{Trigger material used}
A hi-fi trigger material was done, a very basic quiz app, keeping it as simple as possible (see Application Implementation, Iteration 2). All of the devices (tablets and smartphones) that I had available were brought to Zambia.

I added Josefina's questions to the app, and installed the app to all of the devices. This process was repeated for all the days, Sunday-Friday.

\subsubsection{Design workshop \#1 in Zambia}
The coach training started with me having a design workshop with the coaches, not showing them the app that I had created. The co-creation workshop was made to identify important functionality in the minds of the coaches.

\begin{enumerate}
\item Since the knowledge about smartphones and apps were low, I started by introducing these topics.
\item All were familiar with Facebook, so thus I showed the Facebook app. Me wanting to know what the app would look like if the coaches would have designed the app, I first needed to train them how to design an app via drawing wireframes.
\item Using postits, they started with during limited time drawing the start view from the Facebook app.
\item Then, they were asked to draw what they thought happened on the friend icon click, drawing the view on another postit.
\item Then, the mission of the YoungDrive app was described. They were then divided into two teams, having limited time to draw the best imaginable YoungDrive coach quiz app they could. First, they designed the app from the top of their heads. They then pitched their results to each other.
\item On the next iteration, they were to suggest and design improvements how the app should be designed to improve learning, not only assessment. They then again pitched their results to each other.
\end{enumerate}

\subsubsection{Assessment via quiz}
At the end of each day, the app was used to test the coaches' knowledge. Each coach got either a smartphone, tablet or computer. The coach first took the quiz for the most recent session, and could then choose what to do next.

As there were no back-end developed, Josefina by hand documented the scores of each coach, writing the name of the coach, the session, number of correct answers, and what questions had been answered wrong.

Josefina then, when planning the next day, looked at the statistics, looking for trends that would inform the sessions for the following day.

She also evaluated the quality of the questions, before creating the new question sets for the next day.

\subsubsection{Experimenting with quiz before or after the session}
Since the coaches appreciated the app so much, we felt tempted to try what would happen with fun and learning if we tried using the app \textit{before} a session instead of only after. During the rest of the week, we continued, finally finding preferences and tendencies from the coaches, via observation, interviews, and survey.

\subsubsection{Experimenting with design of questions}
During the week, extra tests were done to test the following:

\begin{itemize}
\item Number of questions per quiz
\item Single-answer questions or multiple-answer questions
\item Framing of questions
\item Challenge level of questions
\item Determining what made a question hard
\end{itemize}

\subsubsection{Interviews with Josefina}
At the end of each day, an evaluation interview was held with Josefina. At the end of the week, a final interview was held.

At the end of Day 5, Josefina and I discussed what it would look like to not record the answers manually, but pushing the results online. A co-creation workshop was held, where she drew an Educator Dashboard.


\subsection{Iteration 3}

Iteration 3 had an even more detailed scope. Since the app now succeeds with the first use case, the coach training, not the focus could be on "learning at distance".

A requirement from Josefina was also to test if the app created in Zambia could work also in Uganda. All the quiz questions would need to be converted from the new manual to the old manual, since both structure and content had changed.

To test on all of the coaches in Uganda, it would be preferable if data collection would happen via the app instead of manually, since there would be more than 10 test subjects, which had been the limit in Zambia.

A future requirement was that quiz responses would need be available to the teacher. This means that there needs to be a database, but also a login, so individuals are traceable.

How can login and the database be implemented in the best possible way?

The insights on learning also needed to be considered:
\begin{itemize}
  \item Are coaches really learning via the app, especially learning to be better coaches?
  \begin{itemize}
    \item How can questions be formulated in a way that teaches entrepreneurship, which is so practical?
  \end{itemize}
  \item How can the current multiple-choice quiz app be improved, to:
  \begin{itemize}
  \item reduce guessing
  \item improve confidence
  \item encourage learning
  \end{itemize}
\end{itemize}

Thus, the study design of Iteration 3 became very important. A lot of development and ideation was done.

Also, instead of only testing the app in Tororo, a test was held in Kampala, to get feedback from an entrepreneurship student.

For the interactions, a big app test was held, and also a co-creation workshop was held.

Before the workshop, the wished functionality and goals were well formulated. It was also discussed beforehand how to best design the workshop, together with Linköping University and Expedition Mondial.

Questionnaire 3 was created, used after the test. After interviewed in a big group, they were divided into co-creation workshop groups, with a presentation in the end.

\subsubsection{Aim}

To get an app suitable for learning, it was determined that the pedagogical model behind the app needed to change, emphasising feedback.

The aim was to score higher on Bloom's revised taxonomy, while still including multiple-choice questions in the app.

\subsubsection*{Trigger material}

Josefina was given a task to create a quiz "Are you ready for Session 9?". The aim of this quiz, was partly to score higher on \textit{Bloom's revised taxonomy}, partly to test if Correct Structure and Time Management could be assessed using multiple-choice.

Also, the questions were translated from the new manual into the old manual, which is used in Uganda.

\subsubsection*{Interactions}

There was another partner meeting, with Plan International and Community Vision present. There was an app test with all of the coaches, "Testing the YoungDrive coach app", followed up by splitting into six workshop groups based on solving different problems discovered during the test.

The following day, there were three field visits to CBTs, observing how they prepared themselves for a youth session, and then testing the app for assessing and becoming prepared for a session.

The last day, there was a co-refinement workshop ("Usability Improvements") and one co-creation workshop ("Educator Dashboard") held in parallel, with 3 CBTs and 1 project leader respectively.


\subsection{Iteration 4: Uganda Summative Test}

The focus of iteration 4 was a summative test. First, a pre-test was carried out in paper, including questions about the coach and an entrepreneurship quiz, based on a well-known study \citep{general-entrepreneurship-quiz}, see Appendix \ref{cha:pre-test}. During the test, this was the first time that the app could send data to the server. Data was sent whenever a quiz was started, and whenever a quiz was finished. The group was divided into two, the ones who brought manuals and they who did not. Those that had brought manuals, could use these with the app, see figure \ref{fig:appevaluation}.

\begin{figure}[h]
    \centering
    \includegraphics[width=0.7\textwidth]{appevaluation.jpg}
    \caption{Coaches answering the app questions for topic quiz 3 on Financial Literacy, and the coach guide quiz 9 on Action Plan.}
    \label{fig:appevaluation}
\end{figure}

After the test, every coach was divided into one or three groups, on random. In these groups, they were asked:

\begin{enumerate}
\item Why do you think you were correct or incorrect?
\item Do they like the app?
\item Are you stimulated by the app?
\item What did you like?
\item What did you not like?
\item When do you want to use the app?
\item When are you not able to use the app?
\end{enumerate}

To analyse the paper-submitted data, all of this was combined first into a Google Spreadsheet (the app results were also recorded in paper, but only as a backup). Data collection was done by the app itself, which pushes data to server whenever online (it saves quiz start, and quiz finish).

%The next day, a small app evaluation and co-creation workshop was held for the Educator Dashboard, and the final version of the app. Also, a test was done with the Plan Tororo staff.

%Back in Kampala, a presentation was held with Plan International. Back in Sweden, a presentation was held with the YoungDrive Strategic Management Team.



% Application implementation
\section{Application Implementation}

In this section, the prerequisites for the app is described, from the perspective of the user, stakeholders, and the developer.

\subsection{User Needs}

The technical constraints for the project, would need to affect the technologies used, if the project would be user-centered. On the client side, the app would need to be mobile and web based, consider non-access to internet, and not use a lot of battery, to work for the coaches of YoungDrive. That the app should be simple to use in this cultural setting leaded to design constraints and needs for evaluation.

\subsection{Stakeholder Meeds}

As the project was only three months, and the first month would be without digital development, time constraints were massive. However, to be able to answer how design affects learning, evaluation was needed to be done via data collection.

If no evaluation, there would be no need to write code, instead working with a low-fidelity prototype using pure design tools. Now, a data-driven approach was needed to measure, and therefore an app needed to be developed. On the server side, a database and API would be needed, to pull data from the database and push data from the client. Since internet was not always available, the client must be smart in its usage of pushing and pulling data. This would need to be investigated further into the project.

\subsection{Devices are prepared}
As most of the coaches did not have smartphones or tablets, enough smartpones and tablets were brought with me from Sweden, either donated, borrowed or bought devices. These were a combination of Android and iOS, smartphones and tablets, so the app could be tested on as many platforms as possible. During the user tests, also using a laptop would be tested.

%\subsection{App/Web Development}
%Early in the project, it was thought that existing tools could be used, instead of building the app from scratch. E.g. using existing tools like Knowly or Typeform\footnote{examples include https://showroom.typeform.com/to/ggBJPd and https://showroom.typeform.com/report/njdbt5/dIzi} during the first iterations for understanding users, and during development e.g. the Typeform API (http://typeform.io/). The Typeform API allows developers to create surveys from within their own applications or systems.

\subsection{Choosing frameworks for creating the app}

In the start, Ionic and Meteor were both tested and compared with each other. It was decided that Meteor was the best way forward, partly because it would allow the app to be accessable on the web as well. %\todo{Add from mindmap}

React.js was chosen as the front-end framework, having integration with Meteor and being relatively easy to learn and fast for development.



Since Meteor was chosen, a multiple-choice quiz tutorial in Meteor was used to guide the first version of the app. Modifications were made, for example making it responsive and changing it to YoungDrive's graphic profile.

The app was pushed to GitHub, and first hosted on Meteor free storage, available via youngdrive.meteorapp.com. For Android and iOS, it was made possible to install the app from the computer.  For each day of the training in Zambia, new quizzes were added to the app, which created a belt path (see \ref{progress-payoffs}).

After iteration 2, a different hosting platform was needed when the Meteor free tier was removed, where Heroku was chosen. Staging environment using Heroku allowed changes on specific GitHub branches to deploy updates automatically on Heroku servers. The MongoDB database was created using the Heroku plugin MongoLab. A Meteor build-pack was used to allow Meteor to be used with Heroku.

It was also tested to upload the app to Android Play Store. The neccessary steps from Cordova needed to be followed, screenshots needed to be uploaded, and some administrative tasks. After this, it only took a day for the app to appear on the Play Store, and everything worked satisfactory.


\subsection{Iteration 3}

For me, the user's first feeling of a superpower is a hint of becoming a Certified coach. \todo{This can be commented in the Future work}

On the client, as components grew, there was a need for a client-side router. The Meteor plugin Flow Router was used, as it was very popular with good integrations.

\subsubsection{App for Learning}
This was not much harder than to add new components and functionality for learning. The hard part, was the ideation, deciding on what ideas and what design was the best. For this, see Result \todo{Add reference to Result}.

\subsubsection{Login, Database, and Meteor upgrade}
In order to store data per individual, a database and login would be needed. Meteor upgrade from 1.2 to 1.3 was made to do this easier, but ended up being the reason this was not implemented in Iteration 3. Below, the work is presented.

\textbf{Login}
To record data per user, would require login. This would be a usability issue for most problems, being 1st-time smartphone users. They need to find it intuitive, user-friendly, and be able to remember the password in the future. A lot of different suggestions were through the ideation phase.

The simplest login possible was chosen: a 3-digit code, which was to be given to each coach during the test.

%Jag pratade med flera om detta, Expedition Mondial och Grameen. Från EM lärde jag mig att de trodde min idé med en färdiggjort lista med coachernas namn (vi vet ju vilka som är i Tororo) skulle fungera, och från Grameen fick jag höra om dera erfarenhet att de validerat använda samma approach, med en PIN (längre än 4 siffror dock), men att de inte nailat konceptet ännu, och att de också itererar på sin approach för nästa uppdatering av LedgerLink.

Meteor had limitations with their auto-login module, which is very fast to implement. It forces username and password, and instead I wrote the login myself.

The front-end was not problematic, however, implementing server-client communication so that it worked online and offline, was.

%Tyvärr har också Meteor begränsningar med deras auto-login-modul. Den tvingar både användarnamn och lösenord, och har automatiskt registrering. Går det att stänga av? Jag kan skapa användare och lösenord åt alla, och funderade på hur jag skulle generera lösenord. Ett förslag blev att bara registrera deras förnamn, och sedan skapa lösenordet baserat på T9 med de 6 första bokstäverna utan att berätta det för dem. Sedan tänkte jag på det kulturella, att det kan vara oartigt med förnamn, och bestämde mig för efternamn istället. Hela namnet skulle bli för långt och krångligt.

%Helst skulle jag behöva gå runt Meteors standard-inloggning, och istället ha en enkel login-rullista som den ovan beskrivet, istället för att använda deras standard-lösning.

\textbf{Online database}
If data was to be sent from the client to the server, there needs to be a database with Meteor Collections.

As in version 1 of the app, no results were saved whatsoever, this was new functionality.

An example app was made first, only using Meteor Collections. Meteor's use of Distributed Data Protocol (DDP), made app pushes feel immediate, even though data was not sent until there was Internet access.

However, it was found out that if it took more than 15 minutes to get online, the push would be aborted. For users that are seldom online, this would not be viable.

\textbf{Offline database}
An offline database was needed, and the plugin GroundDB was implemented. As it was cumbersome to get right, pushing the data whenever online, and hard to test (needed to wait 15 minutes each time), this was not ready for the interactions.

\textbf{Upgrading from Meteor 1.2 to 1.3}
Meteor 1.2 had several disadvantages: while it worked for all devices, it did not support React.js

Meteor 1.3 was released, which promised a better developer experience, with JavaScript ES6 support, and access to Node Package Manager (npm), plus official support for React.js.

In 1.2, only some npm packages had been adapted for Meteor, and tools such as Webpack could not be used.

The downsides was discovered after implementation:
\begin{itemize}
\item there were missing backward compatibility to the older of the Android devices
\item Heroku had no Meteor build-pack for 1.3 - a push led the website to crash
\end{itemize}

This meant, that the app would not be able to be installed on many Android devices, and for those devices, a web version would not be available either. As this was unacceptable, the project downgraded to Meteor 1.2 again.

Unfortunately, since the online and offline database had now dependencies on version 1.2, the login and database integration could not be part of iteration 3, but this work needed to be saved for Iteration 4.


For iteration \#4, data collection was done by the app itself, which pushes data to the server whenever online (it saves quiz start, and quiz finish). The server receives JSON data from the client, stored in the MongoDB database hosted on Heroku. Each data point is saved in a database called Results, with the signed in user (from the Users database). In the database, there are collections for Users, Quiz Lists, and Quiz Results.




\section{Data Analysis Theory}

\todo{Lägg till overall data table}

%Methods to choose from for analysing (i.e. what I did during the interactions, to test and analyze my app)

%Partly data collection done via app, but also all the observations

\subsection{Iteration \#1}
%Here I used:


\subsection{Iteration \#2}
%Here I used:


\subsection{Iteration \#3}
%Here I used:


\subsection{Iteration \#4}

Data analysis is done first by a general overview in Google Sheets, by statistical analysis in R, and by a parallel coordinates visualization. The process to do this, is described below.

\subsubsection{Data Acquisition from Server}

It was desired to store the data in Google Sheets, thus it was necessary to collect the MongoDB database content, and convert JSON format into a Google Sheets-readable format, like CSV.

Multiple approaches were tried, and the Google Chrome extension called Magic Json by agaze\_dev\_team (last updated October 29, 2015) %https://chrome.google.com/webstore/detail/magic-json/cajifcebjiflndefndbnoeenjpiiiagm?hl=en
was the one that worked without problems. \citep{agaze}.

\subsubsection{Data Acquisition from Pre-Study}

The Pre-study data acquisition was done by instead of looking at the paper-submitted pre-study evaluation forms, using the data processed into Google Sheets.

\subsubsection{Data Enhancement of Server Results}

This section presents how data from the server was processed, to enable visualization mapping.

To make the data easier to work with, the columns were reordered, and made sortable and filterable.

Some columns were given conditional formatting, so it would be easier to spot irregularities. After this, some observations could be made.

\todo{Lägg till bild "results-colored.png" (finns på skrivbordet)}

To be able to compare the test results with the pre-test results, it was clear that it would not be viable to test every dimension against every dimension.

Instead, since goals of the app evaluation had been predefined in the following way, the quiz results were summarized into a new sheet so that the following could be derived:

\begin{itemize}
\item \% correct 1st try
\item number of tries until 100\%
\item number of tries until 100\% in 1 try
\end{itemize}

These could be calculated by having columns for:

\begin{itemize}
  \item Quiz 3
  \begin{itemize}
    \item Start time training
    \item \% correct 1st try
    \item number of tries until 100\% in 1 try
    \item Time difference start to end time certification
  \end{itemize}
  \item Quiz 9
  \begin{itemize}
    \item Start time training
    \item \% correct 1st try
    \item Time difference start to end 1st try
    \item Time difference start to passed training
    \item Time difference 1st try to certified
  \end{itemize}
\end{itemize}

Then, to see trends, I again added color scales. With ordinal values, a sequential color scheme is used (e.g. fastest time, from green to red), and with nominal values (like if they are female or male) where there is no right value, a qualitative color scheme is used. Now, it was easier to spot outliers and trends.

\subsubsection{Date Enhancement of Pre-study Results}
To see differences in answers more clearly, the data from the pre-study was made sortable and filterable. Then, the data was resampeled for each column that hade numerable (sortable) data in text instead of numbers, so e.g. "The day before" was changed to -1 and "The same day" to 0. In a similar way, school level was divided into four different groups, from 0 to 3, where 0 meant secondary, year unknown, 1 meant lower secondary, 2 meant upper secondary, and 3 meant tertiary.

After this, each column was given conditional formats using a color scale, using Google Sheets built-in functionality. This gave a visual way to quickly get a overview of the pre-test data.

\subsubsection{Data Enhancement by joining Pre-test and Results Summary}

I joined the summary sheet and the pre-quiz sheet, meaning I had created a multiple-variate data set (serveral dimensions that I needed to compare with several dimensions).

I met with my university supervisors, so they could further support me in how to properly analyze the data. Since the two control groups showd similar means on the pre-quiz results, the two control groups were determined comparable.

To meet the challenges of using Google Sheets, a multivariate analyzation software or a visualization was suggested to discover the data in less time.

It was hard to determine a suitable multivariate analysis software suitable when having so few data points. Principle Component Analysis or Cohen's kappa would not be suitable, neither was it believed applicable to do Linear correlation on all dimensions.

After discussion with other Master thesis students working with analysing data from various disciplines, parallel coordinates was suggested. It would allow me to very quickly filter the data, find correlations, and distinguish outliers and common characteristics.

To guide the usage of the parallel coordinates (as there is so much to discover in the data set), using R to do Logistic correlation was also done. A disadvantage with this method, is that to be statistically significant, many data points may be needed, and it was now known before-hand if the method would be useful. Probably, parallel coordinates would be the best method with analysing a small multi-variate data set.

\subsection{Visualization Mapping}
The goal with visualization mapping is to generate renderable data, in my case for the parallel coordinates visualization.

Thus, I added a new spreadsheet, specific for visualizing the data.

I deleted columns that would serve no visual purpose (e.g. timestamps), gave all cells data values (even N/A when undefined), deleting users that did not have data, and shortened the column names so they would fit on the screen.

The data was then exported from the Google Sheet into CSV.

\subsection{Rendering}

For rendering, the JavaScript library D3.js was chosen. It supports data-driven documents for visualizing data with HTML, SVG and CSS. It supports both JSON and CSV data.

A visual framework for multidimensional detectives for D3.js was found, called "Parcoords.js", written by Chang Kai (2012).
% https://syntagmatic.github.io/parallel-coordinates/
% Chang, K. (2012). Parallel Coordinates toolkit : Parcoords.js 0.1. Parallel Coordinates toolkit. Retrieved September 8, 2012, from http://syntagmatic.github.com/parallel-coordinates/
% Kosara, R. (2010, May 13). Parallel Coordinates. Eagereyes.org. Retrieved September 8, 2012, from http://eagereyes.org/techniques/parallel-coordinates
% Tricaud, S. (2008). Picviz: finding a needle in a haystack. Proceedings WASL, San Diego. Retrieved from http://www.usenix.org/events/wasl08/tech/full_papers/tricaud/tricaud.pdf

The example code from "Linking with a Data Table" provided the basis for the rendering. It would be a great benefit to bee able to see both a parallel coordinates visualization, and to see the same values present in the Google Sheet. %https://syntagmatic.github.io/parallel-coordinates/examples/table.html

I replaced the example CSV file with the exported Google Sheets data in CSV.

Eventually, I also changed the colors, and added to the example the toolkit's functionality to drag the axes titles around to reorder the dimensions, since the goal was to quickly compare and find correlations.

\todo{Add parallell coordinates visualization example}


