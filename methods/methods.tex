\chapter{Methods and Implementation}\label{cha:Method}

% längsta avsnittet i rapporten. Den består av en redogörelse av ditt arbete och den visar hur du kommer fram till dina resultat.

% Undvik egna synpunkter. Dessa framförs i Inledningskapitlet och Diskussions-kapitlet

% Ange källa till figurer i slutet. T.ex.  Source: Expedition Mondial.

% Metoddelen beskriver tillvägagångssättet – intervjuer, observationer, litteraturstudier, laborationer och så vidare. Motivera varför en viss metod valdes och vilka eventuella svårigheter som har förekommit. Metoden ska vara replikerbar, vilket innebär att en annan skribent ska kunna göra om studien med hjälp av informationen i metoddelen. Det finns en mängd böcker om olika vetenskapliga metoder. Till exempel kan en intervju utföras på en mängd olika sätt. I rapporter inom humaniora brukar metoddelen vara mer utförlig än i en teknisk rapport.

To understand the users of the app and the design possibilities, the setting and research context is described, together with a description of the collaborators and participants of the master thesis. Subsequently, study design and data collection is presented, followed by a data analysis framework that presents methods used to analyse quantitative and qualitative data. Finally, the application implementation is described.

\section{YoungDrive, Terminology and Limitations}

In this section, the organizational structure of YoungDrive and the master thesis is described, also explaining why there is an opportunity to innovate in regards to mobile technology in Uganda.

\subsection{Social Innovation and Social Entrepreneurship in Uganda} % https://www.linkedin.com/pulse/social-innovation-entrepreneurship-uganda-why-mobile-services-nissar?trk=prof-post

    This section will present background on working with mobile learning platforms, and understanding the society of entrepreneurs in Uganda.

    \subsubsection{Why Uganda is the world's most entrepreneurial country}
    According to Nissar \citep{nissar}, some facts related to entrepreneurship in Uganda are:

    \begin{itemize}
      \item Uganda is the world's most entrepreneurial country. (28\% of the population are entrepreneurs)
        \item Uganda has the second youngest population in the world (77\% of all Ugandans are below 30)
        \item Uganda has a very high unemployment rate (64 \% of people between 18–30 are unemployed)
    \end{itemize}

    % Ytterligare beskrivning av land: http://www.sun-connect-news.org/countries/uganda/

    With a high unemployment rate and little or none social security, starting a business is for many young entrepreneurs simply a tool for survival. But tough conditions can also lead to creativity, and there are as well many innovative entrepreneurs with great ideas and the aim to create positive social impact.

    As Mitchel says about entrepreneurship \citep{mitchel}, the motivation of entrepreneurship does not need to be solely wealth accumulation anymore. The activity of entrepreneurship contributes to society, in a way that is not captuted by the commercial entrepreneurship literature.

    No matter the reason of starting a business, Uganda's many entrepreneurs are contributing to the national society by boosting the economy and creating new jobs.

    \subsubsection{Why mobile services are growing rapidly in Uganda}
    One of the reasons is that the country has invested heavily in communication networks, even connecting remote rural villages with fibre optic cables and thereby connecting them to a world of information.

    As much as 65\% of the adults in Uganda owns a cell phone, which has allowed many areas in the country to skip the landline stage of development and jump right to the digital age.

    For those who hasn’t electricity at home, there are plentiful of charging booths for mobiles all over the country.

    \subsubsection{Mobile services and social innovations}
    The wide use of mobile phones has lead the way for the development of several innovative mobile services and in many cases the mobile service are way ahead of us  \citep{nissar}. In Sweden mobile banking services that allows us to transfer money through our mobile phones were made popular with Swish, introduced in 2012. In Kenya people have had similar services for the last 10 years.


\subsection{YoungDrive}

    \cite{youngdrive-web} is based on a Swedish concept, and had previously had a pilot in Botswana, when tasked with running the entrepreneurship module of A working future for Plan International. The organization fosters and educates young entrepreneurs in developing countries. They train the coaches, provide training material, and support the coaches via direction and direct support through co-project leaders and Youth Mentors.

    YoungDrive moves an entrepreneur to location, becoming country manager and "teacher". The teacher educates project leaders during four days, followed by educating coaches, which then roll out the training to the youth groups during 10 sessions, 1 session per week in average. The Community Based Trainers (CBTs) also rolls out other trainings, often simultaneously.

    For the future of YoungDrive, they want to make the CBT's even better, and collect and take use of data (monitoring and evaluation). Another motivation is scaling and monetization, as Plan International wants to increase the project to more countries, with an increased digital focus, and YoungDrive wants to be independent of project funding (i.e. a social enterprise).

    Given the above, this was a great time to introduce digital enablers for YoungDrive, where there previously had been no technology-focus, especially towards CBT's and Youth Mentors. The master thesis is the first project which focuses on digital enablers for YoungDrive.


\subsection{Roles within YoungDrive}

The \textit{country manager} trains the project leaders. It is also the main person responsible for partnerships and the quality of the YoungDrive program in the respective country. In Uganda, the country manager is Iliana Björling. She is located in the Uganda capital, Kampala, which is a strategic location because it is the same city in which the national office of the main partner, Plan International, is located. In Zambia, the country manager is Josefina Lönn, who previously was project leader in Kampala, and has held all the trainings up to this point. Now, she leads the operations and has trained the coaches in Zambia, in the new role of country manager and project leader.

The \textit{project leaders} trains the coaches and oversee the coaches, manages the coach training, and also collaborates with local stakeholders for quality assurance and to oversee daily operations.

The \textit{coaches} trains the youth. In Zambia, a coach only has responsibility for training youth in the YoungDrive program. In Uganda, this is called a \textit{Youth Mentor (YM)}, in contrast to being a \textit{Community Based Trainer (CBT)}, which also trains the youth in other programs and leads the youth saving groups. Most of the CBT's in Uganda holds sessions together with a Youth Mentor, or divides work between them, instead of being alone. The coaches are often volunteers, receiving a small scholarship from the partner organization. They are often business owners themselves. The coaches could be described as social entrepreneurs \citep{mitchel}. Many of the YoungDrive coaches (and youth) are driven by that their business can have an impact on their community, \textit{as well} as take them out of unemployment or increase their current livelihood.

The \textit{youth} are the ones receiving the training from the CBTs and the YMs, being encouraged to start their own businesses.


\subsection{Mobile Technology in Uganda's Rural Areas}\label{sec:mobile-uganda}

One of the reasons why mobile services are growing rapidly in Uganda is that the country has invested heavily in communication networks, even connecting remote rural villages with fibre optic cables and thereby connecting them to a world of information.

As much as 65\% of the adults in Uganda owns a cell phone, which has allowed many areas in the country to skip the landline stage of development and jump right to the digital age. For those who hasn’t electricity at home, there are available charging booths for mobiles all over the country.

The wide use of mobile phones in Uganda and other developing countries has lead the way for the development of several innovative mobile services and in many cases the mobile service are way ahead of us  \citep{nissar}. In Sweden mobile banking services that allows us to transfer money through our mobile phones were made popular with Swish, introduced in 2012. In the neighbouring country Kenya, people have had similar services for the last 10 years, and mobile money is since long also common in Uganda.

A prominent example of an app that has previously been developed with the target group in mind is Ledger Link \citep{ledgerlink}. This mobile banking service empowers, developed in partnership with a bank, allows saving groups in rural areas such as Tororo to save money remotely. It is developed with human-centered design methods, and has won several awards.

%For the future of YoungDrive, they want to make the CBT's even better, and collect and take use of data (monitoring and evaluation). Another motivation is scaling and monetization, as Plan International wants to increase the project to more countries, with an increased digital focus, and YoungDrive wants to be independent of project funding (i.e. a social enterprise). This was a great time to introduce digital enablers, where there previously had been no technology-focus, especially towards CBT's and Youth Mentors. The master thesis is the first project which focuses on digital enablers for YoungDrive.

%\subsection{Hybrid App Development}

The history of app and web development is rich and increasingly intertwined. First, websites were developed for desktop only, and when smartphones became popular, they were made responsive.

With today's possibilities of native mobile development or developing a native app using web technologies, there are numerous viable alternatives available if an app should function on several devices, depending on budget and preferences.

One of the main argument for developing an app in web technologies, is that the whole application, including the server, can be written in one programming language, JavaScript (full-stack).

Tools such as Apache Cordova can compile JavaScript applications into native apps. Thus, they can appear on Apple iOS and Android Play Store, as well as on the web, or installable offline on a smartphone from the computer.

JavaScript is developing rapidly as a language, as well as its ecosystem of frameworks and tools. Frameworks have emerged and matured, like Meteor.js, which makes building full-stack applications in JavaScript reliable and fast.

Previously, web hosting has been troublesome for JavaScript server applications. Today, tools such as Meteor.js and Heroku have introduced free and paid hosting for such applications, with smart bindings to code platforms such as GitHub, which makes collaboration and version handling easy.




\subsubsection{The Current Author}
It is needed to take on several roles in the project by the current author: most notably that of a project leader, designer and developer. It is needed to balance stakeholders' different opinions and requirements, and caring for the vision (see section \ref{aGoodDesigner} A Good Designer). The motivation doing the master thesis is three-fold: learn as much as possible, create a successful project, and finish the master thesis.

There are two groups, with the current author included in both of them, which gather at the end of each sprint for a check-up meeting. The Expert group consisted of Expedition Mondial and LiU Innovation. Expedition Mondial could help with the design process, and LiU Innovation could offer input on social innovation. The meetings mostly lasted for one hour. The Partner group consisted Iliana Björling from YoungDrive, and Lena Tibell and Konrad Schönborn from Linköping University. In Partner meetings, The Insighs from each iteration was presented and discussed. Then possible decisions were laid out, followed by discussing the alternatives. Outside of these groups, these people can also give advice in certain situations. For specific areas, there are also some experts which have been beneficial during the projects. Below, the whole team is explained: % Then I tell them about which decisions has been taken and why.

\subsubsection{Supervisors}
The supervisors are from YoungDrive and Linköping University. The YoungDrive team consists of Iliana Björling, founder of YoungDrive, and Josefina Lönn, country manager in Zambia. They are both helpful in giving knowledge on the entrepreneurship education program, and giving support. The Linköping University team consists of Lena Tibell, Professor, and Konrad Schönborn, Doctor, within the Department of Visual Learning and Communication.

\subsubsection{Stakeholders}
The stakeholders are considered YoungDrive and Plan International. \textbf{YoungDrive} is the client of the work, and their needs should be satisfied. This person is mainly represented by Iliana Björling, who is part of the YoungDrive Strategic Management Team. Using service design, the project leaders in Uganda and Zambia, are also considered stakeholders: Josefina Lönn in Zambia, and the two co-project leaders in Uganda. Finally, the most important stakeholder of all according to service design, is the actual users: the coaches. They should be the main consideration of the work.

\textbf{Plan International} is the organization allowing for all the interactions with the end users in Uganda. A similar organization is operational in Zambia. They are the ones that are providing facilities, organizes transport, etcetera. They in turn, have the organization Community Vision, which organizes the coaches. If Plan International or Community Vision does not appreciate of the project and the collaboration, then the interactions with the coaches will not be possible.

%\textbf{Linköping University} is a stakeholder, as the supervisor (Lena Tibell) and examinator (Camilla Forsell) determine if the work is a valid master thesis or not. Also, LiU Innovation is interested in supporting continued work with the project, and their representative Peter Gahnström gives advice on social innovation and how this project can continue in the future during expert meetings.

\subsubsection{Experts}
Since the development country context is new to the current author, there are also specific experts advised in the project. For design process, Susanna Nissar and Erik Widmark from Expedition Mondial has supported with all of their knowledge within service design. Julien Tantege, Research Specialist at Grameen Foundation, has been kind to offer support before and during the work, sharing their insights from related work, and giving feedback during ideation. She has experience doing technical development for rural areas. For pedagogical development, Henrik Lundmark from edtech startup Knownly in Sweden has given support with regards to building skills within digital learning. For feedback for how the work relates to social innovation, Peter Gahnström at LiU Innovation has offered feedback.


\section{Describing the YoungDrive Coaches and the Research Context}

The biggest challenge with regards to time constraints and cultural differences is that it is difficult to understand the target group of the app. Therefore, the whole design and development process will take place in Uganda, with several interactions with the intended users. The work was carried out from Hive Colab, a co-working space and an innovation hub. The work is done mainly from Kampala, because that is where YoungDrive is situated, meaning that there is still a long distance to the coaches and youth in Tororo, which is located near the Kenyan border. Another challenge with being in Uganda compared to Sweden is that internet speed and access is worse, especially outside Kampala.

 The interactions took place in either Uganda or Zambia, in the locations where training of the coaches and youth takes place.  There were a number of resources made available to support the work, for example the YoungDrive manuals. Each youth is given a \textit{Participant manual}, describing each week of the 10-week YoungDrive program. Coaches are also given a \textit{Coach guide}, which describes how to carry out and teach each week's topic during the youth training. As most of the coaches did not have smartphones or tablets, four smartphones (3 Android, 1 iOS) and ten tablets (3 Android, 7 iOS) were brought from Sweden. All of these devices had a web browser and access to an app store. These were either donated, borrowed or bought devices. During the user tests, also using a laptop would be tested. The following section describes the Ugandian and Zambian coaches and their businesses.

\subsection{Social Characteristics and Businesses in Uganda}

According to statistics gathered by YoungDrive during 2015 evaluations \citep{youngdrive-statistics}, there are a number of considerations to make regarding the coaches in Uganda. This regards entrepreneurship experience, technical access, and language, see a summary in figure \ref{fig:ydStatistics}. Taken together the coaches' and project leaders' technical skills are currently low, and this needs to be in consideration when designing the app. Regarding language, English can be used in the coach app.

\begin{figure}[h]
    \centering
    \includegraphics[width=1.0\textwidth]{ydStatistics.png}
    \caption{Table showing entrepreneurship experience, technical access, and language between Tororo coaches and the project leaders in Uganda and Zambia. All of the Tororo coaches run a business (with a majority running more than one). This means, they do have practical experience of running a business outside of the YoungDrive coach training. While all have a cell phone, smartphones are very uncommon - only 3 uses Internet on the phone, every day or weekly (mostly for Facebook or email). Regarding power, none has power at home, 3/26 has solar, and only 4/26 can write on a computer. While about half of the asked Uganda youth can not understand (129/225), read (133/225) or write (132/225) English, most of the coaches in Uganda are proficient. These characteristics are similar for youth and project leaders as well.}.
    \label{fig:ydStatistics}
\end{figure}

The coaches in Tororo are divided into three different regions. Based on region, income and experience, they run different kinds of businesses. \footnote{In Uganda and Zambia, a small-scale business is typically not registered. Thus, the coaches' definition of a business can be more generous.} In Tororo, the coaches' businesses range from: pineapple, water melon, onion, chili, bakery, catering, corn, beans, fabric, plastic products, bird farm, milk, fish, ground nuts, cabbage, tomato, hairdresser, sewer, shop and rice. For photograph of environment, see figure \ref{fig:tororo}.

\begin{figure}[h]
    \centering
    \includegraphics[width=0.7\textwidth]{stayover.jpg}
    \caption{One of the co-project leaders showing the rural part of Tororo, where crops are growing close to where he lives.}.
    \label{fig:tororo}
\end{figure}

In Tororo, there are 2 Project Leaders. one project leader's business ranges from: bakery, corn, pig farm and plastic products. The other person's business ranges from: silver fish, beans, corn, and bird farm. In comparison, in Kamuli, there are 4 Project Leaders. Their businesses ranges from: selling office supply, motorcycle taxi, bird farm, pig farm, green pepper, corn, cabbage, tomato, aubergine, chipati ("bread"), chilli, and charging of cellphones.

Among the youth, the top 8 most popular businesses in Tororo, with 134 respondents, are corn, cassava ("potato"), saloon, fish, making of bricks, beans, brooms and rope. These range from 9 for corn (6.7\%) to 5 for rope (3.7\%).

\subsection{Social Characteristics and Businesses in Zambia}
During the visit in Zambia, the coaches had not yet formed their youth groups, and started their own teaching. Regarding characteristics, ages ranged from 21-39 years old (26.8 average). Other data available about the coaches were notes of the Zambian coach job interviews, which could be compared with during quiz analysis \citep{yd-zambia-interviews}. Compared with Uganda, 9 out of 10 had business experience.

%3 were mentioned as being shy during the interviews. They lived from 10-90 minutes outside of town (33 minutes average).

%Regarding motivations for being a coach, 50\% had an emphasis on benefiting the community, and 90\% had personal reasons.



%Care for oneself:
%\begin{itemize}
%\item Learn business %(7)
%\item More skills %(2)
%\item Get idea
%\item Expand business
%\item Benefit CV
%\end{itemize}
%
%Care for community:
%\begin{itemize}
%\item Empower %(2)
%\item Teach business
%\item Leadership
%\item Share
%\item Stop bad behaviours of youth
%\end{itemize}

%regarding experience, 8 had trained youth before, 8 had been a leader before, 9 had business experience. Regarding YoungDrive, they said they could handle training between 8-30 youth (19.8 on average) per group. They could have 1-5 groups per coach (average 3.0), totalling a range between 8-101.5 youth (average 59.8).


\section{Study Design and Data Collection}

  As a computer expert with social skills needing to design and develop an app for an unfamiliar cultural and socio-economic context, it was needed to quickly become a good designer. The technical aspect of the project was but one. It was needed to learn how to develop hybrid apps in JavaScript that worked offline, and had an online back-end. However, those are merely the technical demands.

  It was needed to quickly become a good designer, not mainly from a perspective of graphic design or interaction design, but \textit{how} to explore, design, and implement what the user needs from the requirements "fun, user-friendly, and good for learning". There was also a need to evaluate the effectiveness of the implementations, to assess learning and the interaction design aspects of desirability, utility, usability and pleasurability. The approach used to learn design from these perspectives was to read extensive literature, consult a diverse set of experts, and be humble and curious in interactions with the end-users and stakeholders. In the following section, the creation and implementation for a suitable design process is described, together with the study design and data collection for each iteration of the project.

%Har gått igenom planeringsrapporten lite noggrannare idag och ser två saker som vi kanske ska borde fånga upp under arbetets gång.

% Under 2 Purpose står det ett upplevelsemål från Young Drive. Bör vi mäta detta upplevelsemål om det stämmer med deltagarnas faktiska upplevelse, d v s ska vi försöka få in det under 3 Research Questions?

% På våra avstämningsmöten borde vi också följa upp dina Research Questions så att kundinteraktionerna och servicedesignmetoden tyligt leder dig framåt mot dessa mål.
%* Reflektioner på vilka designprinciper som bör väljas? (utifrån kundinteraktioner)
%* Reflektioner angående tekniska begränsningar?
%* Reflektioner på processen?

\subsubsection{Creation of Design Process}
As there was a unfamiliar target group - mostly young Ugandians with little or no experience of smartphones - service design thinking would benefit true understanding of cultural context and in-depth empathy for the end users.

Tools and methodology in service design were chosen with the help of Expedition Mondial in Stockholm, who provided education and coaching.

At the same time, the end result would be a digital artefact (an app), which is not common in service design.

While this product could be though of as a service, the tools and methodology would benefit to borrow from Agile methodology and Interaction design.

I'm the computer expert kind of designer \citep{lowgren}, adjusted to agile methodology and interaction design, but aspiring to be a socio-technical expert. Expedition Mondial are experienced with service design, aspiring to be more of computer experts.

This led to the joined development of a Digital Service Design method, co-created by the both.

%Expedition Mondial helped with a method for creating a MVP of the digital support for the coaches, so that the app was developed from the perspective of the end users and the education and a "learning by doing" mentality.

%The suggested design process was designed with them after a start-up meeting on Skype, and an education day in Stockholm. During that day a crash course in service design was given, then creating a common plan for the future work based on my needs (see Appendix: Original Time Plan \todo{Add reference}). They also recommended service design literature. These were the methods chosen in each iteration.

The result is that the design and development phase in Uganda is an iterative process with the human in focus. The process is built on top of service design process and methodology, while in-line with digital design practices.

\begin{figure}[h]
    \centering
    \includegraphics[width=1.0\textwidth]{iterativeProcess.png}
    \caption{The iteration timeline shows the process from iteration 1-4, by different colors. Each iteration leads to a loop, with the design situation as the focus, in which interactions can happen that is related to either research ("What's ...?") and an app test ("How can we ...?"). This output from the loop, then gives force to the next iteration.}
    \label{fig:iterative-process}
\end{figure}


\subsubsection{Implementation of Design Process}
See figure \ref{fig:iterative-process}. There were four iterations. The first iteration follows Service Design, not starting the app development, while the other three follows the new methodology, Digital Service Design.

In iteration 1, there is a very broad scope, without digital focus, where iteration 2, 3 and 4 introduces and narrows down the project into a digital solution.

Expedition Mondial gave support in each iteration, helping with refinements of each iteration as learnings happened along the way, and they were able to educate me during the different stages with methodologies whenever necessary.


%\subsection{Implementation of methods for data collection and data analysis}

This section describes the implementation of methods for each iteration's interactions, in regards to collecting and analysing qualitative and quantitative data. Figure \ref{fig:methods} is made to assist the reader in which methods were used together.

%study design, application development, and for data analysis theory.

\begin{figure}[h]
    \centering
    \includegraphics[width=1.0\textwidth]{iterativeProcess.png}
    \caption{Should this be a table?}
    \label{fig:methods}
\end{figure}

\todo{Complete the Methods and Analysis table}

%Methods w/ Analysis
%----
%
%5 Interviews - Notes - Questionnaire for Customer Journey
%1 Customer Journey - Clustering - Personas
%2 Youth Sessions - Shadowing - Needs
%1 Coach Stayover - Empathy - Design Ethnography
%
%5 Training Days - Observations - Understanding entrepreneurship training
%5 Hackathon Days - Interviews - Notes and Sketches
%2 App Workshops - Co-Creation/Co-Refinement - Sketches and Needs
%
%3 Field
%
%Acitivity
%App test observations (group)
%
%App test observation (individual)
%- Affective reactions (5 Why's, think aloud)
%
%Analysis: Interaction design evaluation (desirability, usability, utility, %pleasurability)
%
%Customer Journey Map
%- Activities
%- Behaviour
%
%Written responses (individual)
%- Right/wrong
%- Time
%- Number of tries
%
%Interviews
%- New insights
%
%Data Collection w/ Analysis
%----
%
%Customer Journey Map w/ clustering
%
%Pre-study w/ Quantative analysis
%
%Written quiz responses w/ Quantative analysis
%
%Digital quiz responses / Quantative analysis + Statistical analysis + %Parallell coordinates
%
%Quiz questions 1 w/ Bloom analysis
%Quiz questions 2 w/ Bloom analysis

%\input{implementation/iteration-1}

%\input{implementation/iteration-2}

%\input{implementation/iteration-3}

%\input{implementation/iteration-4}


\subsection{Iteration 1}

% How was this iteration designed?

Following the service design sequencing, the first iteration had a very broad scope and truly is a service design iteration: "From your perspective, what is it like being a coach?". \footnote{A coach meaning either a Community Based Trainer (carrying out all the trainings), or a YoungDrive coach, depending on who was asked the question.}

Lowgren's though about how to start the project was used, meaning that the purpose was to get a preliminary understanding of all important aspects, and build relationships with all stakeholders.

Insights depended heavily on interviews with all the stakeholders  (2 with Plan International, 3 with YoungDrive), and local experts (1 visit each at Grameen Foundation and Designers without Borders, 1 workshop with Mango Tree), since no Interactions with users had been made yet. Also, I immersed myself with the Uganda tech scene as possible, from the new home and office in Kampala, working at the tech hub and co-working space Hive Colab.

Ideation were about creating a questionnaire guide for the interviews, a co-creation workshop using "Customer Journey Map", and identifying how the app test should be designed to test their existing knowledge (and be informed of the design preferences of the YoungDrive app).

Trigger material was the finished questionnaire guide (constructed with Expedition Mondial) a written plan for the co-creation workshop ("A day as a coach"), and a written plan for testing the quiz app Quizoid and the language learning app Duolingo, and a schedule for the interactions.

The interactions were focused on design ethnology, getting to know and learn from people in a different culture, namely the coaches. The focus was on the their needs, motivations, and context.

To accomplish these, four days were spent in Tororo, with one day of travel. There were four face-to-face-interviews,
one meeting with Plan, one meeting with the local partners, two workshops, one coach stay-over, and two youth session visits.


\subsection{Iteration 2}

This time, the iteration has a more detailed scope, with a hypothesis on what needs the app should meet in the end, and create lo-fi and hi-fi trigger material to meet those needs.

A co-creation workshop started the interactions, followed by repeated app tests at minimum one session per day, always followed by a feedback round, so the app and the tomorrow's question set creation could be improved for the next day. At the end of the week, there was a co-refinement workshop of the current hi-fi material, and also lo-fi material for the new version of the app.

\subsubsection*{Creation of questions}
Project leader Josefina in Zambia refined Iliana's first question sets, prepared for my visit in Zambia. Josefina created question sets with Bloom at the back of her head, also taking into account the structure and the order of the coach manuals, what it means being a coach within the topic, and lastly scenarios.

\subsubsection{Trigger material used}
A hi-fi trigger material was done, a very basic quiz app, keeping it as simple as possible (see Application Implementation, Iteration 2). All of the devices (tablets and smartphones) that I had available were brought to Zambia.

I added Josefina's questions to the app, and installed the app to all of the devices. This process was repeated for all the days, Sunday-Friday.

\subsubsection{Design workshop \#1 in Zambia}
The coach training started with me having a design workshop with the coaches, not showing them the app that I had created. The co-creation workshop was made to identify important functionality in the minds of the coaches.

\begin{enumerate}
\item Since the knowledge about smartphones and apps were low, I started by introducing these topics.
\item All were familiar with Facebook, so thus I showed the Facebook app. Me wanting to know what the app would look like if the coaches would have designed the app, I first needed to train them how to design an app via drawing wireframes.
\item Using postits, they started with during limited time drawing the start view from the Facebook app.
\item Then, they were asked to draw what they thought happened on the friend icon click, drawing the view on another postit.
\item Then, the mission of the YoungDrive app was described. They were then divided into two teams, having limited time to draw the best imaginable YoungDrive coach quiz app they could. First, they designed the app from the top of their heads. They then pitched their results to each other.
\item On the next iteration, they were to suggest and design improvements how the app should be designed to improve learning, not only assessment. They then again pitched their results to each other.
\end{enumerate}

\subsubsection{Assessment via quiz}
At the end of each day, the app was used to test the coaches' knowledge. Each coach got either a smartphone, tablet or computer. The coach first took the quiz for the most recent session, and could then choose what to do next.

As there were no back-end developed, Josefina by hand documented the scores of each coach, writing the name of the coach, the session, number of correct answers, and what questions had been answered wrong.

Josefina then, when planning the next day, looked at the statistics, looking for trends that would inform the sessions for the following day.

She also evaluated the quality of the questions, before creating the new question sets for the next day.

\subsubsection{Experimenting with quiz before or after the session}
Since the coaches appreciated the app so much, we felt tempted to try what would happen with fun and learning if we tried using the app \textit{before} a session instead of only after. During the rest of the week, we continued, finally finding preferences and tendencies from the coaches, via observation, interviews, and survey.

\subsubsection{Experimenting with design of questions}
During the week, extra tests were done to test the following:

\begin{itemize}
\item Number of questions per quiz
\item Single-answer questions or multiple-answer questions
\item Framing of questions
\item Challenge level of questions
\item Determining what made a question hard
\end{itemize}

\subsubsection{Interviews with Josefina}
At the end of each day, an evaluation interview was held with Josefina. At the end of the week, a final interview was held.

At the end of Day 5, Josefina and I discussed what it would look like to not record the answers manually, but pushing the results online. A co-creation workshop was held, where she drew an Educator Dashboard.


\subsection{Iteration 3}

Iteration 3 had an even more detailed scope. Since the app now succeeds with the first use case, the coach training, not the focus could be on "learning at distance".

A requirement from Josefina was also to test if the app created in Zambia could work also in Uganda. All the quiz questions would need to be converted from the new manual to the old manual, since both structure and content had changed.

To test on all of the coaches in Uganda, it would be preferable if data collection would happen via the app instead of manually, since there would be more than 10 test subjects, which had been the limit in Zambia.

A future requirement was that quiz responses would need be available to the teacher. This means that there needs to be a database, but also a login, so individuals are traceable.

How can login and the database be implemented in the best possible way?

The insights on learning also needed to be considered:
\begin{itemize}
  \item Are coaches really learning via the app, especially learning to be better coaches?
  \begin{itemize}
    \item How can questions be formulated in a way that teaches entrepreneurship, which is so practical?
  \end{itemize}
  \item How can the current multiple-choice quiz app be improved, to:
  \begin{itemize}
  \item reduce guessing
  \item improve confidence
  \item encourage learning
  \end{itemize}
\end{itemize}

Thus, the study design of Iteration 3 became very important. A lot of development and ideation was done.

Also, instead of only testing the app in Tororo, a test was held in Kampala, to get feedback from an entrepreneurship student.

For the interactions, a big app test was held, and also a co-creation workshop was held.

Before the workshop, the wished functionality and goals were well formulated. It was also discussed beforehand how to best design the workshop, together with Linköping University and Expedition Mondial.

Questionnaire 3 was created, used after the test. After interviewed in a big group, they were divided into co-creation workshop groups, with a presentation in the end.

\subsubsection{Aim}

To get an app suitable for learning, it was determined that the pedagogical model behind the app needed to change, emphasising feedback.

The aim was to score higher on Bloom's revised taxonomy, while still including multiple-choice questions in the app.

\subsubsection*{Trigger material}

Josefina was given a task to create a quiz "Are you ready for Session 9?". The aim of this quiz, was partly to score higher on \textit{Bloom's revised taxonomy}, partly to test if Correct Structure and Time Management could be assessed using multiple-choice.

Also, the questions were translated from the new manual into the old manual, which is used in Uganda.

\subsubsection*{Interactions}

There was another partner meeting, with Plan International and Community Vision present. There was an app test with all of the coaches, "Testing the YoungDrive coach app", followed up by splitting into six workshop groups based on solving different problems discovered during the test.

The following day, there were three field visits to CBTs, observing how they prepared themselves for a youth session, and then testing the app for assessing and becoming prepared for a session.

The last day, there was a co-refinement workshop ("Usability Improvements") and one co-creation workshop ("Educator Dashboard") held in parallel, with 3 CBTs and 1 project leader respectively.


\subsection{Iteration 4: Uganda Summative Test}

The focus of iteration 4 was a summative test. First, a pre-test was carried out in paper, including questions about the coach and an entrepreneurship quiz, based on a well-known study \citep{general-entrepreneurship-quiz}, see Appendix \ref{cha:pre-test}. During the test, this was the first time that the app could send data to the server. Data was sent whenever a quiz was started, and whenever a quiz was finished. The group was divided into two, the ones who brought manuals and they who did not. Those that had brought manuals, could use these with the app, see figure \ref{fig:appevaluation}.

\begin{figure}[h]
    \centering
    \includegraphics[width=0.7\textwidth]{appevaluation.jpg}
    \caption{Coaches answering the app questions for topic quiz 3 on Financial Literacy, and the coach guide quiz 9 on Action Plan.}
    \label{fig:appevaluation}
\end{figure}

After the test, every coach was divided into one or three groups, on random. In these groups, they were asked:

\begin{enumerate}
\item Why do you think you were correct or incorrect?
\item Do they like the app?
\item Are you stimulated by the app?
\item What did you like?
\item What did you not like?
\item When do you want to use the app?
\item When are you not able to use the app?
\end{enumerate}

To analyse the paper-submitted data, all of this was combined first into a Google Spreadsheet (the app results were also recorded in paper, but only as a backup). Data collection was done by the app itself, which pushes data to server whenever online (it saves quiz start, and quiz finish).

%The next day, a small app evaluation and co-creation workshop was held for the Educator Dashboard, and the final version of the app. Also, a test was done with the Plan Tororo staff.

%Back in Kampala, a presentation was held with Plan International. Back in Sweden, a presentation was held with the YoungDrive Strategic Management Team.



\section{Data Analysis Theory}

\todo{Lägg till overall data table}

%Methods to choose from for analysing (i.e. what I did during the interactions, to test and analyze my app)

%Partly data collection done via app, but also all the observations

\subsection{Iteration \#1}
%Here I used:


\subsection{Iteration \#2}
%Here I used:


\subsection{Iteration \#3}
%Here I used:


\subsection{Iteration \#4}

Data analysis is done first by a general overview in Google Sheets, by statistical analysis in R, and by a parallel coordinates visualization. The process to do this, is described below.

\subsubsection{Data Acquisition from Server}

It was desired to store the data in Google Sheets, thus it was necessary to collect the MongoDB database content, and convert JSON format into a Google Sheets-readable format, like CSV.

Multiple approaches were tried, and the Google Chrome extension called Magic Json by agaze\_dev\_team (last updated October 29, 2015) %https://chrome.google.com/webstore/detail/magic-json/cajifcebjiflndefndbnoeenjpiiiagm?hl=en
was the one that worked without problems. \citep{agaze}.

\subsubsection{Data Acquisition from Pre-Study}

The Pre-study data acquisition was done by instead of looking at the paper-submitted pre-study evaluation forms, using the data processed into Google Sheets.

\subsubsection{Data Enhancement of Server Results}

This section presents how data from the server was processed, to enable visualization mapping.

To make the data easier to work with, the columns were reordered, and made sortable and filterable.

Some columns were given conditional formatting, so it would be easier to spot irregularities. After this, some observations could be made.

\todo{Lägg till bild "results-colored.png" (finns på skrivbordet)}

To be able to compare the test results with the pre-test results, it was clear that it would not be viable to test every dimension against every dimension.

Instead, since goals of the app evaluation had been predefined in the following way, the quiz results were summarized into a new sheet so that the following could be derived:

\begin{itemize}
\item \% correct 1st try
\item number of tries until 100\%
\item number of tries until 100\% in 1 try
\end{itemize}

These could be calculated by having columns for:

\begin{itemize}
  \item Quiz 3
  \begin{itemize}
    \item Start time training
    \item \% correct 1st try
    \item number of tries until 100\% in 1 try
    \item Time difference start to end time certification
  \end{itemize}
  \item Quiz 9
  \begin{itemize}
    \item Start time training
    \item \% correct 1st try
    \item Time difference start to end 1st try
    \item Time difference start to passed training
    \item Time difference 1st try to certified
  \end{itemize}
\end{itemize}

Then, to see trends, I again added color scales. With ordinal values, a sequential color scheme is used (e.g. fastest time, from green to red), and with nominal values (like if they are female or male) where there is no right value, a qualitative color scheme is used. Now, it was easier to spot outliers and trends.

\subsubsection{Date Enhancement of Pre-study Results}
To see differences in answers more clearly, the data from the pre-study was made sortable and filterable. Then, the data was resampeled for each column that hade numerable (sortable) data in text instead of numbers, so e.g. "The day before" was changed to -1 and "The same day" to 0. In a similar way, school level was divided into four different groups, from 0 to 3, where 0 meant secondary, year unknown, 1 meant lower secondary, 2 meant upper secondary, and 3 meant tertiary.

After this, each column was given conditional formats using a color scale, using Google Sheets built-in functionality. This gave a visual way to quickly get a overview of the pre-test data.

\subsubsection{Data Enhancement by joining Pre-test and Results Summary}

I joined the summary sheet and the pre-quiz sheet, meaning I had created a multiple-variate data set (serveral dimensions that I needed to compare with several dimensions).

I met with my university supervisors, so they could further support me in how to properly analyze the data. Since the two control groups showd similar means on the pre-quiz results, the two control groups were determined comparable.

To meet the challenges of using Google Sheets, a multivariate analyzation software or a visualization was suggested to discover the data in less time.

It was hard to determine a suitable multivariate analysis software suitable when having so few data points. Principle Component Analysis or Cohen's kappa would not be suitable, neither was it believed applicable to do Linear correlation on all dimensions.

After discussion with other Master thesis students working with analysing data from various disciplines, parallel coordinates was suggested. It would allow me to very quickly filter the data, find correlations, and distinguish outliers and common characteristics.

To guide the usage of the parallel coordinates (as there is so much to discover in the data set), using R to do Logistic correlation was also done. A disadvantage with this method, is that to be statistically significant, many data points may be needed, and it was now known before-hand if the method would be useful. Probably, parallel coordinates would be the best method with analysing a small multi-variate data set.

\subsection{Visualization Mapping}
The goal with visualization mapping is to generate renderable data, in my case for the parallel coordinates visualization.

Thus, I added a new spreadsheet, specific for visualizing the data.

I deleted columns that would serve no visual purpose (e.g. timestamps), gave all cells data values (even N/A when undefined), deleting users that did not have data, and shortened the column names so they would fit on the screen.

The data was then exported from the Google Sheet into CSV.

\subsection{Rendering}

For rendering, the JavaScript library D3.js was chosen. It supports data-driven documents for visualizing data with HTML, SVG and CSS. It supports both JSON and CSV data.

A visual framework for multidimensional detectives for D3.js was found, called "Parcoords.js", written by Chang Kai (2012).
% https://syntagmatic.github.io/parallel-coordinates/
% Chang, K. (2012). Parallel Coordinates toolkit : Parcoords.js 0.1. Parallel Coordinates toolkit. Retrieved September 8, 2012, from http://syntagmatic.github.com/parallel-coordinates/
% Kosara, R. (2010, May 13). Parallel Coordinates. Eagereyes.org. Retrieved September 8, 2012, from http://eagereyes.org/techniques/parallel-coordinates
% Tricaud, S. (2008). Picviz: finding a needle in a haystack. Proceedings WASL, San Diego. Retrieved from http://www.usenix.org/events/wasl08/tech/full_papers/tricaud/tricaud.pdf

The example code from "Linking with a Data Table" provided the basis for the rendering. It would be a great benefit to bee able to see both a parallel coordinates visualization, and to see the same values present in the Google Sheet. %https://syntagmatic.github.io/parallel-coordinates/examples/table.html

I replaced the example CSV file with the exported Google Sheets data in CSV.

Eventually, I also changed the colors, and added to the example the toolkit's functionality to drag the axes titles around to reorder the dimensions, since the goal was to quickly compare and find correlations.

\todo{Add parallell coordinates visualization example}



% Application implementation
\section{Application Implementation}

In this section, the prerequisites for the app is described, from the perspective of the user, stakeholders, and the developer.

\subsection{User Needs}

The technical constraints for the project, would need to affect the technologies used, if the project would be user-centered. On the client side, the app would need to be mobile and web based, consider non-access to internet, and not use a lot of battery, to work for the coaches of YoungDrive. That the app should be simple to use in this cultural setting leaded to design constraints and needs for evaluation.

\subsection{Stakeholder Meeds}

As the project was only three months, and the first month would be without digital development, time constraints were massive. However, to be able to answer how design affects learning, evaluation was needed to be done via data collection.

If no evaluation, there would be no need to write code, instead working with a low-fidelity prototype using pure design tools. Now, a data-driven approach was needed to measure, and therefore an app needed to be developed. On the server side, a database and API would be needed, to pull data from the database and push data from the client. Since internet was not always available, the client must be smart in its usage of pushing and pulling data. This would need to be investigated further into the project.

\subsection{Devices are prepared}
As most of the coaches did not have smartphones or tablets, enough smartpones and tablets were brought with me from Sweden, either donated, borrowed or bought devices. These were a combination of Android and iOS, smartphones and tablets, so the app could be tested on as many platforms as possible. During the user tests, also using a laptop would be tested.

%\subsection{App/Web Development}
%Early in the project, it was thought that existing tools could be used, instead of building the app from scratch. E.g. using existing tools like Knowly or Typeform\footnote{examples include https://showroom.typeform.com/to/ggBJPd and https://showroom.typeform.com/report/njdbt5/dIzi} during the first iterations for understanding users, and during development e.g. the Typeform API (http://typeform.io/). The Typeform API allows developers to create surveys from within their own applications or systems.

\subsection{Choosing frameworks for creating the app}

In the start, Ionic and Meteor were both tested and compared with each other. It was decided that Meteor was the best way forward, partly because it would allow the app to be accessable on the web as well. %\todo{Add from mindmap}

React.js was chosen as the front-end framework, having integration with Meteor and being relatively easy to learn and fast for development.



Since Meteor was chosen, a multiple-choice quiz tutorial in Meteor was used to guide the first version of the app. Modifications were made, for example making it responsive and changing it to YoungDrive's graphic profile.

The app was pushed to GitHub, and first hosted on Meteor free storage, available via youngdrive.meteorapp.com. For Android and iOS, it was made possible to install the app from the computer.  For each day of the training in Zambia, new quizzes were added to the app, which created a belt path (see \ref{progress-payoffs}).

After iteration 2, a different hosting platform was needed when the Meteor free tier was removed, where Heroku was chosen. Staging environment using Heroku allowed changes on specific GitHub branches to deploy updates automatically on Heroku servers. The MongoDB database was created using the Heroku plugin MongoLab. A Meteor build-pack was used to allow Meteor to be used with Heroku.

It was also tested to upload the app to Android Play Store. The neccessary steps from Cordova needed to be followed, screenshots needed to be uploaded, and some administrative tasks. After this, it only took a day for the app to appear on the Play Store, and everything worked satisfactory.


\subsection{Iteration 3}

For me, the user's first feeling of a superpower is a hint of becoming a Certified coach. \todo{This can be commented in the Future work}

On the client, as components grew, there was a need for a client-side router. The Meteor plugin Flow Router was used, as it was very popular with good integrations.

\subsubsection{App for Learning}
This was not much harder than to add new components and functionality for learning. The hard part, was the ideation, deciding on what ideas and what design was the best. For this, see Result \todo{Add reference to Result}.

\subsubsection{Login, Database, and Meteor upgrade}
In order to store data per individual, a database and login would be needed. Meteor upgrade from 1.2 to 1.3 was made to do this easier, but ended up being the reason this was not implemented in Iteration 3. Below, the work is presented.

\textbf{Login}
To record data per user, would require login. This would be a usability issue for most problems, being 1st-time smartphone users. They need to find it intuitive, user-friendly, and be able to remember the password in the future. A lot of different suggestions were through the ideation phase.

The simplest login possible was chosen: a 3-digit code, which was to be given to each coach during the test.

%Jag pratade med flera om detta, Expedition Mondial och Grameen. Från EM lärde jag mig att de trodde min idé med en färdiggjort lista med coachernas namn (vi vet ju vilka som är i Tororo) skulle fungera, och från Grameen fick jag höra om dera erfarenhet att de validerat använda samma approach, med en PIN (längre än 4 siffror dock), men att de inte nailat konceptet ännu, och att de också itererar på sin approach för nästa uppdatering av LedgerLink.

Meteor had limitations with their auto-login module, which is very fast to implement. It forces username and password, and instead I wrote the login myself.

The front-end was not problematic, however, implementing server-client communication so that it worked online and offline, was.

%Tyvärr har också Meteor begränsningar med deras auto-login-modul. Den tvingar både användarnamn och lösenord, och har automatiskt registrering. Går det att stänga av? Jag kan skapa användare och lösenord åt alla, och funderade på hur jag skulle generera lösenord. Ett förslag blev att bara registrera deras förnamn, och sedan skapa lösenordet baserat på T9 med de 6 första bokstäverna utan att berätta det för dem. Sedan tänkte jag på det kulturella, att det kan vara oartigt med förnamn, och bestämde mig för efternamn istället. Hela namnet skulle bli för långt och krångligt.

%Helst skulle jag behöva gå runt Meteors standard-inloggning, och istället ha en enkel login-rullista som den ovan beskrivet, istället för att använda deras standard-lösning.

\textbf{Online database}
If data was to be sent from the client to the server, there needs to be a database with Meteor Collections.

As in version 1 of the app, no results were saved whatsoever, this was new functionality.

An example app was made first, only using Meteor Collections. Meteor's use of Distributed Data Protocol (DDP), made app pushes feel immediate, even though data was not sent until there was Internet access.

However, it was found out that if it took more than 15 minutes to get online, the push would be aborted. For users that are seldom online, this would not be viable.

\textbf{Offline database}
An offline database was needed, and the plugin GroundDB was implemented. As it was cumbersome to get right, pushing the data whenever online, and hard to test (needed to wait 15 minutes each time), this was not ready for the interactions.

\textbf{Upgrading from Meteor 1.2 to 1.3}
Meteor 1.2 had several disadvantages: while it worked for all devices, it did not support React.js

Meteor 1.3 was released, which promised a better developer experience, with JavaScript ES6 support, and access to Node Package Manager (npm), plus official support for React.js.

In 1.2, only some npm packages had been adapted for Meteor, and tools such as Webpack could not be used.

The downsides was discovered after implementation:
\begin{itemize}
\item there were missing backward compatibility to the older of the Android devices
\item Heroku had no Meteor build-pack for 1.3 - a push led the website to crash
\end{itemize}

This meant, that the app would not be able to be installed on many Android devices, and for those devices, a web version would not be available either. As this was unacceptable, the project downgraded to Meteor 1.2 again.

Unfortunately, since the online and offline database had now dependencies on version 1.2, the login and database integration could not be part of iteration 3, but this work needed to be saved for Iteration 4.


For iteration \#4, data collection was done by the app itself, which pushes data to the server whenever online (it saves quiz start, and quiz finish). The server receives JSON data from the client, stored in the MongoDB database hosted on Heroku. Each data point is saved in a database called Results, with the signed in user (from the Users database). In the database, there are collections for Users, Quiz Lists, and Quiz Results.



