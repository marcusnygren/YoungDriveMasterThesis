\section{Application Implementation}

In this section, the prerequisites for the app is described, from the perspective of the user, stakeholders, and the developer.

\subsection{User needs}

The technical constraints for the project, would need to affect the technologies used, if the project would be user-centered.

On the client side, the app would need to be mobile and web based, consider non-access to internet, and not use a lot of battery, to work for the coaches of YoungDrive.

That the app should be simple to use in this cultural setting leaded to design constraints and needs for evaluation.

\subsection{Stakeholder needs}

As the project was only three months, and the first month would be without digital development, time constraints were massive. However, to be able to answer research question \#2, evaluation needed to be done via data collection.

If no evaluation, there would be no need to write code, instead working with a lo-fi prototype using pure design tools. Now, a data-driven approach was needed to measure, and therefore an app needed to be developed.

On the server side, a database and API would be needed, to pull data from the database and push data from the client. Since internet was not always available, the client must be smart in its usage of pushing and pulling data. This would need to be investigated further into the project.

\subsection{Devices to be Used}
As most of the coaches did not have smartphones or tablets, four smartphones (3 Android, 1 iOS) and ten tablets (3 Android, 7 iOS) were brought from Sweden. All of these devices had a web browser and access to an app store. These were either donated, borrowed or bought devices. During the user tests, also using a laptop would be tested.

%\subsection{App/Web Development}
%Early in the project, it was thought that existing tools could be used, instead of building the app from scratch. E.g. using existing tools like Knowly or Typeform\footnote{examples include https://showroom.typeform.com/to/ggBJPd and https://showroom.typeform.com/report/njdbt5/dIzi} during the first iterations for understanding users, and during development e.g. the Typeform API (http://typeform.io/). The Typeform API allows developers to create surveys from within their own applications or systems.

\subsection{Choosing Frameworks for Creating the App}

A JavaScript framework helps and speeds up the creation of building web apps. In the start of the project, Meteor \citep{meteor} and Ionic Framework \citep{ionic} were both tested and compared with each other. It was decided that Meteor was the best way forward, partly because it would allow the app to be accessible on the web as well. %\todo{Add from mindmap}

React \citep{react} (a JavaScript library for building user interfaces) was chosen as the front-end framework, having integration with Meteor and being relatively easy to learn and fast for development.


\subsection{Iteration \#2}
Here, the work and result from iteration \#2 is presented.

\subsubsection{Staging environment using Heroku}
Needed when the Meteor free tier was removed. Connected to deploy from GitHub branches automatically. Could have benefitted from CI, passing tests before ready for production. Solved this by having a stage environment (since April 19th) where stage is YoungDrive-beta (branch Iteration 4), and YoungDrive is master.



For iteration 3, as components grew, there was a need for a client-side router. The Meteor plugin Flow Router was used, as it was very popular with good integrations.

For iteration 3, there was also a need to store data per individual, partly because the feature was prioritized from YoungDrive, but also because of the purpose of data collection. In order to store data per individual, a database and login would be needed. Because of technical difficulties, login and automatic data collection was not implemented until iteration 4, which can be read more about in the Discussion in \ref{backwards-capability}.

\subsection{Login}
To record data per user, would require login. This would be a usability issue for most problems, being 1st-time smartphone users. They need to find it intuitive, user-friendly, and be able to remember the password in the future. A lot of different suggestions were through the ideation phase.

The simplest login possible was chosen, after evaluation and discussion with experts: a 3-digit code, which was to be given to each coach during the test.

%Jag pratade med flera om detta, Expedition Mondial och Grameen. Från EM lärde jag mig att de trodde min idé med en färdiggjort lista med coachernas namn (vi vet ju vilka som är i Tororo) skulle fungera, och från Grameen fick jag höra om dera erfarenhet att de validerat använda samma approach, med en PIN (längre än 4 siffror dock), men att de inte nailat konceptet ännu, och att de också itererar på sin approach för nästa uppdatering av LedgerLink.

Meteor had limitations with their auto-login module, which is very fast to implement. It forces username and password, and instead I wrote the login myself. This was

To summarize, the front-end was not problematic, however, implementing server-client communication so that it worked online and offline, was.

%Tyvärr har också Meteor begränsningar med deras auto-login-modul. Den tvingar både användarnamn och lösenord, och har automatiskt registrering. Går det att stänga av? Jag kan skapa användare och lösenord åt alla, och funderade på hur jag skulle generera lösenord. Ett förslag blev att bara registrera deras förnamn, och sedan skapa lösenordet baserat på T9 med de 6 första bokstäverna utan att berätta det för dem. Sedan tänkte jag på det kulturella, att det kan vara oartigt med förnamn, och bestämde mig för efternamn istället. Hela namnet skulle bli för långt och krångligt.

%Helst skulle jag behöva gå runt Meteors standard-inloggning, och istället ha en enkel login-rullista som den ovan beskrivet, istället för att använda deras standard-lösning.

\subsubsection{Online and offline database}
If data was to be sent from the client to the server, there needs to be a database with Meteor Collections. An example app was made first, only using Meteor Collections. Meteor's use of Distributed Data Protocol (DDP), made app pushes feel immediate, even though data was not sent until there was Internet access.

However, it was found out that if it took more than 15 minutes to get online, the push would be aborted. For users that are seldom online, this would not be viable.

An offline database was needed, and the plugin GroundDB was implemented. As it was cumbersome to get right, pushing the data whenever online, and hard to test (needed to wait 15 minutes each time), this was not ready for the interactions until Iteration 4. As a consequence, until iteration 4 of the app, no results were saved online via the app whatsoever.



For iteration \#4, data collection was done by the app itself, which pushes data to server whenever online (it saves quiz start, and quiz finish). The server receives JSON data from the client, stored in the MongoDB database hosted on Heroku. Each data point is saved in a database called Results, with the signed in user (from the Users database). In the database, there are collections for Users, Quiz Lists, and Quiz Results.


