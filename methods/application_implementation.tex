\section{Application Implementation}

In this section, the prerequisites for the app is described, from the perspective of the user, stakeholders, and the developer.

\subsection{User Needs}

The technical constraints for the project, would need to affect the technologies used, if the project would be user-centered. On the client side, the app would need to be mobile and web based, consider non-access to internet, and not use a lot of battery, to work for the coaches of YoungDrive. That the app should be simple to use in this cultural setting leaded to design constraints and needs for evaluation.

\subsection{Stakeholder Meeds}

As the project was only three months, and the first month would be without digital development, time constraints were massive. However, to be able to answer how design affects learning, evaluation was needed to be done via data collection.

If no evaluation, there would be no need to write code, instead working with a low-fidelity prototype using pure design tools. Now, a data-driven approach was needed to measure, and therefore an app needed to be developed. On the server side, a database and API would be needed, to pull data from the database and push data from the client. Since internet was not always available, the client must be smart in its usage of pushing and pulling data. This would need to be investigated further into the project.

\subsection{Devices are prepared}
As most of the coaches did not have smartphones or tablets, enough smartpones and tablets were brought with me from Sweden, either donated, borrowed or bought devices. These were a combination of Android and iOS, smartphones and tablets, so the app could be tested on as many platforms as possible. During the user tests, also using a laptop would be tested.

%\subsection{App/Web Development}
%Early in the project, it was thought that existing tools could be used, instead of building the app from scratch. E.g. using existing tools like Knowly or Typeform\footnote{examples include https://showroom.typeform.com/to/ggBJPd and https://showroom.typeform.com/report/njdbt5/dIzi} during the first iterations for understanding users, and during development e.g. the Typeform API (http://typeform.io/). The Typeform API allows developers to create surveys from within their own applications or systems.

\subsection{Choosing frameworks for creating the app}

In the start, Ionic and Meteor were both tested and compared with each other. It was decided that Meteor was the best way forward, partly because it would allow the app to be accessable on the web as well. %\todo{Add from mindmap}

React.js was chosen as the front-end framework, having integration with Meteor and being relatively easy to learn and fast for development.



Since Meteor was chosen, a multiple-choice quiz tutorial in Meteor was used to guide the first version of the app. Modifications were made, for example making it responsive and changing it to YoungDrive's graphic profile.

The app was pushed to GitHub, and first hosted on Meteor free storage, available via youngdrive.meteorapp.com. For Android and iOS, it was made possible to install the app from the computer.  For each day of the training in Zambia, new quizzes were added to the app, which created a belt path (see \ref{progress-payoffs}).

After iteration 2, a different hosting platform was needed when the Meteor free tier was removed, where Heroku was chosen. Staging environment using Heroku allowed changes on specific GitHub branches to deploy updates automatically on Heroku servers. The MongoDB database was created using the Heroku plugin MongoLab. A Meteor build-pack was used to allow Meteor to be used with Heroku.

It was also tested to upload the app to Android Play Store. The neccessary steps from Cordova needed to be followed, screenshots needed to be uploaded, and some administrative tasks. After this, it only took a day for the app to appear on the Play Store, and everything worked satisfactory.


\subsection{Iteration 3}

For me, the user's first feeling of a superpower is a hint of becoming a Certified coach. \todo{This can be commented in the Future work}

On the client, as components grew, there was a need for a client-side router. The Meteor plugin Flow Router was used, as it was very popular with good integrations.

\subsubsection{App for Learning}
This was not much harder than to add new components and functionality for learning. The hard part, was the ideation, deciding on what ideas and what design was the best. For this, see Result \todo{Add reference to Result}.

\subsubsection{Login, Database, and Meteor upgrade}
In order to store data per individual, a database and login would be needed. Meteor upgrade from 1.2 to 1.3 was made to do this easier, but ended up being the reason this was not implemented in Iteration 3. Below, the work is presented.

\textbf{Login}
To record data per user, would require login. This would be a usability issue for most problems, being 1st-time smartphone users. They need to find it intuitive, user-friendly, and be able to remember the password in the future. A lot of different suggestions were through the ideation phase.

The simplest login possible was chosen: a 3-digit code, which was to be given to each coach during the test.

%Jag pratade med flera om detta, Expedition Mondial och Grameen. Från EM lärde jag mig att de trodde min idé med en färdiggjort lista med coachernas namn (vi vet ju vilka som är i Tororo) skulle fungera, och från Grameen fick jag höra om dera erfarenhet att de validerat använda samma approach, med en PIN (längre än 4 siffror dock), men att de inte nailat konceptet ännu, och att de också itererar på sin approach för nästa uppdatering av LedgerLink.

Meteor had limitations with their auto-login module, which is very fast to implement. It forces username and password, and instead I wrote the login myself.

The front-end was not problematic, however, implementing server-client communication so that it worked online and offline, was.

%Tyvärr har också Meteor begränsningar med deras auto-login-modul. Den tvingar både användarnamn och lösenord, och har automatiskt registrering. Går det att stänga av? Jag kan skapa användare och lösenord åt alla, och funderade på hur jag skulle generera lösenord. Ett förslag blev att bara registrera deras förnamn, och sedan skapa lösenordet baserat på T9 med de 6 första bokstäverna utan att berätta det för dem. Sedan tänkte jag på det kulturella, att det kan vara oartigt med förnamn, och bestämde mig för efternamn istället. Hela namnet skulle bli för långt och krångligt.

%Helst skulle jag behöva gå runt Meteors standard-inloggning, och istället ha en enkel login-rullista som den ovan beskrivet, istället för att använda deras standard-lösning.

\textbf{Online database}
If data was to be sent from the client to the server, there needs to be a database with Meteor Collections.

As in version 1 of the app, no results were saved whatsoever, this was new functionality.

An example app was made first, only using Meteor Collections. Meteor's use of Distributed Data Protocol (DDP), made app pushes feel immediate, even though data was not sent until there was Internet access.

However, it was found out that if it took more than 15 minutes to get online, the push would be aborted. For users that are seldom online, this would not be viable.

\textbf{Offline database}
An offline database was needed, and the plugin GroundDB was implemented. As it was cumbersome to get right, pushing the data whenever online, and hard to test (needed to wait 15 minutes each time), this was not ready for the interactions.

\textbf{Upgrading from Meteor 1.2 to 1.3}
Meteor 1.2 had several disadvantages: while it worked for all devices, it did not support React.js

Meteor 1.3 was released, which promised a better developer experience, with JavaScript ES6 support, and access to Node Package Manager (npm), plus official support for React.js.

In 1.2, only some npm packages had been adapted for Meteor, and tools such as Webpack could not be used.

The downsides was discovered after implementation:
\begin{itemize}
\item there were missing backward compatibility to the older of the Android devices
\item Heroku had no Meteor build-pack for 1.3 - a push led the website to crash
\end{itemize}

This meant, that the app would not be able to be installed on many Android devices, and for those devices, a web version would not be available either. As this was unacceptable, the project downgraded to Meteor 1.2 again.

Unfortunately, since the online and offline database had now dependencies on version 1.2, the login and database integration could not be part of iteration 3, but this work needed to be saved for Iteration 4.


For iteration \#4, data collection was done by the app itself, which pushes data to the server whenever online (it saves quiz start, and quiz finish). The server receives JSON data from the client, stored in the MongoDB database hosted on Heroku. Each data point is saved in a database called Results, with the signed in user (from the Users database). In the database, there are collections for Users, Quiz Lists, and Quiz Results.


