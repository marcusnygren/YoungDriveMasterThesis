\subsection{Methods to Design for Motivation}

Social psychology can guide the design, when there is a wish to make people behave differently. A big research area is motivational psychology.

With a compelling context, the users are already motivated. Their motivation, is to become better.

Sierra \cite{sierra}, instead suggests the focus to be how to help users progress (see "Progress and payoffs"), and what pulls them off (see "Cognitive load theory").

\subsubsection{Cognitive load theory}

Sierra argues working on what stops people, matters more than working on what entices them. Thus, a focus needs to be identifying and removing blocks.

Sierra \cite{sierra} describes how humans have scarce cognitive resources, and how to design for these.

Cognitive load theory research is divided into three areas: intrinsic CBT, extrinsic CBT, and germane CBT. Below, to design for these are described.

Intrinsic CBT, needs to be dealt with if the effort is too high. Sierra \cite{sierra}describes two strategies. She first says that according to deliberate practice, if you can not get to 95\% reliability within three 45-90 minute sessions, split skills that can be done with effort into sub-skills. The purpose is to reduce time spent practising being mediocre.

Extrinsic CBT, the way presented to a learner, should be handled via designing to support cognitive resources, Sierra says \cite{sierra}.

Scaffolding is a technique to step by step remove the support wheels for the user, e.g. present information in different ways. Gates' \cite{gates} report shows that in their research, each category of scaffolding demonstrated significant effects on learning.

Also, reduce cognitive leaks by e.g. don't make them memorise, and make the thing you want the user to do, the most likely thing to do (affordances). Everything that takes willpower, reduces cognitive leaks.

Germane CBT, is the work put into creating a permanent store of knowledge. To support cognitive resources, escape the brain's spam filter by making the information essential. Either by designing for the compelling context, or desining for just-in-time learning versus just-in-case, Sierra says. \cite{sierra}

\subsubsection{Progress and payoffs}

Sierra aruges that to pull users forward, to stay motivated, progress and payoffs are essential. Both of these, are investigated in terms of motivational psychology.

The feeling of progress can be emphasised by a path with guidelines to help the user know where they are at each step, e.g. for a training.

The best payoff, is a intrinsically rewarding experiences, according to Sierra \cite{sierra}.

It is superb to gamification, says Sierra \cite{sierra}. This is in-line with self-determination theory, where e.g. Pink \cite{pink} says that the surprising truth about what motivates us is that drive is fostered by autonomy, mastery and purpose. The most efficient way is therefore to design for having intrinsically rewarding experiences.

Caring for the compelling context, why the user wants to learn the skill, are helpful strategies. Other strategies are flow, mentioned before, or to give high pay-off tips, helping the user progress in a fair way.

Gates \cite{sierra} says that simple gamification as well as more sophisticated game mechanics can prove effective. However, they add that it should be investigated if "simple gamification" (e.g. contingent point and badges connected to learning activities) more frequently focus on lower-order learning outcomes, compared to studies with more sophisticated game mechanics.
