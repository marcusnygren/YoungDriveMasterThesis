\subsection{Digital Service Design} \label{digital-service-design}

The method combines the benefits of Service Design, Agile Methodologies (namely SCRUM) and Interaction Design. Its purpose is to contribute a holistic approach to the digital design solution for a specific target group. The methodology was co-created by the current author and Expedition Mondial for the master thesis \citep{nissar}.

\subsubsection{The 4 stages of a "Service Sprint"}
In Digital Service Design, an \textit{iteration} is called a \textit{service sprint}. Each iteration includes four steps: insights, ideation, trigger material and interactions. Each steps borrows a number of best practices from agile development or interaction design. There are also new methods, like how a \textit{field hackathon} includes \textit{mini service sprints} each day. Below, the four steps are presented.

\subsubsection{Step 1 - Insights: Analysis, Retrospective \& Stakeholder feedback}
  Insights consists of \textit{analysis} (service design), but also a \textit{retrospective} (SCRUM) and \textit{stakeholder meeting} (service design). In the analysis, the app is evaluated (in terms of interaction design - pleasurability, usability, utility and desirability), and quantitative data is processed (often by clustering data points) and compared with qualitative data (quiz results and questionnaires). This produces an analysis overview of the result. In the \textit{retrospective}, the design process is evaluated ("start doing, stop doing, continue doing"), and changes to the design process are suggested for the following iteration.

    Both the result analysis and the design process analysis is then presented during two stakeholder meetings (service design), structured as \textit{sprint demo}s" (SCRUM), with the purpose of getting feedback. The first "Expert meeting" informs the next iteration's design process, while the second "Partner meeting" informs the next iteration's delivery. From the new insights, a \textit{product backlog} (SCRUM) is converted from needs and ideas into \textit{stories} balancing 1) user needs and 2) stakeholder needs.

\subsubsection{Step 2 - Ideation: Planning Interactions and Delivery}
  Ideation consists of doing \textit{sprint planning} (SCRUM) for the trigger material (a \textit{lo-fi} or/and a \textit{hi-fi prototype}) and the interactions (where tests and workshops and field visits happen).

    \begin{itemize}
    \item Trigger material
      \begin{enumerate}
      \item Ideas are formulated which would satisfy the user needs. This is often a iterative process, which happens in dialogue with chosen experts and entrepreneurs in technology, design and education.
      \item To plan implementation of the ideas, every technical task are laid out, measured in time and prioritized. The least prioritized tasks can thus be cut or moved to the next iteration, in case it is necessary.
      \end{enumerate}
    \item Interactions planning
      \begin{enumerate}
      \item If the technical planning has been realistic, it is time to determine what this iteration's interactions should look like. How will this be tested?
      \item The interactions activities are chosen (what, how, when), so that these are communicated to the local partner, who may schedule the days that will be visited, and solves the needs to the best of their ability.
      \end{enumerate}
    \end{itemize}

  \subsubsection{Step 3 - Trigger Material}
  Trigger material is about preparing the interactions (field visits, interviews, app tests, workshops) and creating the low-fidelity (pen and paper) and high-fidelity prototype (developed app) to be tested with the users. To track the progress and plan effectively, each day starts by a daily stand-up, where today's targets are set, ending by reflecting if the targets were met. If they were not, either the design process needs to change, or something needs to be cut short.

  \subsubsection{Step 4 - Interactions: with "Service Mini-Sprints"}
  Interactions always consists of a sprint demo with the users with the low-fidelity or high-fidelity prototype. During the development process, these are \textit{formative tests}, while for final app evaluation, this is a \textit{summative test}. \textbf{Group tests} are facilitated as workshops. Often, a scenario is presented, devices are given, results are submitted, followed by an open discussion. \textbf{Individual tests} are facilitated in the field (using the before, during, after technique). I observe how the coach does the job today, tests and observes if the app fits into the process, followed by an interview.

    These tests always informs what steps to be taken next, both in terms of app development and interactions. Instead of waiting for the next iteration to do these changes, often a so called Service Mini-Sprint is done.

    In a \textbf{service mini-sprint}\label{mini-sprint}, the insights gathered during the day allows for last-minute adjustments of coming pre-planned workshops (\textit{co-define}, \textit{co-create} or \textit{co-refine}) or field visits (change of interview questions), that can sometimes happen the same day. To take advantage of the precious time with the coaches, at the end of the day, app improvements are made and tomorrow's design process revisited. This means, that already the next day, an improved version of the app can be tested. Similarly, improvements to a workshop format can be improved. These mini-sprints allows for very fast iterations, which can sometimes accelerate the outcome of the visit.
