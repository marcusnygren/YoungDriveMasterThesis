\subsection{Digital Service Design}

As there was a unfamiliar target group - mostly young Ugandians with little or no experience of smartphones - service design thinking would benefit true understanding of cultural context and in-depth empathy for the end users.

Tools and methodology in service design were chosen with the help of Expedition Mondial in Stockholm, who provided education and coaching.

At the same time, the end result would be a digital artefact, which is not common in service design. While the app could be though of as a service, the tools and methodology would need to take this in mind. More suitable approaches would be Agile methodology and Interaction design. These areas, were familiar to me as a computer expert.

This led to the joined development of a Digital Service Design method, created by me and Expedition Mondial. The method combines the benefits of Service Design, Agile Methodologies (namely SCRUM) and Interaction Design. Its purpose was to contribute a holistic approach to the design solution for the specific target group. \cite{nissar-linkedin}

\subsubsection{A "Service Sprint"}
In Digital Service Design, an iteration is called a "service sprint". Similar to service design, it includes four steps: insights, ideation, trigger material and interactions. It has added methodologies from both agile development and interaction design, making the process more suitable. For example, interactions can include mini-service sprints.

\subsubsection{Insights: Analysis, Retrospective \& Stakeholder feedback}
  Insights consists of analysis (service design), but also a retrospective (SCRUM) and stakeholder meeting (service design).

    In the analysis, the app is evaluated (in terms of interaction design - pleasurability, usability, utility and desirability), and quantative data is processed (often by clustering data points) and compared with qualitative data (quiz results and questionnaires). This produces an analysis overview of the result.

    In the retrospective, the design process is evaluated ("start doing, stop doing, continue doing"), and changes to the design process are suggested for the following iteration.

    Both the result analysis and the design process analysis is then presented during two stakeholder meetings (service design), structured as "sprint demo's" (SCRUM), with the purpose of getting feedback.

    The first "Expert meeting" informs the next iteration's design process (with Expedition Mondial), while the second "Partner meeting" informs the next iteration's delivery (with Linköping University and YougnDrive).

    From the insights, a product backlog (SCRUM) is filled with needs and ideas informed by 1) user needs and 2) stakeholder needs.

\subsubsection{Ideation: planning interactions and delivery}
  Ideation consists of a sprint planning (SCRUM). There is one technical planning part, where this sprint's most important user needs from the product backlog are reformulated into stories. There is one test planning part, where interactions are determined and booked.

    \textbf{Technical planning: }

      Ideas are formulated which would satisfy the user needs. This is often a iterative process, which happens in dialogue with chosen experts and entrepreneurs in technology, design and education.

      To plan implementation of the ideas, every technical task are laid out, measured in time and prioritized. The least prioritized tasks can thus be cut or moved to the next iteration, in case it is necessary.

    \textbf{Interactions planning: }

      If the technical planning has been realistic, it is time to determine what this iteration's interactions should look like. How will this be tested?

      The interactions activities are chosen (what, how, when), so that these are communicated to Plan International, who schedules the days I will visit, and solves the needs to the best of their ability.

  \subsubsection{Trigger material}
  Trigger material is about preparing the interactions (field visits, interviews, app tests, workshops) and creating the lo-fi (pen and paper) and hi-fi prototype (developed app) to be tested with the users.

  To track the progress and plan effectively, each day starts by a daily standup, where today's targets are set, ending by reflecting if the targets were met. If they were not, either the design process needs to change, or something needs to be cut short.

  \subsubsection{Interactions: with "Service Mini-Sprints"}
  Interactions always consists of a sprint demo with the users with the lo-fi or hi-fi prototype. During the development process, these are formative tests, while for final app evaluation, this is a summative test.

    Group tests are facilitated as workshops. Often, a scenario is presented, devices are given, results are submitted, followed by an open discussion.

    Field tests are facilitated as naturally as possible (using the before, during, after technique). I observe how the coach does the job today, tests and observes if the app fits into the process, followed by an interview.

    These tests always informs what steps to be taken next, both in terms of app development and interactions. Instead of waiting for the next iteration to do these changes, I do what I call a "Service Mini-Sprint".

    \subsubsection{Service Mini-Sprints}
    The insights gathered during the day allows for last-minute adjustments of coming pre-planned workshops (co-define, co-create or co-refine) or field visits (change of interview questions), that can sometimes happen the same day.

    To take advantage of the precious time with the coaches, at the end of the day, app improvements are made and tomorrow's design process revisited.

    This means, that already the next day, an improved version of the app can be tested. If I was not satisfied with a workshop format, it has been modified.

    These mini-sprints allows for very fast iterations, which can sometimes accelerate the outcome of the visit.
