\subsection{Interdisciplinary Design Process}

%\subsection{Digital Learning}

%\citep{edtech-clark}
%\citep{edtech-sjoden}
%\citep{edtech-dangelo}

\subsubsection{Mobile Learning}

\textbf{The use of deliberates practices on a mobile learning environment}

TODO, superbra artikel

\textbf{An experiment for improving students performance in secondary and tertiary education by means of m-learning auto-assessment}

Luis de-Marcos

% Proved successfull, everyone improved their knowledge. I can use a similar methology, and compare to the development context.

--

Huang et al. (Huang et al., 2008) indicated the common problems encountered in m-learning applications: (1) software integration, (2)
limitations of the web browser, (3) interface usability, (4) reduced size of the screen, and (5) limitation of the battery life. Such limitations are
of particular relevance when the application is intended to run on students’ personal phones; in this case, decisions need to be taken in an
attempt to alleviate the impact such issues may have. Of the problems listed above, item 2 can be mitigated by developing a mobile
application that does not run on the web browser. Items 3 and 4 can be alleviated by designing an interface that minimizes the amount of
information displayed and the input required from the user. This was the main reason for preferring multiple-choice questions to other
kinds of questions, since these questions can usually be stated in a few lines and require the selection of one or more choices. A few mobile
phone buttons can then be programmed to select/unselect each option. Moreover, various experimental studies (Chen, 2010; Ventouras,
Triantis, Tsiakas, \& Stergiopoulos, 2010) support the validity of this assessment method. Solving the problem represented by item 5 was
beyond the scope of this study; however, students were advised to charge their devices before taking the tests and teachers were advised to
design tests of no more than approximately 10 questions, in order to reduce connection times to a maximum of 20 min. Finally, item 1 was
especially difficult to tackle. When the technological framework was set up, our decision was to define the minimal software requirements
that handheld devices would have to meet in order to run the application.

% NTA Digital, Om Digitalt Lärande, Att lära med digitala verktyg
% http://ntadigital.se/teacher/tutorings/2

Interaction design talks about the creation of digital artefacts specifically. When it comes to the design process, it is influenced by related areas such as human-computer science, and more recently human-centered design.

However, various disciplines suggests different design processes. For example, agile development suggest how do develop software efficiently.

Whenever a project is multi-disciplinary, various design processes may need to be combined. Whenever this happens, design thinking becomes a skill essential to thoughtfully design the process.

Löwgren \cite{lowgren} writes about design thinking and useful techniques in general, from his interaction design perspective.

Service design thinking connects various fields of activity \cite{stickdorn}, and it's methodology relies on being close to the users.

While interaction design talks about the creation of digital artefacts specifically, service design talks about the creation of services.

As some digital artefacts are used within a service, or can be thought of as both a product and service simultaneously, the combination of the two are very useful. Service design could help the designer be aware of how such a artefact would need to interplay with its physical environment.

Each discipline holds efficient methods and tools, that can be modified to suit the specific situation even better. From the field of graphic design, mental models are usable. From interaction design, desirability, utility, usability and pleasurability are useful principles. While not naturally a part of service design, these have been useful in service design projects previously. \cite{stickdorn}

In difficult situations, this places demands on the designer. This is where design thinking becomes relevant.

Here, relevant methods and tools are briefly described, and what it means to be a good designer.

\subsubsection{A good designer}

The result of a method can not be better than the people engaging in carrying out the process \cite{lowgren}.

With its user-centered and T-shaped focus \cite{stickdorn}, service design can be said to equip the designer with tools both for reasoning and design ethnography.

This is neccesary, as a good designer can deal with the complexities of design: a satisfactory (and surprising) solution or design can be achieved while working in a highly restricted situation.

\subsubsection{How to deal with relationships and roles}
According to Löwgren, "real" design is about finding ways to design a project within the existing preconditions and limitations \cite{lowgren}.

While a researcher is interested in reality, a designer is interested in what reality could become. \cite{lowgren} Being thoughtful means conceptual clarity from the designer, caring for the vision, and being equipped with appropriate tools of reasoning.

There are three roles as interaction designer in particular can take: the computer expert, the socio-technical expert, and the political agent. The trend is increasingly towards socio-technical experts \cite{lowgren}, the middle ground.

This seems to be a perfect fit with service design, where interaction design is both technical skills and design, and service design can be both design and ethnography. Even more importantly, service design suggests making the whole process co-creative, involving all stakeholders. \cite{stickdorn}

\subsubsection{Thinking of a product as a service}

Service design thinking is described as a process of designing, rather than to its outcome.

A service's intent is to meet customer needs. If it does, it will be used frequently, and recommended. \cite{stickdorn}

As this is often not the case, service design can be applicable to fields including social design, product design, graphic design and interaction design.

The result can be a product service hybrid. When designed and considered well, service design shapes the value proposition and desirability of the product for the better.

\subsubsection{Starting the project}

Löwgren writes about the beginning of a project: This is where the designer gets involved in design work, establishes a preliminary understanding of the situation, navigates through available information, and initiates all neccessary relationships with clients, users, decision makers, and so forth. Based on all this, she creates a design proposal. \cite{lowgren}.
