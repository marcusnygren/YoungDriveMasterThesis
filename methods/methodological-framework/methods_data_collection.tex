\subsection{Data Analysis} \todo{Konrad: include the qualitative methods of analysis}

This section presents relevant methods for data analysis.

\subsubsection{Visualizing Data}

Here, each step of the visualization pipeline is presented, allowing analysis of data.

The Visualization Pipeline describes the process of generating an image from the data: \citep{timo-ropinski-liu}

\begin{enumerate}
\item Data acquisition ($\,\to\,$data are given)
\item Data enhancement ($\,\to\,$ data are processed)
\item Visualization mapping ($\,\to\,$ data are mapped to for example a geometry)
\item Rendering ($\,\to\,$ images generated)
\end{enumerate}

% Timo Ropinski, Scientific Visualization Group, Linköping University, TNM067 - Scientific Visualization, 9/12/2014)
% https://drive.google.com/drive/u/0/folders/0BzlK1PD8EE75bHIxcXRQNWpRMm8

Data acquisition presents how data was acquired.

Data enhancement explains how the data was processed.

Visualization mapping is the process of mapping data to e.g. a geometry.

Finally, rendering allows images to be generated, presented in 2D.

\subsubsection{Calculating Correlation}

Calculate means, follow formula. Cumbersome to do with all of the axises against all the agises.

This can be done in Google Sheets as well as the R programming language.

\textbf{In Google Sheets: }

It is clear that analysis in Google Sheets can only go so far. It can be greatly helpful to sort by multiple columns (e.g. first by Manual?, then by School level, then by Quiz 3). However, it takes a long time to filter the data on multiple parameters, and the work easily becomes tedious. For some applications, it may not be viable to discover the data using this approach.

One approach is to calculate and compare means on a "control"  with a response variable.

\textbf{Psuedo-code in R would be: }

\begin{verbatim}
x1 = c(1,2,3,1,5,6)
x2 = c(2,3,4,NA,6,7)
cor(x = x1, y = x2)
cor.test(x1,x2)
\end{verbatim}

\subsubsection{Visualizing Correlation}

In Google Sheets, color scale can be used to give different column values different colors.

It is still hard to compare all of the axises towards all the axises, and it is not a scientific approach.

\todo{Include figure}

In R programming language there are more powerful tools for visualizing correlation, e.g. using a "Correlation Heatmap".

Psuedo-code in R would be:

\begin{verbatim}
random_matrix <- matrix(rnorm(100), nrow = 10, ncol = 10)
random_matrix[1,1] <- NA
colnames(random_matrix) <- paste("V",1:10)
cor_mat <- cor(random_matrix)
heatmap(cor_mat, keep.dendro = FALSE)
\end{verbatim}

The result would be:

\todo{Include figure}

\subsubsection{Calculating Logistic Regression}

A limitation with correlation is that only two dimensions can be compared with each other.

With multiple-variable data, Logistic Regression is helpful if our response variable can be a logistical dimension (e.g. women or male, used manual or not), while linear regression needs to be used if it is a linear or nominal scale (e.g. age and city respectively).

In either case, the first step is to determine a response variable: the variable I want to compare against, e.g. is there a difference between men and women? In my case, it needs to be a quantative measure of: "Have you learned anything?".

If I add more variable, e.g. also adding if a manual was used, this is called my "control". It is possible to add as many controls as possible.

If linear regression, then I need to determine a quantative measure of ("How much have you learned?").

In Google Sheets, this is not effective to do. R, however, is a very suitable tool.

First, the data is loaded, e.g. as a CSV file. Then, we tell R which the N/A values are, e.g. "N/A" or "Vet ej". We use this to filter the data.

Then, each column we want to use is converted into a factor.

When factors, a model can be created, e.g. using the General Linear Model. A different family can be selected, e.g. binomial.

Then it is possible for R to show this data, showing the coefficient, the Pr value, and others. See code below.

\begin{verbatim}
mydata <- read.csv("Development/R/quizResults.csv", na.strings = c("N/A", "Vet ej"))

mydata$y = ifelse(test = is.na(mydata$Quiz.9..y.n.1st),  yes = 0, no = 1)

mydata$y <- as.factor(mydata$y)
mydata$Help <- as.factor(mydata$Help)
mydata$Sex <- as.factor(mydata$Sex)

mymodel <- glm(formula = y ~ Pre.test.score + Sex, data = mydata, family = 'binomial')

summary(mymodel)

plot()
\end{verbatim}

For analysis, looking at the summary, coefficient (e.g. -1.0704) shows either a negative or positive correlation (in this case -7\%) for what I compare with as a response variable.

To be significantly significant, a common measure is that the Pr value ("the p-value") needs to be higher than 0.05. If the p-value is higher than 0.05, meaning it is significant with a 95\% probability.

\subsubsection{Analysing data with a Parallel Coordinates Visualization}

To learn how to analyse the data, Une-terre \citep{une-terre} was consulted. % http://une-terre.blogspot.se/2012/09/parallel-coordinates-read-out-patterns.html
He writes "||-coords are a data visualisation which allow you to "read out" the relationships and trends between your dimensions. Positive relationship (correlation), negative relationship (invert), or no relationship (random)."
