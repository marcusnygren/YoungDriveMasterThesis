\subsection{Enabling Data Collection}

For iteration 3, as components grew, there was a need for a client-side router. The Meteor plugin Flow Router was used, as it was very popular with good integrations. For iteration 3, there was also a need to store data per individual, partly because the feature was prioritized from YoungDrive, but also because of the purpose of data collection. In order to store data per individual, a database and login would be needed. Because of technical difficulties, login and automatic data collection was not implemented until iteration 4, which is further described in the Discussion, section \ref{backwards-capability}.

\subsubsection{Login}
To record data per user, would require login. This would be a usability issue for most problems, being first-time smartphone users. They need to find it intuitive, user-friendly, and be able to remember the password in the future. A lot of different suggestions were through the ideation phase. The simplest login possible was chosen, after evaluation and discussion with experts: a 3-digit code, which was to be given to each coach during the test. Meteor had limitations with their auto-login module, which is very fast to implement. Thus, a 3-digit login was written manually. To summarize, the front-end was not problematic, however, implementing server-client communication so that it worked online and offline, was.

%Jag pratade med flera om detta, Expedition Mondial och Grameen. Från EM lärde jag mig att de trodde min idé med en färdiggjort lista med coachernas namn (vi vet ju vilka som är i Tororo) skulle fungera, och från Grameen fick jag höra om dera erfarenhet att de validerat använda samma approach, med en PIN (längre än 4 siffror dock), men att de inte nailat konceptet ännu, och att de också itererar på sin approach för nästa uppdatering av LedgerLink.

%Tyvärr har också Meteor begränsningar med deras auto-login-modul. Den tvingar både användarnamn och lösenord, och har automatiskt registrering. Går det att stänga av? Jag kan skapa användare och lösenord åt alla, och funderade på hur jag skulle generera lösenord. Ett förslag blev att bara registrera deras förnamn, och sedan skapa lösenordet baserat på T9 med de 6 första bokstäverna utan att berätta det för dem. Sedan tänkte jag på det kulturella, att det kan vara oartigt med förnamn, och bestämde mig för efternamn istället. Hela namnet skulle bli för långt och krångligt.

%Helst skulle jag behöva gå runt Meteors standard-inloggning, och istället ha en enkel login-rullista som den ovan beskrivet, istället för att använda deras standard-lösning.

\subsubsection{Online and Offline Database}
If data was to be sent from the client to the server, there needs to be a database with Meteor Collections. An example app was made first, only using Meteor Collections. Meteor's use of Distributed Data Protocol (DDP), made app pushes feel immediate, even though data was not sent until there was Internet access.

However, it was found out that if it took more than 15 minutes to get online, the push would be aborted. For users that are seldom online, this would not be viable. An offline database was needed, and the plugin GroundDB was implemented. As it was cumbersome to get right, pushing the data whenever online, and hard to test (needed to wait 15 minutes each time), this was not ready for the interactions until Iteration 4. As a consequence, until iteration 4 of the app, no results were saved online via the app whatsoever.

