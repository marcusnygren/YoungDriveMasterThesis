\subsection{Devices to be used}
As most of the coaches did not have smartphones or tablets, four smartphones (3 Android, 1 iOS) and ten tablets (3 Android, 7 iOS) were brought from Sweden. All of these devices had a web browser and access to an app store. These were either donated, borrowed or bought devices. During the user tests, also using a laptop would be tested.

%\subsection{App/Web Development}
%Early in the project, it was thought that existing tools could be used, instead of building the app from scratch. E.g. using existing tools like Knowly or Typeform\footnote{examples include https://showroom.typeform.com/to/ggBJPd and https://showroom.typeform.com/report/njdbt5/dIzi} during the first iterations for understanding users, and during development e.g. the Typeform API (http://typeform.io/). The Typeform API allows developers to create surveys from within their own applications or systems.

\subsection{Choosing frameworks for creating the app}

A JavaScript framework helps and speeds up the creation of building web apps. In the start of the project, Meteor \citep{meteor} and Ionic Framework \citep{ionic} were both tested and compared with each other. It was decided that Meteor was the best way forward, partly because it would allow the app to be accessible on the web as well. %\todo{Add from mindmap}

React \citep{react} (a JavaScript library for building user interfaces) was chosen as the front-end framework, having integration with Meteor and being relatively easy to learn and fast for development.
