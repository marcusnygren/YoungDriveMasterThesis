\subsection{Devices are prepared}
As most of the coaches did not have smartphones or tablets, enough smartpones and tablets were brought with me from Sweden, either donated, borrowed or bought devices. These were a combination of Android and iOS, smartphones and tablets, so the app could be tested on as many platforms as possible. During the user tests, also using a laptop would be tested.

%\subsection{App/Web Development}
%Early in the project, it was thought that existing tools could be used, instead of building the app from scratch. E.g. using existing tools like Knowly or Typeform\footnote{examples include https://showroom.typeform.com/to/ggBJPd and https://showroom.typeform.com/report/njdbt5/dIzi} during the first iterations for understanding users, and during development e.g. the Typeform API (http://typeform.io/). The Typeform API allows developers to create surveys from within their own applications or systems.

\subsection{Choosing frameworks for creating the app}

In the start, Ionic and Meteor were both tested and compared with each other. It was decided that Meteor was the best way forward, partly because it would allow the app to be accessable on the web as well. %\todo{Add from mindmap}

React.js was chosen as the front-end framework, having integration with Meteor and being relatively easy to learn and fast for development.
