\subsection{Implementation of Learning Methods}

\subsubsection{Compelling context}
My compelling context is that I want to help you become an even better coach.

The better user point of view: don’t just make a better coach training app - make a better user of coach training material.

For me, this means:

"Given a teaching situation among the youth group, a great coach can teach an entrepreneurship topic more consistent with what the coach material said."

"Given a question in the app, a great coach will get the right answer more often, and increasingly leverage the correct answer to their coach situation."

\subsubsection{Deliberate practice}

Help them practice right. Goal: design practice exercises that will take a fine-grained task from unreliable to 95\% reliability, within one to three 45-90-minute sessions.

\subsubsection{Considerations for Entrepreneurship Education}
The scope of the app is to examine and strengthen the entrepreneurship the student already has. One important goal is to give good feedback.

The YoungDrive's entrepreneurship education methodology goes hand in hand with the presented theory. It's mottos are: "Dream big, start small", "Learning by doing" and "We have fun!" \cite{youngdrive}.

Both in regards to designing for the users and for the above reason, the app should be a complement to YoundDrive's existing training material and the structure of the program.

A challenging part of the work is that YoungDrive consists of both the practical skills of the entrepreneur, theoretical material of running a business, and an entrepreneurial mindset. Therefore, both how to assess knowledge, and build habits, needs to be examined.

\subsubsection{Learning from Assessment}

Lorum ipsum
