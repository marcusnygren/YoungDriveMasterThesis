\section{Collaborators for the Master Thesis}\label{sec:collaborators}

Collaborators in the project are the current author, supervisors, stakeholders and experts. Below, the responsibilities of these are more clearly laid out.

\subsubsection{The Current Author}
It is needed to take on several roles in the project by the current author: most notably that of a project leader, designer and developer. It is needed to balance stakeholders' different opinions and requirements, and caring for the vision in order for the project to be successful (see section \ref{aGoodDesigner} A Good Designer).

There are two groups, with the current author included in both of them, which gather at the end of each sprint for a check-up meeting. The Expert group consisted of Expedition Mondial and LiU Innovation. Expedition Mondial could help with the design process, and LiU Innovation could offer input on social innovation. The meetings mostly lasted for one hour. The Partner group consisted Iliana Björling from YoungDrive, and Lena Tibell and Konrad Schönborn from Linköping University. In Partner meetings, The Insighs from each iteration was presented and discussed. Then possible decisions were laid out, followed by discussing the alternatives. Outside of these groups, these people can also give advice in certain situations. For specific areas, there are also some experts which have been beneficial during the projects. Below, the whole team is explained: % Then I tell them about which decisions has been taken and why.

\subsubsection{Supervisors}
The supervisors are from YoungDrive and Linköping University. The YoungDrive team consists of Iliana Björling, founder of YoungDrive, and Josefina Lönn, country manager in Zambia. They are both helpful in giving knowledge on the entrepreneurship education program, and giving support. The Linköping University team consists of Lena Tibell, Professor, and Konrad Schönborn, Doctor, within the Department of Visual Learning and Communication.

\subsubsection{Stakeholders}
The stakeholders are considered YoungDrive and Plan International. \textbf{YoungDrive} is the client of the work, and their needs should be satisfied. This person is mainly represented by Iliana Björling, who is part of the YoungDrive Strategic Management Team. Using service design, the project leaders in Uganda and Zambia, are also considered stakeholders: Josefina Lönn in Zambia, and the two co-project leaders in Uganda. Finally, the most important stakeholder of all according to service design, is the actual users: the coaches. They should be the main consideration of the work.

\textbf{Plan International} is the organization allowing for all the interactions with the end users in Uganda. A similar organization is operational in Zambia. They are the ones that are providing facilities, organizes transport, etcetera. They in turn, have the organization Community Vision, which organizes the coaches. If Plan International or Community Vision does not appreciate of the project and the collaboration, then the interactions with the coaches will not be possible.

%\textbf{Linköping University} is a stakeholder, as the supervisor (Lena Tibell) and examinator (Camilla Forsell) determine if the work is a valid master thesis or not. Also, LiU Innovation is interested in supporting continued work with the project, and their representative Peter Gahnström gives advice on social innovation and how this project can continue in the future during expert meetings.

\subsubsection{Advisors}
Since the development country context is new to the current author, there are also specific experts advised in the project. For design process, Susanna Nissar and Erik Widmark from Expedition Mondial has supported with all of their knowledge within service design. Julien Tantege, Research Specialist at Grameen Foundation, has been kind to offer support before and during the work, sharing their insights from related work, and giving feedback during ideation. She has experience doing technical development for rural areas. For pedagogical development, Henrik Marklund from edtech startup Knownly in Sweden has given support with regards to building skills within digital learning. For feedback for how the work relates to social innovation, Peter Gahnström at LiU Innovation has offered feedback.
