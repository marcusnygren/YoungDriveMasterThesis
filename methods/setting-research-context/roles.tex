\subsection{Roles within YoungDrive}

The \textit{country manager} trains the project leaders. It is also the main person responsible for partnerships and the quality of the YoungDrive program in the respective country. In Uganda, the country manager is Iliana Björling. She is located in the Uganda capital, Kampala, which is a strategic location because it is the same city in which the national office of the main partner, Plan International, is located. In Zambia, the country manager is Josefina Lönn, who previously was the project leader in Kampala, and has held all the trainings up to this point. Now, she leads the operations and has trained the coaches in Zambia, in the new role of country manager and project leader.

The \textit{project leaders} train the coaches and oversee the coaches, manages the coach training, and also collaborates with local stakeholders for quality assurance and to oversee daily operations.

The \textit{coaches} train the youth. In Zambia, a coach only has responsibility for training youth in the YoungDrive program. In Uganda, this is called a \textit{Youth Mentor (YM)}, in contrast to being a \textit{Community Based Trainer (CBT)}, which also trains the youth in other programs and leads the youth saving groups. Most of the CBT's in Uganda holds sessions together with a Youth Mentor, or divides work between them, instead of being alone. The coaches are often volunteers, receiving a small scholarship from the partner organization. They are often business owners themselves. The coaches could be described as social entrepreneurs \citep{mitchel}. Many of the YoungDrive coaches (and youth) are driven by that their business can have an impact on their community, \textit{as well} as take them out of unemployment or increase their current livelihood.

The \textit{youth} are the ones receiving the training from the CBTs and the YMs, being encouraged to start their own businesses.
