\begin{acknowledgments}
%Fakta om rapportens tillkomst.
%Tacka personer som hjälpt till med informations- eller språkgranskning.
%Säg inget om resultat, slutsatser, syfte eller andra uppgifter

  Due to a chain of lucky events, this master thesis took the approach of combining service design, thoughtful interaction design, technology, learning effectiveness research, and entrepreneurship.

For service design, I want to thank Peter Gahnström at LiU Innovation, who led me to Expedition Mondial, and there I especially want to thank Susanna Nissar for being a great tutor.

Thoughtful interaction design was introduced thanks to a recommendation from Lena Tibell and Konrad Schönborn, to go to Jonas Löwgren's first test lecture about interaction design at Linköping University, which led me to looking deeper into his literature, and discovering the book Thoughtful Interaction Deisgn, which was a perfect fit into the world of interaction design for me, coming from an engineering perspective. It showed me what being a great designer really meant, and I was compelled.

For technology, I wish to thank Grameen Foundation in Kampala for their advise and expertise.

For learning effectiveness research, I want to thank Lena Tibell and Konrad Schönborn from a learning perspective, and Henrik Marklund at Knowly for a digital learning perspective. Annika Silvervarg at LiU was also helpful for giving me ,litterature tips within learning and gamification.

For entrepreneurship education, I have so many to thank. But by biggest thanks goes to Konrad Schönborn, and the YoungDrive and Plan team consisting of Iliana Björling, Josefina Lönn and Gerald Emoyo at Plan.

Thank you for your contributions! Lastly but not least, I want to thank my fiancee, Linnea Rothin, who still is the best service designer and coach for my master thesis that could ever be. It was you that put me right on the master thesis (and also allowed me to travel for three months across the globe).

Thank you.

% Se t.ex. exjobbet "Development of handheld mobile applications for the public sector" av Fredrik Bergström och Gustav Engvall

  \addvspace{1em}
  \begin{flushright}
    \textit{%
      Linköping, Januari 2020\\
      N N och M M%
    }
  \end{flushright}
\end{acknowledgments}
