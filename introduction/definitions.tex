\section{Definitions} % operationalise the terms used and the variables that you will measure / investigate. e.g.: "entrepreneurship", "entrepreneur eduction", "training", "effectiveness", "coaching", ... (and so on...) (i.e. the ideas you will expose in section 5 below)

The following words will be defined according to scientific definitions in the final report.

\textit{Entrepreneurship} is the act of creating new businesses. An \textit{entrepreneur education} is when an entrepreneur goes trough training. 
\textit{Training} can be both physical and digital training, but always has the purpose to improve the skills or knowledge of the trained.
\textit{Effectiveness} is about keeping the same quality with less means (economical, physical, time resources, etc).
\textit{Coaching} is the activity in which a person is helped by being asked questions and support, often by a person.
A {\textit{digital tool}} is an electronic help for a person, designed to solve or assist a person in solving a task that otherwise would have been more cumbersome. A \textit{digital education}, is an education which takes place on an electronic device, either partly or fully.
An {app} or \textit{application} is a kind of digital tool, and can often be downloaded from an app store, either on mobile or web.

In the final report, \textit{entrepreneurship education} can be defined according to Ruskovaara (2015), and I can also use Liñán, F. (2004). The author talks about Intention-based models of entrepreneurship education. Piccolla Impresa/Small Business, 3(1), 11-35, contains a definition which may be useful as well.

% Learni

\textit{Formative assessment} (given to you, for your own sake) instead of \textit{summative assessment} (given to the employee, for the employee's sake). You have to secure that it's a process. You have to see that there is an effect! "Assessing for Learning" :) - much debated, has drawbacks. Feedback is one of the most effective ways for learning.