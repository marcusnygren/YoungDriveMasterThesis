\section{Background}

    %\input{introduction/background/a_working_future}

    %\subsection{Digital Education}

    In this section, the background of the master thesis is presented.

    The first section describes the opportunities for entrepreneurship in Uganda, followed by how Plan International and YoungDrive uses this to tackle child poverty by fostering and educating youth in starting their own businesssess. This section concludes by how digital learning and digital tools becomes increasingly demanded, which is why this master thesis has emerged.

    The final section so gives a background to the status of digital education today, followed by related work to the thesis.

    All of this together leads up to the purpose of the master thesis.

    \subsection{Social Innovation and Social Entrepreneurship in Uganda} % https://www.linkedin.com/pulse/social-innovation-entrepreneurship-uganda-why-mobile-services-nissar?trk=prof-post

    This section will present background on working with mobile learning platforms, and understanding the society of entrepreneurs in Uganda.

    \subsubsection{Why Uganda is the world's most entrepreneurial country}
    Some facts about Uganda in terms of entrepreneurship are \cite{nissar}:

    \begin{itemize}
      \item Uganda is the world's most entrepreneurial country. (28\% av of the population are entrepreneurs)
        \item Uganda has the second youngest population in the world (77\% of all Ugandans are below 30)
        \item Uganda has a very high unemployment rate (64 \% of people between 18–30 are unemployed)
    \end{itemize}

    % Ytterligare beskrivning av land: http://www.sun-connect-news.org/countries/uganda/

    With a high unemployment rate and little or none social security, starting a business is for many young entrepreneurs simply a tool for survival.

    But tough conditions can also lead to creativity, and there are as well many innovative entrepreneurs with great ideas and the aim to create positive social impact.

    No matter the reason of starting a business, Uganda's many entrepreneurs are contributing to the national society by boosting the economy and creating new jobs.

    \subsubsection{Why mobile services is growing fast in Uganda}
    One of the reasons is that the country has invested heavily in communication networks, even connecting remote rural villages with fibre optic cables and thereby connecting them to a world of information.

    As much as 65\% of the adults in Uganda owns a cell phone, which has allowed many areas in the country to skip the landline stage of development and jump right to the digital age.

    For those who hasn’t electricity at home, there are plentiful of charging booths for mobiles all over the country.

    \subsubsection{Mobile services and social innovations}
    The wide use of mobile phones has lead the way for the development of several innovative mobile services and in many cases the mobile service are way ahead of us. In Sweden mobile banking services that allows us to transfer money through our mobile phones were made popular with Swish, introduced in 2012. In Kenya people have had similar services for the last 10 years.


    \subsection{The Client: YoungDrive}

In this section, the project that the client YoungDrive is in is first described, and then how YoungDrive  fit into the structure with its entrepreneurship education program. In the last part, future plans of YoungDrive and A working future is presented, giving relevancy to the field of digital education.

\subsubsection{A working future}

Plan International works towards eliminating child poverty, and their project A working future, supported by SIDA 2012-2016, tackles unemployment among youth in rural areas.

It runs for 12 000 youth in Kamuli and Tororo.

\subsubsection{VSLA groups and CBT's}

Because of high tuition fees, saving (financial literacy) and earning (practicing vocational skills) are central.

VSLA groups have existed for many years, where a group starts a village savings and loans group together. A democratic process makes the group independent of e.g. banks, which rates are in general high and which may not even borrow money, either because of long distances to the bank or of no previous financial history.

For Plan International, VSLA groups have been successful in several countries for a long time. However, while the groups were skilled with saving, they did not always spend the money in the most strategic way.

Plan's pilot with A working future, was to introduce trainings on top of the VSLA structure.

Where Community Based Trainers (CBTs) were previously only responsible for hosting the groups, not they were trained and tasked with carrying out different programs: like agriculture, financial literacy, and the most successful program to date, focusing on running own businessess, YoungDrive.

\subsubsection{YoungDrive}

YoungDrive is based on a Swedish concept, and had previously had a pilot in Botswana, when tasked with running the entrepreneurship module of A working future. The organization foster and educate young entrepreneurs in developing countries. They train the CBT's, provide training material, and support the CBT's via direction and direct support via co-project leaders and Youth Mentors (YMs).

YoungDrive moves an entrepreneur to location, becoming country manager. Then, she educates project leaders during four days, followed by educating CBT's, which then roll out the training to the youth groups during 10 sessions, 1 session per week in average. The CBT's also rolls out other trainings, often simultaneously.

\subsubsection{Social entrepreneurship}

The CBT's are often volunteers, receiving a small scholarship from Plan International. They are often business owners themselves.

Thus, the CBT's can be described as social entrepreneurs. As Mitchel says about entrepreneurship \cite{mitchel}, motivation does not need to be wealth accumulation anymore. The activity of entrepreneurship contributes to society, in a way that is not caputed by the commercial entrepreneurship literature.

Many of the YoungDrive participants are driven by that their business can have an impact on their community, as well as take them out of unemployment or increase their current livelihood.

\subsubsection{Future plans}

For the future of YoungDrive, they want to make the CBT's even better, and collect and take us of data (monitoring and evaluation). Another motivation is scaling and monetization, as Plan International wants to increase the project to more countries, with an increased digital focus, and YoungDrive wants to be independent of project funding (i.e. a social enterprise). This was a great time to introduce digital enablers, where there previously had been no technology-focus, especially towards CBT's and YM's.


    \subsection{Digital Education}

    In recent time, e-learning has had a tremendous impact both outside and inside the classroom. With a growing teacher interest, research so far shows that digital education is hard, risky and possibly rewarding. Thus, digital education shows both great potential and great considerations.

    \subsubsection{Brining research into reality}

    Gates \cite{gates} has done a comprehensive study, which motivates why a digital tool or game is a good thing by showing a .33 standard deviations in intrapersonal learning outcomes, relative to non-game instructional conditions. They also conclude, that design rather than medium alone predicts learning outcomes.

    Much of the research to date on digital games has focused on proof-of-concept studies and media comparisons. The study's comparison, is to focus on how theoretically-driven decisions influence learning outcomes: for the broad diversity of learners, within and beyond the classroom.

    \subsubsection{Caring for the context}
    Luckin \cite{luckin} emphasises the need to care for the context. Stickdorn \cite{stickdorn} exemplifies how the design process should be altered when the context is social innovation.

    Service design in a social innovation context is called "social design", and is a new field. \cite{stickdorn}. No longer is service design solely focused on creating and promoting consumer goods, but to offer services to society. The design process should be designed to tackle a social issue, or with the intent to improve human lives. The focus is on delivering positive impact.

    \subsubsection{E-assessment}
    There are numerous examples of developments in e-assessment using mobile environments, as well as immersive environments and social and collaborative environments.

    Interest in formative e-assessment is increasing. A large amount of development has taken place on diagnostic testing environments, that allow teachers and learners to assess present performance against prior performance. \cite{luckin}

    Luckin says that further consideration should  be given to how technology can be used to enable the assessment of knowledge and skills not usually distinguished within current curricula. \cite{luckin} One such example would be entrepreneurship.

    \subsection{Related work}

\subsubsection{Caring for the context}

One great example of a mobile banking service that is a true social innovation is Ledger Link, developed by Grameen Foundation in collaboration with Barclays Bank. This mobile banking service empowers saving groups in rural areas to save money. It is developed with human centered design methods and were lucky to meet up with Juliet, Julius and Joseph, three of the persons behind it, during our visit. \cite{nissar-linkedin}

One great example of an education service that is true social innovation is iSchool, developed by iSchool Zambia. Their app platform is designed to fit the Zambia school curriculum to the point, accessible as a home edition, pupil edition and teacher edition.

\subsubsection{E-assessment / M-learning}

Two studies within e-assessment have been done that this master thesis is inspired by. One uses deliberate practices on a mobile learning environement \cite{yengin}. The other focused on and further validated the research of various experimental studies, that multiple-choice can be a viable auto-assessment method to improving student learning, especially for m-learning \cite{de-marcos}.

