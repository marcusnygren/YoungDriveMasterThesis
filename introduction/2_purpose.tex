\section{Purpose}
%Under rubriken Syfte redogör du för själva frågeställningen och syftet med rapporten. Förklara vad rapporten ska handla om och vad du hoppas kunna uppnå för resultat. Definiera begrepp som inte är kända för målgruppen.

\subsection{Entrepreneurial tools}
In order for young ambitious entrepreneurs to build sustainable enterprises they need to have basic entrepreneurial skills. This is where Marcus Nygren’s mobile learning platform comes into the picture.

The entrepreneurship education is an initiative of Illiana Björling from Young Drive, in collaboration with Plan International, and it has educated, supported and inspired 12 000 Ugandan youth in the process of starting their own businesses.

\subsection{The app}

By training coaches that can carry out the education in larger groups of entrepreneurs, the education reach many young people at the same time. Marcus Nygren’s mobile learning platform will improve the effect even more.

The statistics are promising: One year after the entrepreneurship education 73\% of the participants are running profitable businesses and they have employed 1,5 persons in average as well. \cite{nissar-linkedin}

\subsubsection{Purpose of the app}
The app has the following purposes:

\begin{itemize}
  \item Validate the coaches' level of knowledge during their education
    \item Train the coaches on distance
    \item Certify all staff
\end{itemize}

Young Drive's experience goal for the app is "It should be easy to understand, pedagogical and enjoyable to use, and the coaches should think it is fun and meaningful to learn via the app".

\subsection{Larger scope}
The entrepreneurship education program YoungDrive requests two digital modules, to reach even better results and to be able to scale up the operations to more locations with confidence. The overall aim of my Master thesis is to do a Minimum Viable Product (MVP) of module 1, the Coach module. The master thesis is about how to design an app for entrepreneurship education, including evaluating it's effectiveness towards the coaches. The result is an app which the coaches use during and after the coach training.
