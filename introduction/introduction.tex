\chapter{Introduction}\label{cha:intro}

%Ge läsaren en introduktion till rapporten genom att sätta in läsaren i ämnet. Presentera ej resultat eller detaljerad information.

%Inledningsvis, gör:
%Redogör för syftet och frågeställningar
%Redogör den använda metoden
%Bakgrundsbeskrivning
%- sätt in läsaren i ämnet
%- tidigare forskningen på området
%Redogörelse av resultat, referera och relatera till tidigare forskningen
%Knyt ihop resultaten, analysera dessa, och sätt in resultaten i ett vidare perspektiv

The following master thesis was carried out in the Masters program of Media Technology and Engineering at the Department of Science and Technology, Linköping University. The work has been carried out for the non-profit entrepreneurship academy YoungDrive in Uganda and Zambia. The organization teaches starting a business to youth in developing countries, with YoungDrive coaches now educating 12 000 youth within Plan International's project in Uganda, and now also running a coach training in Zambia. An application for the entrepreneurship coaches is thought to help coaches assess what they know, and learn to become even better with teaching entrepreneurship to the youth.

There are both opportunities and challenges for mobile tools for rural areas of Uganda. While entrepreneurship activity is high, and mobile services are growing rapidly (see \ref{sec:mobile-uganda}), the two fields have not yet been combined for entrepreneurial learning in a developing country context. To properly answer YoungDrive's need, entrepreneurship education and digital education has been researched together with how to develop apps for web and mobile platforms simultaneously. The research questions focuses on the process and result of developing an app used by first-time smartphone users for entrepreneurship learning.

\section{Purpose}
\label{purpose}
%Under rubriken Syfte redogör du för själva frågeställningen och syftet med rapporten. Förklara vad rapporten ska handla om och vad du hoppas kunna uppnå för resultat. Definiera begrepp som inte är kända för målgruppen.

In order for young ambitious entrepreneurs to build sustainable enterprises they need to have basic entrepreneurial skills. A mobile learning platform could asses and teach those skills either complementing a physical entrepreneurship education, or being accessible to the entrepreneur whenever and wherever deemed most important.

\subsection{An App for the Entrepreneurship Coaches}
The entrepreneurship education YoungDrive is an initiative of Illiana Björling from YoungDrive, now collaboration with Plan International. Within the project "A working future", they have educated, supported and inspired 12 000 Ugandan youth in the process of starting their own businesses \citep{nissar}.

The overall aim of the master thesis is to develop a working prototype of an entrepreneurship coach training app. The master thesis is about how to design an app for entrepreneurship education, including evaluating it's effectiveness towards the coaches. The app should be suited for the coach training, and have multiple-choice questions as its basis mechanics.

\subsection{Definition of Success for the App}

By training coaches that can carry out the education in larger groups of entrepreneurs, the education reach many young people at the same time. A mobile learning platform is predicted to improve the effect of the training even more, by fulfilling the following purposes:

\begin{itemize}
  \item Validate the coaches' level of knowledge during their education
    \item Train the coaches on distance
    \item Certify all staff
\end{itemize}

YoungDrive's experience goal for the app is "It should be easy to understand, pedagogical and enjoyable to use, and the coaches should think it is fun and meaningful to learn via the app" \citep{youngdrive-masterthesis-idea}.


% \section{Limitations}
%Du kan också redogöra för avgränsningar här. Dessa kanske du inte kan precisera i början av arbetet utan de växer fram under arbetets gång.

The following things will be overlooked:
\begin{itemize}
% Please rephrase, solely on material repetition
\item Solely existing YoungDrive teaching material will tested using the app, not new material, or other entrepreneurship programs.
\item The app will be a compliment to the physical YoungDrive training, not a replacement. This would be interesting continued work.
\item Ideally, the master thesis would include measuring how app usage affected their youth session quality, measured by the coach, the youth, and co-project leaders.
\item If this would have been the case, there could have been three different control groups: A, using the app and the YoungDrive training, B, using only the YoungDrive training, and C, using only the app.
\end{itemize}


%\section{Definitions} % operationalise the terms used and the variables that you will measure / investigate. e.g.: "entrepreneurship", "entrepreneur eduction", "training", "effectiveness", "coaching", ... (and so on...) (i.e. the ideas you will expose in section 5 below)

The following definitions of words will be used while reading:

\subsubsection{Design situation}
\textit{Entrepreneurship} is the act of creating new businesses.

An \textit{entrepreneur education} is when an entrepreneur goes trough training. (It can be defined according to Ruskovaara (2015) or Liñán, F. (2004) as well. The author talks about Intention-based models of entrepreneurship education. Piccolla Impresa/Small Business, 3(1), 11-35, contains a definition which may be useful as well.)

\textit{Training} can be both physical and digital training, but always has the purpose to improve the skills or knowledge of the trained.

\textit{Effectiveness} is about keeping the same quality with less means (economical, physical, time resources, etc).

\textit{Coaching} is the activity in which a person is helped by being asked questions and support, often by a person.

\subsubsection{Digital development}
A {\textit{digital tool}} is an electronic help for a person, designed to solve or assist a person in solving a task that otherwise would have been more cumbersome.
A \textit{digital education}, is an education which takes place on an electronic device, either partly or fully.
An \textit{app} or \textit{application} is a kind of digital tool, and can often be downloaded from an app store, either on mobile or web.

\subsubsection{Design process}
\textit{Interaction design} describes the creation of digital artefacts.

\subsubsection{Learning}
\textit{Formative assessment} (given to you, for your own sake) instead of \textit{summative assessment} (given to the employee, for the employee's sake). You have to secure that it's a process. You have to see that there is an effect! "Assessing for Learning" :) - much debated, has drawbacks. Feedback is one of the most effective ways for learning.


\section{Research questions} 
% beskrivning av vilka frågeställningar arbetet syftar till att besvara

There are two main research questions. The first question is the main focus of the master thesis (75\%), whereas question \#2 should stand for circa 25\% of the work.

\begin{enumerate}
    \item How can a design process of an application for entrepreneurial learning be implemented in a developing country context? %How can an app be designed and developed to train entrepreneurs? % = An app for entrepreneurship education
    \begin{enumerate}
        \item How is the development affected by the technical possibilities?
\item How is the design affected by the contextual  constraints, e.g. young entrepreneurs, entrepreneurship education, and culture? % such as X, Y, Z, \Å?
        % \item How can service design be used in a developing world context when building an app?
        % \item What should the design and development process look like
        \item How can quiz questions be developed to support entrepreneurship learning? % Bloom
    \end{enumerate}
    \item How can user’s feedback be used to inform modifications of the app?
    \item How does design affect usability and learning done via the app? %How effective is the app for training entrepreneurs? % (= evaluation) % how effective is the app in training entrepreneurs? % Reviderat från: "How much more effective becomes the coach in training entrepreneurs via the app?"
    %\begin{enumerate}
		%\item  % of an app with the purpose to train entrepreneurs? % What principles will inform the design of the app?
	\end{enumerate}
%\end{enumerate}

% --

% Innovation is how you connect different parts into something new, says Peter Gahnström, LiU Innovation.

% App development in place in Uganda with the target group, combined with service design, means that I have innovation height. The result using this methodology may be an interesting contribution to the field of service design within a development context.

% The method I am using becomes the scalability of the method, it may be possible to repeat. When I get statistics that my method works, it may be possible to a lot more, and other organizations (e.g. innovation offices at universities) might be able to use my results.

% \chapter{Theory}\label{cha:Theory}
%

\section{Theoretical Background}

To understand how to reach the objectives of the project, this chapter presents background and related work.

%Part 1-2 deals with the design situation, part 3-4 gives introductions to relevant topics, and part 5 presents related work.

%\subsubsection{Part 1-2: Design situation}
%For design situation, the client context is described. This also includes a motivation for digital learning, and related work to the thesis. The first section describes the opportunities for entrepreneurship in Uganda, followed by how Plan International and YoungDrive uses this to tackle child poverty by fostering and educating youth in starting their own businesses. This section concludes by how digital learning and digital tools becomes increasingly demanded, which is why this master thesis has emerged.

%\subsubsection{Part 3-4: Relevant topics}
%Next an introduction is given into entrepreneurship education, digital learning, and hybrid app development. Examples within digital tools are named that sometimes have considered a developing country context. Two studies within e-assessment are named, which have combined learning theory to a mobile or computer platform.

%\subsection{Design situation}

    %\subsection{A Working Future, Plan Uganda}

"With funding and technical support from SIDA, Plan International Sweden and Accenture Development Partnership (ADP), Plan International Uganda is implementing A Working Future – Uganda (AWF), a three and a half year project intended to support employment and improve the economic empowerment of youth, with the targeted age group being 12,000 15 to 25 year-old young people.  Savings Groups are used as an entry point to communities, a mechanism for organising youth and a platform for financial education and capital build-up.

To help youth move into the local economy, AWF provides practical entrepreneurship training and post-training mentoring.  The greatest impact has been on individual income generating activities where youth have undertaken a new activity or diversified or somehow improved existing activities.  An informal inventory of activities revealed a good variety of businesses in small trade, buying and selling agricultural produce, food processing and some animal-raising.  Many of these are the result of market research rather than youth just copying activities already existing in their communities.

A lot of these businesses can be started on a modest scale with small amounts of capital that youth can access from their Savings Groups.  Most members seem to have a plan for scaling up and show great discipline in reinvesting profits to build a larger capital base.  One field assessment showed that many had doubled or multiplied their individual investment several times in a two to four month period.

Forging relations with private sector businesses to help youth take advantage of markets outside their communities is another feature of AWF.  A micro franchise relationship has grown between a large manufacturer of household products and Savings Groups members who serve as sales agents in their communities.  Other marketing and technical links exist with a large commercial producer of poultry and pigs and a distributor of solar lighting products."
% https://plan-international.org/youth-savings-group-uganda
% http://www.sida.se/English/where-we-work/Africa/Uganda/examples-of-results/A-working-future/
% http://comvisug.org/
% http://www.ugandangodirectory.org/index.php/Microfinance/386-community-vision

% http://awards.ixda.org/entry/2014/ledger-link/
% Craft
%"We used paper prototypes to co-design with the members and learn how the group would use the app during meetings. We encouraged them to modify the low-fi interface and work with us to make sure it would suit their needs.
%There are nearly 40 languages in Uganda; creating many local-language versions of the app wasn’t an option. As is common in Uganda, financial transactions often make use of English phrases, and each group included some members who speak basic English, so we created the app in English. However, members often speak what they called “village English,” which required us to work closely with them to choose appropriate labels.
%The application is only part of the offer; to provide a new technology, the team could not rely on individuals discovering the application themselves. We first tested to determine whether groups understood the need for a sophisticated phone (they did); and whether they would be interested in purchasing this phone (they were).
%Training and support processes are just as important as the app. To successfully train groups, we leveraged Grameen Foundation’s experience instructing poor farmers to use technology and worked through community organizations who were trusted by the groups.

% http://www.comvisug.org/AWFU-achievements.php
%  To enhance safety of the group savings through alternative saving measures, COMVIS in partnership with PLAN Uganda have brought on board GRAMEENand AIRTEL companies with a product of AIRTEL WEZA where groups save their money in a group Simcard. Both the Youth Mentors and CBTs were trained on how these transactions are to be done.

    %\subsection{Digital Education}

    % The statistics are promising: One year after the entrepreneurship education 73\% of the participants are running profitable businesses and they have employed 1,5 persons in average as well.

    \subsection{Social Innovation and Social Entrepreneurship in Uganda} % https://www.linkedin.com/pulse/social-innovation-entrepreneurship-uganda-why-mobile-services-nissar?trk=prof-post

    This section will present background on working with mobile learning platforms, and understanding the society of entrepreneurs in Uganda.

    \subsubsection{Why Uganda is the world's most entrepreneurial country}
    According to Nissar \citep{nissar}, some facts related to entrepreneurship in Uganda are:

    \begin{itemize}
      \item Uganda is the world's most entrepreneurial country. (28\% of the population are entrepreneurs)
        \item Uganda has the second youngest population in the world (77\% of all Ugandans are below 30)
        \item Uganda has a very high unemployment rate (64 \% of people between 18–30 are unemployed)
    \end{itemize}

    % Ytterligare beskrivning av land: http://www.sun-connect-news.org/countries/uganda/

    With a high unemployment rate and little or none social security, starting a business is for many young entrepreneurs simply a tool for survival. But tough conditions can also lead to creativity, and there are as well many innovative entrepreneurs with great ideas and the aim to create positive social impact.

    As Mitchel says about entrepreneurship \citep{mitchel}, the motivation of entrepreneurship does not need to be solely wealth accumulation anymore. The activity of entrepreneurship contributes to society, in a way that is not captuted by the commercial entrepreneurship literature.

    No matter the reason of starting a business, Uganda's many entrepreneurs are contributing to the national society by boosting the economy and creating new jobs.

    \subsubsection{Why mobile services are growing rapidly in Uganda}
    One of the reasons is that the country has invested heavily in communication networks, even connecting remote rural villages with fibre optic cables and thereby connecting them to a world of information.

    As much as 65\% of the adults in Uganda owns a cell phone, which has allowed many areas in the country to skip the landline stage of development and jump right to the digital age.

    For those who hasn’t electricity at home, there are plentiful of charging booths for mobiles all over the country.

    \subsubsection{Mobile services and social innovations}
    The wide use of mobile phones has lead the way for the development of several innovative mobile services and in many cases the mobile service are way ahead of us  \citep{nissar}. In Sweden mobile banking services that allows us to transfer money through our mobile phones were made popular with Swish, introduced in 2012. In Kenya people have had similar services for the last 10 years.


    \subsection{Mobile Technology in Uganda's Rural Areas}\label{sec:mobile-uganda}

One of the reasons why mobile services are growing rapidly in Uganda is that the country has invested heavily in communication networks, even connecting remote rural villages with fibre optic cables and thereby connecting them to a world of information.

As much as 65\% of the adults in Uganda owns a cell phone, which has allowed many areas in the country to skip the landline stage of development and jump right to the digital age. For those who hasn’t electricity at home, there are available charging booths for mobiles all over the country.

The wide use of mobile phones in Uganda and other developing countries has lead the way for the development of several innovative mobile services and in many cases the mobile service are way ahead of us  \citep{nissar}. In Sweden mobile banking services that allows us to transfer money through our mobile phones were made popular with Swish, introduced in 2012. In the neighbouring country Kenya, people have had similar services for the last 10 years, and mobile money is since long also common in Uganda.

A prominent example of an app that has previously been developed with the target group in mind is Ledger Link \citep{ledgerlink}. This mobile banking service empowers, developed in partnership with a bank, allows saving groups in rural areas such as Tororo to save money remotely. It is developed with human-centered design methods, and has won several awards.

%For the future of YoungDrive, they want to make the CBT's even better, and collect and take use of data (monitoring and evaluation). Another motivation is scaling and monetization, as Plan International wants to increase the project to more countries, with an increased digital focus, and YoungDrive wants to be independent of project funding (i.e. a social enterprise). This was a great time to introduce digital enablers, where there previously had been no technology-focus, especially towards CBT's and Youth Mentors. The master thesis is the first project which focuses on digital enablers for YoungDrive.

%\input{introduction/background/theory/app_development}



    %\subsection{Entrepreneurship education}

%\input{theory/learning/entrepreneurship/definition}

%\input{theory/learning/entrepreneurship/entrepreneurship-education}

%Kuratko, D. F. (2005). The emergence of entrepreneurship education: Development, trends, and challenges. Entrepreneurship theory and practice, 29(5), 577-598.

%Pittaway, L., & Cope, J. (2007). Entrepreneurship education a systematic review of the evidence. International Small Business Journal, 25(5), 479-510.

%Bae, T. J., Qian, S., Miao, C., & Fiet, J. O. (2014). The relationship between entrepreneurship education and entrepreneurial intentions: A meta‐analytic review. Entrepreneurship Theory and Practice, 38(2), 217-254.

Entrepreneurship education has been a growing field of investigation over the last three decades. While Dickson \citep{dickson} says there are few empirical studies available, examples include among others Kuratko \citep{kuratko}, Pittaway \citep{pittaway} and Bae \citep{bae}.

Especially relevant for this work is the recent interest in interventions for teaching and learning entrepreneurship in the developing world: \citep{oviawe} \citep{iakovleva}

% OVIAWE, M. J. I. (2010). Repositioning Nigerian youths for economic empowerment through entrepreneurship education. European Journal of Educational Studies, 2(2).

%Iakovleva, T., Kolvereid, L., & Stephan, U. (2011). Entrepreneurial intentions in developing and developed countries. Education+ Training, 53(5), 353-370.

First, Oviawe \citep{oviawe} conclude by how teaching of creativity and problem-solving skills seems to be especially beneficial for entrepreneurship in developed countries. In YoungDrive, the youth are tasked with starting their own business from no capital, which fosters creativity and problem-solving skills.

Further, Iakovleva \citep{iakovleva} indicated that respondents from developing countries do have stronger entrepreneurial intentions than those from developed countries. This stems from attitudes, subjective norms, and perceived behavioural control. Their encouragement, is that developing countries need to focus on the development of institutions that
can support entrepreneurial efforts. YoungDrive is one such example.

Ruskovaara and Pihkala \citep{ruskovaara} concludes, that the teacher seems to be the main factor for entrepreneurship education, and that research agrees with them. There seems to be no indication of difference between men and women, nor previous professional teaching experience. They could find that entrepreneurial activity seems to lead to better entrepreneurship education. Dickson's recommendations for enhancing entrepreneurship education practices were mainly two things. First, the playful side of teaching and learning is mentioned. Secondly, they encourage teacher training that develops the competences as a mentor, enabler or coach. \citep{dickson}


    %\subsection{Mobile Learning}

    YoungDrive has asked for a multiple-choice question learning game used for assessment and improvement of the YoungDrive learning objectives during and after the training. In recent times, digital learning (e-learning) has had a tremendous impact both outside and inside the classroom. With a growing teacher interest, research still so far shows that digital education is hard, risky and possibly rewarding \citep{luckin}. Thus, digital education shows both great potential and great considerations. Much of the research on digital learning to date on digital games has focused on proof-of-concept studies and media comparisons \citep{luckin}. A study by \cite{gates} motivates why a digital tool or game is a good thing by showing an increase in intrapersonal learning outcomes, relative to non-game instructional conditions.

    %Luckin \citep{luckin} emphasises the need to care for the context when designing for learning. Stickdorn \citep{stickdorn} exemplifies how the design process should be altered when the context is social innovation. Service design in a social innovation context is called "social design", and is a new field. \citep{stickdorn}. No longer is service design solely focused on creating and promoting consumer goods, but to offer services to society. The design process should be designed to tackle a social issue, or with the intent to improve human lives. The focus is on delivering positive impact.

    A large amount of development has taken place on diagnostic testing environments, that allow teachers and learners to assess present performance against prior performance \citep{luckin}. There are numerous examples of developments in \textit{e-assessment} using mobile environments, as well as immersive environments and social and collaborative environments. One such example is the educational app platform iSchool, developed by iSchool Zambia \citep{ischool}. The app has been praised and made popular as it was designed to fit the Zambia school curriculum to the point, accessible as a home edition, pupil edition and teacher edition. Interest in formative e-assessment is increasing.  It has been shown that multiple-choice tests in e-assessment can be used to good effect \citep{nicol}.

    Two studies within electronic assessment (e-assessment) or mobile learning (m-learning) have been done that this master thesis is inspired by. One uses deliberate practices on a mobile learning environement \citep{yengin}. The other focused on and further validated the research of various experimental studies, that multiple-choice can be a viable auto-assessment method to improving student learning, especially for m-learning \citep{de-marcos}.

    %Nicol, D. (2007). E‐assessment by design: using multiple‐choice tests to good effect. Journal of Further and higher Education, 31(1), 53-64.

    Luckin says that further consideration should be given to how technology can be used to enable the assessment of knowledge and skills not usually distinguished within current curricula \citep{luckin}. \cite{gates} encourages a focus on how theoretically-driven decisions influence learning outcomes: for the broad diversity of learners, within and beyond the classroom. Combining these two, introducing e-assessment of entrepreneurship in a developing country context is a contribution to existing research.


    %\subsection{Hybrid App Development}

The history of app and web development is rich and increasingly intertwined. First, websites were developed for desktop only, and when smartphones became popular, they were made responsive.

With today's possibilities of native mobile development or developing a native app using web technologies, there are numerous viable alternatives available if an app should function on several devices, depending on budget and preferences.

One of the main argument for developing an app in web technologies, is that the whole application, including the server, can be written in one programming language, JavaScript (full-stack).

Tools such as Apache Cordova can compile JavaScript applications into native apps. Thus, they can appear on Apple iOS and Android Play Store, as well as on the web, or installable offline on a smartphone from the computer.

JavaScript is developing rapidly as a language, as well as its ecosystem of frameworks and tools. Frameworks have emerged and matured, like Meteor.js, which makes building full-stack applications in JavaScript reliable and fast.

Previously, web hosting has been troublesome for JavaScript server applications. Today, tools such as Meteor.js and Heroku have introduced free and paid hosting for such applications, with smart bindings to code platforms such as GitHub, which makes collaboration and version handling easy.




%\section{Method}
%Metoddelen beskriver tillvägagångssättet – intervjuer, observationer, litteraturstudier, laborationer och så vidare. Motivera varför en viss metod valdes och vilka eventuella svårigheter som har förekommit. Metoden ska vara replikerbar, vilket innebär att en annan skribent ska kunna göra om studien med hjälp av informationen i metoddelen. Det finns en mängd böcker om olika vetenskapliga metoder. Till exempel kan en intervju utföras på en mängd olika sätt. I rapporter inom humaniora brukar metoddelen vara mer utförlig än i en teknisk rapport.

%\section{Discussion} % av källor
% Redogör för dina viktigaste källor och kommentera dem kritiskt. Motivera också ditt val av källor. Det kritiska förhållningssättet är speciellt viktigt då källor från internet används.

%\section{Report Structure} % Structure of the report

% I avsnittet Struktur redogör du kort för hur rapporten är disponerad. Detta hjälper läsaren att hitta i rapporten.

%\subsection{Typography conventions}
