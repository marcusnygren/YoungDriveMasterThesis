% \chapter{Theory}\label{cha:Theory}
%

%\section{Theoretical Background}

%%\subsection{Design situation}

    %\subsection{A Working Future, Plan Uganda}

"With funding and technical support from SIDA, Plan International Sweden and Accenture Development Partnership (ADP), Plan International Uganda is implementing A Working Future – Uganda (AWF), a three and a half year project intended to support employment and improve the economic empowerment of youth, with the targeted age group being 12,000 15 to 25 year-old young people.  Savings Groups are used as an entry point to communities, a mechanism for organising youth and a platform for financial education and capital build-up.

To help youth move into the local economy, AWF provides practical entrepreneurship training and post-training mentoring.  The greatest impact has been on individual income generating activities where youth have undertaken a new activity or diversified or somehow improved existing activities.  An informal inventory of activities revealed a good variety of businesses in small trade, buying and selling agricultural produce, food processing and some animal-raising.  Many of these are the result of market research rather than youth just copying activities already existing in their communities.

A lot of these businesses can be started on a modest scale with small amounts of capital that youth can access from their Savings Groups.  Most members seem to have a plan for scaling up and show great discipline in reinvesting profits to build a larger capital base.  One field assessment showed that many had doubled or multiplied their individual investment several times in a two to four month period.

Forging relations with private sector businesses to help youth take advantage of markets outside their communities is another feature of AWF.  A micro franchise relationship has grown between a large manufacturer of household products and Savings Groups members who serve as sales agents in their communities.  Other marketing and technical links exist with a large commercial producer of poultry and pigs and a distributor of solar lighting products."
% https://plan-international.org/youth-savings-group-uganda
% http://www.sida.se/English/where-we-work/Africa/Uganda/examples-of-results/A-working-future/
% http://comvisug.org/
% http://www.ugandangodirectory.org/index.php/Microfinance/386-community-vision

% http://awards.ixda.org/entry/2014/ledger-link/
% Craft
%"We used paper prototypes to co-design with the members and learn how the group would use the app during meetings. We encouraged them to modify the low-fi interface and work with us to make sure it would suit their needs.
%There are nearly 40 languages in Uganda; creating many local-language versions of the app wasn’t an option. As is common in Uganda, financial transactions often make use of English phrases, and each group included some members who speak basic English, so we created the app in English. However, members often speak what they called “village English,” which required us to work closely with them to choose appropriate labels.
%The application is only part of the offer; to provide a new technology, the team could not rely on individuals discovering the application themselves. We first tested to determine whether groups understood the need for a sophisticated phone (they did); and whether they would be interested in purchasing this phone (they were).
%Training and support processes are just as important as the app. To successfully train groups, we leveraged Grameen Foundation’s experience instructing poor farmers to use technology and worked through community organizations who were trusted by the groups.

% http://www.comvisug.org/AWFU-achievements.php
%  To enhance safety of the group savings through alternative saving measures, COMVIS in partnership with PLAN Uganda have brought on board GRAMEENand AIRTEL companies with a product of AIRTEL WEZA where groups save their money in a group Simcard. Both the Youth Mentors and CBTs were trained on how these transactions are to be done.

    %\subsection{Digital Education}

    % The statistics are promising: One year after the entrepreneurship education 73\% of the participants are running profitable businesses and they have employed 1,5 persons in average as well.

    \subsection{Social Innovation and Social Entrepreneurship in Uganda} % https://www.linkedin.com/pulse/social-innovation-entrepreneurship-uganda-why-mobile-services-nissar?trk=prof-post

    This section will present background on working with mobile learning platforms, and understanding the society of entrepreneurs in Uganda.

    \subsubsection{Why Uganda is the world's most entrepreneurial country}
    According to Nissar \citep{nissar}, some facts related to entrepreneurship in Uganda are:

    \begin{itemize}
      \item Uganda is the world's most entrepreneurial country. (28\% of the population are entrepreneurs)
        \item Uganda has the second youngest population in the world (77\% of all Ugandans are below 30)
        \item Uganda has a very high unemployment rate (64 \% of people between 18–30 are unemployed)
    \end{itemize}

    % Ytterligare beskrivning av land: http://www.sun-connect-news.org/countries/uganda/

    With a high unemployment rate and little or none social security, starting a business is for many young entrepreneurs simply a tool for survival. But tough conditions can also lead to creativity, and there are as well many innovative entrepreneurs with great ideas and the aim to create positive social impact.

    As Mitchel says about entrepreneurship \citep{mitchel}, the motivation of entrepreneurship does not need to be solely wealth accumulation anymore. The activity of entrepreneurship contributes to society, in a way that is not captuted by the commercial entrepreneurship literature.

    No matter the reason of starting a business, Uganda's many entrepreneurs are contributing to the national society by boosting the economy and creating new jobs.

    \subsubsection{Why mobile services are growing rapidly in Uganda}
    One of the reasons is that the country has invested heavily in communication networks, even connecting remote rural villages with fibre optic cables and thereby connecting them to a world of information.

    As much as 65\% of the adults in Uganda owns a cell phone, which has allowed many areas in the country to skip the landline stage of development and jump right to the digital age.

    For those who hasn’t electricity at home, there are plentiful of charging booths for mobiles all over the country.

    \subsubsection{Mobile services and social innovations}
    The wide use of mobile phones has lead the way for the development of several innovative mobile services and in many cases the mobile service are way ahead of us  \citep{nissar}. In Sweden mobile banking services that allows us to transfer money through our mobile phones were made popular with Swish, introduced in 2012. In Kenya people have had similar services for the last 10 years.


    \subsection{Mobile Technology in Uganda's Rural Areas}\label{sec:mobile-uganda}

One of the reasons why mobile services are growing rapidly in Uganda is that the country has invested heavily in communication networks, even connecting remote rural villages with fibre optic cables and thereby connecting them to a world of information.

As much as 65\% of the adults in Uganda owns a cell phone, which has allowed many areas in the country to skip the landline stage of development and jump right to the digital age. For those who hasn’t electricity at home, there are available charging booths for mobiles all over the country.

The wide use of mobile phones in Uganda and other developing countries has lead the way for the development of several innovative mobile services and in many cases the mobile service are way ahead of us  \citep{nissar}. In Sweden mobile banking services that allows us to transfer money through our mobile phones were made popular with Swish, introduced in 2012. In the neighbouring country Kenya, people have had similar services for the last 10 years, and mobile money is since long also common in Uganda.

A prominent example of an app that has previously been developed with the target group in mind is Ledger Link \citep{ledgerlink}. This mobile banking service empowers, developed in partnership with a bank, allows saving groups in rural areas such as Tororo to save money remotely. It is developed with human-centered design methods, and has won several awards.

%For the future of YoungDrive, they want to make the CBT's even better, and collect and take use of data (monitoring and evaluation). Another motivation is scaling and monetization, as Plan International wants to increase the project to more countries, with an increased digital focus, and YoungDrive wants to be independent of project funding (i.e. a social enterprise). This was a great time to introduce digital enablers, where there previously had been no technology-focus, especially towards CBT's and Youth Mentors. The master thesis is the first project which focuses on digital enablers for YoungDrive.

%\subsection{Hybrid App Development}

The history of app and web development is rich and increasingly intertwined. First, websites were developed for desktop only, and when smartphones became popular, they were made responsive.

With today's possibilities of native mobile development or developing a native app using web technologies, there are numerous viable alternatives available if an app should function on several devices, depending on budget and preferences.

One of the main argument for developing an app in web technologies, is that the whole application, including the server, can be written in one programming language, JavaScript (full-stack).

Tools such as Apache Cordova can compile JavaScript applications into native apps. Thus, they can appear on Apple iOS and Android Play Store, as well as on the web, or installable offline on a smartphone from the computer.

JavaScript is developing rapidly as a language, as well as its ecosystem of frameworks and tools. Frameworks have emerged and matured, like Meteor.js, which makes building full-stack applications in JavaScript reliable and fast.

Previously, web hosting has been troublesome for JavaScript server applications. Today, tools such as Meteor.js and Heroku have introduced free and paid hosting for such applications, with smart bindings to code platforms such as GitHub, which makes collaboration and version handling easy.




%Part 1-2 deals with the design situation, part 3-4 gives introductions to relevant topics, and part 5 presents related work.

%\subsubsection{Part 1-2: Design situation}
%For design situation, the client context is described. This also includes a motivation for digital learning, and related work to the thesis. The first section describes the opportunities for entrepreneurship in Uganda, followed by how Plan International and YoungDrive uses this to tackle child poverty by fostering and educating youth in starting their own businesses. This section concludes by how digital learning and digital tools becomes increasingly demanded, which is why this master thesis has emerged.

%\subsubsection{Part 3-4: Relevant topics}
%Next an introduction is given into entrepreneurship education, digital learning, and hybrid app development. Examples within digital tools are named that sometimes have considered a developing country context. Two studies within e-assessment are named, which have combined learning theory to a mobile or computer platform.
