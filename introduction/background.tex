% \chapter{Theory}\label{cha:Theory}
%

\section{Theoretical Background}

To understand how to reach the objectives of the project, this chapter presents background and related work.

%Part 1-2 deals with the design situation, part 3-4 gives introductions to relevant topics, and part 5 presents related work.

%\subsubsection{Part 1-2: Design situation}
%For design situation, the client context is described. This also includes a motivation for digital learning, and related work to the thesis. The first section describes the opportunities for entrepreneurship in Uganda, followed by how Plan International and YoungDrive uses this to tackle child poverty by fostering and educating youth in starting their own businesses. This section concludes by how digital learning and digital tools becomes increasingly demanded, which is why this master thesis has emerged.

%\subsubsection{Part 3-4: Relevant topics}
%Next an introduction is given into entrepreneurship education, digital learning, and hybrid app development. Examples within digital tools are named that sometimes have considered a developing country context. Two studies within e-assessment are named, which have combined learning theory to a mobile or computer platform.

%\subsection{Design situation}

    %\input{introduction/background/a_working_future}

    %\subsection{Digital Education}

    % The statistics are promising: One year after the entrepreneurship education 73\% of the participants are running profitable businesses and they have employed 1,5 persons in average as well.

    %\subsection{Social Innovation and Social Entrepreneurship in Uganda} % https://www.linkedin.com/pulse/social-innovation-entrepreneurship-uganda-why-mobile-services-nissar?trk=prof-post

    This section will present background on working with mobile learning platforms, and understanding the society of entrepreneurs in Uganda.

    \subsubsection{Why Uganda is the world's most entrepreneurial country}
    According to Nissar \citep{nissar}, some facts related to entrepreneurship in Uganda are:

    \begin{itemize}
      \item Uganda is the world's most entrepreneurial country. (28\% of the population are entrepreneurs)
        \item Uganda has the second youngest population in the world (77\% of all Ugandans are below 30)
        \item Uganda has a very high unemployment rate (64 \% of people between 18–30 are unemployed)
    \end{itemize}

    % Ytterligare beskrivning av land: http://www.sun-connect-news.org/countries/uganda/

    With a high unemployment rate and little or none social security, starting a business is for many young entrepreneurs simply a tool for survival. But tough conditions can also lead to creativity, and there are as well many innovative entrepreneurs with great ideas and the aim to create positive social impact.

    As Mitchel says about entrepreneurship \citep{mitchel}, the motivation of entrepreneurship does not need to be solely wealth accumulation anymore. The activity of entrepreneurship contributes to society, in a way that is not captuted by the commercial entrepreneurship literature.

    No matter the reason of starting a business, Uganda's many entrepreneurs are contributing to the national society by boosting the economy and creating new jobs.

    \subsubsection{Why mobile services are growing rapidly in Uganda}
    One of the reasons is that the country has invested heavily in communication networks, even connecting remote rural villages with fibre optic cables and thereby connecting them to a world of information.

    As much as 65\% of the adults in Uganda owns a cell phone, which has allowed many areas in the country to skip the landline stage of development and jump right to the digital age.

    For those who hasn’t electricity at home, there are plentiful of charging booths for mobiles all over the country.

    \subsubsection{Mobile services and social innovations}
    The wide use of mobile phones has lead the way for the development of several innovative mobile services and in many cases the mobile service are way ahead of us  \citep{nissar}. In Sweden mobile banking services that allows us to transfer money through our mobile phones were made popular with Swish, introduced in 2012. In Kenya people have had similar services for the last 10 years.


    %\subsection{Mobile Technology in Uganda's Rural Areas}\label{sec:mobile-uganda}

One of the reasons why mobile services are growing rapidly in Uganda is that the country has invested heavily in communication networks, even connecting remote rural villages with fibre optic cables and thereby connecting them to a world of information.

As much as 65\% of the adults in Uganda owns a cell phone, which has allowed many areas in the country to skip the landline stage of development and jump right to the digital age. For those who hasn’t electricity at home, there are available charging booths for mobiles all over the country.

The wide use of mobile phones in Uganda and other developing countries has lead the way for the development of several innovative mobile services and in many cases the mobile service are way ahead of us  \citep{nissar}. In Sweden mobile banking services that allows us to transfer money through our mobile phones were made popular with Swish, introduced in 2012. In the neighbouring country Kenya, people have had similar services for the last 10 years, and mobile money is since long also common in Uganda.

A prominent example of an app that has previously been developed with the target group in mind is Ledger Link \citep{ledgerlink}. This mobile banking service empowers, developed in partnership with a bank, allows saving groups in rural areas such as Tororo to save money remotely. It is developed with human-centered design methods, and has won several awards.

%For the future of YoungDrive, they want to make the CBT's even better, and collect and take use of data (monitoring and evaluation). Another motivation is scaling and monetization, as Plan International wants to increase the project to more countries, with an increased digital focus, and YoungDrive wants to be independent of project funding (i.e. a social enterprise). This was a great time to introduce digital enablers, where there previously had been no technology-focus, especially towards CBT's and Youth Mentors. The master thesis is the first project which focuses on digital enablers for YoungDrive.

%\subsection{Hybrid App Development}

The history of app and web development is rich and increasingly intertwined. First, websites were developed for desktop only, and when smartphones became popular, they were made responsive.

With today's possibilities of native mobile development or developing a native app using web technologies, there are numerous viable alternatives available if an app should function on several devices, depending on budget and preferences.

One of the main argument for developing an app in web technologies, is that the whole application, including the server, can be written in one programming language, JavaScript (full-stack).

Tools such as Apache Cordova can compile JavaScript applications into native apps. Thus, they can appear on Apple iOS and Android Play Store, as well as on the web, or installable offline on a smartphone from the computer.

JavaScript is developing rapidly as a language, as well as its ecosystem of frameworks and tools. Frameworks have emerged and matured, like Meteor.js, which makes building full-stack applications in JavaScript reliable and fast.

Previously, web hosting has been troublesome for JavaScript server applications. Today, tools such as Meteor.js and Heroku have introduced free and paid hosting for such applications, with smart bindings to code platforms such as GitHub, which makes collaboration and version handling easy.




    \subsection{Entrepreneurship Education}

%\input{theory/learning/entrepreneurship/definition}

%\subsubsection{Entrepreneurship Education}

"Entrepreneurship Education in Schools: Empirical Evidence on the Teacher’s Role" says that "The findings indicate that the training teachers have received in entrepreneurship seems to be the main factor determining the observable entrepreneurship education provided by the teachers."

%\citep{entrepreneurship-pihkala}

In this study, we aimed to bring empirical data into the discussion on entrepreneurship education, as there are still few empirical studies available on the topic area (Dickson et al., 2008),

Also assessment practices that include peer and self-assessment have brought new depth into assignments and their completion. Similarly, activity outside of the classroom (Fayolle \& Gailly, 2008; Kickul et al., 2010; Shepherd, 2004; Solomon, 2007) is stated to have widened learners’ perceptions of their possibilities to be active citi- zens, and to also have clarified the role of different actors in society. In addition to the above, Rae and Carswell (2001) utilize entrepreneurship cases to analyze how the self-confidence and self-awareness of learners have grown.
Fiet (2001a) presented a group of methods and argues that both teachers and learners may become bored in the classroom if the teaching is predictable and the learners encounter no surprises.

Finally, the teacher is the central actor in entrepreneurship education and the teachers’ role in defining the time, frequency, contents and methods of entrepreneurship education is decisive. (Fiet, 2001a; Jones, 2010; Lobler, 2006; Seikkula-Leino, Rusko- vaara, Ikavalko, Mattila, \& Rytkola, 2010; Ruskovaara \& Pihkala, 2013).

% Teachers’ gender and entrepreneurship education practices.
Even though we have found studies with a feminist approach to entrepreneurship edu- cation and studies concerning women entrepreneurs (Komulainen, Keskitalo-Foley, Korhonen, \& Lappalainen, 2010; Korhonen, 2012), their findings do not show any indications of differences or similarities between women and men. According to Bennett (2006), a lecturer’s gender does not play a significant role in inclining entrepreneur- ship education. On this basis, we formulate the following proposition:

Proposition 1: Entrepreneurship education practices do not differ between male and female teachers.

% Teachers’ business enterprise background’s positive effect on entrepreneurship education.

Proposition 2: The stronger the teacher’s business back- ground is, the more he/she is bound to execute entre- preneurship education.

Proposition 3: The more the teacher has work experience, the more he/she is inclined to conduct entrepreneurship education.

Proposition 4: Entrepreneurship education differs between education levels.

Proposition 5: Enterprise-related teacher training positively affects teachers’ entrepreneurship education practices.

% Viktigt för YoungDrive, såklart!
Furthermore, the teacher’s professional teaching experience has no sig- nificance in terms of entrepreneurship education. These findings suggest that as a competence area, entrepreneur- ship education is not dependent on the teacher’s experi- ence as a teacher.

% Elena Ruskovaara is working as a researcher, lecturer, and project manager at Lappeenranta University of Tech- nology. For the past 15 years, she has worked in the field of further education for teachers and has been involved in many national and international entrepreneurship proj- ects. Her main interests are entrepreneurship education and especially the challenges of measuring and evaluating entrepreneurship education.

A large number of useful methods and practices have been discov- ered (Seikkula-Leino, 2007), and training concerning dif- ferent pedagogical solutions could be of great value. For example, the playful side of teaching and learning (Solo- mon, 2007) as well as teacher training that develops the competences of a mentor, enabler, or coach should enhance entrepreneurship education practices. When shifting the focus from Gibb’s (2002) idea of developing students’ understanding of entrepreneurship to the teacher, what are the ways for a teacher to see, feel, do, think, and learn entrepreneurship? How to provide teachers with the skills to cope with, create, and perhaps enjoy uncertainty and complexity?

Dickson et al. (2008) found that entrepreneurship education correlates positively with entrepreneurial activity, but admit the challenges of the long time span between the educational experience and the actual entrepreneurial behavior that follows. This, together with other findings, shows a great need for longi- tudinal research.

% Refer to entrepreneurship definition}

%Kuratko, D. F. (2005). The emergence of entrepreneurship education: Development, trends, and challenges. Entrepreneurship theory and practice, 29(5), 577-598.

%Pittaway, L., & Cope, J. (2007). Entrepreneurship education a systematic review of the evidence. International Small Business Journal, 25(5), 479-510.

%Bae, T. J., Qian, S., Miao, C., & Fiet, J. O. (2014). The relationship between entrepreneurship education and entrepreneurial intentions: A meta‐analytic review. Entrepreneurship Theory and Practice, 38(2), 217-254.

Entrepreneurship education has been a growing field of investigation over the last three decades. While \cite{dickson} says there are few empirical studies available, examples include among others \cite{kuratko}, \cite{pittaway} and \cite{bae}. \cite{oviawe} and \cite{iakovleva} has had an interest in interventions for teaching and learning entrepreneurship in the developing world.

% OVIAWE, M. J. I. (2010). Repositioning Nigerian youths for economic empowerment through entrepreneurship education. European Journal of Educational Studies, 2(2).

%Iakovleva, T., Kolvereid, L., & Stephan, U. (2011). Entrepreneurial intentions in developing and developed countries. Education+ Training, 53(5), 353-370.

First, \cite{oviawe} conclude by how teaching of creativity and problem-solving skills seems to be especially beneficial for entrepreneurship in developed countries. In YoungDrive, the youth are tasked with starting their own business from no capital, which fosters creativity and problem-solving skills.

Further, \cite{iakovleva} indicated that respondents from developing countries do have stronger entrepreneurial intentions than those from developed countries. This stems from attitudes, subjective norms, and perceived behavioural control. Their encouragement, is that developing countries need to focus on the development of institutions that can support entrepreneurial efforts. YoungDrive is one such example.

\cite{ruskovaara} concludes, that the teacher seems to be the main factor for entrepreneurship education, and that research agrees with them. There seems to be no indication of difference between men and women, nor previous professional teaching experience. They could find that entrepreneurial activity seems to lead to better entrepreneurship education. \cite{dickson} recommends mainly two things for enhancing entrepreneurship education practices. First, the playful side of teaching and learning is mentioned. Secondly, they encourage teacher training that develops the competences as a mentor, enabler or coach.


    \subsection{Digital Education}

    In recent times, e-learning has had a tremendous impact both outside and inside the classroom. With a growing teacher interest, research so far shows that digital education is hard, risky and possibly rewarding. \citep{luckin} Thus, digital education shows both great potential and great considerations.

    \subsubsection{Brining research into reality}

    Clark \citep{gates} has done a comprehensive study, which motivates why a digital tool or game is a good thing by showing a .33 standard deviations in intrapersonal learning outcomes, relative to non-game instructional conditions. They also conclude, that design rather than medium alone predicts learning outcomes.

    Much of the research to date on digital games has focused on proof-of-concept studies and media comparisons. The study's encouragement, is to focus on how theoretically-driven decisions influence learning outcomes: for the broad diversity of learners, within and beyond the classroom. Literature that have looked at mobile apps for learning specifically, are Godwin-Jones \citep{godwin-jones} and Page \citep{page}.

    \subsubsection{Caring for the context}
    Luckin \citep{luckin} emphasises the need to care for the context. Stickdorn \citep{stickdorn} exemplifies how the design process should be altered when the context is social innovation.

    Service design in a social innovation context is called "social design", and is a new field. \citep{stickdorn}. No longer is service design solely focused on creating and promoting consumer goods, but to offer services to society. The design process should be designed to tackle a social issue, or with the intent to improve human lives. The focus is on delivering positive impact.

    \subsubsection{E-assessment}
    There are numerous examples of developments in e-assessment using mobile environments, as well as immersive environments and social and collaborative environments.

    Interest in formative e-assessment is increasing. A large amount of development has taken place on diagnostic testing environments, that allow teachers and learners to assess present performance against prior performance. \citep{luckin} For example, it has been shown that multiple-choice tests in e-assessment can be used to good effect \citep{nicol}.

    %Nicol, D. (2007). E‐assessment by design: using multiple‐choice tests to good effect. Journal of Further and higher Education, 31(1), 53-64.

    Luckin says that further consideration should  be given to how technology can be used to enable the assessment of knowledge and skills not usually distinguished within current curricula. \citep{luckin} One such example would be entrepreneurship.

    \subsection{Hybrid App Development}

The history of app and web development is rich and increasingly intertwined. First, websites were developed for desktop only, and when smartphones became popular, they were made responsive.

With today's possibilities of native mobile development or developing a native app using web technologies, there are numerous viable alternatives available if an app should function on several devices, depending on budget and preferences.

One of the main argument for developing an app in web technologies, is that the whole application, including the server, can be written in one programming language, JavaScript (full-stack).

Tools such as Apache Cordova can compile JavaScript applications into native apps. Thus, they can appear on Apple iOS and Android Play Store, as well as on the web, or installable offline on a smartphone from the computer.

JavaScript is developing rapidly as a language, as well as its ecosystem of frameworks and tools. Frameworks have emerged and matured, like Meteor.js, which makes building full-stack applications in JavaScript reliable and fast.

Previously, web hosting has been troublesome for JavaScript server applications. Today, tools such as Meteor.js and Heroku have introduced free and paid hosting for such applications, with smart bindings to code platforms such as GitHub, which makes collaboration and version handling easy.


