\section{Research Questions}
% beskrivning av vilka frågeställningar arbetet syftar till att besvara

The overall aim of the study is to create and apply a design process of an application for entrepreneurial learning, to be implemented in a developing country context. In response, the following specific research questions were raised:

\begin{enumerate}
    \item How is the development affected by the technical possibilities?

    \item How is the design affected by the contextual  constraints, e.g. young entrepreneurs, entrepreneurship education, and culture? % such as X, Y, Z, \Å?

    % \item How can service design be used in a developing world context when building an app?
    % \item What should the design and development process look like

    \item How can test questions be developed to support entrepreneurship learning? % Bloom

    \item How does design affect usability and learning done via the app? %How effective is the app for training entrepreneurs? % (= evaluation) % how effective is the app in training entrepreneurs? % Reviderat från: "How much more effective becomes the coach in training entrepreneurs via the app?"

    \item How can users' feedback be used to inform modifications of the app?

    %\begin{enumerate}
	%\item  % of an app with the purpose to train entrepreneurs? % What principles will inform the design of the app?
\end{enumerate}
%\end{enumerate}

% --

% Innovation is how you connect different parts into something new, says Peter Gahnström, LiU Innovation.

% App development in place in Uganda with the target group, combined with service design, means that I have innovation height. The result using this methodology may be an interesting contribution to the field of service design within a development context.

% The method I am using becomes the scalability of the method, it may be possible to repeat. When I get statistics that my method works, it may be possible to a lot more, and other organizations (e.g. innovation offices at universities) might be able to use my results.
