\subsection{A Working Future, Plan Uganda}

"With funding and technical support from SIDA, Plan International Sweden and Accenture Development Partnership (ADP), Plan International Uganda is implementing A Working Future – Uganda (AWF), a three and a half year project intended to support employment and improve the economic empowerment of youth, with the targeted age group being 12,000 15 to 25 year-old young people.  Savings Groups are used as an entry point to communities, a mechanism for organising youth and a platform for financial education and capital build-up.

To help youth move into the local economy, AWF provides practical entrepreneurship training and post-training mentoring.  The greatest impact has been on individual income generating activities where youth have undertaken a new activity or diversified or somehow improved existing activities.  An informal inventory of activities revealed a good variety of businesses in small trade, buying and selling agricultural produce, food processing and some animal-raising.  Many of these are the result of market research rather than youth just copying activities already existing in their communities.

A lot of these businesses can be started on a modest scale with small amounts of capital that youth can access from their Savings Groups.  Most members seem to have a plan for scaling up and show great discipline in reinvesting profits to build a larger capital base.  One field assessment showed that many had doubled or multiplied their individual investment several times in a two to four month period.

Forging relations with private sector businesses to help youth take advantage of markets outside their communities is another feature of AWF.  A micro franchise relationship has grown between a large manufacturer of household products and Savings Groups members who serve as sales agents in their communities.  Other marketing and technical links exist with a large commercial producer of poultry and pigs and a distributor of solar lighting products."
% https://plan-international.org/youth-savings-group-uganda
% http://www.sida.se/English/where-we-work/Africa/Uganda/examples-of-results/A-working-future/
% http://comvisug.org/
% http://www.ugandangodirectory.org/index.php/Microfinance/386-community-vision

% http://awards.ixda.org/entry/2014/ledger-link/
% Craft
%"We used paper prototypes to co-design with the members and learn how the group would use the app during meetings. We encouraged them to modify the low-fi interface and work with us to make sure it would suit their needs.
%There are nearly 40 languages in Uganda; creating many local-language versions of the app wasn’t an option. As is common in Uganda, financial transactions often make use of English phrases, and each group included some members who speak basic English, so we created the app in English. However, members often speak what they called “village English,” which required us to work closely with them to choose appropriate labels.
%The application is only part of the offer; to provide a new technology, the team could not rely on individuals discovering the application themselves. We first tested to determine whether groups understood the need for a sophisticated phone (they did); and whether they would be interested in purchasing this phone (they were).
%Training and support processes are just as important as the app. To successfully train groups, we leveraged Grameen Foundation’s experience instructing poor farmers to use technology and worked through community organizations who were trusted by the groups.

% http://www.comvisug.org/AWFU-achievements.php
%  To enhance safety of the group savings through alternative saving measures, COMVIS in partnership with PLAN Uganda have brought on board GRAMEENand AIRTEL companies with a product of AIRTEL WEZA where groups save their money in a group Simcard. Both the Youth Mentors and CBTs were trained on how these transactions are to be done.