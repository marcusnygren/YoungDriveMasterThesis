\subsection{UCLA Learning \& Forgetting Lab}

\subsubsection{New theory of disuse}
Finally, sometimes people cannot recall information at one point in time, but can recall it later. In looking at these situations, it seems that our memories work in strange and unpredictable ways. The function of our memories, however, may be predictable. The New Theory of Disuse (R. A. Bjork \& E. L. Bjork, 1992) posits that there are two indices of memory strength: storage strength (SS) and retrieval strength (RS).

Storage strength is how well learned something is;

retrieval strength is how accessible (or retrievable) something is.

The New Theory of Disuse postulates that RS and SS interact in interesting ways. For example, the more SS information has, the bigger the boost in RS it will receive as a consequence of restudy (e.g., you can relearn a childhood address much more quickly than a new address, even if you can't recall either of them initially). The more RS that information has, the smaller the boost in SS as a consequence of restudy (e.g., simply repeating to-be-learned information over and over again is not very helpful for really learning that information because you are probably practicing rote rehearsal, without forming the deeper connections necessary to improve long-term retention).

\subsubsection{How we learn versus How we think we learn: Desirable Difficulties in Theory and Practise}

\textbf{Introduction to Desirable Difficulties}

Imagine a scenario in which a teacher has students practice different examples of a single type of math problem for an hour in class. By the end of the hour, it may seem—both to the teacher and to the students—that this type of math problem has been mastered. On a test two weeks later, however, the benefit may not be evident. In fact, much to the dismay of the teacher and the students, performance during training is not always representative of long-term learning. In contrast to the story told above, in which an easy training method was followed by poor performance later, imagine that the teacher had interleaved many different types of problems during in-class training drills.

There are, in fact, certain training conditions that are difficult and appear to impede performance during training but that yield greater long-term benefits than their easier training counterparts. R. A. Bjork (1994) dubbed these difficult but effective training conditions desirable difficulties.

Other examples of desirable difficulties (explained in greater detail in later portions of this webpage) include spacing rather than massing repetitions of to-be-learned information (R. A. Bjork \& Allen, 1970; for review, see Cepeda, Pashler, Vul, Wixted, \& Rohrer, 2006), testing rather than re-studying information (Halamish \& R. A. Bjork, 2011; for review, see Roediger \& Karpicke, 2006), and varying the conditions of practice instead of keeping them constant.

\subsubsection{Spacing}
It is common sense that when we want to learn information, we study that information multiple times. The schedules by which we space repetitions can make a huge difference, however, in how well we learn and retain information we study. The spacing effect is the finding that information that is presented repeatedly over spaced intervals is learned much better than information that is repeated without intervals (i.e., massed presentation). This effect is one of the most robust results in all of cognitive psychology and has been shown to be effective over a large range of stimuli and retention intervals from nonsense syllables (Ebbinghaus, 1885) to foreign language learning across many months (Bahrick, Bahrick, Bahrick \& Bahrick, 1993).

reducing the accessibility of information in memory fosters additional learning of that information (see New Theory of Disuse). In this view, increasing the spacing between learning trials enhances learning because it decreases accessibility of the to-be-learned information.

\textbf{Indeed, R. A. Bjork and Allen (1970) found that increasing the difficulty of an intervening distractor task without changing the duration of the spacing interval leads to improved learning, much like a spacing effect.}

accessibility increased, rather than decreased, with delay— R. A. Bjork, Kornell and Cheung (2009) reversed the spacing effect

This theory of spacing would predict that an optimal learning schedule is one with expanding retrieval practice, rather than equally spaced practices. With successive practices, information is better learned and becomes inaccessible more slowly. As the greatest learning occurs when information accessibility is low (but not impossible), increasingly longer lags between retrieval practices should lead to better long-term learning. Indeed, Landauer and R. A. Bjork (1978) showed that expanding retrieval is better than uniform retrieval in two well-controlled experiments.

\subsubsection{Interleaving}
Spacing is one of the most robust, effective ways of improving learning. However, spacing calls for intervals of time in between repetitions, and this may not be the most efficient use of time. Imagine you have three final exams to study for. If you were to space out study of three whole courses, you might as well start your course review before the quarter even begins! Particularly when one has several different things to learn, an effective strategy is to interleave one's study: Study a little bit of history, then a little bit of psychology followed by a chapter of statistics and go back again to history. Repeat (best if in a blocked-randomized order).

Interleaving benefits not only memory for what is studied, but also leads to benefits in the transfer of learned skills (e.g. Carson \& Wiegand, 1979). The theory is that interleaving requires learners to constantly "reload" motor programs (in the case of motor skills) or retrieve strategies/information (in the case of cognitive skills) and allows learners to extract more general rules that aid transfer.

Interleaving is to shuffle the exemplars of a given category between exemplars of other similar categories. Interleaving however has the effect of both creating temporal spacing between exemplars of a given category and creating temporal juxtaposition of exemplars of different categories.

Kornell, Birnbaum, Bjork, and Bjork (in preparation) investigated these separate effects and found that \textbf{temporal juxtaposition, which allows for discrimination processes, is more critical to inductive learning than simple temporal spacing is.}
% So I should shuffle the questions, and categories.

\textbf{OBS:} Additionally, although spacing alone and interleaving alone offer benefits to inductive learning performance (both desirable difficulties, combining the two manipulations does not provide an additive benefit to inductive learning performance (see Interactions between Desirable Difficulties).

\textbf{Perceptual desirable difficulties}
Fluency, or the subjective ease of processing information, can provide learners with a useful basis for judging how well information has been understood. Perceptual variations are among the most obvious--and, sometimes, the most misleading—cues to the fluency of information. For example, when you encounter fonts that are difficult to read or words in very small print, you may experience a sense of disfluency—that is, you may have a feeling that the unusual or small typefaces are more difficult to process than more common typefaces.

These types of perceptual cues often cause people to think that disfluent information will be harder to remember than fluent information. Some research, however, indicates that perceptual disfluency can be a desirable difficulty (Diemand-Yauman, Oppenheimer, \& Vaughan, 2010). The subjective difficulty of processing disfluent information can actually lead people to engage in deeper processing strategies, which then results in higher recall for those items (Alter, Oppenheimer, Epley, \& Eyre, 2007).
\todo{Detta är också nyttig insikt}

\textbf{Interactions between desirable difficulties}
It is not enough to simply provide educators with a list of desirable difficulties and claim that our work in optimizing learning has been completed. It may be that certain combinations of desirable difficulties interact to yield super-additive or sub-additive effects, if the processes by which the desirable difficulties work enhance or interfere with each other. Research into these interactions is therefore important on both practical ("What is best for learning?") and theoretical ("How do these desirable difficulties work to influence memory?") levels.

%Kritik:
At longer spacing intervals, variation actually hurts memory performance.

Furthermore, the results of this study supported the study-phase retrieval theory of spacing over the encoding variability theory of spacing (see Spacing).

Current studies in this line of research include examining the interactions between spacing and generation and spacing and variation with educationally relevant materials, such as text passages and glossary-style definitions. Interactions are also being explored with inductive learning, and Kornell, Birnbaum, Bjork \& Bjork (in preparation) have found a subadditive effect of spacing and interleaving in the induction of butterfly species (see Interleaving).

\subsubsection{Learning concepts and catagories}
Kornell and R. A. Bjork (2008) found that while many students believe that studying examples of a category all at once leads to more effective inductive learning than studying them intermixed with examples of other to-be-learned categories, the opposite is in fact true (see Interleaving).

\subsubsection{Goal-Directed Forgetting: Directed forgetting}
In the typical list-based directed forgetting paradigm (E. L. Bjork \& R. A. Bjork, 1996), a participant will study two lists of words, and is notified after each list whether or not it will be tested later on. If a list is tested after the learner was notified that it would not be tested, the learner will show weaker recall for that list, compared to a baseline condition in which all lists are expected to be tested, demonstrating the costs of directed forgetting. Interestingly, it is commonly found that recall of any list that was expected to be tested will be greater than that of the baseline condition, demonstrating the unexpected benefits of directed forgetting.
\todo{This suggests that if Josefina says to the students that X will be tested, people will pay more attention}

\subsubsection{Metacognition}
Metacognition refers to the subjective awareness of one's own knowledge. In the Bjork Learning and Forgetting Lab, we focus on how this relates to learning and study behaviors, or metamemory. Metamemory consists of both monitoring the state of one's memory (or lack thereof) as well as using this information to control study decisions. Accurate metamemory can be crucial for a student in determining the success of his or her own study program.

\textbf{Fluency and biases}
(students) instead seem to assume that what they know at the time the judgment is made is an accurate reflection of what they will know at a later time point.

\textbf{Awareness of study strategies}
 Zechmeister and Shaughnessy (1980) showed that people tend to think massed repetitions are more effective than spaced repetitions. Massed repetitions lead to greater short-term performance, but impair long-term performance (e.g., Simon \& Bjork, 2001); this dissociation could explain why people think massed repetitions are more effective. Kornell and R. A. Bjork (2008) found that in the case of inductive learning, the belief that massed presentation is better than spaced presentation holds true even after people have taken a test and have done better with spacing! At the same time, other recent work (e.g., Benjamin \& Bird, 2006; Toppino \& Cohen, 2010) has found evidence that people do prefer later re-study to immediate re-study when allowed to choose between the two.
 \todo{Detta är också nyttigt för mig att känna till}