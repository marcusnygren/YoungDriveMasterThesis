    %\subsection{The Client: YoungDrive}

%In this section, the project that the \textit{client} YoungDrive is in is first described, and then how YoungDrive  fit into the structure of the thesis with its entrepreneurship education program. In the last part, future plans of YoungDrive and A working future is presented, giving relevancy to the field of digital education.

%\subsubsection{The Project: A Working Future}

%Plan International works towards eliminating child poverty, and their project A working future, supported by SIDA since the year 2012 until 2016, tackles unemployment among youth in rural areas. The project runs for 12 000 youth in Kamuli and Tororo.

%\subsubsection{The Structure: Youth Savings Groups with Trainings}

%Because of high tuition fees, saving (financial literacy) and earning (practicing vocational skills) are central. VSLA (Village Savings and Loan Associations) groups have existed for many years, where a group starts a village savings and loans group together. A democratic process makes the group independent of banks, which rates are in general high and which may not even borrow money, either because of long distances to the bank or of no previous financial history.

%For Plan International, VSLA groups have been successful in several countries for a long time. However, while the groups were skilled with saving, they did not always spend the money in the most strategic way.

%Plan's pilot with A working future, was to introduce trainings on top of the VSLA structure. Where CBTs (Community Based Trainers) were previously only responsible for hosting the groups, not they were trained and tasked with carrying out different programs: like agriculture, financial literacy, and the most successful program to date, focusing on running own businesses, YoungDrive.
