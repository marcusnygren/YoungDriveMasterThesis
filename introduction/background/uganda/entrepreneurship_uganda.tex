\subsection{Social Innovation and Social Entrepreneurship in Uganda} % https://www.linkedin.com/pulse/social-innovation-entrepreneurship-uganda-why-mobile-services-nissar?trk=prof-post

    This section will present background on working with mobile learning platforms, and understanding the society of entrepreneurs in Uganda.

    \subsubsection{Why Uganda is the world's most entrepreneurial country}
    According to Nissar \citep{nissar}, some facts related to entrepreneurship in Uganda are:

    \begin{itemize}
      \item Uganda is the world's most entrepreneurial country. (28\% of the population are entrepreneurs)
        \item Uganda has the second youngest population in the world (77\% of all Ugandans are below 30)
        \item Uganda has a very high unemployment rate (64 \% of people between 18–30 are unemployed)
    \end{itemize}

    % Ytterligare beskrivning av land: http://www.sun-connect-news.org/countries/uganda/

    With a high unemployment rate and little or none social security, starting a business is for many young entrepreneurs simply a tool for survival. But tough conditions can also lead to creativity, and there are as well many innovative entrepreneurs with great ideas and the aim to create positive social impact.

    As Mitchel says about entrepreneurship \citep{mitchel}, the motivation of entrepreneurship does not need to be solely wealth accumulation anymore. The activity of entrepreneurship contributes to society, in a way that is not captuted by the commercial entrepreneurship literature.

    No matter the reason of starting a business, Uganda's many entrepreneurs are contributing to the national society by boosting the economy and creating new jobs.

    \subsubsection{Why mobile services are growing rapidly in Uganda}
    One of the reasons is that the country has invested heavily in communication networks, even connecting remote rural villages with fibre optic cables and thereby connecting them to a world of information.

    As much as 65\% of the adults in Uganda owns a cell phone, which has allowed many areas in the country to skip the landline stage of development and jump right to the digital age.

    For those who hasn’t electricity at home, there are plentiful of charging booths for mobiles all over the country.

    \subsubsection{Mobile services and social innovations}
    The wide use of mobile phones has lead the way for the development of several innovative mobile services and in many cases the mobile service are way ahead of us  \citep{nissar}. In Sweden mobile banking services that allows us to transfer money through our mobile phones were made popular with Swish, introduced in 2012. In Kenya people have had similar services for the last 10 years.
