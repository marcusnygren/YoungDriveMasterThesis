\subsection{Entrepreneurship in Uganda} % https://www.linkedin.com/pulse/social-innovation-entrepreneurship-uganda-why-mobile-services-nissar?trk=prof-post

    %This section will present background on working with mobile learning platforms, and understanding the society of entrepreneurs in Uganda.

    %\subsubsection{Why Uganda is Ranked the World's Most Entrepreneurial Country}
    According to Nissar \citep{nissar}, some facts related to entrepreneurship in Uganda are:

    \begin{itemize}
      \item Uganda is the world's most entrepreneurial country. (28\% of the population are entrepreneurs)
        \item Uganda has the second youngest population in the world (77\% of all Ugandans are below 30)
        \item Uganda has a very high unemployment rate (64 \% of people between 18–30 are unemployed)
    \end{itemize}

    % Ytterligare beskrivning av land: http://www.sun-connect-news.org/countries/uganda/

    With a high unemployment rate and little or none social security, starting a business is for many young entrepreneurs simply a tool for survival. But tough conditions can also lead to creativity, and there are as well many innovative entrepreneurs with great ideas and the aim to create positive social impact.

    As Mitchel says about entrepreneurship \citep{mitchel}, the motivation of entrepreneurship does not need to be solely wealth accumulation anymore. The activity of entrepreneurship contributes to society, in a way that is not captuted by the commercial entrepreneurship literature. No matter the reason of starting a business, Uganda's many entrepreneurs are contributing to the national society by boosting the economy and creating new jobs.

    \subsection{Entrepreneurship education}

%\subsubsection{Definition}

Från artiklen "The role of Service Design in the Effectual Journey of Social Entrepreneurs" % http://www.servdes.org/conference-2014-lancaster/:

Social Entrepreneurship Background
Entrepreneurship research aims at a better understanding of the highly heterogeneous process phenomenon of entrepreneurship. The term entrepreneur evolved from a French term meaning “one who undertakes or manages” and was used in the 1800s by a French economist to capture the activity of someone who creates value by “shifting economic resources out of an area of lower and into an area of higher productive and greater yield” (Martin \& Osberg, 2009, p. 31). Although the field has been established as a distinct domain of research, there is still no consensus about the object of study in the field with the concept of entrepreneurship being reinterpreted constantly (Cornelius et al., 2006). Some persisting perspectives include a focus on facing uncertainty (Knight, 1921), on introducing new processes and products by innovating (Schumpeter, 1934) and recognizing opportunities (Kirzner, 1978).

Recently, the phenomenon of entrepreneurship is conceived as more multifaceted than in the past (Bruyat \& Julien, 2004) with researchers looking into its role in society and its social dimensions challenging the economic discourse that is dominating the field (Steyaert \& Katz, 2004). Some of the assumptions that stem from the association of the field with economics, for example the fact that motivation of entrepreneurs is mainly wealth accumulation do not appear appropriate (Mitchel et al., 2007) as entrepreneurship is increasingly identified as an activity that contributes to society in other significant ways that are not captured by the commercial entrepreneurship literature (Steyaert \& Katz, 2004).

%\subsubsection{Entrepreneurship Education}

"Entrepreneurship Education in Schools: Empirical Evidence on the Teacher’s Role" says that "The findings indicate that the training teachers have received in entrepreneurship seems to be the main factor determining the observable entrepreneurship education provided by the teachers."

%\citep{entrepreneurship-pihkala}

In this study, we aimed to bring empirical data into the discussion on entrepreneurship education, as there are still few empirical studies available on the topic area (Dickson et al., 2008),

Also assessment practices that include peer and self-assessment have brought new depth into assignments and their completion. Similarly, activity outside of the classroom (Fayolle \& Gailly, 2008; Kickul et al., 2010; Shepherd, 2004; Solomon, 2007) is stated to have widened learners’ perceptions of their possibilities to be active citi- zens, and to also have clarified the role of different actors in society. In addition to the above, Rae and Carswell (2001) utilize entrepreneurship cases to analyze how the self-confidence and self-awareness of learners have grown.
Fiet (2001a) presented a group of methods and argues that both teachers and learners may become bored in the classroom if the teaching is predictable and the learners encounter no surprises.

Finally, the teacher is the central actor in entrepreneurship education and the teachers’ role in defining the time, frequency, contents and methods of entrepreneurship education is decisive. (Fiet, 2001a; Jones, 2010; Lobler, 2006; Seikkula-Leino, Rusko- vaara, Ikavalko, Mattila, \& Rytkola, 2010; Ruskovaara \& Pihkala, 2013).

% Teachers’ gender and entrepreneurship education practices.
Even though we have found studies with a feminist approach to entrepreneurship edu- cation and studies concerning women entrepreneurs (Komulainen, Keskitalo-Foley, Korhonen, \& Lappalainen, 2010; Korhonen, 2012), their findings do not show any indications of differences or similarities between women and men. According to Bennett (2006), a lecturer’s gender does not play a significant role in inclining entrepreneur- ship education. On this basis, we formulate the following proposition:

Proposition 1: Entrepreneurship education practices do not differ between male and female teachers.

% Teachers’ business enterprise background’s positive effect on entrepreneurship education.

Proposition 2: The stronger the teacher’s business back- ground is, the more he/she is bound to execute entre- preneurship education.

Proposition 3: The more the teacher has work experience, the more he/she is inclined to conduct entrepreneurship education.

Proposition 4: Entrepreneurship education differs between education levels.

Proposition 5: Enterprise-related teacher training positively affects teachers’ entrepreneurship education practices.

% Viktigt för YoungDrive, såklart!
Furthermore, the teacher’s professional teaching experience has no sig- nificance in terms of entrepreneurship education. These findings suggest that as a competence area, entrepreneur- ship education is not dependent on the teacher’s experi- ence as a teacher.

% Elena Ruskovaara is working as a researcher, lecturer, and project manager at Lappeenranta University of Tech- nology. For the past 15 years, she has worked in the field of further education for teachers and has been involved in many national and international entrepreneurship proj- ects. Her main interests are entrepreneurship education and especially the challenges of measuring and evaluating entrepreneurship education.

A large number of useful methods and practices have been discov- ered (Seikkula-Leino, 2007), and training concerning dif- ferent pedagogical solutions could be of great value. For example, the playful side of teaching and learning (Solo- mon, 2007) as well as teacher training that develops the competences of a mentor, enabler, or coach should enhance entrepreneurship education practices. When shifting the focus from Gibb’s (2002) idea of developing students’ understanding of entrepreneurship to the teacher, what are the ways for a teacher to see, feel, do, think, and learn entrepreneurship? How to provide teachers with the skills to cope with, create, and perhaps enjoy uncertainty and complexity?

Dickson et al. (2008) found that entrepreneurship education correlates positively with entrepreneurial activity, but admit the challenges of the long time span between the educational experience and the actual entrepreneurial behavior that follows. This, together with other findings, shows a great need for longi- tudinal research.

% Refer to entrepreneurship definition}

%Kuratko, D. F. (2005). The emergence of entrepreneurship education: Development, trends, and challenges. Entrepreneurship theory and practice, 29(5), 577-598.

%Pittaway, L., & Cope, J. (2007). Entrepreneurship education a systematic review of the evidence. International Small Business Journal, 25(5), 479-510.

%Bae, T. J., Qian, S., Miao, C., & Fiet, J. O. (2014). The relationship between entrepreneurship education and entrepreneurial intentions: A meta‐analytic review. Entrepreneurship Theory and Practice, 38(2), 217-254.

Entrepreneurship education has been a growing field of investigation over the last three decades. While Dickson \citep{dickson} says there are few empirical studies available, examples include among others Kuratko \citep{kuratko}, Pittaway \citep{pittaway} and Bae \citep{bae}.

Especially relevant for this work is the recent interest in interventions for teaching and learning entrepreneurship in the developing world: \citep{oviawe} \citep{iakovleva}

% OVIAWE, M. J. I. (2010). Repositioning Nigerian youths for economic empowerment through entrepreneurship education. European Journal of Educational Studies, 2(2).

%Iakovleva, T., Kolvereid, L., & Stephan, U. (2011). Entrepreneurial intentions in developing and developed countries. Education+ Training, 53(5), 353-370.

First, Oviawe \citep{oviawe} conclude by how teaching of creativity and problem-solving skills seems to be especially beneficial for entrepreneurship in developed countries. In YoungDrive, the youth are tasked with starting their own business from no capital, which fosters creativity and problem-solving skills.

Further, Iakovleva \citep{iakovleva} indicated that respondents from developing countries do have stronger entrepreneurial intentions than those from developed countries. This stems from attitudes, subjective norms, and perceived behavioural control. Their encouragement, is that developing countries need to focus on the development of institutions that
can support entrepreneurial efforts. YoungDrive is one such example.

Ruskovaara and Pihkala \citep{ruskovaara} concludes, that the teacher seems to be the main factor for entrepreneurship education, and that research agrees with them. There seems to be no indication of difference between men and women, nor previous professional teaching experience. They could find that entrepreneurial activity seems to lead to better entrepreneurship education. Dickson's recommendations for enhancing entrepreneurship education practices were mainly two things. First, the playful side of teaching and learning is mentioned. Secondly, they encourage teacher training that develops the competences as a mentor, enabler or coach. \citep{dickson}


    \subsection{YoungDrive}

    YoungDrive is based on a Swedish concept, and had previously had a pilot in Botswana, when tasked with running the entrepreneurship module of A working future for Plan International. The organization fosters and educates young entrepreneurs in developing countries. They train the coaches, provide training material, and support the coaches via direction and direct support through co-project leaders and Youth Mentors.

    YoungDrive moves an entrepreneur to location, becoming country manager and "teacher". The teacher educates project leaders during four days, followed by educating coaches, which then roll out the training to the youth groups during 10 sessions, 1 session per week in average. The Community Based Trainers (CBTs) also rolls out other trainings, often simultaneously.

    For the future of YoungDrive, they want to make the CBT's even better, and collect and take use of data (monitoring and evaluation). Another motivation is scaling and monetization, as Plan International wants to increase the project to more countries, with an increased digital focus, and YoungDrive wants to be independent of project funding (i.e. a social enterprise).

    This was a great time to introduce digital enablers for YoungDrive, where there previously had been no technology-focus, especially towards CBT's and Youth Mentors. The master thesis is the first project which focuses on digital enablers for YoungDrive.
