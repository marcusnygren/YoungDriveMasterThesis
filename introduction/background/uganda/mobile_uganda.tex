\subsection{Mobile Technology in Uganda's Rural Areas}\label{sec:mobile-uganda}

One of the reasons why mobile services are growing rapidly in Uganda is that the country has invested heavily in communication networks, even connecting remote rural villages with fibre optic cables and thereby connecting them to a world of information.

As much as 65\% of the adults in Uganda owns a cell phone, which has allowed many areas in the country to skip the landline stage of development and jump right to the digital age. For those who hasn’t electricity at home, there are available charging booths for mobiles all over the country.

The wide use of mobile phones in Uganda and other developing countries has lead the way for the development of several innovative mobile services and in many cases the mobile service are way ahead of us  \citep{nissar}. In Sweden mobile banking services that allows us to transfer money through our mobile phones were made popular with Swish, introduced in 2012. In the neighbouring country Kenya, people have had similar services for the last 10 years, and mobile money is since long also common in Uganda.

A prominent example of an app that has previously been developed with the target group in mind is Ledger Link \citep{ledgerlink}. This mobile banking service empowers, developed in partnership with a bank, allows saving groups in rural areas such as Tororo to save money remotely. It is developed with human-centered design methods, and has won several awards.

%For the future of YoungDrive, they want to make the CBT's even better, and collect and take use of data (monitoring and evaluation). Another motivation is scaling and monetization, as Plan International wants to increase the project to more countries, with an increased digital focus, and YoungDrive wants to be independent of project funding (i.e. a social enterprise). This was a great time to introduce digital enablers, where there previously had been no technology-focus, especially towards CBT's and Youth Mentors. The master thesis is the first project which focuses on digital enablers for YoungDrive.

%\subsection{Hybrid App Development}

The history of app and web development is rich and increasingly intertwined. First, websites were developed for desktop only, and when smartphones became popular, they were made responsive.

With today's possibilities of native mobile development or developing a native app using web technologies, there are numerous viable alternatives available if an app should function on several devices, depending on budget and preferences.

One of the main argument for developing an app in web technologies, is that the whole application, including the server, can be written in one programming language, JavaScript (full-stack).

Tools such as Apache Cordova can compile JavaScript applications into native apps. Thus, they can appear on Apple iOS and Android Play Store, as well as on the web, or installable offline on a smartphone from the computer.

JavaScript is developing rapidly as a language, as well as its ecosystem of frameworks and tools. Frameworks have emerged and matured, like Meteor.js, which makes building full-stack applications in JavaScript reliable and fast.

Previously, web hosting has been troublesome for JavaScript server applications. Today, tools such as Meteor.js and Heroku have introduced free and paid hosting for such applications, with smart bindings to code platforms such as GitHub, which makes collaboration and version handling easy.

