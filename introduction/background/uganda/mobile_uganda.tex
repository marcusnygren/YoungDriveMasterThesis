\subsection{Mobile in Uganda}

One of the reasons why mobile services are growing rapidly in Uganda is that the country has invested heavily in communication networks, even connecting remote rural villages with fibre optic cables and thereby connecting them to a world of information.

As much as 65\% of the adults in Uganda owns a cell phone, which has allowed many areas in the country to skip the landline stage of development and jump right to the digital age. For those who hasn’t electricity at home, there are plentiful of charging booths for mobiles all over the country.

The wide use of mobile phones in Uganda and other developing countries has lead the way for the development of several innovative mobile services and in many cases the mobile service are way ahead of us  \citep{nissar}. In Sweden mobile banking services that allows us to transfer money through our mobile phones were made popular with Swish, introduced in 2012. In the neighbouring country Kenya, people have had similar services for the last 10 years, and mobile money is since long also common in Uganda.

One great example of a mobile banking service that is a true social innovation is Ledger Link \citep{ledgerlink}, developed in collaboration with Barclays Bank. This mobile banking service empowers saving groups in rural areas to save money. It is developed with human-centered design methods, and has won several awards.

%For the future of YoungDrive, they want to make the CBT's even better, and collect and take use of data (monitoring and evaluation). Another motivation is scaling and monetization, as Plan International wants to increase the project to more countries, with an increased digital focus, and YoungDrive wants to be independent of project funding (i.e. a social enterprise). This was a great time to introduce digital enablers, where there previously had been no technology-focus, especially towards CBT's and Youth Mentors. The master thesis is the first project which focuses on digital enablers for YoungDrive.

\subsection{Mobile Learning}

    YoungDrive has asked for a multiple-choice question learning game used for assessment and improvement of the YoungDrive learning objectives during and after the training. In recent times, digital learning (e-learning) has had a tremendous impact both outside and inside the classroom. With a growing teacher interest, research still so far shows that digital education is hard, risky and possibly rewarding \citep{luckin}. Thus, digital education shows both great potential and great considerations. Much of the research on digital learning to date on digital games has focused on proof-of-concept studies and media comparisons \citep{luckin}. A study by \cite{gates} motivates why a digital tool or game is a good thing by showing an increase in intrapersonal learning outcomes, relative to non-game instructional conditions.

    %Luckin \citep{luckin} emphasises the need to care for the context when designing for learning. Stickdorn \citep{stickdorn} exemplifies how the design process should be altered when the context is social innovation. Service design in a social innovation context is called "social design", and is a new field. \citep{stickdorn}. No longer is service design solely focused on creating and promoting consumer goods, but to offer services to society. The design process should be designed to tackle a social issue, or with the intent to improve human lives. The focus is on delivering positive impact.

    A large amount of development has taken place on diagnostic testing environments, that allow teachers and learners to assess present performance against prior performance \citep{luckin}. There are numerous examples of developments in \textit{e-assessment} using mobile environments, as well as immersive environments and social and collaborative environments. One such example is the educational app platform iSchool, developed by iSchool Zambia \citep{ischool}. The app has been praised and made popular as it was designed to fit the Zambia school curriculum to the point, accessible as a home edition, pupil edition and teacher edition. Interest in formative e-assessment is increasing.  It has been shown that multiple-choice tests in e-assessment can be used to good effect \citep{nicol}.

    Two studies within electronic assessment (e-assessment) or mobile learning (m-learning) have been done that this master thesis is inspired by. One uses deliberate practices on a mobile learning environement \citep{yengin}. The other focused on and further validated the research of various experimental studies, that multiple-choice can be a viable auto-assessment method to improving student learning, especially for m-learning \citep{de-marcos}.

    %Nicol, D. (2007). E‐assessment by design: using multiple‐choice tests to good effect. Journal of Further and higher Education, 31(1), 53-64.

    Luckin says that further consideration should be given to how technology can be used to enable the assessment of knowledge and skills not usually distinguished within current curricula \citep{luckin}. \cite{gates} encourages a focus on how theoretically-driven decisions influence learning outcomes: for the broad diversity of learners, within and beyond the classroom. Combining these two, introducing e-assessment of entrepreneurship in a developing country context is a contribution to existing research.


%\subsection{Hybrid App Development}

The history of app and web development is rich and increasingly intertwined. First, websites were developed for desktop only, and when smartphones became popular, they were made responsive.

With today's possibilities of native mobile development or developing a native app using web technologies, there are numerous viable alternatives available if an app should function on several devices, depending on budget and preferences.

One of the main argument for developing an app in web technologies, is that the whole application, including the server, can be written in one programming language, JavaScript (full-stack).

Tools such as Apache Cordova can compile JavaScript applications into native apps. Thus, they can appear on Apple iOS and Android Play Store, as well as on the web, or installable offline on a smartphone from the computer.

JavaScript is developing rapidly as a language, as well as its ecosystem of frameworks and tools. Frameworks have emerged and matured, like Meteor.js, which makes building full-stack applications in JavaScript reliable and fast.

Previously, web hosting has been troublesome for JavaScript server applications. Today, tools such as Meteor.js and Heroku have introduced free and paid hosting for such applications, with smart bindings to code platforms such as GitHub, which makes collaboration and version handling easy.

