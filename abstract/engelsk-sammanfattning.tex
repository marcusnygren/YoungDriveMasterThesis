% Sammanfattningen ska vara informativ och kortfattat belysa arbetes syfte och metod. Den ska också innehålla de viktigaste resultaten och slutsatserna.
% Skriv en kort sammanfattning, men skriv fullständiga meningar och utforma den så att den kan läsas separat. Den får inte innehålla fakta eller uppgifter som inte finns med i själva rapporten. Sammanfattningen skriver du när rapporten är färdig.

Entrepreneurship educations in developing countries have not yet been able to take advantage of digital tools. The Ugandian non-profit YoungDrive has 60 coaches teaching entrepreneurship to 12 000 youth in rural areas. The coaches have a problem during and after their education with assessing and improving their abilities to learn and teach entrepreneurship. The purpose of this study was to investigate how an app can be designed to address this issue.

Methods within service design, agile development and interaction design has been used and combined to construct and analyse interviews, workshops, question sets, and app tests with the coaches in Uganda and Zambia. In total, three months were spent testing and iterating on low-detailed and high-detailed prototypes. The result is a launched hybrid app for Android, iOS and web.

A formative test shows coaches are more reliably correct using an improved design of multiple-choice questions than a standard multiple-choice design. Interviews shows the coaches has become more aware of what they know and do not know, and feels more confidence before their youth lesson with an increased quiz result. Further research should evaluate that the actual quality of the youth lesson improves.

Increasingly well-constructed multiple-choice questions with thoughtful feedback could stimulate creativity and problem-solving, deemed important by entrepreneurship education research. After overcoming usability issues, the final app could reach both low and high-order learning objectives within entrepreneurship.

The app did seemingly improve the quality of entrepreneurship education for the coaches in this specific developing world context. Further research should also investigate the design and implications of a digital-only entrepreneurship education for the coaches, having in mind that the teacher is believed the main factor of entrepreneurship education. As of now, the app is an effective compliment and assistance to the physical training.
