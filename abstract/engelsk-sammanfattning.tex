% Sammanfattningen ska vara informativ och kortfattat belysa arbetes syfte och metod. Den ska också innehålla de viktigaste resultaten och slutsatserna.
% Skriv en kort sammanfattning, men skriv fullständiga meningar och utforma den så att den kan läsas separat. Den får inte innehålla fakta eller uppgifter som inte finns med i själva rapporten. Sammanfattningen skriver du när rapporten är färdig.

Entrepreneurship educations in developing countries has not yet been able to use digital tools. The Ugandian non-profit YoungDrive has 60 coaches teaching entrepreneurship to 12 000 youth in rural areas. The coaches has a problem during and after their education with assessing and improving their abilities to learn and teach entrepreneurship. The purpose of this study was to investigate how an app could address this issue.

Applying relevant research in design, three months were spent testing and iterating on low-detailed and high-detailed prototypes with the coaches in Uganda and Zambia. Service design methodology needed to be adapted to fit the development of a digital service. The result is a launched hybrid app for Android, iOS and web.

A formative test shows coaches are more reliably correct using an improved design of multiple-choice questions than a standard multiple-choice design. Interviews shows the coaches has become more aware of what they know and do not know, and feels more confidence before their youth session with an increased quiz result. Further research should evaluate that the actual quality of the youth session improves.

Just like research suggests, usability issues will block effective learning if not removed. Removing these issues, the final app could reach both low and high-order learning objectives within entrepreneurship. Well-constructed multiple-choice questions with thoughtful feedback could stimulate creativity and problem-solving, deemed important by entrepreneurship education research.

The app did seemingly improve quality of entrepreneurship education in this specific developing world context. Further research should also investigate the design and implications of a 100\% digital entrepreneurship education for the coaches, having in mind that the teacher is believed the main factor of entrepreneurship education. As of now, the app is an effective compliment and assistance to the physical training.
